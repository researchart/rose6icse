\documentclass[titlepage,a4paper,12pt]{article}
\usepackage{config}
\usepackage{listings}
\usepackage{makeidx}
\usepackage{hyperref}
\begin{document}
\section{List of all functions}
\subsection{/+elltool}
\begin{enumerate}
\item \hyperlink{/+elltool/copyconf}{copyconf}
\fontfamily{pcr}
\selectfont
\begin{lstlisting}
% COPYCONF copies configuration confName to configuration toConfName
\end{lstlisting}
\fontfamily{\familydefault}
\selectfont
\item \hyperlink{/+elltool/editconf}{editconf}
\fontfamily{pcr}
\selectfont
\begin{lstlisting}
% EDITCONF edit configuration confName
\end{lstlisting}
\fontfamily{\familydefault}
\selectfont
\item \hyperlink{/+elltool/listconf}{listconf}
\fontfamily{pcr}
\selectfont
\begin{lstlisting}
% LISTCONF gives a list of existing configurations
\end{lstlisting}
\fontfamily{\familydefault}
\selectfont
\item \hyperlink{/+elltool/setconf}{setconf}
\fontfamily{pcr}
\selectfont
\begin{lstlisting}
% SETCONF selects the configuration confName as current
\end{lstlisting}
\fontfamily{\familydefault}
\selectfont
\end{enumerate}
\subsection{/+elltool/+conf}
\begin{enumerate}
\item \hyperlink{/+elltool/+conf/ConfRepoMgr}{ConfRepoMgr}
\fontfamily{pcr}
\selectfont
\begin{lstlisting}
% CONREPOMGR is analogue for elltool.test.configuration.AdaptiveConfRepoManager
% constructed to provide access for elltool.conf.Properties class to 
% local xml files, where information about properties is stored.
\end{lstlisting}
\fontfamily{\familydefault}
\selectfont
\end{enumerate}
\subsection{/+elltool/+conf/+test}
\begin{enumerate}
\item \hyperlink{/+elltool/+conf/+test/run\_tests}{run\_tests}
\fontfamily{pcr}
\selectfont
\begin{lstlisting}
%

\end{lstlisting}
\fontfamily{\familydefault}
\selectfont
\end{enumerate}
\subsection{/+elltool/+conf/+test/+mlunit}
\begin{enumerate}
\item \hyperlink{/+elltool/+conf/+test/+mlunit/PropertiesTestCase}{PropertiesTestCase}
\fontfamily{pcr}
\selectfont
\begin{lstlisting}
%  $Author: <Zakharov Eugene>  <justenterrr@gmail.com> $    $Date: <5 november> $
%  $Copyright: Moscow State University,
%             Faculty of Computational Mathematics and Computer Science,
%             System Analysis Department <2012> $
\end{lstlisting}
\fontfamily{\familydefault}
\selectfont
\end{enumerate}
\subsection{/+elltool/+conf/@ConfPatchRepo}
\begin{enumerate}
\item \hyperlink{/+elltool/+conf/@ConfPatchRepo/ConfPatchRepo}{ConfPatchRepo}
\fontfamily{pcr}
\selectfont
\begin{lstlisting}
%

\end{lstlisting}
\fontfamily{\familydefault}
\selectfont
\item \hyperlink{/+elltool/+conf/@ConfPatchRepo/patch\_001\_dummy\_patch}{patch\_001\_dummy\_patch}
\fontfamily{pcr}
\selectfont
\begin{lstlisting}
%

\end{lstlisting}
\fontfamily{\familydefault}
\selectfont
\item \hyperlink{/+elltool/+conf/@ConfPatchRepo/patch\_002\_add\_log4j}{patch\_002\_add\_log4j}
\fontfamily{pcr}
\selectfont
\begin{lstlisting}
%

\end{lstlisting}
\fontfamily{\familydefault}
\selectfont
\end{enumerate}
\subsection{/+elltool/+conf/@Properties}
\begin{enumerate}
\item \hyperlink{/+elltool/+conf/@Properties/Properties}{Properties}
\fontfamily{pcr}
\selectfont
\begin{lstlisting}
% PROPERTIES is a static class, providing emulation of static properties
% for toolbox.
\end{lstlisting}
\fontfamily{\familydefault}
\selectfont
\item \hyperlink{/+elltool/+conf/@Properties/parseProp}{parseProp}
\fontfamily{pcr}
\selectfont
\begin{lstlisting}
% PARSEPROP parses input into cell array with values of properties listed in
% neededPropNameList. Values are taken from args or, if there no value for
% some property in args, in current Properties.
\end{lstlisting}
\fontfamily{\familydefault}
\selectfont
\end{enumerate}
\subsection{/+elltool/+core/+test}
\begin{enumerate}
\item \hyperlink{/+elltool/+core/+test/run\_tests}{run\_tests}
\fontfamily{pcr}
\selectfont
\begin{lstlisting}
%

\end{lstlisting}
\fontfamily{\familydefault}
\selectfont
\end{enumerate}
\subsection{/+elltool/+core/+test/+mlunit}
\begin{enumerate}
\item \hyperlink{/+elltool/+core/+test/+mlunit/EllSecTCMultiDim}{EllSecTCMultiDim}
\fontfamily{pcr}
\selectfont
\begin{lstlisting}
%  $Author: Igor Samokhin, Lomonosov Moscow State University,
%  Faculty of Computational Mathematics and Cybernetics, System Analysis
%  Department, 31-January-2013, <igorian.vmk@gmail.com>$
%  $Copyright: Moscow State University,
%             Faculty of Computational Mathematics and Computer Science,
%             System Analysis Department 2012 $
\end{lstlisting}
\fontfamily{\familydefault}
\selectfont
\item \hyperlink{/+elltool/+core/+test/+mlunit/EllTCMultiDim}{EllTCMultiDim}
\fontfamily{pcr}
\selectfont
\begin{lstlisting}
%  $Author: Igor Samokhin, Lomonosov Moscow State University,
%  Faculty of Computational Mathematics and Cybernetics, System Analysis
%  Department, 31-January-2013, <igorian.vmk@gmail.com>$
%  $Copyright: Moscow State University,
%             Faculty of Computational Mathematics and Computer Science,
%             System Analysis Department 2013 $
\end{lstlisting}
\fontfamily{\familydefault}
\selectfont
\item \hyperlink{/+elltool/+core/+test/+mlunit/ElliIntUnionTCMultiDim}{ElliIntUnionTCMultiDim}
\fontfamily{pcr}
\selectfont
\begin{lstlisting}
%  $Author: Igor Samokhin, Lomonosov Moscow State University,
%  Faculty of Computational Mathematics and Cybernetics, System Analysis
%  Department, 31-January-2013, <igorian.vmk@gmail.com>$
%  $Copyright: Moscow State University,
%             Faculty of Computational Mathematics and Computer Science,
%             System Analysis Department 2013 $
\end{lstlisting}
\fontfamily{\familydefault}
\selectfont
\item \hyperlink{/+elltool/+core/+test/+mlunit/EllipsoidIntUnionTC}{EllipsoidIntUnionTC}
\fontfamily{pcr}
\selectfont
\begin{lstlisting}
%  $Author: Vadim Kaushanskiy, Moscow State University by M.V. Lomonosov,
%  Faculty of Computational Mathematics and Cybernetics, System Analysis
%  Department, 24-December-2012, <vkaushanskiy@gmail.com>$
\end{lstlisting}
\fontfamily{\familydefault}
\selectfont
\item \hyperlink{/+elltool/+core/+test/+mlunit/EllipsoidSecTestCase}{EllipsoidSecTestCase}
\fontfamily{pcr}
\selectfont
\begin{lstlisting}
%  $Author: Igor Samokhin, Lomonosov Moscow State University,
%  Faculty of Computational Mathematics and Cybernetics, System Analysis
%  Department, 02-November-2012, <igorian.vmk@gmail.com>$
%  $Copyright: Moscow State University,
%             Faculty of Computational Mathematics and Computer Science,
%             System Analysis Department 2012 $
\end{lstlisting}
\fontfamily{\familydefault}
\selectfont
\item \hyperlink{/+elltool/+core/+test/+mlunit/EllipsoidTestCase}{EllipsoidTestCase}
\fontfamily{pcr}
\selectfont
\begin{lstlisting}
%

\end{lstlisting}
\fontfamily{\familydefault}
\selectfont
\item \hyperlink{/+elltool/+core/+test/+mlunit/GenEllipsoidPlotTestCase}{GenEllipsoidPlotTestCase}
\fontfamily{pcr}
\selectfont
\begin{lstlisting}
%

\end{lstlisting}
\fontfamily{\familydefault}
\selectfont
\item \hyperlink{/+elltool/+core/+test/+mlunit/GenEllipsoidTestCase}{GenEllipsoidTestCase}
\fontfamily{pcr}
\selectfont
\begin{lstlisting}
%

\end{lstlisting}
\fontfamily{\familydefault}
\selectfont
\item \hyperlink{/+elltool/+core/+test/+mlunit/HyperplaneTestCase}{HyperplaneTestCase}
\fontfamily{pcr}
\selectfont
\begin{lstlisting}
%  $Author: <Zakharov Eugene>  <justenterrr@gmail.com> $    $Date: <31 october> $
%  $Copyright: Moscow State University,
%             Faculty of Computational Mathematics and Computer Science,
%             System Analysis Department <2012> $
\end{lstlisting}
\fontfamily{\familydefault}
\selectfont
\end{enumerate}
\subsection{/+elltool/+core/@GenEllipsoid}
\begin{enumerate}
\item \hyperlink{/+elltool/+core/@GenEllipsoid/GenEllipsoid}{GenEllipsoid}
\fontfamily{pcr}
\selectfont
\begin{lstlisting}
%  GENELLIPSOID - class of generalized ellipsoids
\end{lstlisting}
\fontfamily{\familydefault}
\selectfont
\item \hyperlink{/+elltool/+core/@GenEllipsoid/checkBigger}{checkBigger}
\fontfamily{pcr}
\selectfont
\begin{lstlisting}
%

\end{lstlisting}
\fontfamily{\familydefault}
\selectfont
\item \hyperlink{/+elltool/+core/@GenEllipsoid/dimension}{dimension}
\fontfamily{pcr}
\selectfont
\begin{lstlisting}
%

\end{lstlisting}
\fontfamily{\familydefault}
\selectfont
\item \hyperlink{/+elltool/+core/@GenEllipsoid/eq}{eq}
\fontfamily{pcr}
\selectfont
\begin{lstlisting}
%  EQ - compares two arrays of ellipsoids
\end{lstlisting}
\fontfamily{\familydefault}
\selectfont
\item \hyperlink{/+elltool/+core/@GenEllipsoid/findAllInfDir}{findAllInfDir}
\fontfamily{pcr}
\selectfont
\begin{lstlisting}
%

\end{lstlisting}
\fontfamily{\familydefault}
\selectfont
\item \hyperlink{/+elltool/+core/@GenEllipsoid/findBasRank}{findBasRank}
\fontfamily{pcr}
\selectfont
\begin{lstlisting}
%

\end{lstlisting}
\fontfamily{\familydefault}
\selectfont
\item \hyperlink{/+elltool/+core/@GenEllipsoid/findConstruction}{findConstruction}
\fontfamily{pcr}
\selectfont
\begin{lstlisting}
%

\end{lstlisting}
\fontfamily{\familydefault}
\selectfont
\item \hyperlink{/+elltool/+core/@GenEllipsoid/findDiffEaND}{findDiffEaND}
\fontfamily{pcr}
\selectfont
\begin{lstlisting}
%

\end{lstlisting}
\fontfamily{\familydefault}
\selectfont
\item \hyperlink{/+elltool/+core/@GenEllipsoid/findDiffFC}{findDiffFC}
\fontfamily{pcr}
\selectfont
\begin{lstlisting}
%

\end{lstlisting}
\fontfamily{\familydefault}
\selectfont
\item \hyperlink{/+elltool/+core/@GenEllipsoid/findDiffINFC}{findDiffINFC}
\fontfamily{pcr}
\selectfont
\begin{lstlisting}
%

\end{lstlisting}
\fontfamily{\familydefault}
\selectfont
\item \hyperlink{/+elltool/+core/@GenEllipsoid/findDiffIaND}{findDiffIaND}
\fontfamily{pcr}
\selectfont
\begin{lstlisting}
%

\end{lstlisting}
\fontfamily{\familydefault}
\selectfont
\item \hyperlink{/+elltool/+core/@GenEllipsoid/findMatProj}{findMatProj}
\fontfamily{pcr}
\selectfont
\begin{lstlisting}
%

\end{lstlisting}
\fontfamily{\familydefault}
\selectfont
\item \hyperlink{/+elltool/+core/@GenEllipsoid/findSpaceBas}{findSpaceBas}
\fontfamily{pcr}
\selectfont
\begin{lstlisting}
%

\end{lstlisting}
\fontfamily{\familydefault}
\selectfont
\item \hyperlink{/+elltool/+core/@GenEllipsoid/findSqrtOfMatrix}{findSqrtOfMatrix}
\fontfamily{pcr}
\selectfont
\begin{lstlisting}
%

\end{lstlisting}
\fontfamily{\familydefault}
\selectfont
\item \hyperlink{/+elltool/+core/@GenEllipsoid/getColorTable}{getColorTable}
\fontfamily{pcr}
\selectfont
\begin{lstlisting}
%

\end{lstlisting}
\fontfamily{\familydefault}
\selectfont
\item \hyperlink{/+elltool/+core/@GenEllipsoid/getIsGoodDirForMat}{getIsGoodDirForMat}
\fontfamily{pcr}
\selectfont
\begin{lstlisting}
%

\end{lstlisting}
\fontfamily{\familydefault}
\selectfont
\item \hyperlink{/+elltool/+core/@GenEllipsoid/inv}{inv}
\fontfamily{pcr}
\selectfont
\begin{lstlisting}
%  INV - create generalized ellipsoid whose matrix in pseudoinverse
%  to the matrix of input generalized ellipsoid
\end{lstlisting}
\fontfamily{\familydefault}
\selectfont
\item \hyperlink{/+elltool/+core/@GenEllipsoid/minkDiffEa}{minkDiffEa}
\fontfamily{pcr}
\selectfont
\begin{lstlisting}
%  MINKDIFFEA - computes tight external ellipsoidal approximation for
%  Minkowsky difference of two generalized ellipsoids
\end{lstlisting}
\fontfamily{\familydefault}
\selectfont
\item \hyperlink{/+elltool/+core/@GenEllipsoid/minkDiffIa}{minkDiffIa}
\fontfamily{pcr}
\selectfont
\begin{lstlisting}
%  MINKDIFFIA - computes tight internal ellipsoidal approximation for
%  Minkowsky difference of two generalized ellipsoids
\end{lstlisting}
\fontfamily{\familydefault}
\selectfont
\item \hyperlink{/+elltool/+core/@GenEllipsoid/minkSumEa}{minkSumEa}
\fontfamily{pcr}
\selectfont
\begin{lstlisting}
%  MINKSUMEA - computes tight external ellipsoidal approximation for
%  Minkowsky sum of the set of generalized ellipsoids
\end{lstlisting}
\fontfamily{\familydefault}
\selectfont
\item \hyperlink{/+elltool/+core/@GenEllipsoid/minkSumIa}{minkSumIa}
\fontfamily{pcr}
\selectfont
\begin{lstlisting}
%  MINKSUMIA - computes tight internal ellipsoidal approximation for
%  Minkowsky sum of the set of generalized ellipsoids
\end{lstlisting}
\fontfamily{\familydefault}
\selectfont
\item \hyperlink{/+elltool/+core/@GenEllipsoid/plot}{plot}
\fontfamily{pcr}
\selectfont
\begin{lstlisting}
%  PLOT - plots ellipsoids in 2D or 3D.
\end{lstlisting}
\fontfamily{\familydefault}
\selectfont
\item \hyperlink{/+elltool/+core/@GenEllipsoid/rho}{rho}
\fontfamily{pcr}
\selectfont
\begin{lstlisting}
%

\end{lstlisting}
\fontfamily{\familydefault}
\selectfont
\end{enumerate}
\subsection{/+elltool/+cvx}
\begin{enumerate}
\item \hyperlink{/+elltool/+cvx/CVXController}{CVXController}
\fontfamily{pcr}
\selectfont
\begin{lstlisting}
%

\end{lstlisting}
\fontfamily{\familydefault}
\selectfont
\end{enumerate}
\subsection{/+elltool/+demo/+test}
\begin{enumerate}
\item \hyperlink{/+elltool/+demo/+test/run\_tests}{run\_tests}
\fontfamily{pcr}
\selectfont
\begin{lstlisting}
%

\end{lstlisting}
\fontfamily{\familydefault}
\selectfont
\end{enumerate}
\subsection{/+elltool/+demo/+test/+mlunit}
\begin{enumerate}
\item \hyperlink{/+elltool/+demo/+test/+mlunit/BasicTestCase}{BasicTestCase}
\fontfamily{pcr}
\selectfont
\begin{lstlisting}
%

\end{lstlisting}
\fontfamily{\familydefault}
\selectfont
\item \hyperlink{/+elltool/+demo/+test/+mlunit/ETManualTC}{ETManualTC}
\fontfamily{pcr}
\selectfont
\begin{lstlisting}
%

\end{lstlisting}
\fontfamily{\familydefault}
\selectfont
\end{enumerate}
\subsection{/+elltool/+doc}
\begin{enumerate}
\item \hyperlink{/+elltool/+doc/collecthelp}{collecthelp}
\fontfamily{pcr}
\selectfont
\begin{lstlisting}
%  COLLECTHELP collects helps of m files in given directory
\end{lstlisting}
\fontfamily{\familydefault}
\selectfont
\item \hyperlink{/+elltool/+doc/run\_helpcollector}{run\_helpcollector}
\fontfamily{pcr}
\selectfont
\begin{lstlisting}
%

\end{lstlisting}
\fontfamily{\familydefault}
\selectfont
\end{enumerate}
\subsection{/+elltool/+linsys}
\begin{enumerate}
\item \hyperlink{/+elltool/+linsys/LinSys}{LinSys}
\fontfamily{pcr}
\selectfont
\begin{lstlisting}
%  Linear system object of the Ellipsoidal Toolbox.
\end{lstlisting}
\fontfamily{\familydefault}
\selectfont
\end{enumerate}
\subsection{/+elltool/+linsys/+test}
\begin{enumerate}
\item \hyperlink{/+elltool/+linsys/+test/run\_tests}{run\_tests}
\fontfamily{pcr}
\selectfont
\begin{lstlisting}
%

\end{lstlisting}
\fontfamily{\familydefault}
\selectfont
\end{enumerate}
\subsection{/+elltool/+linsys/+test/+mlunit}
\begin{enumerate}
\item \hyperlink{/+elltool/+linsys/+test/+mlunit/LinSysTestCase}{LinSysTestCase}
\fontfamily{pcr}
\selectfont
\begin{lstlisting}
%

\end{lstlisting}
\fontfamily{\familydefault}
\selectfont
\end{enumerate}
\subsection{/+elltool/+logging}
\begin{enumerate}
\item \hyperlink{/+elltool/+logging/Log4jConfigurator}{Log4jConfigurator}
\fontfamily{pcr}
\selectfont
\begin{lstlisting}
% LOG4JCONFIGURATOR simplifies log4j configuration, especially when
% Parallel Computing Toolbox is used. In the latter case the class forwards
% the logs of different processees in separate log files
\end{lstlisting}
\fontfamily{\familydefault}
\selectfont
\end{enumerate}
\subsection{/+elltool/+reach}
\begin{enumerate}
\item \hyperlink{/+elltool/+reach/AReach}{AReach}
\fontfamily{pcr}
\selectfont
\begin{lstlisting}
%  $Author: Kirill Mayantsev  <kirill.mayantsev@gmail.com> $  $Date: March-2012 $
%  $Copyright: Moscow State University,
%             Faculty of Computational Mathematics and Computer Science,
%             System Analysis Department 2012 $
\end{lstlisting}
\fontfamily{\familydefault}
\selectfont
\item \hyperlink{/+elltool/+reach/IReach}{IReach}
\fontfamily{pcr}
\selectfont
\begin{lstlisting}
%  $Author: Kirill Mayantsev  <kirill.mayantsev@gmail.com> $  $Date: March-2012 $
%  $Copyright: Moscow State University,
%             Faculty of Computational Mathematics and Computer Science,
%             System Analysis Department 2012 $
\end{lstlisting}
\fontfamily{\familydefault}
\selectfont
\item \hyperlink{/+elltool/+reach/ReachContinuous}{ReachContinuous}
\fontfamily{pcr}
\selectfont
\begin{lstlisting}
%  Continuous reach set library of the Ellipsoidal Toolbox.
\end{lstlisting}
\fontfamily{\familydefault}
\selectfont
\item \hyperlink{/+elltool/+reach/ReachDiscrete}{ReachDiscrete}
\fontfamily{pcr}
\selectfont
\begin{lstlisting}
%  Discrete reach set library of the Ellipsoidal Toolbox.
\end{lstlisting}
\fontfamily{\familydefault}
\selectfont
\end{enumerate}
\subsection{/+elltool/+reach/+test}
\begin{enumerate}
\item \hyperlink{/+elltool/+reach/+test/run\_continuous\_reach\_tests}{run\_continuous\_reach\_tests}
\fontfamily{pcr}
\selectfont
\begin{lstlisting}
%

\end{lstlisting}
\fontfamily{\familydefault}
\selectfont
\item \hyperlink{/+elltool/+reach/+test/run\_discrete\_reach\_tests}{run\_discrete\_reach\_tests}
\fontfamily{pcr}
\selectfont
\begin{lstlisting}
%

\end{lstlisting}
\fontfamily{\familydefault}
\selectfont
\end{enumerate}
\subsection{/+elltool/+reach/+test/+mlunit}
\begin{enumerate}
\item \hyperlink{/+elltool/+reach/+test/+mlunit/ContinuousReachFirstTestCase}{ContinuousReachFirstTestCase}
\fontfamily{pcr}
\selectfont
\begin{lstlisting}
%

\end{lstlisting}
\fontfamily{\familydefault}
\selectfont
\item \hyperlink{/+elltool/+reach/+test/+mlunit/ContinuousReachRegrTestCase}{ContinuousReachRegrTestCase}
\fontfamily{pcr}
\selectfont
\begin{lstlisting}
%

\end{lstlisting}
\fontfamily{\familydefault}
\selectfont
\item \hyperlink{/+elltool/+reach/+test/+mlunit/ContinuousReachTestCase}{ContinuousReachTestCase}
\fontfamily{pcr}
\selectfont
\begin{lstlisting}
%

\end{lstlisting}
\fontfamily{\familydefault}
\selectfont
\item \hyperlink{/+elltool/+reach/+test/+mlunit/DiscreteReachTestCase}{DiscreteReachTestCase}
\fontfamily{pcr}
\selectfont
\begin{lstlisting}
%

\end{lstlisting}
\fontfamily{\familydefault}
\selectfont
\item \hyperlink{/+elltool/+reach/+test/+mlunit/ReachFactory}{ReachFactory}
\fontfamily{pcr}
\selectfont
\begin{lstlisting}
%

\end{lstlisting}
\fontfamily{\familydefault}
\selectfont
\end{enumerate}
\subsection{/+elltool/+test}
\begin{enumerate}
\item \hyperlink{/+elltool/+test/TmpDataManager}{TmpDataManager}
\fontfamily{pcr}
\selectfont
\begin{lstlisting}
%  TMPDATAMANAGER provides a basic functionality for managing temporary
%  data folders, root folder name is determined automatically
\end{lstlisting}
\fontfamily{\familydefault}
\selectfont
\item \hyperlink{/+elltool/+test/copyconf}{copyconf}
\fontfamily{pcr}
\selectfont
\begin{lstlisting}
%

\end{lstlisting}
\fontfamily{\familydefault}
\selectfont
\item \hyperlink{/+elltool/+test/editconf}{editconf}
\fontfamily{pcr}
\selectfont
\begin{lstlisting}
%

\end{lstlisting}
\fontfamily{\familydefault}
\selectfont
\item \hyperlink{/+elltool/+test/listconf}{listconf}
\fontfamily{pcr}
\selectfont
\begin{lstlisting}
%

\end{lstlisting}
\fontfamily{\familydefault}
\selectfont
\item \hyperlink{/+elltool/+test/run\_tests}{run\_tests}
\fontfamily{pcr}
\selectfont
\begin{lstlisting}
%

\end{lstlisting}
\fontfamily{\familydefault}
\selectfont
\item \hyperlink{/+elltool/+test/run\_tests\_remotely}{run\_tests\_remotely}
\fontfamily{pcr}
\selectfont
\begin{lstlisting}
%

\end{lstlisting}
\fontfamily{\familydefault}
\selectfont
\end{enumerate}
\subsection{/+elltool/+test/+configuration}
\begin{enumerate}
\item \hyperlink{/+elltool/+test/+configuration/AdaptiveConfRepoManager}{AdaptiveConfRepoManager}
\fontfamily{pcr}
\selectfont
\begin{lstlisting}
%  ADAPTIVECONFREPOMANAGER is a simplistic extension of
%  AdaptiveConfRepoManager that injects a configuration change
%  repository class equivolent.test.configuration.ConfPatchRepo
%  automatically
\end{lstlisting}
\fontfamily{\familydefault}
\selectfont
\end{enumerate}
\subsection{/+elltool/+test/+configuration/@ConfPatchRepo}
\begin{enumerate}
\item \hyperlink{/+elltool/+test/+configuration/@ConfPatchRepo/ConfPatchRepo}{ConfPatchRepo}
\fontfamily{pcr}
\selectfont
\begin{lstlisting}
%

\end{lstlisting}
\fontfamily{\familydefault}
\selectfont
\item \hyperlink{/+elltool/+test/+configuration/@ConfPatchRepo/patch\_001\_dummy\_patch}{patch\_001\_dummy\_patch}
\fontfamily{pcr}
\selectfont
\begin{lstlisting}
%

\end{lstlisting}
\fontfamily{\familydefault}
\selectfont
\end{enumerate}
\subsection{/+elltool/+test/+logging}
\begin{enumerate}
\item \hyperlink{/+elltool/+test/+logging/Log4jConfigurator}{Log4jConfigurator}
\fontfamily{pcr}
\selectfont
\begin{lstlisting}
% LOG4JCONFIGURATOR simplifies log4j configuration, especially when
% Parallel Computing Toolbox is used. In the latter case the class forwards
% the logs of different processees in separate log files
\end{lstlisting}
\fontfamily{\familydefault}
\selectfont
\end{enumerate}
\subsection{/+gras/+ellapx/+enums}
\begin{enumerate}
\item \hyperlink{/+gras/+ellapx/+enums/EApproxType}{EApproxType}
\fontfamily{pcr}
\selectfont
\begin{lstlisting}
% APXTYPE Summary of this class goes here
\end{lstlisting}
\fontfamily{\familydefault}
\selectfont
\item \hyperlink{/+gras/+ellapx/+enums/EEllUnionTimeDirection}{EEllUnionTimeDirection}
\fontfamily{pcr}
\selectfont
\begin{lstlisting}
% APXTYPE Summary of this class goes here
\end{lstlisting}
\fontfamily{\familydefault}
\selectfont
\item \hyperlink{/+gras/+ellapx/+enums/EProjType}{EProjType}
\fontfamily{pcr}
\selectfont
\begin{lstlisting}
%

\end{lstlisting}
\fontfamily{\familydefault}
\selectfont
\end{enumerate}
\subsection{/+gras/+ellapx/+gen}
\begin{enumerate}
\item \hyperlink{/+gras/+ellapx/+gen/ATightEllApxBuilder}{ATightEllApxBuilder}
\fontfamily{pcr}
\selectfont
\begin{lstlisting}
%

\end{lstlisting}
\fontfamily{\familydefault}
\selectfont
\item \hyperlink{/+gras/+ellapx/+gen/EllApxCollectionBuilder}{EllApxCollectionBuilder}
\fontfamily{pcr}
\selectfont
\begin{lstlisting}
%

\end{lstlisting}
\fontfamily{\familydefault}
\selectfont
\item \hyperlink{/+gras/+ellapx/+gen/IEllApxBuilder}{IEllApxBuilder}
\fontfamily{pcr}
\selectfont
\begin{lstlisting}
%

\end{lstlisting}
\fontfamily{\familydefault}
\selectfont
\end{enumerate}
\subsection{/+gras/+ellapx/+lreachplain}
\begin{enumerate}
\item \hyperlink{/+gras/+ellapx/+lreachplain/ATightEllApxBuilder}{ATightEllApxBuilder}
\fontfamily{pcr}
\selectfont
\begin{lstlisting}
%

\end{lstlisting}
\fontfamily{\familydefault}
\selectfont
\item \hyperlink{/+gras/+ellapx/+lreachplain/ATightIntEllApxBuilder}{ATightIntEllApxBuilder}
\fontfamily{pcr}
\selectfont
\begin{lstlisting}
%

\end{lstlisting}
\fontfamily{\familydefault}
\selectfont
\item \hyperlink{/+gras/+ellapx/+lreachplain/EllTubeDynamicSpaceProjector}{EllTubeDynamicSpaceProjector}
\fontfamily{pcr}
\selectfont
\begin{lstlisting}
% IELLTUBEPROJECTOR Summary of this class goes here
%    Detailed explanation goes here
\end{lstlisting}
\fontfamily{\familydefault}
\selectfont
\item \hyperlink{/+gras/+ellapx/+lreachplain/ExtEllApxBuilder}{ExtEllApxBuilder}
\fontfamily{pcr}
\selectfont
\begin{lstlisting}
%

\end{lstlisting}
\fontfamily{\familydefault}
\selectfont
\item \hyperlink{/+gras/+ellapx/+lreachplain/GoodDirectionSet}{GoodDirectionSet}
\fontfamily{pcr}
\selectfont
\begin{lstlisting}
%

\end{lstlisting}
\fontfamily{\familydefault}
\selectfont
\item \hyperlink{/+gras/+ellapx/+lreachplain/IntEllApxBuilder}{IntEllApxBuilder}
\fontfamily{pcr}
\selectfont
\begin{lstlisting}
%

\end{lstlisting}
\fontfamily{\familydefault}
\selectfont
\item \hyperlink{/+gras/+ellapx/+lreachplain/IntProperEllApxBuilder}{IntProperEllApxBuilder}
\fontfamily{pcr}
\selectfont
\begin{lstlisting}
%

\end{lstlisting}
\fontfamily{\familydefault}
\selectfont
\end{enumerate}
\subsection{/+gras/+ellapx/+lreachplain/+probdef}
\begin{enumerate}
\item \hyperlink{/+gras/+ellapx/+lreachplain/+probdef/AReachContProblemDef}{AReachContProblemDef}
\fontfamily{pcr}
\selectfont
\begin{lstlisting}
%

\end{lstlisting}
\fontfamily{\familydefault}
\selectfont
\item \hyperlink{/+gras/+ellapx/+lreachplain/+probdef/IReachContProblemDef}{IReachContProblemDef}
\fontfamily{pcr}
\selectfont
\begin{lstlisting}
%

\end{lstlisting}
\fontfamily{\familydefault}
\selectfont
\item \hyperlink{/+gras/+ellapx/+lreachplain/+probdef/LReachContProblemDef}{LReachContProblemDef}
\fontfamily{pcr}
\selectfont
\begin{lstlisting}
%

\end{lstlisting}
\fontfamily{\familydefault}
\selectfont
\item \hyperlink{/+gras/+ellapx/+lreachplain/+probdef/ReachContLTIProblemDef}{ReachContLTIProblemDef}
\fontfamily{pcr}
\selectfont
\begin{lstlisting}
%

\end{lstlisting}
\fontfamily{\familydefault}
\selectfont
\end{enumerate}
\subsection{/+gras/+ellapx/+lreachplain/+probdyn}
\begin{enumerate}
\item \hyperlink{/+gras/+ellapx/+lreachplain/+probdyn/AReachProblemDynamics}{AReachProblemDynamics}
\fontfamily{pcr}
\selectfont
\begin{lstlisting}
%

\end{lstlisting}
\fontfamily{\familydefault}
\selectfont
\item \hyperlink{/+gras/+ellapx/+lreachplain/+probdyn/AReachProblemDynamicsInterp}{AReachProblemDynamicsInterp}
\fontfamily{pcr}
\selectfont
\begin{lstlisting}
%

\end{lstlisting}
\fontfamily{\familydefault}
\selectfont
\item \hyperlink{/+gras/+ellapx/+lreachplain/+probdyn/AReachProblemLTIDynamics}{AReachProblemLTIDynamics}
\fontfamily{pcr}
\selectfont
\begin{lstlisting}
%

\end{lstlisting}
\fontfamily{\familydefault}
\selectfont
\item \hyperlink{/+gras/+ellapx/+lreachplain/+probdyn/IReachProblemDynamics}{IReachProblemDynamics}
\fontfamily{pcr}
\selectfont
\begin{lstlisting}
%

\end{lstlisting}
\fontfamily{\familydefault}
\selectfont
\item \hyperlink{/+gras/+ellapx/+lreachplain/+probdyn/LReachProblemDynamicsFactory}{LReachProblemDynamicsFactory}
\fontfamily{pcr}
\selectfont
\begin{lstlisting}
%

\end{lstlisting}
\fontfamily{\familydefault}
\selectfont
\item \hyperlink{/+gras/+ellapx/+lreachplain/+probdyn/LReachProblemDynamicsInterp}{LReachProblemDynamicsInterp}
\fontfamily{pcr}
\selectfont
\begin{lstlisting}
%

\end{lstlisting}
\fontfamily{\familydefault}
\selectfont
\item \hyperlink{/+gras/+ellapx/+lreachplain/+probdyn/LReachProblemLTIDynamics}{LReachProblemLTIDynamics}
\fontfamily{pcr}
\selectfont
\begin{lstlisting}
%

\end{lstlisting}
\fontfamily{\familydefault}
\selectfont
\end{enumerate}
\subsection{/+gras/+ellapx/+lreachuncert}
\begin{enumerate}
\item \hyperlink{/+gras/+ellapx/+lreachuncert/ExtIntEllApxBuilder}{ExtIntEllApxBuilder}
\fontfamily{pcr}
\selectfont
\begin{lstlisting}
%

\end{lstlisting}
\fontfamily{\familydefault}
\selectfont
\end{enumerate}
\subsection{/+gras/+ellapx/+lreachuncert/+probdef}
\begin{enumerate}
\item \hyperlink{/+gras/+ellapx/+lreachuncert/+probdef/AReachContProblemDef}{AReachContProblemDef}
\fontfamily{pcr}
\selectfont
\begin{lstlisting}
%

\end{lstlisting}
\fontfamily{\familydefault}
\selectfont
\item \hyperlink{/+gras/+ellapx/+lreachuncert/+probdef/IReachContProblemDef}{IReachContProblemDef}
\fontfamily{pcr}
\selectfont
\begin{lstlisting}
%

\end{lstlisting}
\fontfamily{\familydefault}
\selectfont
\item \hyperlink{/+gras/+ellapx/+lreachuncert/+probdef/LReachContProblemDef}{LReachContProblemDef}
\fontfamily{pcr}
\selectfont
\begin{lstlisting}
%

\end{lstlisting}
\fontfamily{\familydefault}
\selectfont
\item \hyperlink{/+gras/+ellapx/+lreachuncert/+probdef/ReachContLTIProblemDef}{ReachContLTIProblemDef}
\fontfamily{pcr}
\selectfont
\begin{lstlisting}
%

\end{lstlisting}
\fontfamily{\familydefault}
\selectfont
\end{enumerate}
\subsection{/+gras/+ellapx/+lreachuncert/+probdyn}
\begin{enumerate}
\item \hyperlink{/+gras/+ellapx/+lreachuncert/+probdyn/AReachProblemDynamics}{AReachProblemDynamics}
\fontfamily{pcr}
\selectfont
\begin{lstlisting}
%

\end{lstlisting}
\fontfamily{\familydefault}
\selectfont
\item \hyperlink{/+gras/+ellapx/+lreachuncert/+probdyn/IReachProblemDynamics}{IReachProblemDynamics}
\fontfamily{pcr}
\selectfont
\begin{lstlisting}
%

\end{lstlisting}
\fontfamily{\familydefault}
\selectfont
\item \hyperlink{/+gras/+ellapx/+lreachuncert/+probdyn/LReachProblemDynamicsFactory}{LReachProblemDynamicsFactory}
\fontfamily{pcr}
\selectfont
\begin{lstlisting}
%

\end{lstlisting}
\fontfamily{\familydefault}
\selectfont
\item \hyperlink{/+gras/+ellapx/+lreachuncert/+probdyn/LReachProblemDynamicsInterp}{LReachProblemDynamicsInterp}
\fontfamily{pcr}
\selectfont
\begin{lstlisting}
%

\end{lstlisting}
\fontfamily{\familydefault}
\selectfont
\item \hyperlink{/+gras/+ellapx/+lreachuncert/+probdyn/LReachProblemLTIDynamics}{LReachProblemLTIDynamics}
\fontfamily{pcr}
\selectfont
\begin{lstlisting}
%

\end{lstlisting}
\fontfamily{\familydefault}
\selectfont
\end{enumerate}
\subsection{/+gras/+ellapx/+proj}
\begin{enumerate}
\item \hyperlink{/+gras/+ellapx/+proj/AEllTubePlainProjector}{AEllTubePlainProjector}
\fontfamily{pcr}
\selectfont
\begin{lstlisting}
% IELLTUBEPROJECTOR Summary of this class goes here
%    Detailed explanation goes here
\end{lstlisting}
\fontfamily{\familydefault}
\selectfont
\item \hyperlink{/+gras/+ellapx/+proj/EllTubeCollectionProjector}{EllTubeCollectionProjector}
\fontfamily{pcr}
\selectfont
\begin{lstlisting}
% IELLTUBEPROJECTOR Summary of this class goes here
%    Detailed explanation goes here
\end{lstlisting}
\fontfamily{\familydefault}
\selectfont
\item \hyperlink{/+gras/+ellapx/+proj/EllTubeStaticSpaceProjector}{EllTubeStaticSpaceProjector}
\fontfamily{pcr}
\selectfont
\begin{lstlisting}
% IELLTUBEPROJECTOR Summary of this class goes here
%    Detailed explanation goes here
\end{lstlisting}
\fontfamily{\familydefault}
\selectfont
\item \hyperlink{/+gras/+ellapx/+proj/IEllTubeProjector}{IEllTubeProjector}
\fontfamily{pcr}
\selectfont
\begin{lstlisting}
% IELLTUBEPROJECTOR Summary of this class goes here
%    Detailed explanation goes here
\end{lstlisting}
\fontfamily{\familydefault}
\selectfont
\end{enumerate}
\subsection{/+gras/+ellapx/+smartdb}
\begin{enumerate}
\item \hyperlink{/+gras/+ellapx/+smartdb/F}{F}
\fontfamily{pcr}
\selectfont
\begin{lstlisting}
% Standard fields
\end{lstlisting}
\fontfamily{\familydefault}
\selectfont
\item \hyperlink{/+gras/+ellapx/+smartdb/RelDispConfigurator}{RelDispConfigurator}
\fontfamily{pcr}
\selectfont
\begin{lstlisting}
%

\end{lstlisting}
\fontfamily{\familydefault}
\selectfont
\end{enumerate}
\subsection{/+gras/+ellapx/+smartdb/+rels}
\begin{enumerate}
\item \hyperlink{/+gras/+ellapx/+smartdb/+rels/EllTube}{EllTube}
\fontfamily{pcr}
\selectfont
\begin{lstlisting}
% TestRelation Summary of this class goes here
%    Detailed explanation goes here
\end{lstlisting}
\fontfamily{\familydefault}
\selectfont
\item \hyperlink{/+gras/+ellapx/+smartdb/+rels/EllTubeBasic}{EllTubeBasic}
\fontfamily{pcr}
\selectfont
\begin{lstlisting}
% TestRelation Summary of this class goes here
%    Detailed explanation goes here
\end{lstlisting}
\fontfamily{\familydefault}
\selectfont
\item \hyperlink{/+gras/+ellapx/+smartdb/+rels/EllTubeProj}{EllTubeProj}
\fontfamily{pcr}
\selectfont
\begin{lstlisting}
% TestRelation Summary of this class goes here
%    Detailed explanation goes here
\end{lstlisting}
\fontfamily{\familydefault}
\selectfont
\item \hyperlink{/+gras/+ellapx/+smartdb/+rels/EllTubeProjBasic}{EllTubeProjBasic}
\fontfamily{pcr}
\selectfont
\begin{lstlisting}
%

\end{lstlisting}
\fontfamily{\familydefault}
\selectfont
\item \hyperlink{/+gras/+ellapx/+smartdb/+rels/EllTubeTouchCurveBasic}{EllTubeTouchCurveBasic}
\fontfamily{pcr}
\selectfont
\begin{lstlisting}
% TestRelation Summary of this class goes here
%    Detailed explanation goes here
\end{lstlisting}
\fontfamily{\familydefault}
\selectfont
\item \hyperlink{/+gras/+ellapx/+smartdb/+rels/EllTubeTouchCurveProjBasic}{EllTubeTouchCurveProjBasic}
\fontfamily{pcr}
\selectfont
\begin{lstlisting}
%

\end{lstlisting}
\fontfamily{\familydefault}
\selectfont
\item \hyperlink{/+gras/+ellapx/+smartdb/+rels/EllUnionTube}{EllUnionTube}
\fontfamily{pcr}
\selectfont
\begin{lstlisting}
% TestRelation Summary of this class goes here
%    Detailed explanation goes here
\end{lstlisting}
\fontfamily{\familydefault}
\selectfont
\item \hyperlink{/+gras/+ellapx/+smartdb/+rels/EllUnionTubeBasic}{EllUnionTubeBasic}
\fontfamily{pcr}
\selectfont
\begin{lstlisting}
% TestRelation Summary of this class goes here
%    Detailed explanation goes here
\end{lstlisting}
\fontfamily{\familydefault}
\selectfont
\item \hyperlink{/+gras/+ellapx/+smartdb/+rels/EllUnionTubeStaticProj}{EllUnionTubeStaticProj}
\fontfamily{pcr}
\selectfont
\begin{lstlisting}
% TestRelation Summary of this class goes here
%    Detailed explanation goes here
\end{lstlisting}
\fontfamily{\familydefault}
\selectfont
\item \hyperlink{/+gras/+ellapx/+smartdb/+rels/TypifiedByFieldCodeRel}{TypifiedByFieldCodeRel}
\fontfamily{pcr}
\selectfont
\begin{lstlisting}
% TestRelation Summary of this class goes here
%    Detailed explanation goes here
\end{lstlisting}
\fontfamily{\familydefault}
\selectfont
\end{enumerate}
\subsection{/+gras/+ellapx/+smartdb/+test}
\begin{enumerate}
\item \hyperlink{/+gras/+ellapx/+smartdb/+test/run\_tests}{run\_tests}
\fontfamily{pcr}
\selectfont
\begin{lstlisting}
%

\end{lstlisting}
\fontfamily{\familydefault}
\selectfont
\end{enumerate}
\subsection{/+gras/+ellapx/+smartdb/+test/+mlunit}
\begin{enumerate}
\item \hyperlink{/+gras/+ellapx/+smartdb/+test/+mlunit/SuiteEllTube}{SuiteEllTube}
\fontfamily{pcr}
\selectfont
\begin{lstlisting}
%

\end{lstlisting}
\fontfamily{\familydefault}
\selectfont
\end{enumerate}
\subsection{/+gras/+ellapx/+test}
\begin{enumerate}
\item \hyperlink{/+gras/+ellapx/+test/run\_tests}{run\_tests}
\fontfamily{pcr}
\selectfont
\begin{lstlisting}
%

\end{lstlisting}
\fontfamily{\familydefault}
\selectfont
\end{enumerate}
\subsection{/+gras/+ellapx/+uncertcalc}
\begin{enumerate}
\item \hyperlink{/+gras/+ellapx/+uncertcalc/ApproxProblemPropertyBuilder}{ApproxProblemPropertyBuilder}
\fontfamily{pcr}
\selectfont
\begin{lstlisting}
%

\end{lstlisting}
\fontfamily{\familydefault}
\selectfont
\item \hyperlink{/+gras/+ellapx/+uncertcalc/EllApxBuilder}{EllApxBuilder}
\fontfamily{pcr}
\selectfont
\begin{lstlisting}
% IELLTUBEPROJECTOR Summary of this class goes here
%    Detailed explanation goes here
\end{lstlisting}
\fontfamily{\familydefault}
\selectfont
\item \hyperlink{/+gras/+ellapx/+uncertcalc/EllTubeProjectorBuilder}{EllTubeProjectorBuilder}
\fontfamily{pcr}
\selectfont
\begin{lstlisting}
% IELLTUBEPROJECTOR Summary of this class goes here
%    Detailed explanation goes here
\end{lstlisting}
\fontfamily{\familydefault}
\selectfont
\item \hyperlink{/+gras/+ellapx/+uncertcalc/copyconf}{copyconf}
\fontfamily{pcr}
\selectfont
\begin{lstlisting}
%

\end{lstlisting}
\fontfamily{\familydefault}
\selectfont
\item \hyperlink{/+gras/+ellapx/+uncertcalc/copysysconf}{copysysconf}
\fontfamily{pcr}
\selectfont
\begin{lstlisting}
%

\end{lstlisting}
\fontfamily{\familydefault}
\selectfont
\item \hyperlink{/+gras/+ellapx/+uncertcalc/editconf}{editconf}
\fontfamily{pcr}
\selectfont
\begin{lstlisting}
%

\end{lstlisting}
\fontfamily{\familydefault}
\selectfont
\item \hyperlink{/+gras/+ellapx/+uncertcalc/editsysconf}{editsysconf}
\fontfamily{pcr}
\selectfont
\begin{lstlisting}
%

\end{lstlisting}
\fontfamily{\familydefault}
\selectfont
\item \hyperlink{/+gras/+ellapx/+uncertcalc/listconf}{listconf}
\fontfamily{pcr}
\selectfont
\begin{lstlisting}
%

\end{lstlisting}
\fontfamily{\familydefault}
\selectfont
\item \hyperlink{/+gras/+ellapx/+uncertcalc/listsysconf}{listsysconf}
\fontfamily{pcr}
\selectfont
\begin{lstlisting}
%

\end{lstlisting}
\fontfamily{\familydefault}
\selectfont
\item \hyperlink{/+gras/+ellapx/+uncertcalc/run}{run}
\fontfamily{pcr}
\selectfont
\begin{lstlisting}
%

\end{lstlisting}
\fontfamily{\familydefault}
\selectfont
\item \hyperlink{/+gras/+ellapx/+uncertcalc/updateallconf}{updateallconf}
\fontfamily{pcr}
\selectfont
\begin{lstlisting}
%

\end{lstlisting}
\fontfamily{\familydefault}
\selectfont
\end{enumerate}
\subsection{/+gras/+ellapx/+uncertcalc/+conf}
\begin{enumerate}
\item \hyperlink{/+gras/+ellapx/+uncertcalc/+conf/ConfRepoMgr}{ConfRepoMgr}
\fontfamily{pcr}
\selectfont
\begin{lstlisting}
%

\end{lstlisting}
\fontfamily{\familydefault}
\selectfont
\item \hyperlink{/+gras/+ellapx/+uncertcalc/+conf/IConfRepoMgr}{IConfRepoMgr}
\fontfamily{pcr}
\selectfont
\begin{lstlisting}
%

\end{lstlisting}
\fontfamily{\familydefault}
\selectfont
\end{enumerate}
\subsection{/+gras/+ellapx/+uncertcalc/+conf/+sysdef}
\begin{enumerate}
\item \hyperlink{/+gras/+ellapx/+uncertcalc/+conf/+sysdef/AConfRepoMgr}{AConfRepoMgr}
\fontfamily{pcr}
\selectfont
\begin{lstlisting}
%

\end{lstlisting}
\fontfamily{\familydefault}
\selectfont
\item \hyperlink{/+gras/+ellapx/+uncertcalc/+conf/+sysdef/ConfRepoMgr}{ConfRepoMgr}
\fontfamily{pcr}
\selectfont
\begin{lstlisting}
%

\end{lstlisting}
\fontfamily{\familydefault}
\selectfont
\end{enumerate}
\subsection{/+gras/+ellapx/+uncertcalc/+conf/+sysdef/+test}
\begin{enumerate}
\item \hyperlink{/+gras/+ellapx/+uncertcalc/+conf/+sysdef/+test/ConfRepoMgr}{ConfRepoMgr}
\fontfamily{pcr}
\selectfont
\begin{lstlisting}
%

\end{lstlisting}
\fontfamily{\familydefault}
\selectfont
\item \hyperlink{/+gras/+ellapx/+uncertcalc/+conf/+sysdef/+test/run\_tests}{run\_tests}
\fontfamily{pcr}
\selectfont
\begin{lstlisting}
%

\end{lstlisting}
\fontfamily{\familydefault}
\selectfont
\end{enumerate}
\subsection{/+gras/+ellapx/+uncertcalc/+conf/+sysdef/+test/+mlunit}
\begin{enumerate}
\item \hyperlink{/+gras/+ellapx/+uncertcalc/+conf/+sysdef/+test/+mlunit/SuiteBasic}{SuiteBasic}
\fontfamily{pcr}
\selectfont
\begin{lstlisting}
%

\end{lstlisting}
\fontfamily{\familydefault}
\selectfont
\end{enumerate}
\subsection{/+gras/+ellapx/+uncertcalc/+conf/+sysdef/@ConfPatchRepo}
\begin{enumerate}
\item \hyperlink{/+gras/+ellapx/+uncertcalc/+conf/+sysdef/@ConfPatchRepo/ConfPatchRepo}{ConfPatchRepo}
\fontfamily{pcr}
\selectfont
\begin{lstlisting}
%

\end{lstlisting}
\fontfamily{\familydefault}
\selectfont
\item \hyperlink{/+gras/+ellapx/+uncertcalc/+conf/+sysdef/@ConfPatchRepo/patch\_001\_remove\_garbage}{patch\_001\_remove\_garbage}
\fontfamily{pcr}
\selectfont
\begin{lstlisting}
%

\end{lstlisting}
\fontfamily{\familydefault}
\selectfont
\item \hyperlink{/+gras/+ellapx/+uncertcalc/+conf/+sysdef/@ConfPatchRepo/patch\_002\_add\_description}{patch\_002\_add\_description}
\fontfamily{pcr}
\selectfont
\begin{lstlisting}
%

\end{lstlisting}
\fontfamily{\familydefault}
\selectfont
\end{enumerate}
\subsection{/+gras/+ellapx/+uncertcalc/+conf/@ConfPatchRepo}
\begin{enumerate}
\item \hyperlink{/+gras/+ellapx/+uncertcalc/+conf/@ConfPatchRepo/ConfPatchRepo}{ConfPatchRepo}
\fontfamily{pcr}
\selectfont
\begin{lstlisting}
%

\end{lstlisting}
\fontfamily{\familydefault}
\selectfont
\item \hyperlink{/+gras/+ellapx/+uncertcalc/+conf/@ConfPatchRepo/patch\_001\_make\_proj\_spec\_logical}{patch\_001\_make\_proj\_spec\_logical}
\fontfamily{pcr}
\selectfont
\begin{lstlisting}
%

\end{lstlisting}
\fontfamily{\familydefault}
\selectfont
\item \hyperlink{/+gras/+ellapx/+uncertcalc/+conf/@ConfPatchRepo/patch\_002\_remove\_redundant\_stuff}{patch\_002\_remove\_redundant\_stuff}
\fontfamily{pcr}
\selectfont
\begin{lstlisting}
%

\end{lstlisting}
\fontfamily{\familydefault}
\selectfont
\item \hyperlink{/+gras/+ellapx/+uncertcalc/+conf/@ConfPatchRepo/patch\_003\_is\_plotting\_enabled}{patch\_003\_is\_plotting\_enabled}
\fontfamily{pcr}
\selectfont
\begin{lstlisting}
%

\end{lstlisting}
\fontfamily{\familydefault}
\selectfont
\item \hyperlink{/+gras/+ellapx/+uncertcalc/+conf/@ConfPatchRepo/patch\_004\_multiple\_int\_ell\_apx\_schemas}{patch\_004\_multiple\_int\_ell\_apx\_schemas}
\fontfamily{pcr}
\selectfont
\begin{lstlisting}
%

\end{lstlisting}
\fontfamily{\familydefault}
\selectfont
\item \hyperlink{/+gras/+ellapx/+uncertcalc/+conf/@ConfPatchRepo/patch\_005\_ext\_ell\_apx\_schema}{patch\_005\_ext\_ell\_apx\_schema}
\fontfamily{pcr}
\selectfont
\begin{lstlisting}
%

\end{lstlisting}
\fontfamily{\familydefault}
\selectfont
\item \hyperlink{/+gras/+ellapx/+uncertcalc/+conf/@ConfPatchRepo/patch\_006\_calc\_precision}{patch\_006\_calc\_precision}
\fontfamily{pcr}
\selectfont
\begin{lstlisting}
%

\end{lstlisting}
\fontfamily{\familydefault}
\selectfont
\item \hyperlink{/+gras/+ellapx/+uncertcalc/+conf/@ConfPatchRepo/patch\_007\_make\_space\_list\_a\_vector}{patch\_007\_make\_space\_list\_a\_vector}
\fontfamily{pcr}
\selectfont
\begin{lstlisting}
%

\end{lstlisting}
\fontfamily{\familydefault}
\selectfont
\item \hyperlink{/+gras/+ellapx/+uncertcalc/+conf/@ConfPatchRepo/patch\_008\_add\_reference\_to\_sysdef}{patch\_008\_add\_reference\_to\_sysdef}
\fontfamily{pcr}
\selectfont
\begin{lstlisting}
%

\end{lstlisting}
\fontfamily{\familydefault}
\selectfont
\item \hyperlink{/+gras/+ellapx/+uncertcalc/+conf/@ConfPatchRepo/patch\_009\_add\_scale\_factors}{patch\_009\_add\_scale\_factors}
\fontfamily{pcr}
\selectfont
\begin{lstlisting}
%

\end{lstlisting}
\fontfamily{\familydefault}
\selectfont
\item \hyperlink{/+gras/+ellapx/+uncertcalc/+conf/@ConfPatchRepo/patch\_010\_rename\_scale\_factors}{patch\_010\_rename\_scale\_factors}
\fontfamily{pcr}
\selectfont
\begin{lstlisting}
%

\end{lstlisting}
\fontfamily{\familydefault}
\selectfont
\item \hyperlink{/+gras/+ellapx/+uncertcalc/+conf/@ConfPatchRepo/patch\_011\_add\_view\_angle\_prop}{patch\_011\_add\_view\_angle\_prop}
\fontfamily{pcr}
\selectfont
\begin{lstlisting}
%

\end{lstlisting}
\fontfamily{\familydefault}
\selectfont
\item \hyperlink{/+gras/+ellapx/+uncertcalc/+conf/@ConfPatchRepo/patch\_012\_rename\_ell\_apx\_schemas}{patch\_012\_rename\_ell\_apx\_schemas}
\fontfamily{pcr}
\selectfont
\begin{lstlisting}
%

\end{lstlisting}
\fontfamily{\familydefault}
\selectfont
\item \hyperlink{/+gras/+ellapx/+uncertcalc/+conf/@ConfPatchRepo/patch\_013\_disable\_uncertainty\_regime\_by\_default}{patch\_013\_disable\_uncertainty\_regime\_by\_default}
\fontfamily{pcr}
\selectfont
\begin{lstlisting}
%

\end{lstlisting}
\fontfamily{\familydefault}
\selectfont
\item \hyperlink{/+gras/+ellapx/+uncertcalc/+conf/@ConfPatchRepo/patch\_014\_internal\_external\_apx\_params}{patch\_014\_internal\_external\_apx\_params}
\fontfamily{pcr}
\selectfont
\begin{lstlisting}
%

\end{lstlisting}
\fontfamily{\familydefault}
\selectfont
\item \hyperlink{/+gras/+ellapx/+uncertcalc/+conf/@ConfPatchRepo/patch\_015\_internal\_external\_apx\_addparams}{patch\_015\_internal\_external\_apx\_addparams}
\fontfamily{pcr}
\selectfont
\begin{lstlisting}
%

\end{lstlisting}
\fontfamily{\familydefault}
\selectfont
\item \hyperlink{/+gras/+ellapx/+uncertcalc/+conf/@ConfPatchRepo/patch\_016\_is\_good\_curves\_separately}{patch\_016\_is\_good\_curves\_separately}
\fontfamily{pcr}
\selectfont
\begin{lstlisting}
%

\end{lstlisting}
\fontfamily{\familydefault}
\selectfont
\item \hyperlink{/+gras/+ellapx/+uncertcalc/+conf/@ConfPatchRepo/patch\_017\_gen\_props\_mat\_calc\_mode}{patch\_017\_gen\_props\_mat\_calc\_mode}
\fontfamily{pcr}
\selectfont
\begin{lstlisting}
%

\end{lstlisting}
\fontfamily{\familydefault}
\selectfont
\item \hyperlink{/+gras/+ellapx/+uncertcalc/+conf/@ConfPatchRepo/patch\_018\_remove\_uncert\_int\_apx\_schema}{patch\_018\_remove\_uncert\_int\_apx\_schema}
\fontfamily{pcr}
\selectfont
\begin{lstlisting}
%

\end{lstlisting}
\fontfamily{\familydefault}
\selectfont
\end{enumerate}
\subsection{/+gras/+ellapx/+uncertcalc/+log}
\begin{enumerate}
\item \hyperlink{/+gras/+ellapx/+uncertcalc/+log/Log4jConfigurator}{Log4jConfigurator}
\fontfamily{pcr}
\selectfont
\begin{lstlisting}
% LOG4JCONFIGURATOR simplifies log4j configuration, especially when
% Parallel Computing Toolbox is used. In the latter case the class forwards
% the logs of different processees in separate log files
\end{lstlisting}
\fontfamily{\familydefault}
\selectfont
\end{enumerate}
\subsection{/+gras/+ellapx/+uncertcalc/+test}
\begin{enumerate}
\item \hyperlink{/+gras/+ellapx/+uncertcalc/+test/run\_tests}{run\_tests}
\fontfamily{pcr}
\selectfont
\begin{lstlisting}
%

\end{lstlisting}
\fontfamily{\familydefault}
\selectfont
\item \hyperlink{/+gras/+ellapx/+uncertcalc/+test/updateallconf}{updateallconf}
\fontfamily{pcr}
\selectfont
\begin{lstlisting}
%  UPDATEALLCONF updates all the configurations in the nested packages
\end{lstlisting}
\fontfamily{\familydefault}
\selectfont
\end{enumerate}
\subsection{/+gras/+ellapx/+uncertcalc/+test/+comp}
\begin{enumerate}
\item \hyperlink{/+gras/+ellapx/+uncertcalc/+test/+comp/copyconf}{copyconf}
\fontfamily{pcr}
\selectfont
\begin{lstlisting}
%

\end{lstlisting}
\fontfamily{\familydefault}
\selectfont
\item \hyperlink{/+gras/+ellapx/+uncertcalc/+test/+comp/editconf}{editconf}
\fontfamily{pcr}
\selectfont
\begin{lstlisting}
%

\end{lstlisting}
\fontfamily{\familydefault}
\selectfont
\item \hyperlink{/+gras/+ellapx/+uncertcalc/+test/+comp/editconftemplate}{editconftemplate}
\fontfamily{pcr}
\selectfont
\begin{lstlisting}
%

\end{lstlisting}
\fontfamily{\familydefault}
\selectfont
\item \hyperlink{/+gras/+ellapx/+uncertcalc/+test/+comp/listconfs}{listconfs}
\fontfamily{pcr}
\selectfont
\begin{lstlisting}
%  UPDATECONFTEMPLATE updates the specified template configuration
\end{lstlisting}
\fontfamily{\familydefault}
\selectfont
\item \hyperlink{/+gras/+ellapx/+uncertcalc/+test/+comp/run\_tests}{run\_tests}
\fontfamily{pcr}
\selectfont
\begin{lstlisting}
%

\end{lstlisting}
\fontfamily{\familydefault}
\selectfont
\item \hyperlink{/+gras/+ellapx/+uncertcalc/+test/+comp/updateallconf}{updateallconf}
\fontfamily{pcr}
\selectfont
\begin{lstlisting}
%  UPDATEALLCONF updates all the configurations in the nested packages
\end{lstlisting}
\fontfamily{\familydefault}
\selectfont
\item \hyperlink{/+gras/+ellapx/+uncertcalc/+test/+comp/updateconftemplate}{updateconftemplate}
\fontfamily{pcr}
\selectfont
\begin{lstlisting}
%  UPDATECONFTEMPLATE updates the specified template configuration
\end{lstlisting}
\fontfamily{\familydefault}
\selectfont
\end{enumerate}
\subsection{/+gras/+ellapx/+uncertcalc/+test/+comp/+conf}
\begin{enumerate}
\item \hyperlink{/+gras/+ellapx/+uncertcalc/+test/+comp/+conf/ConfRepoMgr}{ConfRepoMgr}
\fontfamily{pcr}
\selectfont
\begin{lstlisting}
%

\end{lstlisting}
\fontfamily{\familydefault}
\selectfont
\end{enumerate}
\subsection{/+gras/+ellapx/+uncertcalc/+test/+comp/+conf/+sysdef}
\begin{enumerate}
\item \hyperlink{/+gras/+ellapx/+uncertcalc/+test/+comp/+conf/+sysdef/ConfRepoMgr}{ConfRepoMgr}
\fontfamily{pcr}
\selectfont
\begin{lstlisting}
%

\end{lstlisting}
\fontfamily{\familydefault}
\selectfont
\end{enumerate}
\subsection{/+gras/+ellapx/+uncertcalc/+test/+comp/+mlunit}
\begin{enumerate}
\item \hyperlink{/+gras/+ellapx/+uncertcalc/+test/+comp/+mlunit/SuiteCompare}{SuiteCompare}
\fontfamily{pcr}
\selectfont
\begin{lstlisting}
%

\end{lstlisting}
\fontfamily{\familydefault}
\selectfont
\end{enumerate}
\subsection{/+gras/+ellapx/+uncertcalc/+test/+regr}
\begin{enumerate}
\item \hyperlink{/+gras/+ellapx/+uncertcalc/+test/+regr/copyconf}{copyconf}
\fontfamily{pcr}
\selectfont
\begin{lstlisting}
%

\end{lstlisting}
\fontfamily{\familydefault}
\selectfont
\item \hyperlink{/+gras/+ellapx/+uncertcalc/+test/+regr/editconf}{editconf}
\fontfamily{pcr}
\selectfont
\begin{lstlisting}
%

\end{lstlisting}
\fontfamily{\familydefault}
\selectfont
\item \hyperlink{/+gras/+ellapx/+uncertcalc/+test/+regr/editconftemplate}{editconftemplate}
\fontfamily{pcr}
\selectfont
\begin{lstlisting}
%

\end{lstlisting}
\fontfamily{\familydefault}
\selectfont
\item \hyperlink{/+gras/+ellapx/+uncertcalc/+test/+regr/editsysconf}{editsysconf}
\fontfamily{pcr}
\selectfont
\begin{lstlisting}
%

\end{lstlisting}
\fontfamily{\familydefault}
\selectfont
\item \hyperlink{/+gras/+ellapx/+uncertcalc/+test/+regr/listconfs}{listconfs}
\fontfamily{pcr}
\selectfont
\begin{lstlisting}
%  UPDATECONFTEMPLATE updates the specified template configuration
\end{lstlisting}
\fontfamily{\familydefault}
\selectfont
\item \hyperlink{/+gras/+ellapx/+uncertcalc/+test/+regr/run\_regr\_tests}{run\_regr\_tests}
\fontfamily{pcr}
\selectfont
\begin{lstlisting}
%

\end{lstlisting}
\fontfamily{\familydefault}
\selectfont
\item \hyperlink{/+gras/+ellapx/+uncertcalc/+test/+regr/run\_support\_function\_tests}{run\_support\_function\_tests}
\fontfamily{pcr}
\selectfont
\begin{lstlisting}
%  $Author: Kirill Mayantsev  <kirill.mayantsev@gmail.com> $  $Date: 2-11-2012 $
%  $Copyright: Moscow State University,
%              Faculty of Computational Mathematics and Computer Science,
%              System Analysis Department 2012 $
\end{lstlisting}
\fontfamily{\familydefault}
\selectfont
\item \hyperlink{/+gras/+ellapx/+uncertcalc/+test/+regr/run\_tests}{run\_tests}
\fontfamily{pcr}
\selectfont
\begin{lstlisting}
%

\end{lstlisting}
\fontfamily{\familydefault}
\selectfont
\item \hyperlink{/+gras/+ellapx/+uncertcalc/+test/+regr/updateallconf}{updateallconf}
\fontfamily{pcr}
\selectfont
\begin{lstlisting}
%  UPDATEALLCONF updates all the configurations in the nested packages
\end{lstlisting}
\fontfamily{\familydefault}
\selectfont
\item \hyperlink{/+gras/+ellapx/+uncertcalc/+test/+regr/updateconftemplate}{updateconftemplate}
\fontfamily{pcr}
\selectfont
\begin{lstlisting}
%  UPDATECONFTEMPLATE updates the specified template configuration
\end{lstlisting}
\fontfamily{\familydefault}
\selectfont
\end{enumerate}
\subsection{/+gras/+ellapx/+uncertcalc/+test/+regr/+conf}
\begin{enumerate}
\item \hyperlink{/+gras/+ellapx/+uncertcalc/+test/+regr/+conf/ConfRepoMgr}{ConfRepoMgr}
\fontfamily{pcr}
\selectfont
\begin{lstlisting}
%

\end{lstlisting}
\fontfamily{\familydefault}
\selectfont
\end{enumerate}
\subsection{/+gras/+ellapx/+uncertcalc/+test/+regr/+conf/+sysdef}
\begin{enumerate}
\item \hyperlink{/+gras/+ellapx/+uncertcalc/+test/+regr/+conf/+sysdef/ConfRepoMgr}{ConfRepoMgr}
\fontfamily{pcr}
\selectfont
\begin{lstlisting}
%

\end{lstlisting}
\fontfamily{\familydefault}
\selectfont
\end{enumerate}
\subsection{/+gras/+ellapx/+uncertcalc/+test/+regr/+mlunit}
\begin{enumerate}
\item \hyperlink{/+gras/+ellapx/+uncertcalc/+test/+regr/+mlunit/SuiteBasic}{SuiteBasic}
\fontfamily{pcr}
\selectfont
\begin{lstlisting}
%

\end{lstlisting}
\fontfamily{\familydefault}
\selectfont
\item \hyperlink{/+gras/+ellapx/+uncertcalc/+test/+regr/+mlunit/SuiteRegression}{SuiteRegression}
\fontfamily{pcr}
\selectfont
\begin{lstlisting}
%

\end{lstlisting}
\fontfamily{\familydefault}
\selectfont
\item \hyperlink{/+gras/+ellapx/+uncertcalc/+test/+regr/+mlunit/SuiteSupportFunction}{SuiteSupportFunction}
\fontfamily{pcr}
\selectfont
\begin{lstlisting}
%  $Author: Kirill Mayantsev  <kirill.mayantsev@gmail.com> $  $Date: 2-11-2012 $
%  $Copyright: Moscow State University,
%              Faculty of Computational Mathematics and Computer Science,
%              System Analysis Department 2012 $
\end{lstlisting}
\fontfamily{\familydefault}
\selectfont
\end{enumerate}
\subsection{/+gras/+gen}
\begin{enumerate}
\item \hyperlink{/+gras/+gen/MatVector}{MatVector}
\fontfamily{pcr}
\selectfont
\begin{lstlisting}
% MATVECTOR Summary of this class goes here
%    Detailed explanation goes here
\end{lstlisting}
\fontfamily{\familydefault}
\selectfont
\item \hyperlink{/+gras/+gen/ProgressCmdDisplayer}{ProgressCmdDisplayer}
\fontfamily{pcr}
\selectfont
\begin{lstlisting}
%

\end{lstlisting}
\fontfamily{\familydefault}
\selectfont
\item \hyperlink{/+gras/+gen/SquareMatVector}{SquareMatVector}
\fontfamily{pcr}
\selectfont
\begin{lstlisting}
% MATVECTOR Summary of this class goes here
%    Detailed explanation goes here
\end{lstlisting}
\fontfamily{\familydefault}
\selectfont
\item \hyperlink{/+gras/+gen/minadv}{minadv}
\fontfamily{pcr}
\selectfont
\begin{lstlisting}
%  MINADV works in the same way as the built-in min function but returns
%  indMinSize as a second argument which equals 1 if all minimum elements
%  can be taken from leftArray, 2 if they all can be taken from rightArray
%  and 0 otherwise
\end{lstlisting}
\fontfamily{\familydefault}
\selectfont
\item \hyperlink{/+gras/+gen/sortrowstol}{sortrowstol}
\fontfamily{pcr}
\selectfont
\begin{lstlisting}
%  SORTROWSTOL sorts rows of input numeric matrix in ascending order with a
%  specified precision i.e. sorting [1 2;1+1e-14 1] with tol>=1-14 would put
%  the first row on the second position while the built-in sortrows function
%  would keep the order of rows unchanged, the functions only looks at the
%  neighboring values and doesn't calculate pairwise distances (as in pdist
%  function) for speed
\end{lstlisting}
\fontfamily{\familydefault}
\selectfont
\end{enumerate}
\subsection{/+gras/+gen/+test}
\begin{enumerate}
\item \hyperlink{/+gras/+gen/+test/run\_tests}{run\_tests}
\fontfamily{pcr}
\selectfont
\begin{lstlisting}
%

\end{lstlisting}
\fontfamily{\familydefault}
\selectfont
\end{enumerate}
\subsection{/+gras/+gen/+test/+mlunit}
\begin{enumerate}
\item \hyperlink{/+gras/+gen/+test/+mlunit/SuiteBasic}{SuiteBasic}
\fontfamily{pcr}
\selectfont
\begin{lstlisting}
%

\end{lstlisting}
\fontfamily{\familydefault}
\selectfont
\end{enumerate}
\subsection{/+gras/+geom}
\begin{enumerate}
\item \hyperlink{/+gras/+geom/circlepart}{circlepart}
\fontfamily{pcr}
\selectfont
\begin{lstlisting}
%  CIRCLEPART builds a partition of unit circle into a specified number of
%  points within a specified angle range
\end{lstlisting}
\fontfamily{\familydefault}
\selectfont
\end{enumerate}
\subsection{/+gras/+geom/+ell}
\begin{enumerate}
\item \hyperlink{/+gras/+geom/+ell/ellvolume}{ellvolume}
\fontfamily{pcr}
\selectfont
\begin{lstlisting}
%  ELLVOLUME calculates a volume of ellipsoid
\end{lstlisting}
\fontfamily{\familydefault}
\selectfont
\end{enumerate}
\subsection{/+gras/+geom/+ell/+test}
\begin{enumerate}
\item \hyperlink{/+gras/+geom/+ell/+test/run\_tests}{run\_tests}
\fontfamily{pcr}
\selectfont
\begin{lstlisting}
%

\end{lstlisting}
\fontfamily{\familydefault}
\selectfont
\end{enumerate}
\subsection{/+gras/+geom/+ell/+test/+mlunit}
\begin{enumerate}
\item \hyperlink{/+gras/+geom/+ell/+test/+mlunit/SuiteBasic}{SuiteBasic}
\fontfamily{pcr}
\selectfont
\begin{lstlisting}
%

\end{lstlisting}
\fontfamily{\familydefault}
\selectfont
\end{enumerate}
\subsection{/+gras/+geom/+sup}
\begin{enumerate}
\item \hyperlink{/+gras/+geom/+sup/sup2boundary2}{sup2boundary2}
\fontfamily{pcr}
\selectfont
\begin{lstlisting}
%  SUP2BOUNDARY2 approximates aMat boundary of 3d set using aMat support
%  function values defined for the directions from aMat triangulated unit
%  sphere
\end{lstlisting}
\fontfamily{\familydefault}
\selectfont
\item \hyperlink{/+gras/+geom/+sup/sup2boundary3}{sup2boundary3}
\fontfamily{pcr}
\selectfont
\begin{lstlisting}
%  SUP2BOUNDARY3 approximates aMat boundary of 3d set using aMat support
%  function values defined for the directions from aMat triangulated unit
%  sphere
\end{lstlisting}
\fontfamily{\familydefault}
\selectfont
\item \hyperlink{/+gras/+geom/+sup/supgeomdiff2d}{supgeomdiff2d}
\fontfamily{pcr}
\selectfont
\begin{lstlisting}
%  SUPGEOMDIFF2D calculates support function of two 2-dimensional
%  convex sets defined by their support functions
\end{lstlisting}
\fontfamily{\familydefault}
\selectfont
\end{enumerate}
\subsection{/+gras/+geom/+sup/+test}
\begin{enumerate}
\item \hyperlink{/+gras/+geom/+sup/+test/qint2}{qint2}
\fontfamily{pcr}
\selectfont
\begin{lstlisting}
%

\end{lstlisting}
\fontfamily{\familydefault}
\selectfont
\item \hyperlink{/+gras/+geom/+sup/+test/run\_tests}{run\_tests}
\fontfamily{pcr}
\selectfont
\begin{lstlisting}
%

\end{lstlisting}
\fontfamily{\familydefault}
\selectfont
\end{enumerate}
\subsection{/+gras/+geom/+sup/+test/+mlunit}
\begin{enumerate}
\item \hyperlink{/+gras/+geom/+sup/+test/+mlunit/SuiteBasic}{SuiteBasic}
\fontfamily{pcr}
\selectfont
\begin{lstlisting}
%

\end{lstlisting}
\fontfamily{\familydefault}
\selectfont
\end{enumerate}
\subsection{/+gras/+geom/+test}
\begin{enumerate}
\item \hyperlink{/+gras/+geom/+test/run\_tests}{run\_tests}
\fontfamily{pcr}
\selectfont
\begin{lstlisting}
%

\end{lstlisting}
\fontfamily{\familydefault}
\selectfont
\end{enumerate}
\subsection{/+gras/+geom/+tri}
\begin{enumerate}
\item \hyperlink{/+gras/+geom/+tri/elltubetri}{elltubetri}
\fontfamily{pcr}
\selectfont
\begin{lstlisting}
%  ELLTUBETRI builds a triangulation of ellipsoidal tube
\end{lstlisting}
\fontfamily{\familydefault}
\selectfont
\item \hyperlink{/+gras/+geom/+tri/icosahedron}{icosahedron}
\fontfamily{pcr}
\selectfont
\begin{lstlisting}
%  ICOSAHEDRON generates a triangulation corresponding to Icosahedron's
%  surface
\end{lstlisting}
\fontfamily{\familydefault}
\selectfont
\item \hyperlink{/+gras/+geom/+tri/isface}{isface}
\fontfamily{pcr}
\selectfont
\begin{lstlisting}
%  ISFACE checks if the specified faces belong to the given triangulation
\end{lstlisting}
\fontfamily{\familydefault}
\selectfont
\item \hyperlink{/+gras/+geom/+tri/istriequal}{istriequal}
\fontfamily{pcr}
\selectfont
\begin{lstlisting}
%  ISTRIEQUAL checks if the matrices specify the same triangulation
%  (permunations of edge orders directions, vertices in faces do not count)
\end{lstlisting}
\fontfamily{\familydefault}
\selectfont
\item \hyperlink{/+gras/+geom/+tri/mapface2edge}{mapface2edge}
\fontfamily{pcr}
\selectfont
\begin{lstlisting}
%  MAPFACE2EDGE creates a mapping from faces to edges
\end{lstlisting}
\fontfamily{\familydefault}
\selectfont
\item \hyperlink{/+gras/+geom/+tri/shrinkfacetri}{shrinkfacetri}
\fontfamily{pcr}
\selectfont
\begin{lstlisting}
%  SHRINKFACETRI shrinks faces of 3D triangulation space down to a
%  degree where a length of each edge is less or equal to a specified value
\end{lstlisting}
\fontfamily{\familydefault}
\selectfont
\item \hyperlink{/+gras/+geom/+tri/spheretri}{spheretri}
\fontfamily{pcr}
\selectfont
\begin{lstlisting}
%  SPHERETRI builds a triangulation of a unit sphere based on recursive
%  partitioning each of Icosahedron faces into 4 triangles with vertices in
%  the middles of original face edgeMidMat
\end{lstlisting}
\fontfamily{\familydefault}
\selectfont
\end{enumerate}
\subsection{/+gras/+geom/+tri/+test}
\begin{enumerate}
\item \hyperlink{/+gras/+geom/+tri/+test/run\_tests}{run\_tests}
\fontfamily{pcr}
\selectfont
\begin{lstlisting}
%

\end{lstlisting}
\fontfamily{\familydefault}
\selectfont
\item \hyperlink{/+gras/+geom/+tri/+test/spheretri}{spheretri}
\fontfamily{pcr}
\selectfont
\begin{lstlisting}
%  SPHERETRI builds a triangulation of a unit sphere based on recursive
%  partitioning each of Icosahedron faces into 4 triangles with vertices in
%  the middles of original face edgeMidMat
\end{lstlisting}
\fontfamily{\familydefault}
\selectfont
\end{enumerate}
\subsection{/+gras/+geom/+tri/+test/+mlunit}
\begin{enumerate}
\item \hyperlink{/+gras/+geom/+tri/+test/+mlunit/SuiteTri}{SuiteTri}
\fontfamily{pcr}
\selectfont
\begin{lstlisting}
%

\end{lstlisting}
\fontfamily{\familydefault}
\selectfont
\end{enumerate}
\subsection{/+gras/+geom/+tri/+test/srebuild3d}
\begin{enumerate}
\item \hyperlink{/+gras/+geom/+tri/+test/srebuild3d/build}{build}
\fontfamily{pcr}
\selectfont
\begin{lstlisting}
%

\end{lstlisting}
\fontfamily{\familydefault}
\selectfont
\end{enumerate}
\subsection{/+gras/+interp}
\begin{enumerate}
\item \hyperlink{/+gras/+interp/AMatrixCubicSpline}{AMatrixCubicSpline}
\fontfamily{pcr}
\selectfont
\begin{lstlisting}
%  $Author: Peter Gagarinov  <pgagarinov@gmail.com> $	$Date: 2011-08$
%  $Copyright: Moscow State University,
%             Faculty of Computational Mathematics and Computer Science,
%             System Analysis Department 2011 $
\end{lstlisting}
\fontfamily{\familydefault}
\selectfont
\item \hyperlink{/+gras/+interp/MatrixColCubicSpline}{MatrixColCubicSpline}
\fontfamily{pcr}
\selectfont
\begin{lstlisting}
%  $Author: Peter Gagarinov  <pgagarinov@gmail.com> $	$Date: 2011-08$
%  $Copyright: Moscow State University,
%             Faculty of Computational Mathematics and Computer Science,
%             System Analysis Department 2011 $
\end{lstlisting}
\fontfamily{\familydefault}
\selectfont
\item \hyperlink{/+gras/+interp/MatrixColTriuCubicSpline}{MatrixColTriuCubicSpline}
\fontfamily{pcr}
\selectfont
\begin{lstlisting}
%  $Author: Peter Gagarinov  <pgagarinov@gmail.com> $	$Date: 2011-08$
%  $Copyright: Moscow State University,
%             Faculty of Computational Mathematics and Computer Science,
%             System Analysis Department 2011 $
\end{lstlisting}
\fontfamily{\familydefault}
\selectfont
\item \hyperlink{/+gras/+interp/MatrixColTriuSymmCubicSpline}{MatrixColTriuSymmCubicSpline}
\fontfamily{pcr}
\selectfont
\begin{lstlisting}
%  $Author: Peter Gagarinov  <pgagarinov@gmail.com> $	$Date: 2011-08$
%  $Copyright: Moscow State University,
%             Faculty of Computational Mathematics and Computer Science,
%             System Analysis Department 2011 $
\end{lstlisting}
\fontfamily{\familydefault}
\selectfont
\item \hyperlink{/+gras/+interp/MatrixInterpolantFactory}{MatrixInterpolantFactory}
\fontfamily{pcr}
\selectfont
\begin{lstlisting}
%

\end{lstlisting}
\fontfamily{\familydefault}
\selectfont
\item \hyperlink{/+gras/+interp/MatrixRowCubicSpline}{MatrixRowCubicSpline}
\fontfamily{pcr}
\selectfont
\begin{lstlisting}
%  $Author: Peter Gagarinov  <pgagarinov@gmail.com> $	$Date: 2011-08$
%  $Copyright: Moscow State University,
%             Faculty of Computational Mathematics and Computer Science,
%             System Analysis Department 2011 $
\end{lstlisting}
\fontfamily{\familydefault}
\selectfont
\item \hyperlink{/+gras/+interp/NNDefMatCholCubicSpline}{NNDefMatCholCubicSpline}
\fontfamily{pcr}
\selectfont
\begin{lstlisting}
%  $Author: Peter Gagarinov  <pgagarinov@gmail.com> $	$Date: 2011-10$
%  $Copyright: Moscow State University,
%             Faculty of Computational Mathematics and Computer Science,
%             System Analysis Department 2011 $
\end{lstlisting}
\fontfamily{\familydefault}
\selectfont
\item \hyperlink{/+gras/+interp/PosDefMatCholCubicSpline}{PosDefMatCholCubicSpline}
\fontfamily{pcr}
\selectfont
\begin{lstlisting}
%  $Author: Peter Gagarinov  <pgagarinov@gmail.com> $	$Date: 2011-08$
%  $Copyright: Moscow State University,
%             Faculty of Computational Mathematics and Computer Science,
%             System Analysis Department 2011 $
\end{lstlisting}
\fontfamily{\familydefault}
\selectfont
\item \hyperlink{/+gras/+interp/SplineMatrixOperations}{SplineMatrixOperations}
\fontfamily{pcr}
\selectfont
\begin{lstlisting}
%

\end{lstlisting}
\fontfamily{\familydefault}
\selectfont
\end{enumerate}
\subsection{/+gras/+interp/+test}
\begin{enumerate}
\item \hyperlink{/+gras/+interp/+test/run\_tests}{run\_tests}
\fontfamily{pcr}
\selectfont
\begin{lstlisting}
%

\end{lstlisting}
\fontfamily{\familydefault}
\selectfont
\end{enumerate}
\subsection{/+gras/+interp/+test/+mlunit}
\begin{enumerate}
\item \hyperlink{/+gras/+interp/+test/+mlunit/SuiteBasic}{SuiteBasic}
\fontfamily{pcr}
\selectfont
\begin{lstlisting}
%

\end{lstlisting}
\fontfamily{\familydefault}
\selectfont
\end{enumerate}
\subsection{/+gras/+la}
\begin{enumerate}
\item \hyperlink{/+gras/+la/ismatposdef}{ismatposdef}
\fontfamily{pcr}
\selectfont
\begin{lstlisting}
%  ISMATPOSDEF  checks if qMat is positive definite
\end{lstlisting}
\fontfamily{\familydefault}
\selectfont
\item \hyperlink{/+gras/+la/ismatsymm}{ismatsymm}
\fontfamily{pcr}
\selectfont
\begin{lstlisting}
%  ISMATSYMM  checks if qMat is symmetric
\end{lstlisting}
\fontfamily{\familydefault}
\selectfont
\item \hyperlink{/+gras/+la/matorth}{matorth}
\fontfamily{pcr}
\selectfont
\begin{lstlisting}
%  MATORTH generates an orthogonal matrix that contains in its first k
%  columns orthogonalized vectors specified on input as [n,k] matrix
\end{lstlisting}
\fontfamily{\familydefault}
\selectfont
\item \hyperlink{/+gras/+la/mlorthtransl}{mlorthtransl}
\fontfamily{pcr}
\selectfont
\begin{lstlisting}
%  MLORTHTRANSL generates a set of orthogonal matrices that translate each of
%  the given vectors into a corresponding another vector from another set
\end{lstlisting}
\fontfamily{\familydefault}
\selectfont
\item \hyperlink{/+gras/+la/orthtransl}{orthtransl}
\fontfamily{pcr}
\selectfont
\begin{lstlisting}
%  ORTHTRANSL generates an orthogonal matrix that translates a specified
%  vector to another vector that is collinear to the second specified vector
\end{lstlisting}
\fontfamily{\familydefault}
\selectfont
\item \hyperlink{/+gras/+la/orthtranslhaus}{orthtranslhaus}
\fontfamily{pcr}
\selectfont
\begin{lstlisting}
%  ORTHTRANSLHAUS generates an orthogonal matrix that translates a specified
%  vector to another vector that is collinear to the second specified vector
%  using the Hausholder method:
%    w=srcVec-dstVec;
%    oMat=I-2*w*w.'./(w.'*w)
\end{lstlisting}
\fontfamily{\familydefault}
\selectfont
\item \hyperlink{/+gras/+la/orthtranslmaxdir}{orthtranslmaxdir}
\fontfamily{pcr}
\selectfont
\begin{lstlisting}
%  ORTHTRANSLMAXDIR generates an orthogonal matrix oMat that translates
%  vector srcVec to another vector that is collinear to the second 
%  specified vector dstVec. The matrix is chosen to maximize 
%  (oMat*srcMaxVec,dstMaxVec)
\end{lstlisting}
\fontfamily{\familydefault}
\selectfont
\item \hyperlink{/+gras/+la/orthtranslmaxtr}{orthtranslmaxtr}
\fontfamily{pcr}
\selectfont
\begin{lstlisting}
%  ORTHTRANSLMAXVOL generates an orthogonal matrix oMat that translates 
%  a specified vector srcVec to another vector that is collinear to 
%  the second specified vector dstVec
%  The matrix S is chosen to maximize Tr(oMat*maxMat) where maxMat
%  is specified
\end{lstlisting}
\fontfamily{\familydefault}
\selectfont
\item \hyperlink{/+gras/+la/sqrtm}{sqrtm}
\fontfamily{pcr}
\selectfont
\begin{lstlisting}
%  SQRTM generates a square root from matrix QMat 
%  Input:
%       QMat: double[nDims, nDims]
%  Output:
%    QsqrtMat: double[nDims,nDims]
\end{lstlisting}
\fontfamily{\familydefault}
\selectfont
\end{enumerate}
\subsection{/+gras/+la/+test}
\begin{enumerate}
\item \hyperlink{/+gras/+la/+test/run\_tests}{run\_tests}
\fontfamily{pcr}
\selectfont
\begin{lstlisting}
%

\end{lstlisting}
\fontfamily{\familydefault}
\selectfont
\end{enumerate}
\subsection{/+gras/+la/+test/+mlunit}
\begin{enumerate}
\item \hyperlink{/+gras/+la/+test/+mlunit/BasicTestCase}{BasicTestCase}
\fontfamily{pcr}
\selectfont
\begin{lstlisting}
%  $Author: Vadim Kaushanskiy, Moscow State University by M.V. Lomonosov,
%  Faculty of Computational Mathematics and Cybernetics, System Analysis
%  Department, 1-November-2012, <vkaushanskiy@gmail.com>$
\end{lstlisting}
\fontfamily{\familydefault}
\selectfont
\item \hyperlink{/+gras/+la/+test/+mlunit/SuiteOrthTransl}{SuiteOrthTransl}
\fontfamily{pcr}
\selectfont
\begin{lstlisting}
%

\end{lstlisting}
\fontfamily{\familydefault}
\selectfont
\end{enumerate}
\subsection{/+gras/+mat}
\begin{enumerate}
\item \hyperlink{/+gras/+mat/AConstMatrixFunction}{AConstMatrixFunction}
\fontfamily{pcr}
\selectfont
\begin{lstlisting}
%

\end{lstlisting}
\fontfamily{\familydefault}
\selectfont
\item \hyperlink{/+gras/+mat/AMatrixBinaryOpFunc}{AMatrixBinaryOpFunc}
\fontfamily{pcr}
\selectfont
\begin{lstlisting}
%

\end{lstlisting}
\fontfamily{\familydefault}
\selectfont
\item \hyperlink{/+gras/+mat/AMatrixOpFunc}{AMatrixOpFunc}
\fontfamily{pcr}
\selectfont
\begin{lstlisting}
%

\end{lstlisting}
\fontfamily{\familydefault}
\selectfont
\item \hyperlink{/+gras/+mat/AMatrixOperations}{AMatrixOperations}
\fontfamily{pcr}
\selectfont
\begin{lstlisting}
%

\end{lstlisting}
\fontfamily{\familydefault}
\selectfont
\item \hyperlink{/+gras/+mat/AMatrixTernaryOpFunc}{AMatrixTernaryOpFunc}
\fontfamily{pcr}
\selectfont
\begin{lstlisting}
%

\end{lstlisting}
\fontfamily{\familydefault}
\selectfont
\item \hyperlink{/+gras/+mat/AMatrixUnaryOpFunc}{AMatrixUnaryOpFunc}
\fontfamily{pcr}
\selectfont
\begin{lstlisting}
%

\end{lstlisting}
\fontfamily{\familydefault}
\selectfont
\item \hyperlink{/+gras/+mat/CompositeMatrixOperations}{CompositeMatrixOperations}
\fontfamily{pcr}
\selectfont
\begin{lstlisting}
%

\end{lstlisting}
\fontfamily{\familydefault}
\selectfont
\item \hyperlink{/+gras/+mat/ConstMatrixFunctionFactory}{ConstMatrixFunctionFactory}
\fontfamily{pcr}
\selectfont
\begin{lstlisting}
%

\end{lstlisting}
\fontfamily{\familydefault}
\selectfont
\item \hyperlink{/+gras/+mat/IMatrixFunction}{IMatrixFunction}
\fontfamily{pcr}
\selectfont
\begin{lstlisting}
%

\end{lstlisting}
\fontfamily{\familydefault}
\selectfont
\item \hyperlink{/+gras/+mat/IMatrixOperations}{IMatrixOperations}
\fontfamily{pcr}
\selectfont
\begin{lstlisting}
%

\end{lstlisting}
\fontfamily{\familydefault}
\selectfont
\item \hyperlink{/+gras/+mat/MatrixOperationsFactory}{MatrixOperationsFactory}
\fontfamily{pcr}
\selectfont
\begin{lstlisting}
%

\end{lstlisting}
\fontfamily{\familydefault}
\selectfont
\end{enumerate}
\subsection{/+gras/+mat/+fcnlib}
\begin{enumerate}
\item \hyperlink{/+gras/+mat/+fcnlib/ConstColFunction}{ConstColFunction}
\fontfamily{pcr}
\selectfont
\begin{lstlisting}
%

\end{lstlisting}
\fontfamily{\familydefault}
\selectfont
\item \hyperlink{/+gras/+mat/+fcnlib/ConstMatrixFunction}{ConstMatrixFunction}
\fontfamily{pcr}
\selectfont
\begin{lstlisting}
%

\end{lstlisting}
\fontfamily{\familydefault}
\selectfont
\item \hyperlink{/+gras/+mat/+fcnlib/ConstRowFunction}{ConstRowFunction}
\fontfamily{pcr}
\selectfont
\begin{lstlisting}
%

\end{lstlisting}
\fontfamily{\familydefault}
\selectfont
\item \hyperlink{/+gras/+mat/+fcnlib/MatrixBinaryTimesFunc}{MatrixBinaryTimesFunc}
\fontfamily{pcr}
\selectfont
\begin{lstlisting}
%

\end{lstlisting}
\fontfamily{\familydefault}
\selectfont
\item \hyperlink{/+gras/+mat/+fcnlib/MatrixExpFunc}{MatrixExpFunc}
\fontfamily{pcr}
\selectfont
\begin{lstlisting}
%

\end{lstlisting}
\fontfamily{\familydefault}
\selectfont
\item \hyperlink{/+gras/+mat/+fcnlib/MatrixExpTimeFunc}{MatrixExpTimeFunc}
\fontfamily{pcr}
\selectfont
\begin{lstlisting}
%

\end{lstlisting}
\fontfamily{\familydefault}
\selectfont
\item \hyperlink{/+gras/+mat/+fcnlib/MatrixInvFunc}{MatrixInvFunc}
\fontfamily{pcr}
\selectfont
\begin{lstlisting}
%

\end{lstlisting}
\fontfamily{\familydefault}
\selectfont
\item \hyperlink{/+gras/+mat/+fcnlib/MatrixLRDivideVecFunc}{MatrixLRDivideVecFunc}
\fontfamily{pcr}
\selectfont
\begin{lstlisting}
%

\end{lstlisting}
\fontfamily{\familydefault}
\selectfont
\item \hyperlink{/+gras/+mat/+fcnlib/MatrixLRTimesFunc}{MatrixLRTimesFunc}
\fontfamily{pcr}
\selectfont
\begin{lstlisting}
%

\end{lstlisting}
\fontfamily{\familydefault}
\selectfont
\item \hyperlink{/+gras/+mat/+fcnlib/MatrixMakeSymmetricFunc}{MatrixMakeSymmetricFunc}
\fontfamily{pcr}
\selectfont
\begin{lstlisting}
%

\end{lstlisting}
\fontfamily{\familydefault}
\selectfont
\item \hyperlink{/+gras/+mat/+fcnlib/MatrixMinEigValFunc}{MatrixMinEigValFunc}
\fontfamily{pcr}
\selectfont
\begin{lstlisting}
%

\end{lstlisting}
\fontfamily{\familydefault}
\selectfont
\item \hyperlink{/+gras/+mat/+fcnlib/MatrixMinusFunc}{MatrixMinusFunc}
\fontfamily{pcr}
\selectfont
\begin{lstlisting}
%

\end{lstlisting}
\fontfamily{\familydefault}
\selectfont
\item \hyperlink{/+gras/+mat/+fcnlib/MatrixPInvFunc}{MatrixPInvFunc}
\fontfamily{pcr}
\selectfont
\begin{lstlisting}
%

\end{lstlisting}
\fontfamily{\familydefault}
\selectfont
\item \hyperlink{/+gras/+mat/+fcnlib/MatrixPlusFunc}{MatrixPlusFunc}
\fontfamily{pcr}
\selectfont
\begin{lstlisting}
%

\end{lstlisting}
\fontfamily{\familydefault}
\selectfont
\item \hyperlink{/+gras/+mat/+fcnlib/MatrixSqrtFunc}{MatrixSqrtFunc}
\fontfamily{pcr}
\selectfont
\begin{lstlisting}
%

\end{lstlisting}
\fontfamily{\familydefault}
\selectfont
\item \hyperlink{/+gras/+mat/+fcnlib/MatrixTernaryTimesFunc}{MatrixTernaryTimesFunc}
\fontfamily{pcr}
\selectfont
\begin{lstlisting}
%

\end{lstlisting}
\fontfamily{\familydefault}
\selectfont
\item \hyperlink{/+gras/+mat/+fcnlib/MatrixTransposeFunc}{MatrixTransposeFunc}
\fontfamily{pcr}
\selectfont
\begin{lstlisting}
%

\end{lstlisting}
\fontfamily{\familydefault}
\selectfont
\item \hyperlink{/+gras/+mat/+fcnlib/MatrixTriuFunc}{MatrixTriuFunc}
\fontfamily{pcr}
\selectfont
\begin{lstlisting}
%

\end{lstlisting}
\fontfamily{\familydefault}
\selectfont
\item \hyperlink{/+gras/+mat/+fcnlib/QuadraticFormSqrtFunc}{QuadraticFormSqrtFunc}
\fontfamily{pcr}
\selectfont
\begin{lstlisting}
%

\end{lstlisting}
\fontfamily{\familydefault}
\selectfont
\end{enumerate}
\subsection{/+gras/+mat/+symb}
\begin{enumerate}
\item \hyperlink{/+gras/+mat/+symb/MatrixSFBinaryProd}{MatrixSFBinaryProd}
\fontfamily{pcr}
\selectfont
\begin{lstlisting}
%  $Author: Peter Gagarinov  <pgagarinov@gmail.com> $	$Date: 2011-12-12$
%  $Copyright: Moscow State University,
%             Faculty of Computational Mathematics and Computer Science,
%             System Analysis Department 2011 $
\end{lstlisting}
\fontfamily{\familydefault}
\selectfont
\item \hyperlink{/+gras/+mat/+symb/MatrixSFBinaryProdByVec}{MatrixSFBinaryProdByVec}
\fontfamily{pcr}
\selectfont
\begin{lstlisting}
%  $Author: Peter Gagarinov  <pgagarinov@gmail.com> $	$Date: 2011-12-12$
%  $Copyright: Moscow State University,
%             Faculty of Computational Mathematics and Computer Science,
%             System Analysis Department 2011 $
\end{lstlisting}
\fontfamily{\familydefault}
\selectfont
\item \hyperlink{/+gras/+mat/+symb/MatrixSFTripleProd}{MatrixSFTripleProd}
\fontfamily{pcr}
\selectfont
\begin{lstlisting}
%

\end{lstlisting}
\fontfamily{\familydefault}
\selectfont
\item \hyperlink{/+gras/+mat/+symb/MatrixSymbFormulaBased}{MatrixSymbFormulaBased}
\fontfamily{pcr}
\selectfont
\begin{lstlisting}
%  $Author: Peter Gagarinov  <pgagarinov@gmail.com> $	$Date: 2011-12-12$
%  $Copyright: Moscow State University,
%             Faculty of Computational Mathematics and Computer Science,
%             System Analysis Department 2011 $
\end{lstlisting}
\fontfamily{\familydefault}
\selectfont
\item \hyperlink{/+gras/+mat/+symb/iscellofstringconst}{iscellofstringconst}
\fontfamily{pcr}
\selectfont
\begin{lstlisting}
%

\end{lstlisting}
\fontfamily{\familydefault}
\selectfont
\end{enumerate}
\subsection{/+gras/+mat/+test}
\begin{enumerate}
\item \hyperlink{/+gras/+mat/+test/run\_tests}{run\_tests}
\fontfamily{pcr}
\selectfont
\begin{lstlisting}
%

\end{lstlisting}
\fontfamily{\familydefault}
\selectfont
\end{enumerate}
\subsection{/+gras/+mat/+test/+mlunit}
\begin{enumerate}
\item \hyperlink{/+gras/+mat/+test/+mlunit/SuiteBasic}{SuiteBasic}
\fontfamily{pcr}
\selectfont
\begin{lstlisting}
%

\end{lstlisting}
\fontfamily{\familydefault}
\selectfont
\item \hyperlink{/+gras/+mat/+test/+mlunit/SuiteOp}{SuiteOp}
\fontfamily{pcr}
\selectfont
\begin{lstlisting}
%

\end{lstlisting}
\fontfamily{\familydefault}
\selectfont
\end{enumerate}
\subsection{/+gras/+ode}
\begin{enumerate}
\item \hyperlink{/+gras/+ode/MatrixODESolver}{MatrixODESolver}
\fontfamily{pcr}
\selectfont
\begin{lstlisting}
%

\end{lstlisting}
\fontfamily{\familydefault}
\selectfont
\item \hyperlink{/+gras/+ode/MatrixSysODESolver}{MatrixSysODESolver}
\fontfamily{pcr}
\selectfont
\begin{lstlisting}
%

\end{lstlisting}
\fontfamily{\familydefault}
\selectfont
\item \hyperlink{/+gras/+ode/ode113reg}{ode113reg}
\fontfamily{pcr}
\selectfont
\begin{lstlisting}
%

\end{lstlisting}
\fontfamily{\familydefault}
\selectfont
\item \hyperlink{/+gras/+ode/ode45reg}{ode45reg}
\fontfamily{pcr}
\selectfont
\begin{lstlisting}
%  ODE45REG is an extension of built-in ode45 solver capable of solving ODEs
%  with right hand-side functions having a limited definition area
\end{lstlisting}
\fontfamily{\familydefault}
\selectfont
\end{enumerate}
\subsection{/+gras/+ode/+test}
\begin{enumerate}
\item \hyperlink{/+gras/+ode/+test/run\_tests}{run\_tests}
\fontfamily{pcr}
\selectfont
\begin{lstlisting}
%

\end{lstlisting}
\fontfamily{\familydefault}
\selectfont
\end{enumerate}
\subsection{/+gras/+ode/+test/+mlunit}
\begin{enumerate}
\item \hyperlink{/+gras/+ode/+test/+mlunit/SuiteBasic}{SuiteBasic}
\fontfamily{pcr}
\selectfont
\begin{lstlisting}
%

\end{lstlisting}
\fontfamily{\familydefault}
\selectfont
\end{enumerate}
\subsection{/+gras/+ode/private}
\begin{enumerate}
\item \hyperlink{/+gras/+ode/private/odearguments}{odearguments}
\fontfamily{pcr}
\selectfont
\begin{lstlisting}
%

\end{lstlisting}
\fontfamily{\familydefault}
\selectfont
\item \hyperlink{/+gras/+ode/private/odenonnegative}{odenonnegative}
\fontfamily{pcr}
\selectfont
\begin{lstlisting}
%

\end{lstlisting}
\fontfamily{\familydefault}
\selectfont
\end{enumerate}
\subsection{/+gras/+test}
\begin{enumerate}
\item \hyperlink{/+gras/+test/TmpDataManager}{TmpDataManager}
\fontfamily{pcr}
\selectfont
\begin{lstlisting}
%  TMPDATAMANAGER provides a basic functionality for managing temporary
%  data folders, root folder name is determined automatically
\end{lstlisting}
\fontfamily{\familydefault}
\selectfont
\item \hyperlink{/+gras/+test/editconf}{editconf}
\fontfamily{pcr}
\selectfont
\begin{lstlisting}
%

\end{lstlisting}
\fontfamily{\familydefault}
\selectfont
\item \hyperlink{/+gras/+test/run\_tests}{run\_tests}
\fontfamily{pcr}
\selectfont
\begin{lstlisting}
%

\end{lstlisting}
\fontfamily{\familydefault}
\selectfont
\item \hyperlink{/+gras/+test/run\_tests\_remotely}{run\_tests\_remotely}
\fontfamily{pcr}
\selectfont
\begin{lstlisting}
%

\end{lstlisting}
\fontfamily{\familydefault}
\selectfont
\end{enumerate}
\subsection{/+gras/+test/+configuration}
\begin{enumerate}
\item \hyperlink{/+gras/+test/+configuration/AdaptiveConfRepoManager}{AdaptiveConfRepoManager}
\fontfamily{pcr}
\selectfont
\begin{lstlisting}
%  ADAPTIVECONFREPOMANAGER is a simplistic extension of
%  AdaptiveConfRepoManager that injects a configuration change
%  repository class equivolent.test.configuration.ConfPatchRepo
%  automatically
\end{lstlisting}
\fontfamily{\familydefault}
\selectfont
\end{enumerate}
\subsection{/+gras/+test/+configuration/@ConfPatchRepo}
\begin{enumerate}
\item \hyperlink{/+gras/+test/+configuration/@ConfPatchRepo/ConfPatchRepo}{ConfPatchRepo}
\fontfamily{pcr}
\selectfont
\begin{lstlisting}
%

\end{lstlisting}
\fontfamily{\familydefault}
\selectfont
\item \hyperlink{/+gras/+test/+configuration/@ConfPatchRepo/patch\_001\_dummy\_patch}{patch\_001\_dummy\_patch}
\fontfamily{pcr}
\selectfont
\begin{lstlisting}
%

\end{lstlisting}
\fontfamily{\familydefault}
\selectfont
\end{enumerate}
\subsection{/+gras/+test/+logging}
\begin{enumerate}
\item \hyperlink{/+gras/+test/+logging/Log4jConfigurator}{Log4jConfigurator}
\fontfamily{pcr}
\selectfont
\begin{lstlisting}
% LOG4JCONFIGURATOR simplifies log4j configuration, especially when
% Parallel Computing Toolbox is used. In the latter case the class forwards
% the logs of different processees in separate log files
\end{lstlisting}
\fontfamily{\familydefault}
\selectfont
\end{enumerate}
\subsection{/+gras/+test/+mlunit}
\begin{enumerate}
\item \hyperlink{/+gras/+test/+mlunit/SuiteBasic}{SuiteBasic}
\fontfamily{pcr}
\selectfont
\begin{lstlisting}
%

\end{lstlisting}
\fontfamily{\familydefault}
\selectfont
\end{enumerate}
\subsection{/elltoolboxcore/@ellipsoid}
\begin{enumerate}
\item \hyperlink{/elltoolboxcore/@ellipsoid/checkIsMe}{checkIsMe}
\fontfamily{pcr}
\selectfont
\begin{lstlisting}
%

\end{lstlisting}
\fontfamily{\familydefault}
\selectfont
\item \hyperlink{/elltoolboxcore/@ellipsoid/contains}{contains}
\fontfamily{pcr}
\selectfont
\begin{lstlisting}
%  CONTAINS - checks if one ellipsoid contains the other.
%             The condition for E1 = firstEllArr to contain
%             E2 = secondEllArr is
%             min(rho(l | E1) - rho(l | E2)) > 0, subject to <l, l> = 1.
\end{lstlisting}
\fontfamily{\familydefault}
\selectfont
\item \hyperlink{/elltoolboxcore/@ellipsoid/contents}{contents}
\fontfamily{pcr}
\selectfont
\begin{lstlisting}
%  Ellipsoid library of the Ellipsoidal Toolbox.
\end{lstlisting}
\fontfamily{\familydefault}
\selectfont
\item \hyperlink{/elltoolboxcore/@ellipsoid/dimension}{dimension}
\fontfamily{pcr}
\selectfont
\begin{lstlisting}
%  DIMENSION - returns the dimension of the space in which the ellipsoid
%              is defined and the actual dimension of the ellipsoid.
\end{lstlisting}
\fontfamily{\familydefault}
\selectfont
\item \hyperlink{/elltoolboxcore/@ellipsoid/disp}{disp}
\fontfamily{pcr}
\selectfont
\begin{lstlisting}
%  DISP - Displays ellipsoid object.
\end{lstlisting}
\fontfamily{\familydefault}
\selectfont
\item \hyperlink{/elltoolboxcore/@ellipsoid/display}{display}
\fontfamily{pcr}
\selectfont
\begin{lstlisting}
%  DISPLAY - Displays the details of the ellipsoid object.
\end{lstlisting}
\fontfamily{\familydefault}
\selectfont
\item \hyperlink{/elltoolboxcore/@ellipsoid/distance}{distance}
\fontfamily{pcr}
\selectfont
\begin{lstlisting}
%  DISTANCE - computes distance between the given ellipsoid (or array of
%             ellipsoids) to the specified object (or arrays of objects):
%             vector, ellipsoid, hyperplane or polytope.
\end{lstlisting}
\fontfamily{\familydefault}
\selectfont
\item \hyperlink{/elltoolboxcore/@ellipsoid/double}{double}
\fontfamily{pcr}
\selectfont
\begin{lstlisting}
%  DOUBLE - returns parameters of the ellipsoid.
\end{lstlisting}
\fontfamily{\familydefault}
\selectfont
\item \hyperlink{/elltoolboxcore/@ellipsoid/ellbndr\_2d}{ellbndr\_2d}
\fontfamily{pcr}
\selectfont
\begin{lstlisting}
%

\end{lstlisting}
\fontfamily{\familydefault}
\selectfont
\item \hyperlink{/elltoolboxcore/@ellipsoid/ellbndr\_3d}{ellbndr\_3d}
\fontfamily{pcr}
\selectfont
\begin{lstlisting}
%

\end{lstlisting}
\fontfamily{\familydefault}
\selectfont
\item \hyperlink{/elltoolboxcore/@ellipsoid/ellintersection\_ia}{ellintersection\_ia}
\fontfamily{pcr}
\selectfont
\begin{lstlisting}
%  ELLINTERSECTION_IA - computes maximum volume ellipsoid that is
%                       contained in the intersection of
%                       given ellipsoids.
\end{lstlisting}
\fontfamily{\familydefault}
\selectfont
\item \hyperlink{/elltoolboxcore/@ellipsoid/ellipsoid}{ellipsoid}
\fontfamily{pcr}
\selectfont
\begin{lstlisting}
%  ELLIPSOID - constructor of the ellipsoid object.
\end{lstlisting}
\fontfamily{\familydefault}
\selectfont
\item \hyperlink{/elltoolboxcore/@ellipsoid/ellunion\_ea}{ellunion\_ea}
\fontfamily{pcr}
\selectfont
\begin{lstlisting}
%  ELLUNION_EA - computes minimum volume ellipsoid that contains union
%                of given ellipsoids.
\end{lstlisting}
\fontfamily{\familydefault}
\selectfont
\item \hyperlink{/elltoolboxcore/@ellipsoid/eq}{eq}
\fontfamily{pcr}
\selectfont
\begin{lstlisting}
%  EQ - compares two arrays of ellipsoids
\end{lstlisting}
\fontfamily{\familydefault}
\selectfont
\item \hyperlink{/elltoolboxcore/@ellipsoid/ge}{ge}
\fontfamily{pcr}
\selectfont
\begin{lstlisting}
%  GE - checks if the first ellipsoid is bigger than the second one.
%       Same as GT.
\end{lstlisting}
\fontfamily{\familydefault}
\selectfont
\item \hyperlink{/elltoolboxcore/@ellipsoid/getAbsTol}{getAbsTol}
\fontfamily{pcr}
\selectfont
\begin{lstlisting}
%  GETABSTOL - gives array the same size as ellArr with values of absTol
%    properties for each ellipsoid in ellArr
\end{lstlisting}
\fontfamily{\familydefault}
\selectfont
\item \hyperlink{/elltoolboxcore/@ellipsoid/getNPlot2dPoints}{getNPlot2dPoints}
\fontfamily{pcr}
\selectfont
\begin{lstlisting}
%  GETNPLOT2DPOINTS - gives value of nPlot2dPoints property
%    of ellipsoids in ellArr
\end{lstlisting}
\fontfamily{\familydefault}
\selectfont
\item \hyperlink{/elltoolboxcore/@ellipsoid/getNPlot3dPoints}{getNPlot3dPoints}
\fontfamily{pcr}
\selectfont
\begin{lstlisting}
%  GETNPLOT3DPOINTS - gives value of nPlot3dPoints property
%    of ellipsoids in ellArr
\end{lstlisting}
\fontfamily{\familydefault}
\selectfont
\item \hyperlink{/elltoolboxcore/@ellipsoid/getProperty}{getProperty}
\fontfamily{pcr}
\selectfont
\begin{lstlisting}
%

\end{lstlisting}
\fontfamily{\familydefault}
\selectfont
\item \hyperlink{/elltoolboxcore/@ellipsoid/getRelTol}{getRelTol}
\fontfamily{pcr}
\selectfont
\begin{lstlisting}
%  GETRELTOL - gives array the same size as ellArr with values of relTol
%              properties for each ellipsoid in ellArr
\end{lstlisting}
\fontfamily{\familydefault}
\selectfont
\item \hyperlink{/elltoolboxcore/@ellipsoid/gt}{gt}
\fontfamily{pcr}
\selectfont
\begin{lstlisting}
%  GT - checks if the first ellipsoid is bigger than the second one.
\end{lstlisting}
\fontfamily{\familydefault}
\selectfont
\item \hyperlink{/elltoolboxcore/@ellipsoid/hpintersection}{hpintersection}
\fontfamily{pcr}
\selectfont
\begin{lstlisting}
%  HPINTERSECTION - computes the intersection of ellipsoid with hyperplane.
\end{lstlisting}
\fontfamily{\familydefault}
\selectfont
\item \hyperlink{/elltoolboxcore/@ellipsoid/intersect}{intersect}
\fontfamily{pcr}
\selectfont
\begin{lstlisting}
%  INTERSECT - checks if the union or intersection of ellipsoids intersects
%              given ellipsoid, hyperplane or polytope.
\end{lstlisting}
\fontfamily{\familydefault}
\selectfont
\item \hyperlink{/elltoolboxcore/@ellipsoid/intersection\_ea}{intersection\_ea}
\fontfamily{pcr}
\selectfont
\begin{lstlisting}
%  INTERSECTION_EA - external ellipsoidal approximation of the
%                    intersection of two ellipsoids, or ellipsoid and
%                    halfspace, or ellipsoid and polytope.
\end{lstlisting}
\fontfamily{\familydefault}
\selectfont
\item \hyperlink{/elltoolboxcore/@ellipsoid/intersection\_ia}{intersection\_ia}
\fontfamily{pcr}
\selectfont
\begin{lstlisting}
%  INTERSECTION_IA - internal ellipsoidal approximation of the
%                    intersection of ellipsoid and ellipsoid,
%                    or ellipsoid and halfspace, or ellipsoid
%                    and polytope.
\end{lstlisting}
\fontfamily{\familydefault}
\selectfont
\item \hyperlink{/elltoolboxcore/@ellipsoid/inv}{inv}
\fontfamily{pcr}
\selectfont
\begin{lstlisting}
%  INV - inverts shape matrices of ellipsoids in the given array.
\end{lstlisting}
\fontfamily{\familydefault}
\selectfont
\item \hyperlink{/elltoolboxcore/@ellipsoid/isbaddirection}{isbaddirection}
\fontfamily{pcr}
\selectfont
\begin{lstlisting}
%  ISBADDIRECTION - checks if ellipsoidal approximations of geometric
%                   difference of two ellipsoids can be computed for
%                   given directions.
%    isBadDirVec = ISBADDIRECTION(fstEll, secEll, dirsMat) - Checks if
%        it is possible to build ellipsoidal approximation of the
%        geometric difference of two ellipsoids fstEll - secEll in
%        directions specified by matrix dirsMat (columns of dirsMat
%        are direction vectors). Type 'help minkdiff_ea' or
%        'help minkdiff_ia' for more information.
\end{lstlisting}
\fontfamily{\familydefault}
\selectfont
\item \hyperlink{/elltoolboxcore/@ellipsoid/isbaddirectionmat}{isbaddirectionmat}
\fontfamily{pcr}
\selectfont
\begin{lstlisting}
%

\end{lstlisting}
\fontfamily{\familydefault}
\selectfont
\item \hyperlink{/elltoolboxcore/@ellipsoid/isbigger}{isbigger}
\fontfamily{pcr}
\selectfont
\begin{lstlisting}
%  ISBIGGER - checks if one ellipsoid would contain the other if their
%             centers would coincide.
\end{lstlisting}
\fontfamily{\familydefault}
\selectfont
\item \hyperlink{/elltoolboxcore/@ellipsoid/isdegenerate}{isdegenerate}
\fontfamily{pcr}
\selectfont
\begin{lstlisting}
%  ISDEGENERATE - checks if the ellipsoid is degenerate.
\end{lstlisting}
\fontfamily{\familydefault}
\selectfont
\item \hyperlink{/elltoolboxcore/@ellipsoid/isempty}{isempty}
\fontfamily{pcr}
\selectfont
\begin{lstlisting}
%  ISEMPTY - checks if the ellipsoid object is empty.
\end{lstlisting}
\fontfamily{\familydefault}
\selectfont
\item \hyperlink{/elltoolboxcore/@ellipsoid/isinside}{isinside}
\fontfamily{pcr}
\selectfont
\begin{lstlisting}
%  ISINSIDE - checks if the intersection of ellipsoids contains the
%             union or intersection of given ellipsoids or polytopes.
\end{lstlisting}
\fontfamily{\familydefault}
\selectfont
\item \hyperlink{/elltoolboxcore/@ellipsoid/isinternal}{isinternal}
\fontfamily{pcr}
\selectfont
\begin{lstlisting}
%  ISINTERNAL - checks if given points belong to the union or intersection
%               of ellipsoids in the given array.
\end{lstlisting}
\fontfamily{\familydefault}
\selectfont
\item \hyperlink{/elltoolboxcore/@ellipsoid/le}{le}
\fontfamily{pcr}
\selectfont
\begin{lstlisting}
%  LE - checks if the second ellipsoid is bigger than the first one.
%       Same as LT.
\end{lstlisting}
\fontfamily{\familydefault}
\selectfont
\item \hyperlink{/elltoolboxcore/@ellipsoid/lt}{lt}
\fontfamily{pcr}
\selectfont
\begin{lstlisting}
%  LT - checks if the second ellipsoid is bigger than the first one.
%       The opposite of GT.
\end{lstlisting}
\fontfamily{\familydefault}
\selectfont
\item \hyperlink{/elltoolboxcore/@ellipsoid/maxeig}{maxeig}
\fontfamily{pcr}
\selectfont
\begin{lstlisting}
%  MAXEIG - return the maximal eigenvalue of the ellipsoid.
\end{lstlisting}
\fontfamily{\familydefault}
\selectfont
\item \hyperlink{/elltoolboxcore/@ellipsoid/mineig}{mineig}
\fontfamily{pcr}
\selectfont
\begin{lstlisting}
%  MINEIG - return the minimal eigenvalue of the ellipsoid.
\end{lstlisting}
\fontfamily{\familydefault}
\selectfont
\item \hyperlink{/elltoolboxcore/@ellipsoid/minkdiff}{minkdiff}
\fontfamily{pcr}
\selectfont
\begin{lstlisting}
%  MINKDIFF - computes geometric (Minkowski) difference of two
%             ellipsoids in 2D or 3D.
\end{lstlisting}
\fontfamily{\familydefault}
\selectfont
\item \hyperlink{/elltoolboxcore/@ellipsoid/minkdiff\_ea}{minkdiff\_ea}
\fontfamily{pcr}
\selectfont
\begin{lstlisting}
%  MINKDIFF_EA - computation of external approximating ellipsoids
%                of the geometric difference of two ellipsoids along
%                given directions.
\end{lstlisting}
\fontfamily{\familydefault}
\selectfont
\item \hyperlink{/elltoolboxcore/@ellipsoid/minkdiff\_ia}{minkdiff\_ia}
\fontfamily{pcr}
\selectfont
\begin{lstlisting}
%  MINKDIFF_IA - computation of internal approximating ellipsoids
%                of the geometric difference of two ellipsoids along
%                given directions.
\end{lstlisting}
\fontfamily{\familydefault}
\selectfont
\item \hyperlink{/elltoolboxcore/@ellipsoid/minkmp}{minkmp}
\fontfamily{pcr}
\selectfont
\begin{lstlisting}
%  MINKMP - computes and plots geometric (Minkowski) sum of the
%           geometric difference of two ellipsoids and the geometric
%           sum of n ellipsoids in 2D or 3D:
%           (E - Em) + (E1 + E2 + ... + En),
%           where E = firstEll, Em = secondEll,
%           E1, E2, ..., En - are ellipsoids in sumEllArr
\end{lstlisting}
\fontfamily{\familydefault}
\selectfont
\item \hyperlink{/elltoolboxcore/@ellipsoid/minkmp\_ea}{minkmp\_ea}
\fontfamily{pcr}
\selectfont
\begin{lstlisting}
%  MINKMP_EA - computation of external approximating ellipsoids
%              of (E - Em) + (E1 + ... + En) along given directions.
%              where E = fstEll, Em = secEll,
%              E1, E2, ..., En - are ellipsoids in sumEllArr
\end{lstlisting}
\fontfamily{\familydefault}
\selectfont
\item \hyperlink{/elltoolboxcore/@ellipsoid/minkmp\_ia}{minkmp\_ia}
\fontfamily{pcr}
\selectfont
\begin{lstlisting}
%  MINKMP_IA - computation of internal approximating ellipsoids
%              of (E - Em) + (E1 + ... + En) along given directions.
%              where E = fstEll, Em = secEll,
%              E1, E2, ..., En - are ellipsoids in sumEllArr
\end{lstlisting}
\fontfamily{\familydefault}
\selectfont
\item \hyperlink{/elltoolboxcore/@ellipsoid/minkpm}{minkpm}
\fontfamily{pcr}
\selectfont
\begin{lstlisting}
%  MINKPM - computes and plots geometric (Minkowski) difference
%           of the geometric sum of ellipsoids and a single ellipsoid
%           in 2D or 3D: (E1 + E2 + ... + En) - E,
%           where E = inpEll,
%           E1, E2, ... En - are ellipsoids in inpEllArr.
\end{lstlisting}
\fontfamily{\familydefault}
\selectfont
\item \hyperlink{/elltoolboxcore/@ellipsoid/minkpm\_ea}{minkpm\_ea}
\fontfamily{pcr}
\selectfont
\begin{lstlisting}
%  MINKPM_EA - computation of external approximating ellipsoids
%              of (E1 + E2 + ... + En) - E along given directions.
%              where E = inpEll,
%              E1, E2, ... En - are ellipsoids in inpEllArr.
\end{lstlisting}
\fontfamily{\familydefault}
\selectfont
\item \hyperlink{/elltoolboxcore/@ellipsoid/minkpm\_ia}{minkpm\_ia}
\fontfamily{pcr}
\selectfont
\begin{lstlisting}
%  MINKPM_IA - computation of internal approximating ellipsoids
%              of (E1 + E2 + ... + En) - E along given directions.
%              where E = inpEll,
%              E1, E2, ... En - are ellipsoids in inpEllArr.
\end{lstlisting}
\fontfamily{\familydefault}
\selectfont
\item \hyperlink{/elltoolboxcore/@ellipsoid/minksum}{minksum}
\fontfamily{pcr}
\selectfont
\begin{lstlisting}
%  MINKSUM - computes geometric (Minkowski) sum of ellipsoids in 2D or 3D.
\end{lstlisting}
\fontfamily{\familydefault}
\selectfont
\item \hyperlink{/elltoolboxcore/@ellipsoid/minksum\_ea}{minksum\_ea}
\fontfamily{pcr}
\selectfont
\begin{lstlisting}
%  MINKSUM_EA - computation of external approximating ellipsoids
%               of the geometric sum of ellipsoids along given directions.
\end{lstlisting}
\fontfamily{\familydefault}
\selectfont
\item \hyperlink{/elltoolboxcore/@ellipsoid/minksum\_ia}{minksum\_ia}
\fontfamily{pcr}
\selectfont
\begin{lstlisting}
%  MINKSUM_IA - computation of internal approximating ellipsoids
%               of the geometric sum of ellipsoids along given directions.
\end{lstlisting}
\fontfamily{\familydefault}
\selectfont
\item \hyperlink{/elltoolboxcore/@ellipsoid/minus}{minus}
\fontfamily{pcr}
\selectfont
\begin{lstlisting}
%  MINUS - overloaded operator '-'
\end{lstlisting}
\fontfamily{\familydefault}
\selectfont
\item \hyperlink{/elltoolboxcore/@ellipsoid/move2origin}{move2origin}
\fontfamily{pcr}
\selectfont
\begin{lstlisting}
%  MOVE2ORIGIN - moves ellipsoids in the given array to the origin.
\end{lstlisting}
\fontfamily{\familydefault}
\selectfont
\item \hyperlink{/elltoolboxcore/@ellipsoid/mtimes}{mtimes}
\fontfamily{pcr}
\selectfont
\begin{lstlisting}
%  MTIMES - overloaded operator '*'.
\end{lstlisting}
\fontfamily{\familydefault}
\selectfont
\item \hyperlink{/elltoolboxcore/@ellipsoid/my\_color\_table}{my\_color\_table}
\fontfamily{pcr}
\selectfont
\begin{lstlisting}
%

\end{lstlisting}
\fontfamily{\familydefault}
\selectfont
\item \hyperlink{/elltoolboxcore/@ellipsoid/ne}{ne}
\fontfamily{pcr}
\selectfont
\begin{lstlisting}
%  NE - the opposite of EQ
\end{lstlisting}
\fontfamily{\familydefault}
\selectfont
\item \hyperlink{/elltoolboxcore/@ellipsoid/parameters}{parameters}
\fontfamily{pcr}
\selectfont
\begin{lstlisting}
%  PARAMETERS - returns parameters of the ellipsoid.
\end{lstlisting}
\fontfamily{\familydefault}
\selectfont
\item \hyperlink{/elltoolboxcore/@ellipsoid/plot}{plot}
\fontfamily{pcr}
\selectfont
\begin{lstlisting}
%  PLOT - plots ellipsoids in 2D or 3D.
\end{lstlisting}
\fontfamily{\familydefault}
\selectfont
\item \hyperlink{/elltoolboxcore/@ellipsoid/plot3}{plot3}
\fontfamily{pcr}
\selectfont
\begin{lstlisting}
%  PLOT3 - plots ellipsoids in 2D or 3D.
\end{lstlisting}
\fontfamily{\familydefault}
\selectfont
\item \hyperlink{/elltoolboxcore/@ellipsoid/plus}{plus}
\fontfamily{pcr}
\selectfont
\begin{lstlisting}
%  PLUS - overloaded operator '+'
\end{lstlisting}
\fontfamily{\familydefault}
\selectfont
\item \hyperlink{/elltoolboxcore/@ellipsoid/polar}{polar}
\fontfamily{pcr}
\selectfont
\begin{lstlisting}
%  POLAR - computes the polar ellipsoids.
\end{lstlisting}
\fontfamily{\familydefault}
\selectfont
\item \hyperlink{/elltoolboxcore/@ellipsoid/projection}{projection}
\fontfamily{pcr}
\selectfont
\begin{lstlisting}
%  PROJECTION - computes projection of the ellipsoid onto the given subspace.
\end{lstlisting}
\fontfamily{\familydefault}
\selectfont
\item \hyperlink{/elltoolboxcore/@ellipsoid/regularize}{regularize}
\fontfamily{pcr}
\selectfont
\begin{lstlisting}
%

\end{lstlisting}
\fontfamily{\familydefault}
\selectfont
\item \hyperlink{/elltoolboxcore/@ellipsoid/rho}{rho}
\fontfamily{pcr}
\selectfont
\begin{lstlisting}
%  RHO - computes the values of the support function for given ellipsoid
% 	and given direction.
\end{lstlisting}
\fontfamily{\familydefault}
\selectfont
\item \hyperlink{/elltoolboxcore/@ellipsoid/rm\_bad\_directions}{rm\_bad\_directions}
\fontfamily{pcr}
\selectfont
\begin{lstlisting}
%

\end{lstlisting}
\fontfamily{\familydefault}
\selectfont
\item \hyperlink{/elltoolboxcore/@ellipsoid/shape}{shape}
\fontfamily{pcr}
\selectfont
\begin{lstlisting}
%  SHAPE - modifies the shape matrix of the ellipsoid without
%    changing its center.
\end{lstlisting}
\fontfamily{\familydefault}
\selectfont
\item \hyperlink{/elltoolboxcore/@ellipsoid/trace}{trace}
\fontfamily{pcr}
\selectfont
\begin{lstlisting}
%  TRACE - returns the trace of the ellipsoid.
\end{lstlisting}
\fontfamily{\familydefault}
\selectfont
\item \hyperlink{/elltoolboxcore/@ellipsoid/uminus}{uminus}
\fontfamily{pcr}
\selectfont
\begin{lstlisting}
%  UMINUS - changes the sign of the center of ellipsoid.
\end{lstlisting}
\fontfamily{\familydefault}
\selectfont
\item \hyperlink{/elltoolboxcore/@ellipsoid/volume}{volume}
\fontfamily{pcr}
\selectfont
\begin{lstlisting}
%  VOLUME - returns the volume of the ellipsoid.
\end{lstlisting}
\fontfamily{\familydefault}
\selectfont
\end{enumerate}
\subsection{/elltoolboxcore/@hyperplane}
\begin{enumerate}
\item \hyperlink{/elltoolboxcore/@hyperplane/checkIsMe}{checkIsMe}
\fontfamily{pcr}
\selectfont
\begin{lstlisting}
%  CHECKISME - determine whether input object is hyperplane. And display
%              message and abort function if input object
%              is not hyperplane
\end{lstlisting}
\fontfamily{\familydefault}
\selectfont
\item \hyperlink{/elltoolboxcore/@hyperplane/contains}{contains}
\fontfamily{pcr}
\selectfont
\begin{lstlisting}
%  CONTAINS - checks if given vectors belong to the hyperplanes.
\end{lstlisting}
\fontfamily{\familydefault}
\selectfont
\item \hyperlink{/elltoolboxcore/@hyperplane/contents}{contents}
\fontfamily{pcr}
\selectfont
\begin{lstlisting}
%  Hyperplane object of the Ellipsoidal Toolbox.
\end{lstlisting}
\fontfamily{\familydefault}
\selectfont
\item \hyperlink{/elltoolboxcore/@hyperplane/dimension}{dimension}
\fontfamily{pcr}
\selectfont
\begin{lstlisting}
%  DIMENSION - returns dimensions of hyperplanes in the array.
\end{lstlisting}
\fontfamily{\familydefault}
\selectfont
\item \hyperlink{/elltoolboxcore/@hyperplane/display}{display}
\fontfamily{pcr}
\selectfont
\begin{lstlisting}
%  DISPLAY - Displays hyperplane object.
\end{lstlisting}
\fontfamily{\familydefault}
\selectfont
\item \hyperlink{/elltoolboxcore/@hyperplane/double}{double}
\fontfamily{pcr}
\selectfont
\begin{lstlisting}
%  DOUBLE - return parameters of hyperplane - normal vector and shift.
\end{lstlisting}
\fontfamily{\familydefault}
\selectfont
\item \hyperlink{/elltoolboxcore/@hyperplane/eq}{eq}
\fontfamily{pcr}
\selectfont
\begin{lstlisting}
%  EQ - check if two hyperplanes are the same.
\end{lstlisting}
\fontfamily{\familydefault}
\selectfont
\item \hyperlink{/elltoolboxcore/@hyperplane/getAbsTol}{getAbsTol}
\fontfamily{pcr}
\selectfont
\begin{lstlisting}
%  GETABSTOL - gives array the same size as hplaneArr with values
%              of absTol properties for each hyperplane in hplaneArr.
\end{lstlisting}
\fontfamily{\familydefault}
\selectfont
\item \hyperlink{/elltoolboxcore/@hyperplane/hyperplane}{hyperplane}
\fontfamily{pcr}
\selectfont
\begin{lstlisting}
%  HYPERPLANE - creates hyperplane structure
%               (or array of hyperplane structures).
\end{lstlisting}
\fontfamily{\familydefault}
\selectfont
\item \hyperlink{/elltoolboxcore/@hyperplane/isempty}{isempty}
\fontfamily{pcr}
\selectfont
\begin{lstlisting}
%  ISEMPTY - checks if hyperplanes in H are empty.
\end{lstlisting}
\fontfamily{\familydefault}
\selectfont
\item \hyperlink{/elltoolboxcore/@hyperplane/isparallel}{isparallel}
\fontfamily{pcr}
\selectfont
\begin{lstlisting}
%  ISPARALLEL - check if two hyperplanes are parallel.
\end{lstlisting}
\fontfamily{\familydefault}
\selectfont
\item \hyperlink{/elltoolboxcore/@hyperplane/ne}{ne}
\fontfamily{pcr}
\selectfont
\begin{lstlisting}
%  NE - The opposite of EQ.
\end{lstlisting}
\fontfamily{\familydefault}
\selectfont
\item \hyperlink{/elltoolboxcore/@hyperplane/parameters}{parameters}
\fontfamily{pcr}
\selectfont
\begin{lstlisting}
%  PARAMETERS - return parameters of hyperplane - normal vector and shift.
\end{lstlisting}
\fontfamily{\familydefault}
\selectfont
\item \hyperlink{/elltoolboxcore/@hyperplane/plot}{plot}
\fontfamily{pcr}
\selectfont
\begin{lstlisting}
%  PLOT - plots hyperplanes in 2D or 3D.
\end{lstlisting}
\fontfamily{\familydefault}
\selectfont
\item \hyperlink{/elltoolboxcore/@hyperplane/uminus}{uminus}
\fontfamily{pcr}
\selectfont
\begin{lstlisting}
%  UMINUS - switch signs of normal vector and the shift scalar
%           to the opposite.
\end{lstlisting}
\fontfamily{\familydefault}
\selectfont
\end{enumerate}
\subsection{/elltoolboxcore/auxiliary}
\begin{enumerate}
\item \hyperlink{/elltoolboxcore/auxiliary/ell\_enclose}{ell\_enclose}
\fontfamily{pcr}
\selectfont
\begin{lstlisting}
%  ELL_ENCLOSE - computes minimum volume ellipsoid that contains given vectors.
\end{lstlisting}
\fontfamily{\familydefault}
\selectfont
\item \hyperlink{/elltoolboxcore/auxiliary/ell\_fusionlambda}{ell\_fusionlambda}
\fontfamily{pcr}
\selectfont
\begin{lstlisting}
%  ELL_FUSIONLAMBDA - function whose root in the interval (0, 1) determines
%                     the minimal volume ellipsoid overapproximating the
%                     intersection of two ellipsoids.
\end{lstlisting}
\fontfamily{\familydefault}
\selectfont
\item \hyperlink{/elltoolboxcore/auxiliary/ell\_inv}{ell\_inv}
\fontfamily{pcr}
\selectfont
\begin{lstlisting}
%  ELL_INV - computes matrix inverse treating ill-conditioned matrices properly.
\end{lstlisting}
\fontfamily{\familydefault}
\selectfont
\item \hyperlink{/elltoolboxcore/auxiliary/ell\_simdiag}{ell\_simdiag}
\fontfamily{pcr}
\selectfont
\begin{lstlisting}
%  ELL_SIMDIAG - computes the transformation matrix that simultaneously
%                diagonalizes two symmetric matrices.
\end{lstlisting}
\fontfamily{\familydefault}
\selectfont
\item \hyperlink{/elltoolboxcore/auxiliary/ell\_unitball}{ell\_unitball}
\fontfamily{pcr}
\selectfont
\begin{lstlisting}
%  ELL_UNITBALL - creates unit ball object
\end{lstlisting}
\fontfamily{\familydefault}
\selectfont
\item \hyperlink{/elltoolboxcore/auxiliary/ell\_valign}{ell\_valign}
\fontfamily{pcr}
\selectfont
\begin{lstlisting}
%  ELL_VALIGN - given two vectors in R^n, computes orthogonal matrix that rotates
%               the second vector making it parallel to the first one.
\end{lstlisting}
\fontfamily{\familydefault}
\selectfont
\item \hyperlink{/elltoolboxcore/auxiliary/hyperplane2polytope}{hyperplane2polytope}
\fontfamily{pcr}
\selectfont
\begin{lstlisting}
%  HYPERPLANE2POLYTOPE - converts array of hyperplanes into polytope
\end{lstlisting}
\fontfamily{\familydefault}
\selectfont
\item \hyperlink{/elltoolboxcore/auxiliary/polytope2hyperplane}{polytope2hyperplane}
\fontfamily{pcr}
\selectfont
\begin{lstlisting}
%  POLYTOPE2HYPERPLANE - converts given polytope object into the array
%                        of hyperplanes.
\end{lstlisting}
\fontfamily{\familydefault}
\selectfont
\end{enumerate}
\subsection{/elltoolboxcore/control/auxiliary}
\begin{enumerate}
\item \hyperlink{/elltoolboxcore/control/auxiliary/ell\_center\_ode}{ell\_center\_ode}
\fontfamily{pcr}
\selectfont
\begin{lstlisting}
%  ELL_CENTER_ODE - ODE for the center of the reach set.
\end{lstlisting}
\fontfamily{\familydefault}
\selectfont
\item \hyperlink{/elltoolboxcore/control/auxiliary/ell\_eedist\_ode}{ell\_eedist\_ode}
\fontfamily{pcr}
\selectfont
\begin{lstlisting}
%  ELL_EEDIST_ODE - ODE for the shape matrix of the external ellipsoid
%                   for system with disturbance.
\end{lstlisting}
\fontfamily{\familydefault}
\selectfont
\item \hyperlink{/elltoolboxcore/control/auxiliary/ell\_eesm\_ode}{ell\_eesm\_ode}
\fontfamily{pcr}
\selectfont
\begin{lstlisting}
%  ELL_EESM_ODE - ODE for the shape matrix of the external ellipsoid.
\end{lstlisting}
\fontfamily{\familydefault}
\selectfont
\item \hyperlink{/elltoolboxcore/control/auxiliary/ell\_iedist\_ode}{ell\_iedist\_ode}
\fontfamily{pcr}
\selectfont
\begin{lstlisting}
%  ELL_IEDIST_ODE - ODE for the shape matrix of the internal ellipsoid
%                   for system with disturbance.
\end{lstlisting}
\fontfamily{\familydefault}
\selectfont
\item \hyperlink{/elltoolboxcore/control/auxiliary/ell\_iesm\_ode}{ell\_iesm\_ode}
\fontfamily{pcr}
\selectfont
\begin{lstlisting}
%  ELL_IESM_ODE - ODE for the shape matrix of the internal ellipsoid.
\end{lstlisting}
\fontfamily{\familydefault}
\selectfont
\item \hyperlink{/elltoolboxcore/control/auxiliary/ell\_ode\_solver}{ell\_ode\_solver}
\fontfamily{pcr}
\selectfont
\begin{lstlisting}
%  ELL_ODE_SOLVER - caller for particular ODE solver.
\end{lstlisting}
\fontfamily{\familydefault}
\selectfont
\item \hyperlink{/elltoolboxcore/control/auxiliary/ell\_regularize}{ell\_regularize}
\fontfamily{pcr}
\selectfont
\begin{lstlisting}
%  ELL_REGULARIZE - regularization of singular matrix.
\end{lstlisting}
\fontfamily{\familydefault}
\selectfont
\item \hyperlink{/elltoolboxcore/control/auxiliary/ell\_stm\_ode}{ell\_stm\_ode}
\fontfamily{pcr}
\selectfont
\begin{lstlisting}
%  ELL_STM_ODE - ODE for state transition matrix.
\end{lstlisting}
\fontfamily{\familydefault}
\selectfont
\item \hyperlink{/elltoolboxcore/control/auxiliary/ell\_value\_extract}{ell\_value\_extract}
\fontfamily{pcr}
\selectfont
\begin{lstlisting}
%  ELL_VALUE_EXTRACT - extracts matrix value from ppform or vector array.
\end{lstlisting}
\fontfamily{\familydefault}
\selectfont
\item \hyperlink{/elltoolboxcore/control/auxiliary/iesm\_ode}{iesm\_ode}
\fontfamily{pcr}
\selectfont
\begin{lstlisting}
%  ELL_IESM_ODE - ODE for the shape matrix of the internal ellipsoid.
\end{lstlisting}
\fontfamily{\familydefault}
\selectfont
\end{enumerate}
\subsection{/elltoolboxcore/demo}
\begin{enumerate}
\item \hyperlink{/elltoolboxcore/demo/ell\_demo0}{ell\_demo0}
\fontfamily{pcr}
\selectfont
\begin{lstlisting}
%  Demo of the ellipsoidal calculus.
\end{lstlisting}
\fontfamily{\familydefault}
\selectfont
\item \hyperlink{/elltoolboxcore/demo/ell\_demo1}{ell\_demo1}
\fontfamily{pcr}
\selectfont
\begin{lstlisting}
%  Demo of the ellipsoidal calculus.
\end{lstlisting}
\fontfamily{\familydefault}
\selectfont
\item \hyperlink{/elltoolboxcore/demo/ell\_demo2}{ell\_demo2}
\fontfamily{pcr}
\selectfont
\begin{lstlisting}
%  Demo of the ellipsoid visualization.
\end{lstlisting}
\fontfamily{\familydefault}
\selectfont
\item \hyperlink{/elltoolboxcore/demo/ell\_demo3}{ell\_demo3}
\fontfamily{pcr}
\selectfont
\begin{lstlisting}
%  Reachability Demo.
\end{lstlisting}
\fontfamily{\familydefault}
\selectfont
\end{enumerate}
\subsection{/elltoolboxcore/graphics}
\begin{enumerate}
\item \hyperlink{/elltoolboxcore/graphics/ell\_plot}{ell\_plot}
\fontfamily{pcr}
\selectfont
\begin{lstlisting}
%  Description:
%  ------------
\end{lstlisting}
\fontfamily{\familydefault}
\selectfont
\item \hyperlink{/elltoolboxcore/graphics/ell\_square\_facets}{ell\_square\_facets}
\fontfamily{pcr}
\selectfont
\begin{lstlisting}
%  ELL_SQUARE_FACETS - generates square facets to be used in PATCH function call.
\end{lstlisting}
\fontfamily{\familydefault}
\selectfont
\item \hyperlink{/elltoolboxcore/graphics/ell\_triag\_facets}{ell\_triag\_facets}
\fontfamily{pcr}
\selectfont
\begin{lstlisting}
%  ELL_TRIAG_FACETS - generates triangular facets to be used in PATCH function call.
\end{lstlisting}
\fontfamily{\familydefault}
\selectfont
\end{enumerate}
\subsection{/elltoolboxcore/solvers/gradient}
\begin{enumerate}
\item \hyperlink{/elltoolboxcore/solvers/gradient/ell\_nlfnlc}{ell\_nlfnlc}
\fontfamily{pcr}
\selectfont
\begin{lstlisting}
%  ELL_NLFNLC - computes minimum of nonlinear function with nonlinear constraints.
\end{lstlisting}
\fontfamily{\familydefault}
\selectfont
\end{enumerate}
\subsection{/elltoolboxcore/solvers/gradient/private}
\begin{enumerate}
\item \hyperlink{/elltoolboxcore/solvers/gradient/private/compute\_direction}{compute\_direction}
\fontfamily{pcr}
\selectfont
\begin{lstlisting}
%  COMPUTE_DIRECTION - computes a search direction in a subspace defined by Z.
\end{lstlisting}
\fontfamily{\familydefault}
\selectfont
\item \hyperlink{/elltoolboxcore/solvers/gradient/private/nlcp\_solve}{nlcp\_solve}
\fontfamily{pcr}
\selectfont
\begin{lstlisting}
%  NLCP_SOLVE - nonlinear function minimizer under nonlinear constraints.
\end{lstlisting}
\fontfamily{\familydefault}
\selectfont
\item \hyperlink{/elltoolboxcore/solvers/gradient/private/qps}{qps}
\fontfamily{pcr}
\selectfont
\begin{lstlisting}
%  QPS - Quadratic programming problem.
\end{lstlisting}
\fontfamily{\familydefault}
\selectfont
\end{enumerate}
\section{List of authors}
\subsection{/+elltool}
\begin{enumerate}
\item \hyperlink{/+elltool/copyconf}{copyconf}
\fontfamily{pcr}
\selectfont
\begin{lstlisting}
% $Author: Zakharov Eugene  <justenterrr@gmail.com> $    $Date: 17-november-2012 $
% $Copyright: Moscow State University,
%             Faculty of Computational Mathematics and Computer Science,
%             System Analysis Department 2012 $
% 
%

\end{lstlisting}
\fontfamily{\familydefault}
\selectfont
\item \hyperlink{/+elltool/editconf}{editconf}
\fontfamily{pcr}
\selectfont
\begin{lstlisting}
% $Author: Zakharov Eugene  <justenterrr@gmail.com> $    $Date: 17-november-2012 $
% $Copyright: Moscow State University,
%             Faculty of Computational Mathematics and Computer Science,
%             System Analysis Department 2012 $
% 
%

\end{lstlisting}
\fontfamily{\familydefault}
\selectfont
\item \hyperlink{/+elltool/listconf}{listconf}
\fontfamily{pcr}
\selectfont
\begin{lstlisting}
% $Author: Zakharov Eugene  <justenterrr@gmail.com> $    $Date: 17-november-2012 $
% $Copyright: Moscow State University,
%             Faculty of Computational Mathematics and Computer Science,
%             System Analysis Department 2012 $
% 
%

\end{lstlisting}
\fontfamily{\familydefault}
\selectfont
\item \hyperlink{/+elltool/setconf}{setconf}
\fontfamily{pcr}
\selectfont
\begin{lstlisting}
% $Author: Zakharov Eugene  <justenterrr@gmail.com> $    $Date: 17-november-2012 $
% $Copyright: Moscow State University,
%             Faculty of Computational Mathematics and Computer Science,
%             System Analysis Department 2012 $
% 
%

\end{lstlisting}
\fontfamily{\familydefault}
\selectfont
\end{enumerate}
\subsection{/+elltool/+conf}
\begin{enumerate}
\item \hyperlink{/+elltool/+conf/ConfRepoMgr}{ConfRepoMgr}
\fontfamily{pcr}
\selectfont
\begin{lstlisting}
%  $Author: <Zakharov Eugene>  <justenterrr@gmail.com> $    $Date: <5 november> $
%  $Copyright: Moscow State University,
%             Faculty of Computational Mathematics and Computer Science,
%             System Analysis Department <2012> $
% 
%

\end{lstlisting}
\fontfamily{\familydefault}
\selectfont
\end{enumerate}
\subsection{/+elltool/+conf/+test}
\begin{enumerate}
\item \hyperlink{/+elltool/+conf/+test/run\_tests}{run\_tests}
\fontfamily{pcr}
\selectfont
\begin{lstlisting}
%

\end{lstlisting}
\fontfamily{\familydefault}
\selectfont
\end{enumerate}
\subsection{/+elltool/+conf/+test/+mlunit}
\begin{enumerate}
\item \hyperlink{/+elltool/+conf/+test/+mlunit/PropertiesTestCase}{PropertiesTestCase}
\fontfamily{pcr}
\selectfont
\begin{lstlisting}
%  $Author: <Zakharov Eugene>  <justenterrr@gmail.com> $    $Date: <5 november> $
%  $Copyright: Moscow State University,
%             Faculty of Computational Mathematics and Computer Science,
%             System Analysis Department <2012> $
% 
%

\end{lstlisting}
\fontfamily{\familydefault}
\selectfont
\end{enumerate}
\subsection{/+elltool/+conf/@ConfPatchRepo}
\begin{enumerate}
\item \hyperlink{/+elltool/+conf/@ConfPatchRepo/ConfPatchRepo}{ConfPatchRepo}
\fontfamily{pcr}
\selectfont
\begin{lstlisting}
%

\end{lstlisting}
\fontfamily{\familydefault}
\selectfont
\item \hyperlink{/+elltool/+conf/@ConfPatchRepo/patch\_001\_dummy\_patch}{patch\_001\_dummy\_patch}
\fontfamily{pcr}
\selectfont
\begin{lstlisting}
%

\end{lstlisting}
\fontfamily{\familydefault}
\selectfont
\item \hyperlink{/+elltool/+conf/@ConfPatchRepo/patch\_002\_add\_log4j}{patch\_002\_add\_log4j}
\fontfamily{pcr}
\selectfont
\begin{lstlisting}
%

\end{lstlisting}
\fontfamily{\familydefault}
\selectfont
\end{enumerate}
\subsection{/+elltool/+conf/@Properties}
\begin{enumerate}
\item \hyperlink{/+elltool/+conf/@Properties/Properties}{Properties}
\fontfamily{pcr}
\selectfont
\begin{lstlisting}
% $Author: Zakharov Eugene  <justenterrr@gmail.com> $    $Date: 5-november-2012 $
% $Author: Peter Gagarinov  <pgagarinov@gmail.com> $    $Date: 25-november-2012 $    
% $Copyright: Moscow State University,
%             Faculty of Computational Mathematics and Computer Science,
%             System Analysis Department 2012 $
% 
%

\end{lstlisting}
\fontfamily{\familydefault}
\selectfont
\item \hyperlink{/+elltool/+conf/@Properties/parseProp}{parseProp}
\fontfamily{pcr}
\selectfont
\begin{lstlisting}
% $Author: Zakharov Eugene  <justenterrr@gmail.com> $    $Date: 5-november-2012 $
% $Copyright: Moscow State University,
%             Faculty of Computational Mathematics and Computer Science,
%             System Analysis Department 2012 $
% 
%

\end{lstlisting}
\fontfamily{\familydefault}
\selectfont
\end{enumerate}
\subsection{/+elltool/+core/+test}
\begin{enumerate}
\item \hyperlink{/+elltool/+core/+test/run\_tests}{run\_tests}
\fontfamily{pcr}
\selectfont
\begin{lstlisting}
%

\end{lstlisting}
\fontfamily{\familydefault}
\selectfont
\end{enumerate}
\subsection{/+elltool/+core/+test/+mlunit}
\begin{enumerate}
\item \hyperlink{/+elltool/+core/+test/+mlunit/EllSecTCMultiDim}{EllSecTCMultiDim}
\fontfamily{pcr}
\selectfont
\begin{lstlisting}
%  $Author: Igor Samokhin, Lomonosov Moscow State University,
%  Faculty of Computational Mathematics and Cybernetics, System Analysis
%  Department, 31-January-2013, <igorian.vmk@gmail.com>$
%  $Copyright: Moscow State University,
%             Faculty of Computational Mathematics and Computer Science,
%             System Analysis Department 2012 $
%

\end{lstlisting}
\fontfamily{\familydefault}
\selectfont
\item \hyperlink{/+elltool/+core/+test/+mlunit/EllTCMultiDim}{EllTCMultiDim}
\fontfamily{pcr}
\selectfont
\begin{lstlisting}
%  $Author: Igor Samokhin, Lomonosov Moscow State University,
%  Faculty of Computational Mathematics and Cybernetics, System Analysis
%  Department, 31-January-2013, <igorian.vmk@gmail.com>$
%  $Copyright: Moscow State University,
%             Faculty of Computational Mathematics and Computer Science,
%             System Analysis Department 2013 $
%

\end{lstlisting}
\fontfamily{\familydefault}
\selectfont
\item \hyperlink{/+elltool/+core/+test/+mlunit/ElliIntUnionTCMultiDim}{ElliIntUnionTCMultiDim}
\fontfamily{pcr}
\selectfont
\begin{lstlisting}
%  $Author: Igor Samokhin, Lomonosov Moscow State University,
%  Faculty of Computational Mathematics and Cybernetics, System Analysis
%  Department, 31-January-2013, <igorian.vmk@gmail.com>$
%  $Copyright: Moscow State University,
%             Faculty of Computational Mathematics and Computer Science,
%             System Analysis Department 2013 $
%

\end{lstlisting}
\fontfamily{\familydefault}
\selectfont
\item \hyperlink{/+elltool/+core/+test/+mlunit/EllipsoidIntUnionTC}{EllipsoidIntUnionTC}
\fontfamily{pcr}
\selectfont
\begin{lstlisting}
%  $Author: Vadim Kaushanskiy, Moscow State University by M.V. Lomonosov,
%  Faculty of Computational Mathematics and Cybernetics, System Analysis
%  Department, 24-December-2012, <vkaushanskiy@gmail.com>$
%

\end{lstlisting}
\fontfamily{\familydefault}
\selectfont
\item \hyperlink{/+elltool/+core/+test/+mlunit/EllipsoidSecTestCase}{EllipsoidSecTestCase}
\fontfamily{pcr}
\selectfont
\begin{lstlisting}
%  $Author: Igor Samokhin, Lomonosov Moscow State University,
%  Faculty of Computational Mathematics and Cybernetics, System Analysis
%  Department, 02-November-2012, <igorian.vmk@gmail.com>$
%  $Copyright: Moscow State University,
%             Faculty of Computational Mathematics and Computer Science,
%             System Analysis Department 2012 $
%

\end{lstlisting}
\fontfamily{\familydefault}
\selectfont
\item \hyperlink{/+elltool/+core/+test/+mlunit/EllipsoidTestCase}{EllipsoidTestCase}
\fontfamily{pcr}
\selectfont
\begin{lstlisting}
%

\end{lstlisting}
\fontfamily{\familydefault}
\selectfont
\item \hyperlink{/+elltool/+core/+test/+mlunit/GenEllipsoidPlotTestCase}{GenEllipsoidPlotTestCase}
\fontfamily{pcr}
\selectfont
\begin{lstlisting}
%

\end{lstlisting}
\fontfamily{\familydefault}
\selectfont
\item \hyperlink{/+elltool/+core/+test/+mlunit/GenEllipsoidTestCase}{GenEllipsoidTestCase}
\fontfamily{pcr}
\selectfont
\begin{lstlisting}
%

\end{lstlisting}
\fontfamily{\familydefault}
\selectfont
\item \hyperlink{/+elltool/+core/+test/+mlunit/HyperplaneTestCase}{HyperplaneTestCase}
\fontfamily{pcr}
\selectfont
\begin{lstlisting}
%  $Author: <Zakharov Eugene>  <justenterrr@gmail.com> $    $Date: <31 october> $
%  $Copyright: Moscow State University,
%             Faculty of Computational Mathematics and Computer Science,
%             System Analysis Department <2012> $
% 
%

\end{lstlisting}
\fontfamily{\familydefault}
\selectfont
\end{enumerate}
\subsection{/+elltool/+core/@GenEllipsoid}
\begin{enumerate}
\item \hyperlink{/+elltool/+core/@GenEllipsoid/GenEllipsoid}{GenEllipsoid}
\fontfamily{pcr}
\selectfont
\begin{lstlisting}
%  $Author: Vitaly Baranov  <vetbar42@gmail.com> $    $Date: Nov-2012$
%  $Copyright: Moscow State University,
%             Faculty of Computational Mathematics and Cybernetics,
%             System Analysis Department 2012 $
% 
%

\end{lstlisting}
\fontfamily{\familydefault}
\selectfont
\item \hyperlink{/+elltool/+core/@GenEllipsoid/checkBigger}{checkBigger}
\fontfamily{pcr}
\selectfont
\begin{lstlisting}
%

\end{lstlisting}
\fontfamily{\familydefault}
\selectfont
\item \hyperlink{/+elltool/+core/@GenEllipsoid/dimension}{dimension}
\fontfamily{pcr}
\selectfont
\begin{lstlisting}
%

\end{lstlisting}
\fontfamily{\familydefault}
\selectfont
\item \hyperlink{/+elltool/+core/@GenEllipsoid/eq}{eq}
\fontfamily{pcr}
\selectfont
\begin{lstlisting}
%  $Author: Peter Gagarinov  <pgagarinov@gmail.com> $    $Date: Dec-2012$
%  $Copyright: Moscow State University,
%             Faculty of Computational Mathematics and Cybernetics,
%             System Analysis Department 2012 $
% 
%

\end{lstlisting}
\fontfamily{\familydefault}
\selectfont
\item \hyperlink{/+elltool/+core/@GenEllipsoid/findAllInfDir}{findAllInfDir}
\fontfamily{pcr}
\selectfont
\begin{lstlisting}
%

\end{lstlisting}
\fontfamily{\familydefault}
\selectfont
\item \hyperlink{/+elltool/+core/@GenEllipsoid/findBasRank}{findBasRank}
\fontfamily{pcr}
\selectfont
\begin{lstlisting}
%

\end{lstlisting}
\fontfamily{\familydefault}
\selectfont
\item \hyperlink{/+elltool/+core/@GenEllipsoid/findConstruction}{findConstruction}
\fontfamily{pcr}
\selectfont
\begin{lstlisting}
%

\end{lstlisting}
\fontfamily{\familydefault}
\selectfont
\item \hyperlink{/+elltool/+core/@GenEllipsoid/findDiffEaND}{findDiffEaND}
\fontfamily{pcr}
\selectfont
\begin{lstlisting}
%

\end{lstlisting}
\fontfamily{\familydefault}
\selectfont
\item \hyperlink{/+elltool/+core/@GenEllipsoid/findDiffFC}{findDiffFC}
\fontfamily{pcr}
\selectfont
\begin{lstlisting}
%

\end{lstlisting}
\fontfamily{\familydefault}
\selectfont
\item \hyperlink{/+elltool/+core/@GenEllipsoid/findDiffINFC}{findDiffINFC}
\fontfamily{pcr}
\selectfont
\begin{lstlisting}
%

\end{lstlisting}
\fontfamily{\familydefault}
\selectfont
\item \hyperlink{/+elltool/+core/@GenEllipsoid/findDiffIaND}{findDiffIaND}
\fontfamily{pcr}
\selectfont
\begin{lstlisting}
%

\end{lstlisting}
\fontfamily{\familydefault}
\selectfont
\item \hyperlink{/+elltool/+core/@GenEllipsoid/findMatProj}{findMatProj}
\fontfamily{pcr}
\selectfont
\begin{lstlisting}
%

\end{lstlisting}
\fontfamily{\familydefault}
\selectfont
\item \hyperlink{/+elltool/+core/@GenEllipsoid/findSpaceBas}{findSpaceBas}
\fontfamily{pcr}
\selectfont
\begin{lstlisting}
%

\end{lstlisting}
\fontfamily{\familydefault}
\selectfont
\item \hyperlink{/+elltool/+core/@GenEllipsoid/findSqrtOfMatrix}{findSqrtOfMatrix}
\fontfamily{pcr}
\selectfont
\begin{lstlisting}
%

\end{lstlisting}
\fontfamily{\familydefault}
\selectfont
\item \hyperlink{/+elltool/+core/@GenEllipsoid/getColorTable}{getColorTable}
\fontfamily{pcr}
\selectfont
\begin{lstlisting}
%

\end{lstlisting}
\fontfamily{\familydefault}
\selectfont
\item \hyperlink{/+elltool/+core/@GenEllipsoid/getIsGoodDirForMat}{getIsGoodDirForMat}
\fontfamily{pcr}
\selectfont
\begin{lstlisting}
%

\end{lstlisting}
\fontfamily{\familydefault}
\selectfont
\item \hyperlink{/+elltool/+core/@GenEllipsoid/inv}{inv}
\fontfamily{pcr}
\selectfont
\begin{lstlisting}
%  $Author: Vitaly Baranov  <vetbar42@gmail.com> $    $Date: Nov-2012$
%  $Copyright: Moscow State University,
%             Faculty of Computational Mathematics and Cybernetics,
%             System Analysis Department 2012 $
% 
% 
%

\end{lstlisting}
\fontfamily{\familydefault}
\selectfont
\item \hyperlink{/+elltool/+core/@GenEllipsoid/minkDiffEa}{minkDiffEa}
\fontfamily{pcr}
\selectfont
\begin{lstlisting}
%  $Author: Vitaly Baranov  <vetbar42@gmail.com> $    $Date: 2012-11$
%  $Copyright: Moscow State University,
%             Faculty of Computational Mathematics and Cybernetics,
%             System Analysis Department 2012 $
% 
% 
%

\end{lstlisting}
\fontfamily{\familydefault}
\selectfont
\item \hyperlink{/+elltool/+core/@GenEllipsoid/minkDiffIa}{minkDiffIa}
\fontfamily{pcr}
\selectfont
\begin{lstlisting}
%  Bibliography:
%  V.V.Shiryaev, 'About internal ellipsoidal approximations of attainability
%  sets of linear systems under uncertanty'. Moscow University Vestnik,
%  Ser.15, Computational mathematics and cybernetics, 2012, N3, p. 20-27.
% 
%

\end{lstlisting}
\fontfamily{\familydefault}
\selectfont
\item \hyperlink{/+elltool/+core/@GenEllipsoid/minkSumEa}{minkSumEa}
\fontfamily{pcr}
\selectfont
\begin{lstlisting}
%  $Author: Vitaly Baranov  <vetbar42@gmail.com> $    $Date: 2012-11$
%  $Copyright: Moscow State University,
%             Faculty of Computational Mathematics and Cybernetics,
%             System Analysis Department 2012 $
% 
% 
%

\end{lstlisting}
\fontfamily{\familydefault}
\selectfont
\item \hyperlink{/+elltool/+core/@GenEllipsoid/minkSumIa}{minkSumIa}
\fontfamily{pcr}
\selectfont
\begin{lstlisting}
%  $Author: Vitaly Baranov  <vetbar42@gmail.com> $    $Date: 2012-11$
%  $Copyright: Moscow State University,
%             Faculty of Computational Mathematics and Cybernetics,
%             System Analysis Department 2012 $
% 
% 
%

\end{lstlisting}
\fontfamily{\familydefault}
\selectfont
\item \hyperlink{/+elltool/+core/@GenEllipsoid/plot}{plot}
\fontfamily{pcr}
\selectfont
\begin{lstlisting}
%  Examples:
%        plot([ell1, ell2, ell3], 'color', [1, 0, 1; 0, 0, 1; 1, 0, 0]);
%        plot([ell1, ell2, ell3], 'color', [1; 0; 1; 0; 0; 1; 1; 0; 0]);
%        plot([ell1, ell2, ell3; ell1, ell2, ell3], 'shade', [1, 1, 1; 1, 1,
%        1]);
%        plot([ell1, ell2, ell3; ell1, ell2, ell3], 'shade', [1; 1; 1; 1; 1;
%        1]);
%        plot([ell1, ell2, ell3], 'shade', 0.5);
%        plot([ell1, ell2, ell3], 'lineWidth', 1.5);
%        plot([ell1, ell2, ell3], 'lineWidth', [1.5, 0.5, 3]);
%

\end{lstlisting}
\fontfamily{\familydefault}
\selectfont
\item \hyperlink{/+elltool/+core/@GenEllipsoid/rho}{rho}
\fontfamily{pcr}
\selectfont
\begin{lstlisting}
%

\end{lstlisting}
\fontfamily{\familydefault}
\selectfont
\end{enumerate}
\subsection{/+elltool/+cvx}
\begin{enumerate}
\item \hyperlink{/+elltool/+cvx/CVXController}{CVXController}
\fontfamily{pcr}
\selectfont
\begin{lstlisting}
%

\end{lstlisting}
\fontfamily{\familydefault}
\selectfont
\end{enumerate}
\subsection{/+elltool/+demo/+test}
\begin{enumerate}
\item \hyperlink{/+elltool/+demo/+test/run\_tests}{run\_tests}
\fontfamily{pcr}
\selectfont
\begin{lstlisting}
%

\end{lstlisting}
\fontfamily{\familydefault}
\selectfont
\end{enumerate}
\subsection{/+elltool/+demo/+test/+mlunit}
\begin{enumerate}
\item \hyperlink{/+elltool/+demo/+test/+mlunit/BasicTestCase}{BasicTestCase}
\fontfamily{pcr}
\selectfont
\begin{lstlisting}
%

\end{lstlisting}
\fontfamily{\familydefault}
\selectfont
\item \hyperlink{/+elltool/+demo/+test/+mlunit/ETManualTC}{ETManualTC}
\fontfamily{pcr}
\selectfont
\begin{lstlisting}
%

\end{lstlisting}
\fontfamily{\familydefault}
\selectfont
\end{enumerate}
\subsection{/+elltool/+doc}
\begin{enumerate}
\item \hyperlink{/+elltool/+doc/collecthelp}{collecthelp}
\fontfamily{pcr}
\selectfont
\begin{lstlisting}
%  $Author: Peter Gagarinov  <pgagarinov@gmail.com> $	$Date: 2013-04-01 $
%  $Copyright: Moscow State University,
%             Faculty of Computational Mathematics and Computer Science,
%             System Analysis Department 2013 $
%

\end{lstlisting}
\fontfamily{\familydefault}
\selectfont
\item \hyperlink{/+elltool/+doc/run\_helpcollector}{run\_helpcollector}
\fontfamily{pcr}
\selectfont
\begin{lstlisting}
%

\end{lstlisting}
\fontfamily{\familydefault}
\selectfont
\end{enumerate}
\subsection{/+elltool/+linsys}
\begin{enumerate}
\item \hyperlink{/+elltool/+linsys/LinSys}{LinSys}
\fontfamily{pcr}
\selectfont
\begin{lstlisting}
%  $Authors: Alex Kurzhanskiy <akurzhan@eecs.berkeley.edu>
%            Ivan Menshikov  <ivan.v.menshikov@gmail.com> $    $Date: 2012 $
%            Kirill Mayantsev  <kirill.mayantsev@gmail.com> $  $Date: March-2012 $
%  $Copyright: Moscow State University,
%             Faculty of Computational Mathematics and Computer Science,
%             System Analysis Department 2012 $
% 
%

\end{lstlisting}
\fontfamily{\familydefault}
\selectfont
\end{enumerate}
\subsection{/+elltool/+linsys/+test}
\begin{enumerate}
\item \hyperlink{/+elltool/+linsys/+test/run\_tests}{run\_tests}
\fontfamily{pcr}
\selectfont
\begin{lstlisting}
%

\end{lstlisting}
\fontfamily{\familydefault}
\selectfont
\end{enumerate}
\subsection{/+elltool/+linsys/+test/+mlunit}
\begin{enumerate}
\item \hyperlink{/+elltool/+linsys/+test/+mlunit/LinSysTestCase}{LinSysTestCase}
\fontfamily{pcr}
\selectfont
\begin{lstlisting}
%

\end{lstlisting}
\fontfamily{\familydefault}
\selectfont
\end{enumerate}
\subsection{/+elltool/+logging}
\begin{enumerate}
\item \hyperlink{/+elltool/+logging/Log4jConfigurator}{Log4jConfigurator}
\fontfamily{pcr}
\selectfont
\begin{lstlisting}
%  $Author: Peter Gagarinov  <pgagarinov@gmail.com> $	$Date: 2011-05-18 $ 
%  $Copyright: Moscow State University,
%             Faculty of Computational Mathematics and Computer Science,
%             System Analysis Department 2011 $
%

\end{lstlisting}
\fontfamily{\familydefault}
\selectfont
\end{enumerate}
\subsection{/+elltool/+reach}
\begin{enumerate}
\item \hyperlink{/+elltool/+reach/AReach}{AReach}
\fontfamily{pcr}
\selectfont
\begin{lstlisting}
%  $Author: Kirill Mayantsev  <kirill.mayantsev@gmail.com> $  $Date: March-2012 $
%  $Copyright: Moscow State University,
%             Faculty of Computational Mathematics and Computer Science,
%             System Analysis Department 2012 $
% 
%

\end{lstlisting}
\fontfamily{\familydefault}
\selectfont
\item \hyperlink{/+elltool/+reach/IReach}{IReach}
\fontfamily{pcr}
\selectfont
\begin{lstlisting}
%  $Author: Kirill Mayantsev  <kirill.mayantsev@gmail.com> $  $Date: March-2012 $
%  $Copyright: Moscow State University,
%             Faculty of Computational Mathematics and Computer Science,
%             System Analysis Department 2012 $
% 
%

\end{lstlisting}
\fontfamily{\familydefault}
\selectfont
\item \hyperlink{/+elltool/+reach/ReachContinuous}{ReachContinuous}
\fontfamily{pcr}
\selectfont
\begin{lstlisting}
%  $Authors: Alex Kurzhanskiy <akurzhan@eecs.berkeley.edu>
%            Kirill Mayantsev  <kirill.mayantsev@gmail.com> $  $Date: March-2012 $
%  $Copyright: Moscow State University,
%             Faculty of Computational Mathematics and Computer Science,
%             System Analysis Department 2012 $
%

\end{lstlisting}
\fontfamily{\familydefault}
\selectfont
\item \hyperlink{/+elltool/+reach/ReachDiscrete}{ReachDiscrete}
\fontfamily{pcr}
\selectfont
\begin{lstlisting}
%  $Authors: Alex Kurzhanskiy <akurzhan@eecs.berkeley.edu>
%            Kirill Mayantsev  <kirill.mayantsev@gmail.com> $  $Date: March-2012 $
%  $Copyright: Moscow State University,
%             Faculty of Computational Mathematics and Computer Science,
%             System Analysis Department 2012 $
%

\end{lstlisting}
\fontfamily{\familydefault}
\selectfont
\end{enumerate}
\subsection{/+elltool/+reach/+test}
\begin{enumerate}
\item \hyperlink{/+elltool/+reach/+test/run\_continuous\_reach\_tests}{run\_continuous\_reach\_tests}
\fontfamily{pcr}
\selectfont
\begin{lstlisting}
%

\end{lstlisting}
\fontfamily{\familydefault}
\selectfont
\item \hyperlink{/+elltool/+reach/+test/run\_discrete\_reach\_tests}{run\_discrete\_reach\_tests}
\fontfamily{pcr}
\selectfont
\begin{lstlisting}
%

\end{lstlisting}
\fontfamily{\familydefault}
\selectfont
\end{enumerate}
\subsection{/+elltool/+reach/+test/+mlunit}
\begin{enumerate}
\item \hyperlink{/+elltool/+reach/+test/+mlunit/ContinuousReachFirstTestCase}{ContinuousReachFirstTestCase}
\fontfamily{pcr}
\selectfont
\begin{lstlisting}
%

\end{lstlisting}
\fontfamily{\familydefault}
\selectfont
\item \hyperlink{/+elltool/+reach/+test/+mlunit/ContinuousReachRegrTestCase}{ContinuousReachRegrTestCase}
\fontfamily{pcr}
\selectfont
\begin{lstlisting}
%

\end{lstlisting}
\fontfamily{\familydefault}
\selectfont
\item \hyperlink{/+elltool/+reach/+test/+mlunit/ContinuousReachTestCase}{ContinuousReachTestCase}
\fontfamily{pcr}
\selectfont
\begin{lstlisting}
%

\end{lstlisting}
\fontfamily{\familydefault}
\selectfont
\item \hyperlink{/+elltool/+reach/+test/+mlunit/DiscreteReachTestCase}{DiscreteReachTestCase}
\fontfamily{pcr}
\selectfont
\begin{lstlisting}
%

\end{lstlisting}
\fontfamily{\familydefault}
\selectfont
\item \hyperlink{/+elltool/+reach/+test/+mlunit/ReachFactory}{ReachFactory}
\fontfamily{pcr}
\selectfont
\begin{lstlisting}
%

\end{lstlisting}
\fontfamily{\familydefault}
\selectfont
\end{enumerate}
\subsection{/+elltool/+test}
\begin{enumerate}
\item \hyperlink{/+elltool/+test/TmpDataManager}{TmpDataManager}
\fontfamily{pcr}
\selectfont
\begin{lstlisting}
%  TMPDATAMANAGER provides a basic functionality for managing temporary
%  data folders, root folder name is determined automatically
% 
%

\end{lstlisting}
\fontfamily{\familydefault}
\selectfont
\item \hyperlink{/+elltool/+test/copyconf}{copyconf}
\fontfamily{pcr}
\selectfont
\begin{lstlisting}
%

\end{lstlisting}
\fontfamily{\familydefault}
\selectfont
\item \hyperlink{/+elltool/+test/editconf}{editconf}
\fontfamily{pcr}
\selectfont
\begin{lstlisting}
%

\end{lstlisting}
\fontfamily{\familydefault}
\selectfont
\item \hyperlink{/+elltool/+test/listconf}{listconf}
\fontfamily{pcr}
\selectfont
\begin{lstlisting}
%

\end{lstlisting}
\fontfamily{\familydefault}
\selectfont
\item \hyperlink{/+elltool/+test/run\_tests}{run\_tests}
\fontfamily{pcr}
\selectfont
\begin{lstlisting}
%

\end{lstlisting}
\fontfamily{\familydefault}
\selectfont
\item \hyperlink{/+elltool/+test/run\_tests\_remotely}{run\_tests\_remotely}
\fontfamily{pcr}
\selectfont
\begin{lstlisting}
%

\end{lstlisting}
\fontfamily{\familydefault}
\selectfont
\end{enumerate}
\subsection{/+elltool/+test/+configuration}
\begin{enumerate}
\item \hyperlink{/+elltool/+test/+configuration/AdaptiveConfRepoManager}{AdaptiveConfRepoManager}
\fontfamily{pcr}
\selectfont
\begin{lstlisting}
%  $Author: Peter Gagarinov <pgagarinov@gmail.com> $	$Date: 2011-05-18 $ 
%  $Copyright: Moscow State University,
%             Faculty of Computational Mathematics and Computer Science,
%             System Analysis Department 2011 $
% 
%

\end{lstlisting}
\fontfamily{\familydefault}
\selectfont
\end{enumerate}
\subsection{/+elltool/+test/+configuration/@ConfPatchRepo}
\begin{enumerate}
\item \hyperlink{/+elltool/+test/+configuration/@ConfPatchRepo/ConfPatchRepo}{ConfPatchRepo}
\fontfamily{pcr}
\selectfont
\begin{lstlisting}
%

\end{lstlisting}
\fontfamily{\familydefault}
\selectfont
\item \hyperlink{/+elltool/+test/+configuration/@ConfPatchRepo/patch\_001\_dummy\_patch}{patch\_001\_dummy\_patch}
\fontfamily{pcr}
\selectfont
\begin{lstlisting}
%

\end{lstlisting}
\fontfamily{\familydefault}
\selectfont
\end{enumerate}
\subsection{/+elltool/+test/+logging}
\begin{enumerate}
\item \hyperlink{/+elltool/+test/+logging/Log4jConfigurator}{Log4jConfigurator}
\fontfamily{pcr}
\selectfont
\begin{lstlisting}
%  $Author: Peter Gagarinov  <pgagarinov@gmail.com> $	$Date: 2011-05-18 $ 
%  $Copyright: Moscow State University,
%             Faculty of Computational Mathematics and Computer Science,
%             System Analysis Department 2011 $
%

\end{lstlisting}
\fontfamily{\familydefault}
\selectfont
\end{enumerate}
\subsection{/+gras/+ellapx/+enums}
\begin{enumerate}
\item \hyperlink{/+gras/+ellapx/+enums/EApproxType}{EApproxType}
\fontfamily{pcr}
\selectfont
\begin{lstlisting}
% APXTYPE Summary of this class goes here
%

\end{lstlisting}
\fontfamily{\familydefault}
\selectfont
\item \hyperlink{/+gras/+ellapx/+enums/EEllUnionTimeDirection}{EEllUnionTimeDirection}
\fontfamily{pcr}
\selectfont
\begin{lstlisting}
% APXTYPE Summary of this class goes here
%

\end{lstlisting}
\fontfamily{\familydefault}
\selectfont
\item \hyperlink{/+gras/+ellapx/+enums/EProjType}{EProjType}
\fontfamily{pcr}
\selectfont
\begin{lstlisting}
%

\end{lstlisting}
\fontfamily{\familydefault}
\selectfont
\end{enumerate}
\subsection{/+gras/+ellapx/+gen}
\begin{enumerate}
\item \hyperlink{/+gras/+ellapx/+gen/ATightEllApxBuilder}{ATightEllApxBuilder}
\fontfamily{pcr}
\selectfont
\begin{lstlisting}
%

\end{lstlisting}
\fontfamily{\familydefault}
\selectfont
\item \hyperlink{/+gras/+ellapx/+gen/EllApxCollectionBuilder}{EllApxCollectionBuilder}
\fontfamily{pcr}
\selectfont
\begin{lstlisting}
%

\end{lstlisting}
\fontfamily{\familydefault}
\selectfont
\item \hyperlink{/+gras/+ellapx/+gen/IEllApxBuilder}{IEllApxBuilder}
\fontfamily{pcr}
\selectfont
\begin{lstlisting}
%

\end{lstlisting}
\fontfamily{\familydefault}
\selectfont
\end{enumerate}
\subsection{/+gras/+ellapx/+lreachplain}
\begin{enumerate}
\item \hyperlink{/+gras/+ellapx/+lreachplain/ATightEllApxBuilder}{ATightEllApxBuilder}
\fontfamily{pcr}
\selectfont
\begin{lstlisting}
%

\end{lstlisting}
\fontfamily{\familydefault}
\selectfont
\item \hyperlink{/+gras/+ellapx/+lreachplain/ATightIntEllApxBuilder}{ATightIntEllApxBuilder}
\fontfamily{pcr}
\selectfont
\begin{lstlisting}
%

\end{lstlisting}
\fontfamily{\familydefault}
\selectfont
\item \hyperlink{/+gras/+ellapx/+lreachplain/EllTubeDynamicSpaceProjector}{EllTubeDynamicSpaceProjector}
\fontfamily{pcr}
\selectfont
\begin{lstlisting}
% IELLTUBEPROJECTOR Summary of this class goes here
%    Detailed explanation goes here
%

\end{lstlisting}
\fontfamily{\familydefault}
\selectfont
\item \hyperlink{/+gras/+ellapx/+lreachplain/ExtEllApxBuilder}{ExtEllApxBuilder}
\fontfamily{pcr}
\selectfont
\begin{lstlisting}
%

\end{lstlisting}
\fontfamily{\familydefault}
\selectfont
\item \hyperlink{/+gras/+ellapx/+lreachplain/GoodDirectionSet}{GoodDirectionSet}
\fontfamily{pcr}
\selectfont
\begin{lstlisting}
%

\end{lstlisting}
\fontfamily{\familydefault}
\selectfont
\item \hyperlink{/+gras/+ellapx/+lreachplain/IntEllApxBuilder}{IntEllApxBuilder}
\fontfamily{pcr}
\selectfont
\begin{lstlisting}
%

\end{lstlisting}
\fontfamily{\familydefault}
\selectfont
\item \hyperlink{/+gras/+ellapx/+lreachplain/IntProperEllApxBuilder}{IntProperEllApxBuilder}
\fontfamily{pcr}
\selectfont
\begin{lstlisting}
%

\end{lstlisting}
\fontfamily{\familydefault}
\selectfont
\end{enumerate}
\subsection{/+gras/+ellapx/+lreachplain/+probdef}
\begin{enumerate}
\item \hyperlink{/+gras/+ellapx/+lreachplain/+probdef/AReachContProblemDef}{AReachContProblemDef}
\fontfamily{pcr}
\selectfont
\begin{lstlisting}
%

\end{lstlisting}
\fontfamily{\familydefault}
\selectfont
\item \hyperlink{/+gras/+ellapx/+lreachplain/+probdef/IReachContProblemDef}{IReachContProblemDef}
\fontfamily{pcr}
\selectfont
\begin{lstlisting}
%

\end{lstlisting}
\fontfamily{\familydefault}
\selectfont
\item \hyperlink{/+gras/+ellapx/+lreachplain/+probdef/LReachContProblemDef}{LReachContProblemDef}
\fontfamily{pcr}
\selectfont
\begin{lstlisting}
%

\end{lstlisting}
\fontfamily{\familydefault}
\selectfont
\item \hyperlink{/+gras/+ellapx/+lreachplain/+probdef/ReachContLTIProblemDef}{ReachContLTIProblemDef}
\fontfamily{pcr}
\selectfont
\begin{lstlisting}
%

\end{lstlisting}
\fontfamily{\familydefault}
\selectfont
\end{enumerate}
\subsection{/+gras/+ellapx/+lreachplain/+probdyn}
\begin{enumerate}
\item \hyperlink{/+gras/+ellapx/+lreachplain/+probdyn/AReachProblemDynamics}{AReachProblemDynamics}
\fontfamily{pcr}
\selectfont
\begin{lstlisting}
%

\end{lstlisting}
\fontfamily{\familydefault}
\selectfont
\item \hyperlink{/+gras/+ellapx/+lreachplain/+probdyn/AReachProblemDynamicsInterp}{AReachProblemDynamicsInterp}
\fontfamily{pcr}
\selectfont
\begin{lstlisting}
%

\end{lstlisting}
\fontfamily{\familydefault}
\selectfont
\item \hyperlink{/+gras/+ellapx/+lreachplain/+probdyn/AReachProblemLTIDynamics}{AReachProblemLTIDynamics}
\fontfamily{pcr}
\selectfont
\begin{lstlisting}
%

\end{lstlisting}
\fontfamily{\familydefault}
\selectfont
\item \hyperlink{/+gras/+ellapx/+lreachplain/+probdyn/IReachProblemDynamics}{IReachProblemDynamics}
\fontfamily{pcr}
\selectfont
\begin{lstlisting}
%

\end{lstlisting}
\fontfamily{\familydefault}
\selectfont
\item \hyperlink{/+gras/+ellapx/+lreachplain/+probdyn/LReachProblemDynamicsFactory}{LReachProblemDynamicsFactory}
\fontfamily{pcr}
\selectfont
\begin{lstlisting}
%

\end{lstlisting}
\fontfamily{\familydefault}
\selectfont
\item \hyperlink{/+gras/+ellapx/+lreachplain/+probdyn/LReachProblemDynamicsInterp}{LReachProblemDynamicsInterp}
\fontfamily{pcr}
\selectfont
\begin{lstlisting}
%

\end{lstlisting}
\fontfamily{\familydefault}
\selectfont
\item \hyperlink{/+gras/+ellapx/+lreachplain/+probdyn/LReachProblemLTIDynamics}{LReachProblemLTIDynamics}
\fontfamily{pcr}
\selectfont
\begin{lstlisting}
%

\end{lstlisting}
\fontfamily{\familydefault}
\selectfont
\end{enumerate}
\subsection{/+gras/+ellapx/+lreachuncert}
\begin{enumerate}
\item \hyperlink{/+gras/+ellapx/+lreachuncert/ExtIntEllApxBuilder}{ExtIntEllApxBuilder}
\fontfamily{pcr}
\selectfont
\begin{lstlisting}
%

\end{lstlisting}
\fontfamily{\familydefault}
\selectfont
\end{enumerate}
\subsection{/+gras/+ellapx/+lreachuncert/+probdef}
\begin{enumerate}
\item \hyperlink{/+gras/+ellapx/+lreachuncert/+probdef/AReachContProblemDef}{AReachContProblemDef}
\fontfamily{pcr}
\selectfont
\begin{lstlisting}
%

\end{lstlisting}
\fontfamily{\familydefault}
\selectfont
\item \hyperlink{/+gras/+ellapx/+lreachuncert/+probdef/IReachContProblemDef}{IReachContProblemDef}
\fontfamily{pcr}
\selectfont
\begin{lstlisting}
%

\end{lstlisting}
\fontfamily{\familydefault}
\selectfont
\item \hyperlink{/+gras/+ellapx/+lreachuncert/+probdef/LReachContProblemDef}{LReachContProblemDef}
\fontfamily{pcr}
\selectfont
\begin{lstlisting}
%

\end{lstlisting}
\fontfamily{\familydefault}
\selectfont
\item \hyperlink{/+gras/+ellapx/+lreachuncert/+probdef/ReachContLTIProblemDef}{ReachContLTIProblemDef}
\fontfamily{pcr}
\selectfont
\begin{lstlisting}
%

\end{lstlisting}
\fontfamily{\familydefault}
\selectfont
\end{enumerate}
\subsection{/+gras/+ellapx/+lreachuncert/+probdyn}
\begin{enumerate}
\item \hyperlink{/+gras/+ellapx/+lreachuncert/+probdyn/AReachProblemDynamics}{AReachProblemDynamics}
\fontfamily{pcr}
\selectfont
\begin{lstlisting}
%

\end{lstlisting}
\fontfamily{\familydefault}
\selectfont
\item \hyperlink{/+gras/+ellapx/+lreachuncert/+probdyn/IReachProblemDynamics}{IReachProblemDynamics}
\fontfamily{pcr}
\selectfont
\begin{lstlisting}
%

\end{lstlisting}
\fontfamily{\familydefault}
\selectfont
\item \hyperlink{/+gras/+ellapx/+lreachuncert/+probdyn/LReachProblemDynamicsFactory}{LReachProblemDynamicsFactory}
\fontfamily{pcr}
\selectfont
\begin{lstlisting}
%

\end{lstlisting}
\fontfamily{\familydefault}
\selectfont
\item \hyperlink{/+gras/+ellapx/+lreachuncert/+probdyn/LReachProblemDynamicsInterp}{LReachProblemDynamicsInterp}
\fontfamily{pcr}
\selectfont
\begin{lstlisting}
%

\end{lstlisting}
\fontfamily{\familydefault}
\selectfont
\item \hyperlink{/+gras/+ellapx/+lreachuncert/+probdyn/LReachProblemLTIDynamics}{LReachProblemLTIDynamics}
\fontfamily{pcr}
\selectfont
\begin{lstlisting}
%

\end{lstlisting}
\fontfamily{\familydefault}
\selectfont
\end{enumerate}
\subsection{/+gras/+ellapx/+proj}
\begin{enumerate}
\item \hyperlink{/+gras/+ellapx/+proj/AEllTubePlainProjector}{AEllTubePlainProjector}
\fontfamily{pcr}
\selectfont
\begin{lstlisting}
% IELLTUBEPROJECTOR Summary of this class goes here
%    Detailed explanation goes here
%

\end{lstlisting}
\fontfamily{\familydefault}
\selectfont
\item \hyperlink{/+gras/+ellapx/+proj/EllTubeCollectionProjector}{EllTubeCollectionProjector}
\fontfamily{pcr}
\selectfont
\begin{lstlisting}
% IELLTUBEPROJECTOR Summary of this class goes here
%    Detailed explanation goes here
%

\end{lstlisting}
\fontfamily{\familydefault}
\selectfont
\item \hyperlink{/+gras/+ellapx/+proj/EllTubeStaticSpaceProjector}{EllTubeStaticSpaceProjector}
\fontfamily{pcr}
\selectfont
\begin{lstlisting}
% IELLTUBEPROJECTOR Summary of this class goes here
%    Detailed explanation goes here
%

\end{lstlisting}
\fontfamily{\familydefault}
\selectfont
\item \hyperlink{/+gras/+ellapx/+proj/IEllTubeProjector}{IEllTubeProjector}
\fontfamily{pcr}
\selectfont
\begin{lstlisting}
% IELLTUBEPROJECTOR Summary of this class goes here
%    Detailed explanation goes here
%

\end{lstlisting}
\fontfamily{\familydefault}
\selectfont
\end{enumerate}
\subsection{/+gras/+ellapx/+smartdb}
\begin{enumerate}
\item \hyperlink{/+gras/+ellapx/+smartdb/F}{F}
\fontfamily{pcr}
\selectfont
\begin{lstlisting}
% Standard fields
%

\end{lstlisting}
\fontfamily{\familydefault}
\selectfont
\item \hyperlink{/+gras/+ellapx/+smartdb/RelDispConfigurator}{RelDispConfigurator}
\fontfamily{pcr}
\selectfont
\begin{lstlisting}
%

\end{lstlisting}
\fontfamily{\familydefault}
\selectfont
\end{enumerate}
\subsection{/+gras/+ellapx/+smartdb/+rels}
\begin{enumerate}
\item \hyperlink{/+gras/+ellapx/+smartdb/+rels/EllTube}{EllTube}
\fontfamily{pcr}
\selectfont
\begin{lstlisting}
% TestRelation Summary of this class goes here
%    Detailed explanation goes here
%

\end{lstlisting}
\fontfamily{\familydefault}
\selectfont
\item \hyperlink{/+gras/+ellapx/+smartdb/+rels/EllTubeBasic}{EllTubeBasic}
\fontfamily{pcr}
\selectfont
\begin{lstlisting}
% TestRelation Summary of this class goes here
%    Detailed explanation goes here
%

\end{lstlisting}
\fontfamily{\familydefault}
\selectfont
\item \hyperlink{/+gras/+ellapx/+smartdb/+rels/EllTubeProj}{EllTubeProj}
\fontfamily{pcr}
\selectfont
\begin{lstlisting}
% TestRelation Summary of this class goes here
%    Detailed explanation goes here
%

\end{lstlisting}
\fontfamily{\familydefault}
\selectfont
\item \hyperlink{/+gras/+ellapx/+smartdb/+rels/EllTubeProjBasic}{EllTubeProjBasic}
\fontfamily{pcr}
\selectfont
\begin{lstlisting}
%

\end{lstlisting}
\fontfamily{\familydefault}
\selectfont
\item \hyperlink{/+gras/+ellapx/+smartdb/+rels/EllTubeTouchCurveBasic}{EllTubeTouchCurveBasic}
\fontfamily{pcr}
\selectfont
\begin{lstlisting}
% TestRelation Summary of this class goes here
%    Detailed explanation goes here
%

\end{lstlisting}
\fontfamily{\familydefault}
\selectfont
\item \hyperlink{/+gras/+ellapx/+smartdb/+rels/EllTubeTouchCurveProjBasic}{EllTubeTouchCurveProjBasic}
\fontfamily{pcr}
\selectfont
\begin{lstlisting}
%

\end{lstlisting}
\fontfamily{\familydefault}
\selectfont
\item \hyperlink{/+gras/+ellapx/+smartdb/+rels/EllUnionTube}{EllUnionTube}
\fontfamily{pcr}
\selectfont
\begin{lstlisting}
% TestRelation Summary of this class goes here
%    Detailed explanation goes here
%

\end{lstlisting}
\fontfamily{\familydefault}
\selectfont
\item \hyperlink{/+gras/+ellapx/+smartdb/+rels/EllUnionTubeBasic}{EllUnionTubeBasic}
\fontfamily{pcr}
\selectfont
\begin{lstlisting}
% TestRelation Summary of this class goes here
%    Detailed explanation goes here
%

\end{lstlisting}
\fontfamily{\familydefault}
\selectfont
\item \hyperlink{/+gras/+ellapx/+smartdb/+rels/EllUnionTubeStaticProj}{EllUnionTubeStaticProj}
\fontfamily{pcr}
\selectfont
\begin{lstlisting}
% TestRelation Summary of this class goes here
%    Detailed explanation goes here
% 
%

\end{lstlisting}
\fontfamily{\familydefault}
\selectfont
\item \hyperlink{/+gras/+ellapx/+smartdb/+rels/TypifiedByFieldCodeRel}{TypifiedByFieldCodeRel}
\fontfamily{pcr}
\selectfont
\begin{lstlisting}
% TestRelation Summary of this class goes here
%    Detailed explanation goes here
%

\end{lstlisting}
\fontfamily{\familydefault}
\selectfont
\end{enumerate}
\subsection{/+gras/+ellapx/+smartdb/+test}
\begin{enumerate}
\item \hyperlink{/+gras/+ellapx/+smartdb/+test/run\_tests}{run\_tests}
\fontfamily{pcr}
\selectfont
\begin{lstlisting}
%

\end{lstlisting}
\fontfamily{\familydefault}
\selectfont
\end{enumerate}
\subsection{/+gras/+ellapx/+smartdb/+test/+mlunit}
\begin{enumerate}
\item \hyperlink{/+gras/+ellapx/+smartdb/+test/+mlunit/SuiteEllTube}{SuiteEllTube}
\fontfamily{pcr}
\selectfont
\begin{lstlisting}
%

\end{lstlisting}
\fontfamily{\familydefault}
\selectfont
\end{enumerate}
\subsection{/+gras/+ellapx/+test}
\begin{enumerate}
\item \hyperlink{/+gras/+ellapx/+test/run\_tests}{run\_tests}
\fontfamily{pcr}
\selectfont
\begin{lstlisting}
%

\end{lstlisting}
\fontfamily{\familydefault}
\selectfont
\end{enumerate}
\subsection{/+gras/+ellapx/+uncertcalc}
\begin{enumerate}
\item \hyperlink{/+gras/+ellapx/+uncertcalc/ApproxProblemPropertyBuilder}{ApproxProblemPropertyBuilder}
\fontfamily{pcr}
\selectfont
\begin{lstlisting}
%

\end{lstlisting}
\fontfamily{\familydefault}
\selectfont
\item \hyperlink{/+gras/+ellapx/+uncertcalc/EllApxBuilder}{EllApxBuilder}
\fontfamily{pcr}
\selectfont
\begin{lstlisting}
% IELLTUBEPROJECTOR Summary of this class goes here
%    Detailed explanation goes here
%

\end{lstlisting}
\fontfamily{\familydefault}
\selectfont
\item \hyperlink{/+gras/+ellapx/+uncertcalc/EllTubeProjectorBuilder}{EllTubeProjectorBuilder}
\fontfamily{pcr}
\selectfont
\begin{lstlisting}
% IELLTUBEPROJECTOR Summary of this class goes here
%    Detailed explanation goes here
%

\end{lstlisting}
\fontfamily{\familydefault}
\selectfont
\item \hyperlink{/+gras/+ellapx/+uncertcalc/copyconf}{copyconf}
\fontfamily{pcr}
\selectfont
\begin{lstlisting}
%

\end{lstlisting}
\fontfamily{\familydefault}
\selectfont
\item \hyperlink{/+gras/+ellapx/+uncertcalc/copysysconf}{copysysconf}
\fontfamily{pcr}
\selectfont
\begin{lstlisting}
%

\end{lstlisting}
\fontfamily{\familydefault}
\selectfont
\item \hyperlink{/+gras/+ellapx/+uncertcalc/editconf}{editconf}
\fontfamily{pcr}
\selectfont
\begin{lstlisting}
%

\end{lstlisting}
\fontfamily{\familydefault}
\selectfont
\item \hyperlink{/+gras/+ellapx/+uncertcalc/editsysconf}{editsysconf}
\fontfamily{pcr}
\selectfont
\begin{lstlisting}
%

\end{lstlisting}
\fontfamily{\familydefault}
\selectfont
\item \hyperlink{/+gras/+ellapx/+uncertcalc/listconf}{listconf}
\fontfamily{pcr}
\selectfont
\begin{lstlisting}
%

\end{lstlisting}
\fontfamily{\familydefault}
\selectfont
\item \hyperlink{/+gras/+ellapx/+uncertcalc/listsysconf}{listsysconf}
\fontfamily{pcr}
\selectfont
\begin{lstlisting}
%

\end{lstlisting}
\fontfamily{\familydefault}
\selectfont
\item \hyperlink{/+gras/+ellapx/+uncertcalc/run}{run}
\fontfamily{pcr}
\selectfont
\begin{lstlisting}
%

\end{lstlisting}
\fontfamily{\familydefault}
\selectfont
\item \hyperlink{/+gras/+ellapx/+uncertcalc/updateallconf}{updateallconf}
\fontfamily{pcr}
\selectfont
\begin{lstlisting}
%

\end{lstlisting}
\fontfamily{\familydefault}
\selectfont
\end{enumerate}
\subsection{/+gras/+ellapx/+uncertcalc/+conf}
\begin{enumerate}
\item \hyperlink{/+gras/+ellapx/+uncertcalc/+conf/ConfRepoMgr}{ConfRepoMgr}
\fontfamily{pcr}
\selectfont
\begin{lstlisting}
%

\end{lstlisting}
\fontfamily{\familydefault}
\selectfont
\item \hyperlink{/+gras/+ellapx/+uncertcalc/+conf/IConfRepoMgr}{IConfRepoMgr}
\fontfamily{pcr}
\selectfont
\begin{lstlisting}
%

\end{lstlisting}
\fontfamily{\familydefault}
\selectfont
\end{enumerate}
\subsection{/+gras/+ellapx/+uncertcalc/+conf/+sysdef}
\begin{enumerate}
\item \hyperlink{/+gras/+ellapx/+uncertcalc/+conf/+sysdef/AConfRepoMgr}{AConfRepoMgr}
\fontfamily{pcr}
\selectfont
\begin{lstlisting}
%

\end{lstlisting}
\fontfamily{\familydefault}
\selectfont
\item \hyperlink{/+gras/+ellapx/+uncertcalc/+conf/+sysdef/ConfRepoMgr}{ConfRepoMgr}
\fontfamily{pcr}
\selectfont
\begin{lstlisting}
%

\end{lstlisting}
\fontfamily{\familydefault}
\selectfont
\end{enumerate}
\subsection{/+gras/+ellapx/+uncertcalc/+conf/+sysdef/+test}
\begin{enumerate}
\item \hyperlink{/+gras/+ellapx/+uncertcalc/+conf/+sysdef/+test/ConfRepoMgr}{ConfRepoMgr}
\fontfamily{pcr}
\selectfont
\begin{lstlisting}
%

\end{lstlisting}
\fontfamily{\familydefault}
\selectfont
\item \hyperlink{/+gras/+ellapx/+uncertcalc/+conf/+sysdef/+test/run\_tests}{run\_tests}
\fontfamily{pcr}
\selectfont
\begin{lstlisting}
%

\end{lstlisting}
\fontfamily{\familydefault}
\selectfont
\end{enumerate}
\subsection{/+gras/+ellapx/+uncertcalc/+conf/+sysdef/+test/+mlunit}
\begin{enumerate}
\item \hyperlink{/+gras/+ellapx/+uncertcalc/+conf/+sysdef/+test/+mlunit/SuiteBasic}{SuiteBasic}
\fontfamily{pcr}
\selectfont
\begin{lstlisting}
%

\end{lstlisting}
\fontfamily{\familydefault}
\selectfont
\end{enumerate}
\subsection{/+gras/+ellapx/+uncertcalc/+conf/+sysdef/@ConfPatchRepo}
\begin{enumerate}
\item \hyperlink{/+gras/+ellapx/+uncertcalc/+conf/+sysdef/@ConfPatchRepo/ConfPatchRepo}{ConfPatchRepo}
\fontfamily{pcr}
\selectfont
\begin{lstlisting}
%

\end{lstlisting}
\fontfamily{\familydefault}
\selectfont
\item \hyperlink{/+gras/+ellapx/+uncertcalc/+conf/+sysdef/@ConfPatchRepo/patch\_001\_remove\_garbage}{patch\_001\_remove\_garbage}
\fontfamily{pcr}
\selectfont
\begin{lstlisting}
%

\end{lstlisting}
\fontfamily{\familydefault}
\selectfont
\item \hyperlink{/+gras/+ellapx/+uncertcalc/+conf/+sysdef/@ConfPatchRepo/patch\_002\_add\_description}{patch\_002\_add\_description}
\fontfamily{pcr}
\selectfont
\begin{lstlisting}
%

\end{lstlisting}
\fontfamily{\familydefault}
\selectfont
\end{enumerate}
\subsection{/+gras/+ellapx/+uncertcalc/+conf/@ConfPatchRepo}
\begin{enumerate}
\item \hyperlink{/+gras/+ellapx/+uncertcalc/+conf/@ConfPatchRepo/ConfPatchRepo}{ConfPatchRepo}
\fontfamily{pcr}
\selectfont
\begin{lstlisting}
%

\end{lstlisting}
\fontfamily{\familydefault}
\selectfont
\item \hyperlink{/+gras/+ellapx/+uncertcalc/+conf/@ConfPatchRepo/patch\_001\_make\_proj\_spec\_logical}{patch\_001\_make\_proj\_spec\_logical}
\fontfamily{pcr}
\selectfont
\begin{lstlisting}
%

\end{lstlisting}
\fontfamily{\familydefault}
\selectfont
\item \hyperlink{/+gras/+ellapx/+uncertcalc/+conf/@ConfPatchRepo/patch\_002\_remove\_redundant\_stuff}{patch\_002\_remove\_redundant\_stuff}
\fontfamily{pcr}
\selectfont
\begin{lstlisting}
%

\end{lstlisting}
\fontfamily{\familydefault}
\selectfont
\item \hyperlink{/+gras/+ellapx/+uncertcalc/+conf/@ConfPatchRepo/patch\_003\_is\_plotting\_enabled}{patch\_003\_is\_plotting\_enabled}
\fontfamily{pcr}
\selectfont
\begin{lstlisting}
%

\end{lstlisting}
\fontfamily{\familydefault}
\selectfont
\item \hyperlink{/+gras/+ellapx/+uncertcalc/+conf/@ConfPatchRepo/patch\_004\_multiple\_int\_ell\_apx\_schemas}{patch\_004\_multiple\_int\_ell\_apx\_schemas}
\fontfamily{pcr}
\selectfont
\begin{lstlisting}
%

\end{lstlisting}
\fontfamily{\familydefault}
\selectfont
\item \hyperlink{/+gras/+ellapx/+uncertcalc/+conf/@ConfPatchRepo/patch\_005\_ext\_ell\_apx\_schema}{patch\_005\_ext\_ell\_apx\_schema}
\fontfamily{pcr}
\selectfont
\begin{lstlisting}
%

\end{lstlisting}
\fontfamily{\familydefault}
\selectfont
\item \hyperlink{/+gras/+ellapx/+uncertcalc/+conf/@ConfPatchRepo/patch\_006\_calc\_precision}{patch\_006\_calc\_precision}
\fontfamily{pcr}
\selectfont
\begin{lstlisting}
%

\end{lstlisting}
\fontfamily{\familydefault}
\selectfont
\item \hyperlink{/+gras/+ellapx/+uncertcalc/+conf/@ConfPatchRepo/patch\_007\_make\_space\_list\_a\_vector}{patch\_007\_make\_space\_list\_a\_vector}
\fontfamily{pcr}
\selectfont
\begin{lstlisting}
%

\end{lstlisting}
\fontfamily{\familydefault}
\selectfont
\item \hyperlink{/+gras/+ellapx/+uncertcalc/+conf/@ConfPatchRepo/patch\_008\_add\_reference\_to\_sysdef}{patch\_008\_add\_reference\_to\_sysdef}
\fontfamily{pcr}
\selectfont
\begin{lstlisting}
%

\end{lstlisting}
\fontfamily{\familydefault}
\selectfont
\item \hyperlink{/+gras/+ellapx/+uncertcalc/+conf/@ConfPatchRepo/patch\_009\_add\_scale\_factors}{patch\_009\_add\_scale\_factors}
\fontfamily{pcr}
\selectfont
\begin{lstlisting}
%

\end{lstlisting}
\fontfamily{\familydefault}
\selectfont
\item \hyperlink{/+gras/+ellapx/+uncertcalc/+conf/@ConfPatchRepo/patch\_010\_rename\_scale\_factors}{patch\_010\_rename\_scale\_factors}
\fontfamily{pcr}
\selectfont
\begin{lstlisting}
%

\end{lstlisting}
\fontfamily{\familydefault}
\selectfont
\item \hyperlink{/+gras/+ellapx/+uncertcalc/+conf/@ConfPatchRepo/patch\_011\_add\_view\_angle\_prop}{patch\_011\_add\_view\_angle\_prop}
\fontfamily{pcr}
\selectfont
\begin{lstlisting}
%

\end{lstlisting}
\fontfamily{\familydefault}
\selectfont
\item \hyperlink{/+gras/+ellapx/+uncertcalc/+conf/@ConfPatchRepo/patch\_012\_rename\_ell\_apx\_schemas}{patch\_012\_rename\_ell\_apx\_schemas}
\fontfamily{pcr}
\selectfont
\begin{lstlisting}
%

\end{lstlisting}
\fontfamily{\familydefault}
\selectfont
\item \hyperlink{/+gras/+ellapx/+uncertcalc/+conf/@ConfPatchRepo/patch\_013\_disable\_uncertainty\_regime\_by\_default}{patch\_013\_disable\_uncertainty\_regime\_by\_default}
\fontfamily{pcr}
\selectfont
\begin{lstlisting}
%

\end{lstlisting}
\fontfamily{\familydefault}
\selectfont
\item \hyperlink{/+gras/+ellapx/+uncertcalc/+conf/@ConfPatchRepo/patch\_014\_internal\_external\_apx\_params}{patch\_014\_internal\_external\_apx\_params}
\fontfamily{pcr}
\selectfont
\begin{lstlisting}
%

\end{lstlisting}
\fontfamily{\familydefault}
\selectfont
\item \hyperlink{/+gras/+ellapx/+uncertcalc/+conf/@ConfPatchRepo/patch\_015\_internal\_external\_apx\_addparams}{patch\_015\_internal\_external\_apx\_addparams}
\fontfamily{pcr}
\selectfont
\begin{lstlisting}
%

\end{lstlisting}
\fontfamily{\familydefault}
\selectfont
\item \hyperlink{/+gras/+ellapx/+uncertcalc/+conf/@ConfPatchRepo/patch\_016\_is\_good\_curves\_separately}{patch\_016\_is\_good\_curves\_separately}
\fontfamily{pcr}
\selectfont
\begin{lstlisting}
%

\end{lstlisting}
\fontfamily{\familydefault}
\selectfont
\item \hyperlink{/+gras/+ellapx/+uncertcalc/+conf/@ConfPatchRepo/patch\_017\_gen\_props\_mat\_calc\_mode}{patch\_017\_gen\_props\_mat\_calc\_mode}
\fontfamily{pcr}
\selectfont
\begin{lstlisting}
%

\end{lstlisting}
\fontfamily{\familydefault}
\selectfont
\item \hyperlink{/+gras/+ellapx/+uncertcalc/+conf/@ConfPatchRepo/patch\_018\_remove\_uncert\_int\_apx\_schema}{patch\_018\_remove\_uncert\_int\_apx\_schema}
\fontfamily{pcr}
\selectfont
\begin{lstlisting}
%

\end{lstlisting}
\fontfamily{\familydefault}
\selectfont
\end{enumerate}
\subsection{/+gras/+ellapx/+uncertcalc/+log}
\begin{enumerate}
\item \hyperlink{/+gras/+ellapx/+uncertcalc/+log/Log4jConfigurator}{Log4jConfigurator}
\fontfamily{pcr}
\selectfont
\begin{lstlisting}
%  $Author: Peter Gagarinov  <pgagarinov@gmail.com> $	$Date: 2011-05-18 $ 
%  $Copyright: Moscow State University,
%             Faculty of Computational Mathematics and Computer Science,
%             System Analysis Department 2011 $
% 
%

\end{lstlisting}
\fontfamily{\familydefault}
\selectfont
\end{enumerate}
\subsection{/+gras/+ellapx/+uncertcalc/+test}
\begin{enumerate}
\item \hyperlink{/+gras/+ellapx/+uncertcalc/+test/run\_tests}{run\_tests}
\fontfamily{pcr}
\selectfont
\begin{lstlisting}
%

\end{lstlisting}
\fontfamily{\familydefault}
\selectfont
\item \hyperlink{/+gras/+ellapx/+uncertcalc/+test/updateallconf}{updateallconf}
\fontfamily{pcr}
\selectfont
\begin{lstlisting}
%  $Author: Peter Gagarinov <pgagarinov@gmail.com> $	$Date: 2012-11-24 $ 
%  $Copyright: Moscow State University,
%             Faculty of Computational Mathematics and Computer Science,
%             System Analysis Department 2012 $
% 
%

\end{lstlisting}
\fontfamily{\familydefault}
\selectfont
\end{enumerate}
\subsection{/+gras/+ellapx/+uncertcalc/+test/+comp}
\begin{enumerate}
\item \hyperlink{/+gras/+ellapx/+uncertcalc/+test/+comp/copyconf}{copyconf}
\fontfamily{pcr}
\selectfont
\begin{lstlisting}
%

\end{lstlisting}
\fontfamily{\familydefault}
\selectfont
\item \hyperlink{/+gras/+ellapx/+uncertcalc/+test/+comp/editconf}{editconf}
\fontfamily{pcr}
\selectfont
\begin{lstlisting}
%

\end{lstlisting}
\fontfamily{\familydefault}
\selectfont
\item \hyperlink{/+gras/+ellapx/+uncertcalc/+test/+comp/editconftemplate}{editconftemplate}
\fontfamily{pcr}
\selectfont
\begin{lstlisting}
%

\end{lstlisting}
\fontfamily{\familydefault}
\selectfont
\item \hyperlink{/+gras/+ellapx/+uncertcalc/+test/+comp/listconfs}{listconfs}
\fontfamily{pcr}
\selectfont
\begin{lstlisting}
%  $Author: Peter Gagarinov <pgagarinov@gmail.com> $	$Date: 2011-09-09 $ 
%  $Copyright: Moscow State University,
%             Faculty of Computational Mathematics and Computer Science,
%             System Analysis Department 2011 $
% 
% 
%

\end{lstlisting}
\fontfamily{\familydefault}
\selectfont
\item \hyperlink{/+gras/+ellapx/+uncertcalc/+test/+comp/run\_tests}{run\_tests}
\fontfamily{pcr}
\selectfont
\begin{lstlisting}
%

\end{lstlisting}
\fontfamily{\familydefault}
\selectfont
\item \hyperlink{/+gras/+ellapx/+uncertcalc/+test/+comp/updateallconf}{updateallconf}
\fontfamily{pcr}
\selectfont
\begin{lstlisting}
%  $Author: Peter Gagarinov <pgagarinov@gmail.com> $	$Date: 2011-09-09 $ 
%  $Copyright: Moscow State University,
%             Faculty of Computational Mathematics and Computer Science,
%             System Analysis Department 2011 $
% 
% 
%

\end{lstlisting}
\fontfamily{\familydefault}
\selectfont
\item \hyperlink{/+gras/+ellapx/+uncertcalc/+test/+comp/updateconftemplate}{updateconftemplate}
\fontfamily{pcr}
\selectfont
\begin{lstlisting}
%  $Author: Peter Gagarinov <pgagarinov@gmail.com> $	$Date: 2011-09-09 $ 
%  $Copyright: Moscow State University,
%             Faculty of Computational Mathematics and Computer Science,
%             System Analysis Department 2011 $
% 
% 
%

\end{lstlisting}
\fontfamily{\familydefault}
\selectfont
\end{enumerate}
\subsection{/+gras/+ellapx/+uncertcalc/+test/+comp/+conf}
\begin{enumerate}
\item \hyperlink{/+gras/+ellapx/+uncertcalc/+test/+comp/+conf/ConfRepoMgr}{ConfRepoMgr}
\fontfamily{pcr}
\selectfont
\begin{lstlisting}
%

\end{lstlisting}
\fontfamily{\familydefault}
\selectfont
\end{enumerate}
\subsection{/+gras/+ellapx/+uncertcalc/+test/+comp/+conf/+sysdef}
\begin{enumerate}
\item \hyperlink{/+gras/+ellapx/+uncertcalc/+test/+comp/+conf/+sysdef/ConfRepoMgr}{ConfRepoMgr}
\fontfamily{pcr}
\selectfont
\begin{lstlisting}
%

\end{lstlisting}
\fontfamily{\familydefault}
\selectfont
\end{enumerate}
\subsection{/+gras/+ellapx/+uncertcalc/+test/+comp/+mlunit}
\begin{enumerate}
\item \hyperlink{/+gras/+ellapx/+uncertcalc/+test/+comp/+mlunit/SuiteCompare}{SuiteCompare}
\fontfamily{pcr}
\selectfont
\begin{lstlisting}
%

\end{lstlisting}
\fontfamily{\familydefault}
\selectfont
\end{enumerate}
\subsection{/+gras/+ellapx/+uncertcalc/+test/+regr}
\begin{enumerate}
\item \hyperlink{/+gras/+ellapx/+uncertcalc/+test/+regr/copyconf}{copyconf}
\fontfamily{pcr}
\selectfont
\begin{lstlisting}
%

\end{lstlisting}
\fontfamily{\familydefault}
\selectfont
\item \hyperlink{/+gras/+ellapx/+uncertcalc/+test/+regr/editconf}{editconf}
\fontfamily{pcr}
\selectfont
\begin{lstlisting}
%

\end{lstlisting}
\fontfamily{\familydefault}
\selectfont
\item \hyperlink{/+gras/+ellapx/+uncertcalc/+test/+regr/editconftemplate}{editconftemplate}
\fontfamily{pcr}
\selectfont
\begin{lstlisting}
%

\end{lstlisting}
\fontfamily{\familydefault}
\selectfont
\item \hyperlink{/+gras/+ellapx/+uncertcalc/+test/+regr/editsysconf}{editsysconf}
\fontfamily{pcr}
\selectfont
\begin{lstlisting}
%

\end{lstlisting}
\fontfamily{\familydefault}
\selectfont
\item \hyperlink{/+gras/+ellapx/+uncertcalc/+test/+regr/listconfs}{listconfs}
\fontfamily{pcr}
\selectfont
\begin{lstlisting}
%  $Author: Peter Gagarinov <pgagarinov@gmail.com> $	$Date: 2011-09-09 $ 
%  $Copyright: Moscow State University,
%             Faculty of Computational Mathematics and Computer Science,
%             System Analysis Department 2011 $
% 
% 
%

\end{lstlisting}
\fontfamily{\familydefault}
\selectfont
\item \hyperlink{/+gras/+ellapx/+uncertcalc/+test/+regr/run\_regr\_tests}{run\_regr\_tests}
\fontfamily{pcr}
\selectfont
\begin{lstlisting}
%

\end{lstlisting}
\fontfamily{\familydefault}
\selectfont
\item \hyperlink{/+gras/+ellapx/+uncertcalc/+test/+regr/run\_support\_function\_tests}{run\_support\_function\_tests}
\fontfamily{pcr}
\selectfont
\begin{lstlisting}
%  $Author: Kirill Mayantsev  <kirill.mayantsev@gmail.com> $  $Date: 2-11-2012 $
%  $Copyright: Moscow State University,
%              Faculty of Computational Mathematics and Computer Science,
%              System Analysis Department 2012 $
%

\end{lstlisting}
\fontfamily{\familydefault}
\selectfont
\item \hyperlink{/+gras/+ellapx/+uncertcalc/+test/+regr/run\_tests}{run\_tests}
\fontfamily{pcr}
\selectfont
\begin{lstlisting}
%

\end{lstlisting}
\fontfamily{\familydefault}
\selectfont
\item \hyperlink{/+gras/+ellapx/+uncertcalc/+test/+regr/updateallconf}{updateallconf}
\fontfamily{pcr}
\selectfont
\begin{lstlisting}
%  $Author: Peter Gagarinov <pgagarinov@gmail.com> $	$Date: 2011-09-09 $ 
%  $Copyright: Moscow State University,
%             Faculty of Computational Mathematics and Computer Science,
%             System Analysis Department 2011 $
% 
% 
%

\end{lstlisting}
\fontfamily{\familydefault}
\selectfont
\item \hyperlink{/+gras/+ellapx/+uncertcalc/+test/+regr/updateconftemplate}{updateconftemplate}
\fontfamily{pcr}
\selectfont
\begin{lstlisting}
%  $Author: Peter Gagarinov <pgagarinov@gmail.com> $	$Date: 2011-09-09 $ 
%  $Copyright: Moscow State University,
%             Faculty of Computational Mathematics and Computer Science,
%             System Analysis Department 2011 $
% 
% 
%

\end{lstlisting}
\fontfamily{\familydefault}
\selectfont
\end{enumerate}
\subsection{/+gras/+ellapx/+uncertcalc/+test/+regr/+conf}
\begin{enumerate}
\item \hyperlink{/+gras/+ellapx/+uncertcalc/+test/+regr/+conf/ConfRepoMgr}{ConfRepoMgr}
\fontfamily{pcr}
\selectfont
\begin{lstlisting}
%

\end{lstlisting}
\fontfamily{\familydefault}
\selectfont
\end{enumerate}
\subsection{/+gras/+ellapx/+uncertcalc/+test/+regr/+conf/+sysdef}
\begin{enumerate}
\item \hyperlink{/+gras/+ellapx/+uncertcalc/+test/+regr/+conf/+sysdef/ConfRepoMgr}{ConfRepoMgr}
\fontfamily{pcr}
\selectfont
\begin{lstlisting}
%

\end{lstlisting}
\fontfamily{\familydefault}
\selectfont
\end{enumerate}
\subsection{/+gras/+ellapx/+uncertcalc/+test/+regr/+mlunit}
\begin{enumerate}
\item \hyperlink{/+gras/+ellapx/+uncertcalc/+test/+regr/+mlunit/SuiteBasic}{SuiteBasic}
\fontfamily{pcr}
\selectfont
\begin{lstlisting}
%

\end{lstlisting}
\fontfamily{\familydefault}
\selectfont
\item \hyperlink{/+gras/+ellapx/+uncertcalc/+test/+regr/+mlunit/SuiteRegression}{SuiteRegression}
\fontfamily{pcr}
\selectfont
\begin{lstlisting}
%

\end{lstlisting}
\fontfamily{\familydefault}
\selectfont
\item \hyperlink{/+gras/+ellapx/+uncertcalc/+test/+regr/+mlunit/SuiteSupportFunction}{SuiteSupportFunction}
\fontfamily{pcr}
\selectfont
\begin{lstlisting}
%  $Author: Kirill Mayantsev  <kirill.mayantsev@gmail.com> $  $Date: 2-11-2012 $
%  $Copyright: Moscow State University,
%              Faculty of Computational Mathematics and Computer Science,
%              System Analysis Department 2012 $
%

\end{lstlisting}
\fontfamily{\familydefault}
\selectfont
\end{enumerate}
\subsection{/+gras/+gen}
\begin{enumerate}
\item \hyperlink{/+gras/+gen/MatVector}{MatVector}
\fontfamily{pcr}
\selectfont
\begin{lstlisting}
% MATVECTOR Summary of this class goes here
%    Detailed explanation goes here
%

\end{lstlisting}
\fontfamily{\familydefault}
\selectfont
\item \hyperlink{/+gras/+gen/ProgressCmdDisplayer}{ProgressCmdDisplayer}
\fontfamily{pcr}
\selectfont
\begin{lstlisting}
%

\end{lstlisting}
\fontfamily{\familydefault}
\selectfont
\item \hyperlink{/+gras/+gen/SquareMatVector}{SquareMatVector}
\fontfamily{pcr}
\selectfont
\begin{lstlisting}
% MATVECTOR Summary of this class goes here
%    Detailed explanation goes here
%

\end{lstlisting}
\fontfamily{\familydefault}
\selectfont
\item \hyperlink{/+gras/+gen/minadv}{minadv}
\fontfamily{pcr}
\selectfont
\begin{lstlisting}
%  $Author: Peter Gagarinov  <pgagarinov@gmail.com> $	$Date: 2011-05-29 $ 
%  $Copyright: Moscow State University,
%             Faculty of Computational Mathematics and Computer Science,
%             System Analysis Department 2011 $
% 
%

\end{lstlisting}
\fontfamily{\familydefault}
\selectfont
\item \hyperlink{/+gras/+gen/sortrowstol}{sortrowstol}
\fontfamily{pcr}
\selectfont
\begin{lstlisting}
%  TODO add support for clustering method based on
%    clusterdata([1;1+1e-14;2;2+1e-14;2-1e-14],'criterion','distance','cutoff',1e-14)
%

\end{lstlisting}
\fontfamily{\familydefault}
\selectfont
\end{enumerate}
\subsection{/+gras/+gen/+test}
\begin{enumerate}
\item \hyperlink{/+gras/+gen/+test/run\_tests}{run\_tests}
\fontfamily{pcr}
\selectfont
\begin{lstlisting}
%

\end{lstlisting}
\fontfamily{\familydefault}
\selectfont
\end{enumerate}
\subsection{/+gras/+gen/+test/+mlunit}
\begin{enumerate}
\item \hyperlink{/+gras/+gen/+test/+mlunit/SuiteBasic}{SuiteBasic}
\fontfamily{pcr}
\selectfont
\begin{lstlisting}
%

\end{lstlisting}
\fontfamily{\familydefault}
\selectfont
\end{enumerate}
\subsection{/+gras/+geom}
\begin{enumerate}
\item \hyperlink{/+gras/+geom/circlepart}{circlepart}
\fontfamily{pcr}
\selectfont
\begin{lstlisting}
%  $Author: Peter Gagarinov  <pgagarinov@gmail.com> $	$Date: 2011-05-31$ 
%  $Copyright: Moscow State University,
%             Faculty of Computational Mathematics and Computer Science,
%             System Analysis Department 2011 $
% 
%

\end{lstlisting}
\fontfamily{\familydefault}
\selectfont
\end{enumerate}
\subsection{/+gras/+geom/+ell}
\begin{enumerate}
\item \hyperlink{/+gras/+geom/+ell/ellvolume}{ellvolume}
\fontfamily{pcr}
\selectfont
\begin{lstlisting}
%  $Author: Peter Gagarinov  <pgagarinov@gmail.com> $	$Date: 2011-12-30 $ 
%  $Copyright: Moscow State University,
%             Faculty of Computational Mathematics and Computer Science,
%             System Analysis Department 2011 $
% 
%

\end{lstlisting}
\fontfamily{\familydefault}
\selectfont
\end{enumerate}
\subsection{/+gras/+geom/+ell/+test}
\begin{enumerate}
\item \hyperlink{/+gras/+geom/+ell/+test/run\_tests}{run\_tests}
\fontfamily{pcr}
\selectfont
\begin{lstlisting}
%

\end{lstlisting}
\fontfamily{\familydefault}
\selectfont
\end{enumerate}
\subsection{/+gras/+geom/+ell/+test/+mlunit}
\begin{enumerate}
\item \hyperlink{/+gras/+geom/+ell/+test/+mlunit/SuiteBasic}{SuiteBasic}
\fontfamily{pcr}
\selectfont
\begin{lstlisting}
%

\end{lstlisting}
\fontfamily{\familydefault}
\selectfont
\end{enumerate}
\subsection{/+gras/+geom/+sup}
\begin{enumerate}
\item \hyperlink{/+gras/+geom/+sup/sup2boundary2}{sup2boundary2}
\fontfamily{pcr}
\selectfont
\begin{lstlisting}
%  $Author: Peter Gagarinov  <pgagarinov@gmail.com> $	$Date: 2011-05-30 $ 
%  $Copyright: Moscow State University,
%             Faculty of Computational Mathematics and Computer Science,
%             System Analysis Department 2011 $
% 
%

\end{lstlisting}
\fontfamily{\familydefault}
\selectfont
\item \hyperlink{/+gras/+geom/+sup/sup2boundary3}{sup2boundary3}
\fontfamily{pcr}
\selectfont
\begin{lstlisting}
%  $Author: Peter Gagarinov  <pgagarinov@gmail.com> $	$Date: 2011-05-30$ 
%  $Copyright: Moscow State University,
%             Faculty of Computational Mathematics and Computer Science,
%             System Analysis Department 2011 $
% 
%

\end{lstlisting}
\fontfamily{\familydefault}
\selectfont
\item \hyperlink{/+gras/+geom/+sup/supgeomdiff2d}{supgeomdiff2d}
\fontfamily{pcr}
\selectfont
\begin{lstlisting}
%  $Author: Peter Gagarinov  <pgagarinov@gmail.com> $	$Date: 2013-01-22 $ 
%  $Copyright: Moscow State University,
%             Faculty of Computational Mathematics and Computer Science,
%             System Analysis Department 2013 $
% 
%

\end{lstlisting}
\fontfamily{\familydefault}
\selectfont
\end{enumerate}
\subsection{/+gras/+geom/+sup/+test}
\begin{enumerate}
\item \hyperlink{/+gras/+geom/+sup/+test/qint2}{qint2}
\fontfamily{pcr}
\selectfont
\begin{lstlisting}
%

\end{lstlisting}
\fontfamily{\familydefault}
\selectfont
\item \hyperlink{/+gras/+geom/+sup/+test/run\_tests}{run\_tests}
\fontfamily{pcr}
\selectfont
\begin{lstlisting}
%

\end{lstlisting}
\fontfamily{\familydefault}
\selectfont
\end{enumerate}
\subsection{/+gras/+geom/+sup/+test/+mlunit}
\begin{enumerate}
\item \hyperlink{/+gras/+geom/+sup/+test/+mlunit/SuiteBasic}{SuiteBasic}
\fontfamily{pcr}
\selectfont
\begin{lstlisting}
%

\end{lstlisting}
\fontfamily{\familydefault}
\selectfont
\end{enumerate}
\subsection{/+gras/+geom/+test}
\begin{enumerate}
\item \hyperlink{/+gras/+geom/+test/run\_tests}{run\_tests}
\fontfamily{pcr}
\selectfont
\begin{lstlisting}
%

\end{lstlisting}
\fontfamily{\familydefault}
\selectfont
\end{enumerate}
\subsection{/+gras/+geom/+tri}
\begin{enumerate}
\item \hyperlink{/+gras/+geom/+tri/elltubetri}{elltubetri}
\fontfamily{pcr}
\selectfont
\begin{lstlisting}
%  $Author: Peter Gagarinov  <pgagarinov@gmail.com> $	$Date: 2009-07 $
%  $Copyright: Moscow State University,
%             Faculty of Computational Mathematics and Computer Science,
%             System Analysis Department 2011 $
%

\end{lstlisting}
\fontfamily{\familydefault}
\selectfont
\item \hyperlink{/+gras/+geom/+tri/icosahedron}{icosahedron}
\fontfamily{pcr}
\selectfont
\begin{lstlisting}
%  $Author: Peter Gagarinov  <pgagarinov@gmail.com> $	$Date: 2011-05-27$ 
%  $Copyright: Moscow State University,
%             Faculty of Computational Mathematics and Computer Science,
%             System Analysis Department 2011 $
% 
%

\end{lstlisting}
\fontfamily{\familydefault}
\selectfont
\item \hyperlink{/+gras/+geom/+tri/isface}{isface}
\fontfamily{pcr}
\selectfont
\begin{lstlisting}
%  $Author: Peter Gagarinov  <pgagarinov@gmail.com> $	$Date: 2011-05-27$ 
%  $Copyright: Moscow State University,
%             Faculty of Computational Mathematics and Computer Science,
%             System Analysis Department 2011 $
%

\end{lstlisting}
\fontfamily{\familydefault}
\selectfont
\item \hyperlink{/+gras/+geom/+tri/istriequal}{istriequal}
\fontfamily{pcr}
\selectfont
\begin{lstlisting}
%  $Author: Peter Gagarinov  <pgagarinov@gmail.com> $	$Date: 2011-05-27$ 
%  $Copyright: Moscow State University,
%             Faculty of Computational Mathematics and Computer Science,
%             System Analysis Department 2011 $
% 
%

\end{lstlisting}
\fontfamily{\familydefault}
\selectfont
\item \hyperlink{/+gras/+geom/+tri/mapface2edge}{mapface2edge}
\fontfamily{pcr}
\selectfont
\begin{lstlisting}
%  $Author: Peter Gagarinov  <pgagarinov@gmail.com> $	$Date: 2011-05-27$ 
%  $Copyright: Moscow State University,
%             Faculty of Computational Mathematics and Computer Science,
%             System Analysis Department 2011 $
% 
%

\end{lstlisting}
\fontfamily{\familydefault}
\selectfont
\item \hyperlink{/+gras/+geom/+tri/shrinkfacetri}{shrinkfacetri}
\fontfamily{pcr}
\selectfont
\begin{lstlisting}
%  $Author: Peter Gagarinov  <pgagarinov@gmail.com> $	$Date: 2011-05-28$ 
%  $Copyright: Moscow State University,
%             Faculty of Computational Mathematics and Computer Science,
%             System Analysis Department 2011 $
% 
%

\end{lstlisting}
\fontfamily{\familydefault}
\selectfont
\item \hyperlink{/+gras/+geom/+tri/spheretri}{spheretri}
\fontfamily{pcr}
\selectfont
\begin{lstlisting}
%  $Author: Peter Gagarinov  <pgagarinov@gmail.com> $	$Date: 2011-05-27$ 
%  $Copyright: Moscow State University,
%             Faculty of Computational Mathematics and Computer Science,
%             System Analysis Department 2011 $
% 
%

\end{lstlisting}
\fontfamily{\familydefault}
\selectfont
\end{enumerate}
\subsection{/+gras/+geom/+tri/+test}
\begin{enumerate}
\item \hyperlink{/+gras/+geom/+tri/+test/run\_tests}{run\_tests}
\fontfamily{pcr}
\selectfont
\begin{lstlisting}
%

\end{lstlisting}
\fontfamily{\familydefault}
\selectfont
\item \hyperlink{/+gras/+geom/+tri/+test/spheretri}{spheretri}
\fontfamily{pcr}
\selectfont
\begin{lstlisting}
%  $Author: Peter Gagarinov  <pgagarinov@gmail.com> $	$Date: 2011-05-21$ 
%  $Copyright: Moscow State University,
%             Faculty of Computational Mathematics and Computer Science,
%             System Analysis Department 2011 $
%

\end{lstlisting}
\fontfamily{\familydefault}
\selectfont
\end{enumerate}
\subsection{/+gras/+geom/+tri/+test/+mlunit}
\begin{enumerate}
\item \hyperlink{/+gras/+geom/+tri/+test/+mlunit/SuiteTri}{SuiteTri}
\fontfamily{pcr}
\selectfont
\begin{lstlisting}
%

\end{lstlisting}
\fontfamily{\familydefault}
\selectfont
\end{enumerate}
\subsection{/+gras/+geom/+tri/+test/srebuild3d}
\begin{enumerate}
\item \hyperlink{/+gras/+geom/+tri/+test/srebuild3d/build}{build}
\fontfamily{pcr}
\selectfont
\begin{lstlisting}
%

\end{lstlisting}
\fontfamily{\familydefault}
\selectfont
\end{enumerate}
\subsection{/+gras/+interp}
\begin{enumerate}
\item \hyperlink{/+gras/+interp/AMatrixCubicSpline}{AMatrixCubicSpline}
\fontfamily{pcr}
\selectfont
\begin{lstlisting}
%  $Author: Peter Gagarinov  <pgagarinov@gmail.com> $	$Date: 2011-08$
%  $Copyright: Moscow State University,
%             Faculty of Computational Mathematics and Computer Science,
%             System Analysis Department 2011 $
% 
%

\end{lstlisting}
\fontfamily{\familydefault}
\selectfont
\item \hyperlink{/+gras/+interp/MatrixColCubicSpline}{MatrixColCubicSpline}
\fontfamily{pcr}
\selectfont
\begin{lstlisting}
%  $Author: Peter Gagarinov  <pgagarinov@gmail.com> $	$Date: 2011-08$
%  $Copyright: Moscow State University,
%             Faculty of Computational Mathematics and Computer Science,
%             System Analysis Department 2011 $
% 
%

\end{lstlisting}
\fontfamily{\familydefault}
\selectfont
\item \hyperlink{/+gras/+interp/MatrixColTriuCubicSpline}{MatrixColTriuCubicSpline}
\fontfamily{pcr}
\selectfont
\begin{lstlisting}
%  $Author: Peter Gagarinov  <pgagarinov@gmail.com> $	$Date: 2011-08$
%  $Copyright: Moscow State University,
%             Faculty of Computational Mathematics and Computer Science,
%             System Analysis Department 2011 $
%     
%

\end{lstlisting}
\fontfamily{\familydefault}
\selectfont
\item \hyperlink{/+gras/+interp/MatrixColTriuSymmCubicSpline}{MatrixColTriuSymmCubicSpline}
\fontfamily{pcr}
\selectfont
\begin{lstlisting}
%  $Author: Peter Gagarinov  <pgagarinov@gmail.com> $	$Date: 2011-08$
%  $Copyright: Moscow State University,
%             Faculty of Computational Mathematics and Computer Science,
%             System Analysis Department 2011 $
%     
%

\end{lstlisting}
\fontfamily{\familydefault}
\selectfont
\item \hyperlink{/+gras/+interp/MatrixInterpolantFactory}{MatrixInterpolantFactory}
\fontfamily{pcr}
\selectfont
\begin{lstlisting}
%

\end{lstlisting}
\fontfamily{\familydefault}
\selectfont
\item \hyperlink{/+gras/+interp/MatrixRowCubicSpline}{MatrixRowCubicSpline}
\fontfamily{pcr}
\selectfont
\begin{lstlisting}
%  $Author: Peter Gagarinov  <pgagarinov@gmail.com> $	$Date: 2011-08$
%  $Copyright: Moscow State University,
%             Faculty of Computational Mathematics and Computer Science,
%             System Analysis Department 2011 $
%     
%

\end{lstlisting}
\fontfamily{\familydefault}
\selectfont
\item \hyperlink{/+gras/+interp/NNDefMatCholCubicSpline}{NNDefMatCholCubicSpline}
\fontfamily{pcr}
\selectfont
\begin{lstlisting}
%  $Author: Peter Gagarinov  <pgagarinov@gmail.com> $	$Date: 2011-10$
%  $Copyright: Moscow State University,
%             Faculty of Computational Mathematics and Computer Science,
%             System Analysis Department 2011 $    
%

\end{lstlisting}
\fontfamily{\familydefault}
\selectfont
\item \hyperlink{/+gras/+interp/PosDefMatCholCubicSpline}{PosDefMatCholCubicSpline}
\fontfamily{pcr}
\selectfont
\begin{lstlisting}
%  $Author: Peter Gagarinov  <pgagarinov@gmail.com> $	$Date: 2011-08$
%  $Copyright: Moscow State University,
%             Faculty of Computational Mathematics and Computer Science,
%             System Analysis Department 2011 $    
%

\end{lstlisting}
\fontfamily{\familydefault}
\selectfont
\item \hyperlink{/+gras/+interp/SplineMatrixOperations}{SplineMatrixOperations}
\fontfamily{pcr}
\selectfont
\begin{lstlisting}
%

\end{lstlisting}
\fontfamily{\familydefault}
\selectfont
\end{enumerate}
\subsection{/+gras/+interp/+test}
\begin{enumerate}
\item \hyperlink{/+gras/+interp/+test/run\_tests}{run\_tests}
\fontfamily{pcr}
\selectfont
\begin{lstlisting}
%

\end{lstlisting}
\fontfamily{\familydefault}
\selectfont
\end{enumerate}
\subsection{/+gras/+interp/+test/+mlunit}
\begin{enumerate}
\item \hyperlink{/+gras/+interp/+test/+mlunit/SuiteBasic}{SuiteBasic}
\fontfamily{pcr}
\selectfont
\begin{lstlisting}
%

\end{lstlisting}
\fontfamily{\familydefault}
\selectfont
\end{enumerate}
\subsection{/+gras/+la}
\begin{enumerate}
\item \hyperlink{/+gras/+la/ismatposdef}{ismatposdef}
\fontfamily{pcr}
\selectfont
\begin{lstlisting}
%  $Author: Vitaly Baranov  <vetbar42@gmail.com> $	$Date: 2013-01-Mar$
%  $Copyright: Lomonosov Moscow State University,
%              Faculty of Computational Mathematics and Cybernetics,
%              Department of System Analysis  2013 $
% 
% 
%

\end{lstlisting}
\fontfamily{\familydefault}
\selectfont
\item \hyperlink{/+gras/+la/ismatsymm}{ismatsymm}
\fontfamily{pcr}
\selectfont
\begin{lstlisting}
%  $Author: Rustam Guliev  <glvrst@gmail.com> $	$Date: 2012-16-11$
%  $Copyright: Moscow State University,
%             Faculty of Computational Mathematics and Cybernetics,
%             System Analysis Department 2012 $
% 
%

\end{lstlisting}
\fontfamily{\familydefault}
\selectfont
\item \hyperlink{/+gras/+la/matorth}{matorth}
\fontfamily{pcr}
\selectfont
\begin{lstlisting}
%  $Author: Peter Gagarinov  <pgagarinov@gmail.com> $	$Date: 2012-06-25$
%  $Copyright: Moscow State University,
%             Faculty of Computational Mathematics and Computer Science,
%             System Analysis Department 2012 $
% 
%

\end{lstlisting}
\fontfamily{\familydefault}
\selectfont
\item \hyperlink{/+gras/+la/mlorthtransl}{mlorthtransl}
\fontfamily{pcr}
\selectfont
\begin{lstlisting}
%  $Author: Peter Gagarinov  <pgagarinov@gmail.com> $	$Date: 2011-05-01$
%  $Copyright: Moscow State University,
%             Faculty of Computational Mathematics and Computer Science,
%             System Analysis Department 2011 $
% 
%

\end{lstlisting}
\fontfamily{\familydefault}
\selectfont
\item \hyperlink{/+gras/+la/orthtransl}{orthtransl}
\fontfamily{pcr}
\selectfont
\begin{lstlisting}
%  $Author: Peter Gagarinov  <pgagarinov@gmail.com> $	$Date: 2012-11-28$
%  $Copyright: Moscow State University,
%             Faculty of Computational Mathematics and Computer Science,
%             System Analysis Department 2012 $
% 
%

\end{lstlisting}
\fontfamily{\familydefault}
\selectfont
\item \hyperlink{/+gras/+la/orthtranslhaus}{orthtranslhaus}
\fontfamily{pcr}
\selectfont
\begin{lstlisting}
%  Output:
%    oMat: double[nDims,nDims]
%  $Author: Peter Gagarinov  <pgagarinov@gmail.com> $	$Date: 2011-05-15$
%  $Copyright: Moscow State University,
%             Faculty of Computational Mathematics and Computer Science,
%             System Analysis Department 2011 $
% 
%

\end{lstlisting}
\fontfamily{\familydefault}
\selectfont
\item \hyperlink{/+gras/+la/orthtranslmaxdir}{orthtranslmaxdir}
\fontfamily{pcr}
\selectfont
\begin{lstlisting}
%  $Author: Peter Gagarinov  <pgagarinov@gmail.com> $	$Date: 2011-05-03$
%  $Copyright: Moscow State University,
%             Faculty of Computational Mathematics and Computer Science,
%             System Analysis Department 2011 $
% 
%

\end{lstlisting}
\fontfamily{\familydefault}
\selectfont
\item \hyperlink{/+gras/+la/orthtranslmaxtr}{orthtranslmaxtr}
\fontfamily{pcr}
\selectfont
\begin{lstlisting}
%  $Author: Peter Gagarinov  <pgagarinov@gmail.com> $	$Date: 2011-05-03$
%  $Copyright: Moscow State University,
%             Faculty of Computational Mathematics and Computer Science,
%             System Analysis Department 2011 $
% 
%

\end{lstlisting}
\fontfamily{\familydefault}
\selectfont
\item \hyperlink{/+gras/+la/sqrtm}{sqrtm}
\fontfamily{pcr}
\selectfont
\begin{lstlisting}
%  $Author: Vadim Kaushanskiy  <vkaushanskiy@gmail.com> $	$Date: 2012-01-11$
%  $Copyright: Moscow State University,
%             Faculty of Computational Mathematics and Cybernetics,
%             System Analysis Department 2012 $
%

\end{lstlisting}
\fontfamily{\familydefault}
\selectfont
\end{enumerate}
\subsection{/+gras/+la/+test}
\begin{enumerate}
\item \hyperlink{/+gras/+la/+test/run\_tests}{run\_tests}
\fontfamily{pcr}
\selectfont
\begin{lstlisting}
%

\end{lstlisting}
\fontfamily{\familydefault}
\selectfont
\end{enumerate}
\subsection{/+gras/+la/+test/+mlunit}
\begin{enumerate}
\item \hyperlink{/+gras/+la/+test/+mlunit/BasicTestCase}{BasicTestCase}
\fontfamily{pcr}
\selectfont
\begin{lstlisting}
%  $Author: Vadim Kaushanskiy, Moscow State University by M.V. Lomonosov,
%  Faculty of Computational Mathematics and Cybernetics, System Analysis
%  Department, 1-November-2012, <vkaushanskiy@gmail.com>$
%

\end{lstlisting}
\fontfamily{\familydefault}
\selectfont
\item \hyperlink{/+gras/+la/+test/+mlunit/SuiteOrthTransl}{SuiteOrthTransl}
\fontfamily{pcr}
\selectfont
\begin{lstlisting}
%

\end{lstlisting}
\fontfamily{\familydefault}
\selectfont
\end{enumerate}
\subsection{/+gras/+mat}
\begin{enumerate}
\item \hyperlink{/+gras/+mat/AConstMatrixFunction}{AConstMatrixFunction}
\fontfamily{pcr}
\selectfont
\begin{lstlisting}
%

\end{lstlisting}
\fontfamily{\familydefault}
\selectfont
\item \hyperlink{/+gras/+mat/AMatrixBinaryOpFunc}{AMatrixBinaryOpFunc}
\fontfamily{pcr}
\selectfont
\begin{lstlisting}
%

\end{lstlisting}
\fontfamily{\familydefault}
\selectfont
\item \hyperlink{/+gras/+mat/AMatrixOpFunc}{AMatrixOpFunc}
\fontfamily{pcr}
\selectfont
\begin{lstlisting}
%

\end{lstlisting}
\fontfamily{\familydefault}
\selectfont
\item \hyperlink{/+gras/+mat/AMatrixOperations}{AMatrixOperations}
\fontfamily{pcr}
\selectfont
\begin{lstlisting}
%

\end{lstlisting}
\fontfamily{\familydefault}
\selectfont
\item \hyperlink{/+gras/+mat/AMatrixTernaryOpFunc}{AMatrixTernaryOpFunc}
\fontfamily{pcr}
\selectfont
\begin{lstlisting}
%

\end{lstlisting}
\fontfamily{\familydefault}
\selectfont
\item \hyperlink{/+gras/+mat/AMatrixUnaryOpFunc}{AMatrixUnaryOpFunc}
\fontfamily{pcr}
\selectfont
\begin{lstlisting}
%

\end{lstlisting}
\fontfamily{\familydefault}
\selectfont
\item \hyperlink{/+gras/+mat/CompositeMatrixOperations}{CompositeMatrixOperations}
\fontfamily{pcr}
\selectfont
\begin{lstlisting}
%

\end{lstlisting}
\fontfamily{\familydefault}
\selectfont
\item \hyperlink{/+gras/+mat/ConstMatrixFunctionFactory}{ConstMatrixFunctionFactory}
\fontfamily{pcr}
\selectfont
\begin{lstlisting}
%

\end{lstlisting}
\fontfamily{\familydefault}
\selectfont
\item \hyperlink{/+gras/+mat/IMatrixFunction}{IMatrixFunction}
\fontfamily{pcr}
\selectfont
\begin{lstlisting}
%

\end{lstlisting}
\fontfamily{\familydefault}
\selectfont
\item \hyperlink{/+gras/+mat/IMatrixOperations}{IMatrixOperations}
\fontfamily{pcr}
\selectfont
\begin{lstlisting}
%

\end{lstlisting}
\fontfamily{\familydefault}
\selectfont
\item \hyperlink{/+gras/+mat/MatrixOperationsFactory}{MatrixOperationsFactory}
\fontfamily{pcr}
\selectfont
\begin{lstlisting}
%

\end{lstlisting}
\fontfamily{\familydefault}
\selectfont
\end{enumerate}
\subsection{/+gras/+mat/+fcnlib}
\begin{enumerate}
\item \hyperlink{/+gras/+mat/+fcnlib/ConstColFunction}{ConstColFunction}
\fontfamily{pcr}
\selectfont
\begin{lstlisting}
%

\end{lstlisting}
\fontfamily{\familydefault}
\selectfont
\item \hyperlink{/+gras/+mat/+fcnlib/ConstMatrixFunction}{ConstMatrixFunction}
\fontfamily{pcr}
\selectfont
\begin{lstlisting}
%

\end{lstlisting}
\fontfamily{\familydefault}
\selectfont
\item \hyperlink{/+gras/+mat/+fcnlib/ConstRowFunction}{ConstRowFunction}
\fontfamily{pcr}
\selectfont
\begin{lstlisting}
%

\end{lstlisting}
\fontfamily{\familydefault}
\selectfont
\item \hyperlink{/+gras/+mat/+fcnlib/MatrixBinaryTimesFunc}{MatrixBinaryTimesFunc}
\fontfamily{pcr}
\selectfont
\begin{lstlisting}
%

\end{lstlisting}
\fontfamily{\familydefault}
\selectfont
\item \hyperlink{/+gras/+mat/+fcnlib/MatrixExpFunc}{MatrixExpFunc}
\fontfamily{pcr}
\selectfont
\begin{lstlisting}
%

\end{lstlisting}
\fontfamily{\familydefault}
\selectfont
\item \hyperlink{/+gras/+mat/+fcnlib/MatrixExpTimeFunc}{MatrixExpTimeFunc}
\fontfamily{pcr}
\selectfont
\begin{lstlisting}
%

\end{lstlisting}
\fontfamily{\familydefault}
\selectfont
\item \hyperlink{/+gras/+mat/+fcnlib/MatrixInvFunc}{MatrixInvFunc}
\fontfamily{pcr}
\selectfont
\begin{lstlisting}
%

\end{lstlisting}
\fontfamily{\familydefault}
\selectfont
\item \hyperlink{/+gras/+mat/+fcnlib/MatrixLRDivideVecFunc}{MatrixLRDivideVecFunc}
\fontfamily{pcr}
\selectfont
\begin{lstlisting}
%

\end{lstlisting}
\fontfamily{\familydefault}
\selectfont
\item \hyperlink{/+gras/+mat/+fcnlib/MatrixLRTimesFunc}{MatrixLRTimesFunc}
\fontfamily{pcr}
\selectfont
\begin{lstlisting}
%

\end{lstlisting}
\fontfamily{\familydefault}
\selectfont
\item \hyperlink{/+gras/+mat/+fcnlib/MatrixMakeSymmetricFunc}{MatrixMakeSymmetricFunc}
\fontfamily{pcr}
\selectfont
\begin{lstlisting}
%

\end{lstlisting}
\fontfamily{\familydefault}
\selectfont
\item \hyperlink{/+gras/+mat/+fcnlib/MatrixMinEigValFunc}{MatrixMinEigValFunc}
\fontfamily{pcr}
\selectfont
\begin{lstlisting}
%

\end{lstlisting}
\fontfamily{\familydefault}
\selectfont
\item \hyperlink{/+gras/+mat/+fcnlib/MatrixMinusFunc}{MatrixMinusFunc}
\fontfamily{pcr}
\selectfont
\begin{lstlisting}
%

\end{lstlisting}
\fontfamily{\familydefault}
\selectfont
\item \hyperlink{/+gras/+mat/+fcnlib/MatrixPInvFunc}{MatrixPInvFunc}
\fontfamily{pcr}
\selectfont
\begin{lstlisting}
%

\end{lstlisting}
\fontfamily{\familydefault}
\selectfont
\item \hyperlink{/+gras/+mat/+fcnlib/MatrixPlusFunc}{MatrixPlusFunc}
\fontfamily{pcr}
\selectfont
\begin{lstlisting}
%

\end{lstlisting}
\fontfamily{\familydefault}
\selectfont
\item \hyperlink{/+gras/+mat/+fcnlib/MatrixSqrtFunc}{MatrixSqrtFunc}
\fontfamily{pcr}
\selectfont
\begin{lstlisting}
%

\end{lstlisting}
\fontfamily{\familydefault}
\selectfont
\item \hyperlink{/+gras/+mat/+fcnlib/MatrixTernaryTimesFunc}{MatrixTernaryTimesFunc}
\fontfamily{pcr}
\selectfont
\begin{lstlisting}
%

\end{lstlisting}
\fontfamily{\familydefault}
\selectfont
\item \hyperlink{/+gras/+mat/+fcnlib/MatrixTransposeFunc}{MatrixTransposeFunc}
\fontfamily{pcr}
\selectfont
\begin{lstlisting}
%

\end{lstlisting}
\fontfamily{\familydefault}
\selectfont
\item \hyperlink{/+gras/+mat/+fcnlib/MatrixTriuFunc}{MatrixTriuFunc}
\fontfamily{pcr}
\selectfont
\begin{lstlisting}
%

\end{lstlisting}
\fontfamily{\familydefault}
\selectfont
\item \hyperlink{/+gras/+mat/+fcnlib/QuadraticFormSqrtFunc}{QuadraticFormSqrtFunc}
\fontfamily{pcr}
\selectfont
\begin{lstlisting}
%

\end{lstlisting}
\fontfamily{\familydefault}
\selectfont
\end{enumerate}
\subsection{/+gras/+mat/+symb}
\begin{enumerate}
\item \hyperlink{/+gras/+mat/+symb/MatrixSFBinaryProd}{MatrixSFBinaryProd}
\fontfamily{pcr}
\selectfont
\begin{lstlisting}
%  $Author: Peter Gagarinov  <pgagarinov@gmail.com> $	$Date: 2011-12-12$
%  $Copyright: Moscow State University,
%             Faculty of Computational Mathematics and Computer Science,
%             System Analysis Department 2011 $
%     
%

\end{lstlisting}
\fontfamily{\familydefault}
\selectfont
\item \hyperlink{/+gras/+mat/+symb/MatrixSFBinaryProdByVec}{MatrixSFBinaryProdByVec}
\fontfamily{pcr}
\selectfont
\begin{lstlisting}
%  $Author: Peter Gagarinov  <pgagarinov@gmail.com> $	$Date: 2011-12-12$
%  $Copyright: Moscow State University,
%             Faculty of Computational Mathematics and Computer Science,
%             System Analysis Department 2011 $
%     
%

\end{lstlisting}
\fontfamily{\familydefault}
\selectfont
\item \hyperlink{/+gras/+mat/+symb/MatrixSFTripleProd}{MatrixSFTripleProd}
\fontfamily{pcr}
\selectfont
\begin{lstlisting}
%

\end{lstlisting}
\fontfamily{\familydefault}
\selectfont
\item \hyperlink{/+gras/+mat/+symb/MatrixSymbFormulaBased}{MatrixSymbFormulaBased}
\fontfamily{pcr}
\selectfont
\begin{lstlisting}
%  $Author: Peter Gagarinov  <pgagarinov@gmail.com> $	$Date: 2011-12-12$
%  $Copyright: Moscow State University,
%             Faculty of Computational Mathematics and Computer Science,
%             System Analysis Department 2011 $    
%

\end{lstlisting}
\fontfamily{\familydefault}
\selectfont
\item \hyperlink{/+gras/+mat/+symb/iscellofstringconst}{iscellofstringconst}
\fontfamily{pcr}
\selectfont
\begin{lstlisting}
%

\end{lstlisting}
\fontfamily{\familydefault}
\selectfont
\end{enumerate}
\subsection{/+gras/+mat/+test}
\begin{enumerate}
\item \hyperlink{/+gras/+mat/+test/run\_tests}{run\_tests}
\fontfamily{pcr}
\selectfont
\begin{lstlisting}
%

\end{lstlisting}
\fontfamily{\familydefault}
\selectfont
\end{enumerate}
\subsection{/+gras/+mat/+test/+mlunit}
\begin{enumerate}
\item \hyperlink{/+gras/+mat/+test/+mlunit/SuiteBasic}{SuiteBasic}
\fontfamily{pcr}
\selectfont
\begin{lstlisting}
%

\end{lstlisting}
\fontfamily{\familydefault}
\selectfont
\item \hyperlink{/+gras/+mat/+test/+mlunit/SuiteOp}{SuiteOp}
\fontfamily{pcr}
\selectfont
\begin{lstlisting}
%

\end{lstlisting}
\fontfamily{\familydefault}
\selectfont
\end{enumerate}
\subsection{/+gras/+ode}
\begin{enumerate}
\item \hyperlink{/+gras/+ode/MatrixODESolver}{MatrixODESolver}
\fontfamily{pcr}
\selectfont
\begin{lstlisting}
%

\end{lstlisting}
\fontfamily{\familydefault}
\selectfont
\item \hyperlink{/+gras/+ode/MatrixSysODESolver}{MatrixSysODESolver}
\fontfamily{pcr}
\selectfont
\begin{lstlisting}
%

\end{lstlisting}
\fontfamily{\familydefault}
\selectfont
\item \hyperlink{/+gras/+ode/ode113reg}{ode113reg}
\fontfamily{pcr}
\selectfont
\begin{lstlisting}
%

\end{lstlisting}
\fontfamily{\familydefault}
\selectfont
\item \hyperlink{/+gras/+ode/ode45reg}{ode45reg}
\fontfamily{pcr}
\selectfont
\begin{lstlisting}
%  $Author: Peter Gagarinov  <pgagarinov@gmail.com> $	$Date: 2011$
%  $Copyright: Moscow State University,
%             Faculty of Computational Mathematics and Computer Science,
%             System Analysis Department 2011 $
% 
%

\end{lstlisting}
\fontfamily{\familydefault}
\selectfont
\end{enumerate}
\subsection{/+gras/+ode/+test}
\begin{enumerate}
\item \hyperlink{/+gras/+ode/+test/run\_tests}{run\_tests}
\fontfamily{pcr}
\selectfont
\begin{lstlisting}
%

\end{lstlisting}
\fontfamily{\familydefault}
\selectfont
\end{enumerate}
\subsection{/+gras/+ode/+test/+mlunit}
\begin{enumerate}
\item \hyperlink{/+gras/+ode/+test/+mlunit/SuiteBasic}{SuiteBasic}
\fontfamily{pcr}
\selectfont
\begin{lstlisting}
%

\end{lstlisting}
\fontfamily{\familydefault}
\selectfont
\end{enumerate}
\subsection{/+gras/+ode/private}
\begin{enumerate}
\item \hyperlink{/+gras/+ode/private/odearguments}{odearguments}
\fontfamily{pcr}
\selectfont
\begin{lstlisting}
%

\end{lstlisting}
\fontfamily{\familydefault}
\selectfont
\item \hyperlink{/+gras/+ode/private/odenonnegative}{odenonnegative}
\fontfamily{pcr}
\selectfont
\begin{lstlisting}
%

\end{lstlisting}
\fontfamily{\familydefault}
\selectfont
\end{enumerate}
\subsection{/+gras/+test}
\begin{enumerate}
\item \hyperlink{/+gras/+test/TmpDataManager}{TmpDataManager}
\fontfamily{pcr}
\selectfont
\begin{lstlisting}
%  TMPDATAMANAGER provides a basic functionality for managing temporary
%  data folders, root folder name is determined automatically
% 
%

\end{lstlisting}
\fontfamily{\familydefault}
\selectfont
\item \hyperlink{/+gras/+test/editconf}{editconf}
\fontfamily{pcr}
\selectfont
\begin{lstlisting}
%

\end{lstlisting}
\fontfamily{\familydefault}
\selectfont
\item \hyperlink{/+gras/+test/run\_tests}{run\_tests}
\fontfamily{pcr}
\selectfont
\begin{lstlisting}
%

\end{lstlisting}
\fontfamily{\familydefault}
\selectfont
\item \hyperlink{/+gras/+test/run\_tests\_remotely}{run\_tests\_remotely}
\fontfamily{pcr}
\selectfont
\begin{lstlisting}
%

\end{lstlisting}
\fontfamily{\familydefault}
\selectfont
\end{enumerate}
\subsection{/+gras/+test/+configuration}
\begin{enumerate}
\item \hyperlink{/+gras/+test/+configuration/AdaptiveConfRepoManager}{AdaptiveConfRepoManager}
\fontfamily{pcr}
\selectfont
\begin{lstlisting}
%  $Author: Peter Gagarinov <pgagarinov@gmail.com> $	$Date: 2011-05-18 $ 
%  $Copyright: Moscow State University,
%             Faculty of Computational Mathematics and Computer Science,
%             System Analysis Department 2011 $
% 
%

\end{lstlisting}
\fontfamily{\familydefault}
\selectfont
\end{enumerate}
\subsection{/+gras/+test/+configuration/@ConfPatchRepo}
\begin{enumerate}
\item \hyperlink{/+gras/+test/+configuration/@ConfPatchRepo/ConfPatchRepo}{ConfPatchRepo}
\fontfamily{pcr}
\selectfont
\begin{lstlisting}
%

\end{lstlisting}
\fontfamily{\familydefault}
\selectfont
\item \hyperlink{/+gras/+test/+configuration/@ConfPatchRepo/patch\_001\_dummy\_patch}{patch\_001\_dummy\_patch}
\fontfamily{pcr}
\selectfont
\begin{lstlisting}
%

\end{lstlisting}
\fontfamily{\familydefault}
\selectfont
\end{enumerate}
\subsection{/+gras/+test/+logging}
\begin{enumerate}
\item \hyperlink{/+gras/+test/+logging/Log4jConfigurator}{Log4jConfigurator}
\fontfamily{pcr}
\selectfont
\begin{lstlisting}
%  $Author: Peter Gagarinov  <pgagarinov@gmail.com> $	$Date: 2011-05-18$
%  $Copyright: Moscow State University,
%             Faculty of Computational Mathematics and Computer Science,
%             System Analysis Department 2011 $
% 
%

\end{lstlisting}
\fontfamily{\familydefault}
\selectfont
\end{enumerate}
\subsection{/+gras/+test/+mlunit}
\begin{enumerate}
\item \hyperlink{/+gras/+test/+mlunit/SuiteBasic}{SuiteBasic}
\fontfamily{pcr}
\selectfont
\begin{lstlisting}
%

\end{lstlisting}
\fontfamily{\familydefault}
\selectfont
\end{enumerate}
\subsection{/elltoolboxcore/@ellipsoid}
\begin{enumerate}
\item \hyperlink{/elltoolboxcore/@ellipsoid/checkIsMe}{checkIsMe}
\fontfamily{pcr}
\selectfont
\begin{lstlisting}
%

\end{lstlisting}
\fontfamily{\familydefault}
\selectfont
\item \hyperlink{/elltoolboxcore/@ellipsoid/contains}{contains}
\fontfamily{pcr}
\selectfont
\begin{lstlisting}
%  $Author: Guliev Rustam <glvrst@gmail.com> $   $Date: Dec-2012$
%  $Copyright: Moscow State University,
%              Faculty of Computational Mathematics and Cybernetics,
%              Science, System Analysis Department 2012 $
% 
%

\end{lstlisting}
\fontfamily{\familydefault}
\selectfont
\item \hyperlink{/elltoolboxcore/@ellipsoid/contents}{contents}
\fontfamily{pcr}
\selectfont
\begin{lstlisting}
%  Author:
%  -------
%     Alex Kurzhanskiy <akurzhan@eecs.berkeley.edu>
% 
%

\end{lstlisting}
\fontfamily{\familydefault}
\selectfont
\item \hyperlink{/elltoolboxcore/@ellipsoid/dimension}{dimension}
\fontfamily{pcr}
\selectfont
\begin{lstlisting}
%  $Author: Guliev Rustam <glvrst@gmail.com> $   $Date: Dec-2012$
%  $Copyright: Moscow State University,
%              Faculty of Computational Mathematics and Cybernetics,
%              Science, System Analysis Department 2012 $
% 
%

\end{lstlisting}
\fontfamily{\familydefault}
\selectfont
\item \hyperlink{/elltoolboxcore/@ellipsoid/disp}{disp}
\fontfamily{pcr}
\selectfont
\begin{lstlisting}
%  $Author: Alex Kurzhanskiy <akurzhan@eecs.berkeley.edu>
%  $Copyright:  The Regents of the University of California 2004-2008 $
%

\end{lstlisting}
\fontfamily{\familydefault}
\selectfont
\item \hyperlink{/elltoolboxcore/@ellipsoid/display}{display}
\fontfamily{pcr}
\selectfont
\begin{lstlisting}
%  $Author: Alex Kurzhanskiy <akurzhan@eecs.berkeley.edu>
%  $Copyright:  The Regents of the University of California 2004-2008 $
%

\end{lstlisting}
\fontfamily{\familydefault}
\selectfont
\item \hyperlink{/elltoolboxcore/@ellipsoid/distance}{distance}
\fontfamily{pcr}
\selectfont
\begin{lstlisting}
%  $Author:  Vitaly Baranov  <vetbar42@gmail.com> $    $Date: 31-10-2012 $
%  $Copyright: Lomonosov Moscow State University,
%             Faculty of Computational Mathematics and Cybernetics,
%             System Analysis Department 2012 $
%  Literature:
%     1. Lin, A. and Han, S. On the Distance between Two Ellipsoids.
%        SIAM Journal on Optimization, 2002, Vol. 13, No. 1 : pp. 298-308
%     2. Stanley Chan, "Numerical method for Finding Minimum Distance to an
%        Ellipsoid". http://videoprocessing.ucsd.edu/~stanleychan/publication/unpublished/Ellipse.pdf
% 
%

\end{lstlisting}
\fontfamily{\familydefault}
\selectfont
\item \hyperlink{/elltoolboxcore/@ellipsoid/double}{double}
\fontfamily{pcr}
\selectfont
\begin{lstlisting}
%  $Author: Guliev Rustam <glvrst@gmail.com> $   $Date: Dec-2012$
%  $Copyright: Moscow State University,
%              Faculty of Computational Mathematics and Cybernetics,
%              Science, System Analysis Department 2012 $
% 
%

\end{lstlisting}
\fontfamily{\familydefault}
\selectfont
\item \hyperlink{/elltoolboxcore/@ellipsoid/ellbndr\_2d}{ellbndr\_2d}
\fontfamily{pcr}
\selectfont
\begin{lstlisting}
%

\end{lstlisting}
\fontfamily{\familydefault}
\selectfont
\item \hyperlink{/elltoolboxcore/@ellipsoid/ellbndr\_3d}{ellbndr\_3d}
\fontfamily{pcr}
\selectfont
\begin{lstlisting}
%

\end{lstlisting}
\fontfamily{\familydefault}
\selectfont
\item \hyperlink{/elltoolboxcore/@ellipsoid/ellintersection\_ia}{ellintersection\_ia}
\fontfamily{pcr}
\selectfont
\begin{lstlisting}
%  $Author: Vadim Kaushanskiy <vkaushanskiy@gmail.com>$ $Date: 10-11-2012$
%  $Copyright: Moscow State University,
%             Faculty of Computational Mathematics and Computer Science,
%             System Analysis Department 2012 $
%

\end{lstlisting}
\fontfamily{\familydefault}
\selectfont
\item \hyperlink{/elltoolboxcore/@ellipsoid/ellipsoid}{ellipsoid}
\fontfamily{pcr}
\selectfont
\begin{lstlisting}
%  $Author: Guliev Rustam <glvrst@gmail.com> $   $Date: Dec-2012$
%  $Copyright: Moscow State University,
%              Faculty of Computational Mathematics and Cybernetics,
%              Science, System Analysis Department 2012 $
% 
%

\end{lstlisting}
\fontfamily{\familydefault}
\selectfont
\item \hyperlink{/elltoolboxcore/@ellipsoid/ellunion\_ea}{ellunion\_ea}
\fontfamily{pcr}
\selectfont
\begin{lstlisting}
%  $Author: Vadim Kaushanskiy <vkaushanskiy@gmail.com>$ $Date: 10-11-2012$
%  $Copyright: Moscow State University,
%             Faculty of Computational Mathematics and Computer Science,
%             System Analysis Department 2012 $
%

\end{lstlisting}
\fontfamily{\familydefault}
\selectfont
\item \hyperlink{/elltoolboxcore/@ellipsoid/eq}{eq}
\fontfamily{pcr}
\selectfont
\begin{lstlisting}
%  $Author: Vadim Kaushansky  <vkaushanskiy@gmail.com> $    $Date: Nov-2012$
%  $Copyright: Moscow State University,
%             Faculty of Computational Mathematics and Cybernetics,
%             System Analysis Department 2012 $
%  $Author: Peter Gagarinov  <pgagarinov@gmail.com> $    $Date: Dec-2012$
%  $Copyright: Moscow State University,
%             Faculty of Computational Mathematics and Cybernetics,
%             System Analysis Department 2012 $
%

\end{lstlisting}
\fontfamily{\familydefault}
\selectfont
\item \hyperlink{/elltoolboxcore/@ellipsoid/ge}{ge}
\fontfamily{pcr}
\selectfont
\begin{lstlisting}
%  $Author: Alex Kurzhanskiy <akurzhan@eecs.berkeley.edu>
%  $Copyright:  The Regents of the University of California 2004-2008 $
%

\end{lstlisting}
\fontfamily{\familydefault}
\selectfont
\item \hyperlink{/elltoolboxcore/@ellipsoid/getAbsTol}{getAbsTol}
\fontfamily{pcr}
\selectfont
\begin{lstlisting}
%  $Author: Zakharov Eugene  <justenterrr@gmail.com> $
%    $Date: 17-november-2012$
%  $Copyright: Moscow State University,
%             Faculty of Computational Arrhematics and Computer Science,
%             System Analysis Department 2012 $
% 
%

\end{lstlisting}
\fontfamily{\familydefault}
\selectfont
\item \hyperlink{/elltoolboxcore/@ellipsoid/getNPlot2dPoints}{getNPlot2dPoints}
\fontfamily{pcr}
\selectfont
\begin{lstlisting}
%  $Author: Zakharov Eugene  <justenterrr@gmail.com> $ 
%    $Date: 17-november-2012$
%  $Copyright: Moscow State University,
%             Faculty of Computational Arrhematics and Computer Science,
%             System Analysis Department 2012 $
% 
%

\end{lstlisting}
\fontfamily{\familydefault}
\selectfont
\item \hyperlink{/elltoolboxcore/@ellipsoid/getNPlot3dPoints}{getNPlot3dPoints}
\fontfamily{pcr}
\selectfont
\begin{lstlisting}
%  $Author: Zakharov Eugene  <justenterrr@gmail.com> $
%    $Date: 17-november-2012$
%  $Copyright: Moscow State University,
%             Faculty of Computational Arrhematics and Computer Science,
%             System Analysis Department 2012 $
% 
%

\end{lstlisting}
\fontfamily{\familydefault}
\selectfont
\item \hyperlink{/elltoolboxcore/@ellipsoid/getProperty}{getProperty}
\fontfamily{pcr}
\selectfont
\begin{lstlisting}
%

\end{lstlisting}
\fontfamily{\familydefault}
\selectfont
\item \hyperlink{/elltoolboxcore/@ellipsoid/getRelTol}{getRelTol}
\fontfamily{pcr}
\selectfont
\begin{lstlisting}
%  $Author: Zakharov Eugene <justenterrr@gmail.com> $
%    $Date: 17-november-2012$
%  $Copyright: Moscow State University,
%             Faculty of Computational Arrhematics and Computer Science,
%             System Analysis Department 2012 $
% 
%

\end{lstlisting}
\fontfamily{\familydefault}
\selectfont
\item \hyperlink{/elltoolboxcore/@ellipsoid/gt}{gt}
\fontfamily{pcr}
\selectfont
\begin{lstlisting}
%  $Author: Guliev Rustam <glvrst@gmail.com> $   $Date: Dec-2012$
%  $Copyright: Moscow State University,
%              Faculty of Computational Mathematics and Cybernetics,
%              Science, System Analysis Department 2012 $
% 
%

\end{lstlisting}
\fontfamily{\familydefault}
\selectfont
\item \hyperlink{/elltoolboxcore/@ellipsoid/hpintersection}{hpintersection}
\fontfamily{pcr}
\selectfont
\begin{lstlisting}
%  $Author: Guliev Rustam <glvrst@gmail.com> $   $Date: Dec-2012$
%  $Copyright: Moscow State University,
%              Faculty of Computational Mathematics and Cybernetics,
%              Science, System Analysis Department 2012 $
% 
%

\end{lstlisting}
\fontfamily{\familydefault}
\selectfont
\item \hyperlink{/elltoolboxcore/@ellipsoid/intersect}{intersect}
\fontfamily{pcr}
\selectfont
\begin{lstlisting}
%  $Author: <Zakharov Eugene>  <justenterrr@gmail.com> $    $Date: March-2013 $
%  $Copyright: Moscow State University,
%             Faculty of Computational Mathematics and Computer Science,
%             System Analysis Department$
% 
%

\end{lstlisting}
\fontfamily{\familydefault}
\selectfont
\item \hyperlink{/elltoolboxcore/@ellipsoid/intersection\_ea}{intersection\_ea}
\fontfamily{pcr}
\selectfont
\begin{lstlisting}
%  $Author: Guliev Rustam <glvrst@gmail.com> $   $Date: Dec-2012$
%  $Copyright: Moscow State University,
%              Faculty of Computational Mathematics and Cybernetics,
%              Science, System Analysis Department 2012 $
% 
%

\end{lstlisting}
\fontfamily{\familydefault}
\selectfont
\item \hyperlink{/elltoolboxcore/@ellipsoid/intersection\_ia}{intersection\_ia}
\fontfamily{pcr}
\selectfont
\begin{lstlisting}
%  $Author: Guliev Rustam <glvrst@gmail.com> $   $Date: Dec-2012$
%  $Copyright: Moscow State University,
%              Faculty of Computational Mathematics and Cybernetics,
%              Science, System Analysis Department 2012 $
% 
%

\end{lstlisting}
\fontfamily{\familydefault}
\selectfont
\item \hyperlink{/elltoolboxcore/@ellipsoid/inv}{inv}
\fontfamily{pcr}
\selectfont
\begin{lstlisting}
%  $Author: Guliev Rustam <glvrst@gmail.com> $   $Date: Dec-2012$
%  $Copyright: Moscow State University,
%              Faculty of Computational Mathematics and Cybernetics,
%              Science, System Analysis Department 2012 $
% 
%

\end{lstlisting}
\fontfamily{\familydefault}
\selectfont
\item \hyperlink{/elltoolboxcore/@ellipsoid/isbaddirection}{isbaddirection}
\fontfamily{pcr}
\selectfont
\begin{lstlisting}
%  $Author: Guliev Rustam <glvrst@gmail.com> $   $Date: Dec-2012$
%  $Copyright: Moscow State University,
%              Faculty of Computational Mathematics and Cybernetics,
%              Science, System Analysis Department 2012 $
% 
%

\end{lstlisting}
\fontfamily{\familydefault}
\selectfont
\item \hyperlink{/elltoolboxcore/@ellipsoid/isbaddirectionmat}{isbaddirectionmat}
\fontfamily{pcr}
\selectfont
\begin{lstlisting}
%

\end{lstlisting}
\fontfamily{\familydefault}
\selectfont
\item \hyperlink{/elltoolboxcore/@ellipsoid/isbigger}{isbigger}
\fontfamily{pcr}
\selectfont
\begin{lstlisting}
%  $Author: Alex Kurzhanskiy <akurzhan@eecs.berkeley.edu>
%  $Copyright:  The Regents of the University of California 2004-2008 $
%

\end{lstlisting}
\fontfamily{\familydefault}
\selectfont
\item \hyperlink{/elltoolboxcore/@ellipsoid/isdegenerate}{isdegenerate}
\fontfamily{pcr}
\selectfont
\begin{lstlisting}
%  $Author: Guliev Rustam <glvrst@gmail.com> $   $Date: Dec-2012$
%  $Copyright: Moscow State University,
%              Faculty of Computational Mathematics and Cybernetics,
%              Science, System Analysis Department 2012 $
% 
%

\end{lstlisting}
\fontfamily{\familydefault}
\selectfont
\item \hyperlink{/elltoolboxcore/@ellipsoid/isempty}{isempty}
\fontfamily{pcr}
\selectfont
\begin{lstlisting}
%  $Author: Guliev Rustam <glvrst@gmail.com> $   $Date: Dec-2012$
%  $Copyright: Moscow State University,
%              Faculty of Computational Mathematics and Cybernetics,
%              Science, System Analysis Department 2012 $
% 
%

\end{lstlisting}
\fontfamily{\familydefault}
\selectfont
\item \hyperlink{/elltoolboxcore/@ellipsoid/isinside}{isinside}
\fontfamily{pcr}
\selectfont
\begin{lstlisting}
%  $Author: Vadim Kaushanskiy <vkaushanskiy@gmail.com>$ $Date: 10-11-2012$
%  $Copyright: Moscow State University,
%             Faculty of Computational Mathematics and Computer Science,
%             System Analysis Department 2012 $
%

\end{lstlisting}
\fontfamily{\familydefault}
\selectfont
\item \hyperlink{/elltoolboxcore/@ellipsoid/isinternal}{isinternal}
\fontfamily{pcr}
\selectfont
\begin{lstlisting}
%  $Author: Alex Kurzhanskiy <akurzhan@eecs.berkeley.edu>
%  $Copyright:  The Regents of the University of California 2004-2008 $
%

\end{lstlisting}
\fontfamily{\familydefault}
\selectfont
\item \hyperlink{/elltoolboxcore/@ellipsoid/le}{le}
\fontfamily{pcr}
\selectfont
\begin{lstlisting}
%  $Author: Alex Kurzhanskiy <akurzhan@eecs.berkeley.edu>
%  $Copyright:  The Regents of the University of California 2004-2008 $
%

\end{lstlisting}
\fontfamily{\familydefault}
\selectfont
\item \hyperlink{/elltoolboxcore/@ellipsoid/lt}{lt}
\fontfamily{pcr}
\selectfont
\begin{lstlisting}
%  $Author: Alex Kurzhanskiy <akurzhan@eecs.berkeley.edu>
%  $Copyright:  The Regents of the University of California 2004-2008 $
%

\end{lstlisting}
\fontfamily{\familydefault}
\selectfont
\item \hyperlink{/elltoolboxcore/@ellipsoid/maxeig}{maxeig}
\fontfamily{pcr}
\selectfont
\begin{lstlisting}
%  $Author: Guliev Rustam <glvrst@gmail.com> $   $Date: Dec-2012$
%  $Copyright: Moscow State University,
%              Faculty of Computational Mathematics and Cybernetics,
%              Science, System Analysis Department 2012 $
% 
%

\end{lstlisting}
\fontfamily{\familydefault}
\selectfont
\item \hyperlink{/elltoolboxcore/@ellipsoid/mineig}{mineig}
\fontfamily{pcr}
\selectfont
\begin{lstlisting}
%  $Author: Guliev Rustam <glvrst@gmail.com> $   $Date: Dec-2012$
%  $Copyright: Moscow State University,
%              Faculty of Computational Mathematics and Cybernetics,
%              Science, System Analysis Department 2012 $
% 
%

\end{lstlisting}
\fontfamily{\familydefault}
\selectfont
\item \hyperlink{/elltoolboxcore/@ellipsoid/minkdiff}{minkdiff}
\fontfamily{pcr}
\selectfont
\begin{lstlisting}
%  $Author: Alex Kurzhanskiy <akurzhan@eecs.berkeley.edu>
%  $Copyright:  The Regents of the University of California 2004-2008 $
%

\end{lstlisting}
\fontfamily{\familydefault}
\selectfont
\item \hyperlink{/elltoolboxcore/@ellipsoid/minkdiff\_ea}{minkdiff\_ea}
\fontfamily{pcr}
\selectfont
\begin{lstlisting}
%  $Author: Guliev Rustam <glvrst@gmail.com> $   $Date: Dec-2012$
%  $Copyright: Moscow State University,
%              Faculty of Computational Mathematics and Cybernetics,
%              Science, System Analysis Department 2012 $
% 
%

\end{lstlisting}
\fontfamily{\familydefault}
\selectfont
\item \hyperlink{/elltoolboxcore/@ellipsoid/minkdiff\_ia}{minkdiff\_ia}
\fontfamily{pcr}
\selectfont
\begin{lstlisting}
%  $Author: Guliev Rustam <glvrst@gmail.com> $   $Date: Dec-2012$
%  $Copyright: Moscow State University,
%              Faculty of Computational Mathematics and Cybernetics,
%              Science, System Analysis Department 2012 $
% 
%

\end{lstlisting}
\fontfamily{\familydefault}
\selectfont
\item \hyperlink{/elltoolboxcore/@ellipsoid/minkmp}{minkmp}
\fontfamily{pcr}
\selectfont
\begin{lstlisting}
%  $Author: Guliev Rustam <glvrst@gmail.com> $   $Date: Nov-2012$
%  $Copyright: Moscow State University,
%              Faculty of Computational Mathematics and Cybernetics,
%              Science, System Analysis Department 2012 $
% 
%

\end{lstlisting}
\fontfamily{\familydefault}
\selectfont
\item \hyperlink{/elltoolboxcore/@ellipsoid/minkmp\_ea}{minkmp\_ea}
\fontfamily{pcr}
\selectfont
\begin{lstlisting}
%  $Author: Guliev Rustam <glvrst@gmail.com> $   $Date: Dec-2012$
%  $Copyright: Moscow State University,
%              Faculty of Computational Mathematics and Cybernetics,
%              Science, System Analysis Department 2012 $
% 
%

\end{lstlisting}
\fontfamily{\familydefault}
\selectfont
\item \hyperlink{/elltoolboxcore/@ellipsoid/minkmp\_ia}{minkmp\_ia}
\fontfamily{pcr}
\selectfont
\begin{lstlisting}
%  $Author: Guliev Rustam <glvrst@gmail.com> $   $Date: Dec-2012$
%  $Copyright: Moscow State University,
%              Faculty of Computational Mathematics and Cybernetics,
%              Science, System Analysis Department 2012 $
% 
%

\end{lstlisting}
\fontfamily{\familydefault}
\selectfont
\item \hyperlink{/elltoolboxcore/@ellipsoid/minkpm}{minkpm}
\fontfamily{pcr}
\selectfont
\begin{lstlisting}
%  $Author: Guliev Rustam <glvrst@gmail.com> $   $Date: Dec-2012$
%  $Copyright: Moscow State University,
%              Faculty of Computational Mathematics and Cybernetics,
%              Science, System Analysis Department 2012 $
% 
%

\end{lstlisting}
\fontfamily{\familydefault}
\selectfont
\item \hyperlink{/elltoolboxcore/@ellipsoid/minkpm\_ea}{minkpm\_ea}
\fontfamily{pcr}
\selectfont
\begin{lstlisting}
%  $Author: Guliev Rustam <glvrst@gmail.com> $   $Date: Dec-2012$
%  $Copyright: Moscow State University,
%              Faculty of Computational Mathematics and Cybernetics,
%              Science, System Analysis Department 2012 $
% 
%

\end{lstlisting}
\fontfamily{\familydefault}
\selectfont
\item \hyperlink{/elltoolboxcore/@ellipsoid/minkpm\_ia}{minkpm\_ia}
\fontfamily{pcr}
\selectfont
\begin{lstlisting}
%  $Author: Guliev Rustam <glvrst@gmail.com> $   $Date: Dec-2012$
%  $Copyright: Moscow State University,
%              Faculty of Computational Mathematics and Cybernetics,
%              Science, System Analysis Department 2012 $
% 
%

\end{lstlisting}
\fontfamily{\familydefault}
\selectfont
\item \hyperlink{/elltoolboxcore/@ellipsoid/minksum}{minksum}
\fontfamily{pcr}
\selectfont
\begin{lstlisting}
%  $Author: Guliev Rustam <glvrst@gmail.com> $   $Date: Dec-2012$
%  $Copyright: Moscow State University,
%              Faculty of Computational Mathematics and Cybernetics,
%              Science, System Analysis Department 2012 $
% 
%

\end{lstlisting}
\fontfamily{\familydefault}
\selectfont
\item \hyperlink{/elltoolboxcore/@ellipsoid/minksum\_ea}{minksum\_ea}
\fontfamily{pcr}
\selectfont
\begin{lstlisting}
%  $Author: Guliev Rustam <glvrst@gmail.com> $   $Date: Dec-2012$
%  $Copyright: Moscow State University,
%              Faculty of Computational Mathematics and Cybernetics,
%              Science, System Analysis Department 2012 $
% 
%

\end{lstlisting}
\fontfamily{\familydefault}
\selectfont
\item \hyperlink{/elltoolboxcore/@ellipsoid/minksum\_ia}{minksum\_ia}
\fontfamily{pcr}
\selectfont
\begin{lstlisting}
%  $Author: Guliev Rustam <glvrst@gmail.com> $   $Date: Dec-2012$
%  $Copyright: Moscow State University,
%              Faculty of Computational Mathematics and Cybernetics,
%              Science, System Analysis Department 2012 $
% 
%

\end{lstlisting}
\fontfamily{\familydefault}
\selectfont
\item \hyperlink{/elltoolboxcore/@ellipsoid/minus}{minus}
\fontfamily{pcr}
\selectfont
\begin{lstlisting}
%  $Author: Guliev Rustam <glvrst@gmail.com> $   $Date: Dec-2012$
%  $Copyright: Moscow State University,
%              Faculty of Computational Mathematics and Cybernetics,
%              Science, System Analysis Department 2012 $
% 
%

\end{lstlisting}
\fontfamily{\familydefault}
\selectfont
\item \hyperlink{/elltoolboxcore/@ellipsoid/move2origin}{move2origin}
\fontfamily{pcr}
\selectfont
\begin{lstlisting}
%  $Author: Guliev Rustam <glvrst@gmail.com> $   $Date: Dec-2012$
%  $Copyright: Moscow State University,
%              Faculty of Computational Mathematics and Cybernetics,
%              Science, System Analysis Department 2012 $
% 
%

\end{lstlisting}
\fontfamily{\familydefault}
\selectfont
\item \hyperlink{/elltoolboxcore/@ellipsoid/mtimes}{mtimes}
\fontfamily{pcr}
\selectfont
\begin{lstlisting}
%  $Author: Guliev Rustam <glvrst@gmail.com> $   $Date: Dec-2012$
%  $Copyright: Moscow State University,
%              Faculty of Computational Mathematics and Cybernetics,
%              Science, System Analysis Department 2012 $
% 
%

\end{lstlisting}
\fontfamily{\familydefault}
\selectfont
\item \hyperlink{/elltoolboxcore/@ellipsoid/my\_color\_table}{my\_color\_table}
\fontfamily{pcr}
\selectfont
\begin{lstlisting}
%

\end{lstlisting}
\fontfamily{\familydefault}
\selectfont
\item \hyperlink{/elltoolboxcore/@ellipsoid/ne}{ne}
\fontfamily{pcr}
\selectfont
\begin{lstlisting}
%  $Author: Alex Kurzhanskiy <akurzhan@eecs.berkeley.edu>
%  $Copyright:  The Regents of the University of California 2004-2008 $
%

\end{lstlisting}
\fontfamily{\familydefault}
\selectfont
\item \hyperlink{/elltoolboxcore/@ellipsoid/parameters}{parameters}
\fontfamily{pcr}
\selectfont
\begin{lstlisting}
%  $Author: Guliev Rustam <glvrst@gmail.com> $   $Date: Nov-2012$
%  $Copyright: Moscow State University,
%              Faculty of Computational Mathematics and Cybernetics,
%              Science, System Analysis Department 2012 $
% 
%

\end{lstlisting}
\fontfamily{\familydefault}
\selectfont
\item \hyperlink{/elltoolboxcore/@ellipsoid/plot}{plot}
\fontfamily{pcr}
\selectfont
\begin{lstlisting}
%     ELLIPSOID/ELLIPSOID.
% 
%

\end{lstlisting}
\fontfamily{\familydefault}
\selectfont
\item \hyperlink{/elltoolboxcore/@ellipsoid/plot3}{plot3}
\fontfamily{pcr}
\selectfont
\begin{lstlisting}
%     ELLIPSOID/ELLIPSOID, ELLIPSOID/PLOT.
% 
%

\end{lstlisting}
\fontfamily{\familydefault}
\selectfont
\item \hyperlink{/elltoolboxcore/@ellipsoid/plus}{plus}
\fontfamily{pcr}
\selectfont
\begin{lstlisting}
%  $Author: Guliev Rustam <glvrst@gmail.com> $   $Date: Dec-2012$
%  $Copyright: Moscow State University,
%              Faculty of Computational Mathematics and Cybernetics,
%              Science, System Analysis Department 2012 $
% 
%

\end{lstlisting}
\fontfamily{\familydefault}
\selectfont
\item \hyperlink{/elltoolboxcore/@ellipsoid/polar}{polar}
\fontfamily{pcr}
\selectfont
\begin{lstlisting}
%  $Author: Guliev Rustam <glvrst@gmail.com> $   $Date: Dec-2012$
%  $Copyright: Moscow State University,
%              Faculty of Computational Mathematics and Cybernetics,
%              Science, System Analysis Department 2012 $
% 
%

\end{lstlisting}
\fontfamily{\familydefault}
\selectfont
\item \hyperlink{/elltoolboxcore/@ellipsoid/projection}{projection}
\fontfamily{pcr}
\selectfont
\begin{lstlisting}
%  $Author: Guliev Rustam <glvrst@gmail.com> $   $Date: Dec-2012$
%  $Copyright: Moscow State University,
%              Faculty of Computational Mathematics and Cybernetics,
%              Science, System Analysis Department 2012 $
% 
%

\end{lstlisting}
\fontfamily{\familydefault}
\selectfont
\item \hyperlink{/elltoolboxcore/@ellipsoid/regularize}{regularize}
\fontfamily{pcr}
\selectfont
\begin{lstlisting}
%

\end{lstlisting}
\fontfamily{\familydefault}
\selectfont
\item \hyperlink{/elltoolboxcore/@ellipsoid/rho}{rho}
\fontfamily{pcr}
\selectfont
\begin{lstlisting}
%  $Author: Guliev Rustam <glvrst@gmail.com> $   $Date: Dec-2012$
%  $Copyright: Moscow State University,
%              Faculty of Computational Mathematics and Cybernetics,
%              Science, System Analysis Department 2012 $
% 
%

\end{lstlisting}
\fontfamily{\familydefault}
\selectfont
\item \hyperlink{/elltoolboxcore/@ellipsoid/rm\_bad\_directions}{rm\_bad\_directions}
\fontfamily{pcr}
\selectfont
\begin{lstlisting}
%

\end{lstlisting}
\fontfamily{\familydefault}
\selectfont
\item \hyperlink{/elltoolboxcore/@ellipsoid/shape}{shape}
\fontfamily{pcr}
\selectfont
\begin{lstlisting}
%  $Author: Guliev Rustam <glvrst@gmail.com> $   $Date: Dec-2012$
%  $Copyright: Moscow State University,
%              Faculty of Computational Mathematics and Cybernetics,
%              Science, System Analysis Department 2012 $
% 
%

\end{lstlisting}
\fontfamily{\familydefault}
\selectfont
\item \hyperlink{/elltoolboxcore/@ellipsoid/trace}{trace}
\fontfamily{pcr}
\selectfont
\begin{lstlisting}
%  $Author: Guliev Rustam <glvrst@gmail.com> $   $Date: Dec-2012$
%  $Copyright: Moscow State University,
%              Faculty of Computational Mathematics and Cybernetics,
%              Science, System Analysis Department 2012 $
% 
%

\end{lstlisting}
\fontfamily{\familydefault}
\selectfont
\item \hyperlink{/elltoolboxcore/@ellipsoid/uminus}{uminus}
\fontfamily{pcr}
\selectfont
\begin{lstlisting}
%  $Author: Guliev Rustam <glvrst@gmail.com> $   $Date: Dec-2012$
%  $Copyright: Moscow State University,
%              Faculty of Computational Mathematics and Cybernetics,
%              Science, System Analysis Department 2012 $
% 
%

\end{lstlisting}
\fontfamily{\familydefault}
\selectfont
\item \hyperlink{/elltoolboxcore/@ellipsoid/volume}{volume}
\fontfamily{pcr}
\selectfont
\begin{lstlisting}
%  $Author: Guliev Rustam <glvrst@gmail.com> $   $Date: Dec-2012$
%  $Copyright: Moscow State University,
%              Faculty of Computational Mathematics and Cybernetics,
%              Science, System Analysis Department 2012 $
% s
%

\end{lstlisting}
\fontfamily{\familydefault}
\selectfont
\end{enumerate}
\subsection{/elltoolboxcore/@hyperplane}
\begin{enumerate}
\item \hyperlink{/elltoolboxcore/@hyperplane/checkIsMe}{checkIsMe}
\fontfamily{pcr}
\selectfont
\begin{lstlisting}
%  $Author: Aushkap Nikolay <n.aushkap@gmail.com> $  $Date: 30-11-2012$
%  $Copyright: Moscow State University,
%    Faculty of Computational Mathematics and Computer Science,
%    System Analysis Department 2012 $
%

\end{lstlisting}
\fontfamily{\familydefault}
\selectfont
\item \hyperlink{/elltoolboxcore/@hyperplane/contains}{contains}
\fontfamily{pcr}
\selectfont
\begin{lstlisting}
%  $Authors:
%    Zakharov Eugene <justenterrr@gmail.com>$ $Date: <31 october>$
%    Aushkap Nikolay <n.aushkap@gmail.com> $  $Date: 30-11-2012$
%  $Copyright: Moscow State University,
%    Faculty of Computational Mathematics and Computer Science,
%    System Analysis Department 2012 $
%

\end{lstlisting}
\fontfamily{\familydefault}
\selectfont
\item \hyperlink{/elltoolboxcore/@hyperplane/contents}{contents}
\fontfamily{pcr}
\selectfont
\begin{lstlisting}
%  Author:
%  -------
%     Alex Kurzhanskiy <akurzhan@eecs.berkeley.edu>
% 
%

\end{lstlisting}
\fontfamily{\familydefault}
\selectfont
\item \hyperlink{/elltoolboxcore/@hyperplane/dimension}{dimension}
\fontfamily{pcr}
\selectfont
\begin{lstlisting}
%  $Author: Aushkap Nikolay <n.aushkap@gmail.com> $  $Date: 30-11-2012$
%  $Copyright: Moscow State University,
%    Faculty of Computational Mathematics and Computer Science,
%    System Analysis Department 2012 $
%

\end{lstlisting}
\fontfamily{\familydefault}
\selectfont
\item \hyperlink{/elltoolboxcore/@hyperplane/display}{display}
\fontfamily{pcr}
\selectfont
\begin{lstlisting}
%  $Author: Aushkap Nikolay <n.aushkap@gmail.com> $  $Date: 07-12-2012$
%  $Copyright: Moscow State University,
%    Faculty of Computational Mathematics and Computer Science,
%    System Analysis Department 2012 $
%

\end{lstlisting}
\fontfamily{\familydefault}
\selectfont
\item \hyperlink{/elltoolboxcore/@hyperplane/double}{double}
\fontfamily{pcr}
\selectfont
\begin{lstlisting}
%  $Author: Aushkap Nikolay <n.aushkap@gmail.com> $  $Date: 30-11-2012$
%  $Copyright: Moscow State University,
%    Faculty of Computational Mathematics and Computer Science,
%    System Analysis Department 2012 $
%

\end{lstlisting}
\fontfamily{\familydefault}
\selectfont
\item \hyperlink{/elltoolboxcore/@hyperplane/eq}{eq}
\fontfamily{pcr}
\selectfont
\begin{lstlisting}
%  $Authors:
%    Peter Gagarinov  <pgagarinov@gmail.com> $ $Date: Dec-2012$
%    Aushkap Nikolay <n.aushkap@gmail.com> $ $Date: Dec-2012$
%  $Copyright: Moscow State University,
%    Faculty of Computational Mathematics and Computer Science,
%    System Analysis Department 2012 $
%

\end{lstlisting}
\fontfamily{\familydefault}
\selectfont
\item \hyperlink{/elltoolboxcore/@hyperplane/getAbsTol}{getAbsTol}
\fontfamily{pcr}
\selectfont
\begin{lstlisting}
%  $Author: Zakharov Eugene <justenterrr@gmail.com>$ $Date: 17-11-2012$
%  $Copyright: Moscow State University,
%    Faculty of Computational Mathematics and Computer Science,
%    System Analysis Department 2012 $
%  
%

\end{lstlisting}
\fontfamily{\familydefault}
\selectfont
\item \hyperlink{/elltoolboxcore/@hyperplane/hyperplane}{hyperplane}
\fontfamily{pcr}
\selectfont
\begin{lstlisting}
%  $Author: Aushkap Nikolay <n.aushkap@gmail.com> $
%    $Date: 30-11-2012$
%  $Copyright: Moscow State University,
%    Faculty of Computational Mathematics and Computer
%    Science, System Analysis Department 2012 $
%

\end{lstlisting}
\fontfamily{\familydefault}
\selectfont
\item \hyperlink{/elltoolboxcore/@hyperplane/isempty}{isempty}
\fontfamily{pcr}
\selectfont
\begin{lstlisting}
%  $Author: Aushkap Nikolay <n.aushkap@gmail.com> $  $Date: 30-11-2012$
%  $Copyright: Moscow State University,
%    Faculty of Computational Mathematics and Computer Science,
%    System Analysis Department 2012 $
%

\end{lstlisting}
\fontfamily{\familydefault}
\selectfont
\item \hyperlink{/elltoolboxcore/@hyperplane/isparallel}{isparallel}
\fontfamily{pcr}
\selectfont
\begin{lstlisting}
%  $Author: Aushkap Nikolay <n.aushkap@gmail.com> $  $Date: 30-11-2012$
%  $Copyright: Moscow State University,
%    Faculty of Computational Mathematics and Computer Science,
%    System Analysis Department 2012 $
%

\end{lstlisting}
\fontfamily{\familydefault}
\selectfont
\item \hyperlink{/elltoolboxcore/@hyperplane/ne}{ne}
\fontfamily{pcr}
\selectfont
\begin{lstlisting}
%  $Authors:
%    Peter Gagarinov  <pgagarinov@gmail.com> $ $Date: Dec-2012$
%    Aushkap Nikolay <n.aushkap@gmail.com> $ $Date: Dec-2012$
%  $Copyright: Moscow State University,
%    Faculty of Computational Mathematics and Computer Science,
%    System Analysis Department 2012 $
%

\end{lstlisting}
\fontfamily{\familydefault}
\selectfont
\item \hyperlink{/elltoolboxcore/@hyperplane/parameters}{parameters}
\fontfamily{pcr}
\selectfont
\begin{lstlisting}
%  $Author: Aushkap Nikolay <n.aushkap@gmail.com> $  $Date: 30-11-2012$
%  $Copyright: Moscow State University,
%    Faculty of Computational Mathematics and Computer Science,
%    System Analysis Department 2012 $
%

\end{lstlisting}
\fontfamily{\familydefault}
\selectfont
\item \hyperlink{/elltoolboxcore/@hyperplane/plot}{plot}
\fontfamily{pcr}
\selectfont
\begin{lstlisting}
%  $Author: <Zakharov Eugene>  <justenterrr@gmail.com> $    $Date: <1 november> $
%  $Copyright: Moscow State University,
%             Faculty of Computational Mathematics and Computer Science,
%             System Analysis Department <2012> $
%

\end{lstlisting}
\fontfamily{\familydefault}
\selectfont
\item \hyperlink{/elltoolboxcore/@hyperplane/uminus}{uminus}
\fontfamily{pcr}
\selectfont
\begin{lstlisting}
%  $Author: Aushkap Nikolay <n.aushkap@gmail.com> $  $Date: 30-11-2012$
%  $Copyright: Moscow State University,
%    Faculty of Computational Mathematics and Computer Science,
%    System Analysis Department 2012 $
%

\end{lstlisting}
\fontfamily{\familydefault}
\selectfont
\end{enumerate}
\subsection{/elltoolboxcore/auxiliary}
\begin{enumerate}
\item \hyperlink{/elltoolboxcore/auxiliary/ell\_enclose}{ell\_enclose}
\fontfamily{pcr}
\selectfont
\begin{lstlisting}
%     ELLIPSOID/ISINTERNAL, ELLUNION_EA;
%     POLYTOPE/getOutterEllipsoid.
% 
%

\end{lstlisting}
\fontfamily{\familydefault}
\selectfont
\item \hyperlink{/elltoolboxcore/auxiliary/ell\_fusionlambda}{ell\_fusionlambda}
\fontfamily{pcr}
\selectfont
\begin{lstlisting}
%  This function is called from ELLIPSOID/INTERSECTION_EA by FZERO.
% 
%

\end{lstlisting}
\fontfamily{\familydefault}
\selectfont
\item \hyperlink{/elltoolboxcore/auxiliary/ell\_inv}{ell\_inv}
\fontfamily{pcr}
\selectfont
\begin{lstlisting}
%     INV, COND.
% 
%

\end{lstlisting}
\fontfamily{\familydefault}
\selectfont
\item \hyperlink{/elltoolboxcore/auxiliary/ell\_simdiag}{ell\_simdiag}
\fontfamily{pcr}
\selectfont
\begin{lstlisting}
%     SVD, GSVD.
% 
%

\end{lstlisting}
\fontfamily{\familydefault}
\selectfont
\item \hyperlink{/elltoolboxcore/auxiliary/ell\_unitball}{ell\_unitball}
\fontfamily{pcr}
\selectfont
\begin{lstlisting}
%     ELLIPSOID/ELLIPSOID
% 
%

\end{lstlisting}
\fontfamily{\familydefault}
\selectfont
\item \hyperlink{/elltoolboxcore/auxiliary/ell\_valign}{ell\_valign}
\fontfamily{pcr}
\selectfont
\begin{lstlisting}
%     SVD, GSVD.
% 
%

\end{lstlisting}
\fontfamily{\familydefault}
\selectfont
\item \hyperlink{/elltoolboxcore/auxiliary/hyperplane2polytope}{hyperplane2polytope}
\fontfamily{pcr}
\selectfont
\begin{lstlisting}
%     HYPERPLANE/HYPERPLANE, POLYTOPE/POLYTOPE, POLYTOPE2HYPERPLANE.
% 
%

\end{lstlisting}
\fontfamily{\familydefault}
\selectfont
\item \hyperlink{/elltoolboxcore/auxiliary/polytope2hyperplane}{polytope2hyperplane}
\fontfamily{pcr}
\selectfont
\begin{lstlisting}
%     POLYTOPE/POLYTOPE, HYPERPLANE/HYPERPLANE, HYPERPLANE2POLYTOPE.
% 
%

\end{lstlisting}
\fontfamily{\familydefault}
\selectfont
\end{enumerate}
\subsection{/elltoolboxcore/control/auxiliary}
\begin{enumerate}
\item \hyperlink{/elltoolboxcore/control/auxiliary/ell\_center\_ode}{ell\_center\_ode}
\fontfamily{pcr}
\selectfont
\begin{lstlisting}
%  ELL_CENTER_ODE - ODE for the center of the reach set.
% 
%

\end{lstlisting}
\fontfamily{\familydefault}
\selectfont
\item \hyperlink{/elltoolboxcore/control/auxiliary/ell\_eedist\_ode}{ell\_eedist\_ode}
\fontfamily{pcr}
\selectfont
\begin{lstlisting}
%  ELL_EEDIST_ODE - ODE for the shape matrix of the external ellipsoid
%                   for system with disturbance.
% 
%

\end{lstlisting}
\fontfamily{\familydefault}
\selectfont
\item \hyperlink{/elltoolboxcore/control/auxiliary/ell\_eesm\_ode}{ell\_eesm\_ode}
\fontfamily{pcr}
\selectfont
\begin{lstlisting}
%  ELL_EESM_ODE - ODE for the shape matrix of the external ellipsoid.
% 
%

\end{lstlisting}
\fontfamily{\familydefault}
\selectfont
\item \hyperlink{/elltoolboxcore/control/auxiliary/ell\_iedist\_ode}{ell\_iedist\_ode}
\fontfamily{pcr}
\selectfont
\begin{lstlisting}
%  ELL_IEDIST_ODE - ODE for the shape matrix of the internal ellipsoid
%                   for system with disturbance.
% 
%

\end{lstlisting}
\fontfamily{\familydefault}
\selectfont
\item \hyperlink{/elltoolboxcore/control/auxiliary/ell\_iesm\_ode}{ell\_iesm\_ode}
\fontfamily{pcr}
\selectfont
\begin{lstlisting}
%  ELL_IESM_ODE - ODE for the shape matrix of the internal ellipsoid.
% 
%

\end{lstlisting}
\fontfamily{\familydefault}
\selectfont
\item \hyperlink{/elltoolboxcore/control/auxiliary/ell\_ode\_solver}{ell\_ode\_solver}
\fontfamily{pcr}
\selectfont
\begin{lstlisting}
%  ELL_ODE_SOLVER - caller for particular ODE solver.
% 
%

\end{lstlisting}
\fontfamily{\familydefault}
\selectfont
\item \hyperlink{/elltoolboxcore/control/auxiliary/ell\_regularize}{ell\_regularize}
\fontfamily{pcr}
\selectfont
\begin{lstlisting}
%  ELL_REGULARIZE - regularization of singular matrix.
% 
%

\end{lstlisting}
\fontfamily{\familydefault}
\selectfont
\item \hyperlink{/elltoolboxcore/control/auxiliary/ell\_stm\_ode}{ell\_stm\_ode}
\fontfamily{pcr}
\selectfont
\begin{lstlisting}
%  ELL_STM_ODE - ODE for state transition matrix.
% 
%

\end{lstlisting}
\fontfamily{\familydefault}
\selectfont
\item \hyperlink{/elltoolboxcore/control/auxiliary/ell\_value\_extract}{ell\_value\_extract}
\fontfamily{pcr}
\selectfont
\begin{lstlisting}
%  ELL_VALUE_EXTRACT - extracts matrix value from ppform or vector array.
% 
%

\end{lstlisting}
\fontfamily{\familydefault}
\selectfont
\item \hyperlink{/elltoolboxcore/control/auxiliary/iesm\_ode}{iesm\_ode}
\fontfamily{pcr}
\selectfont
\begin{lstlisting}
%  ELL_IESM_ODE - ODE for the shape matrix of the internal ellipsoid.
% 
%

\end{lstlisting}
\fontfamily{\familydefault}
\selectfont
\end{enumerate}
\subsection{/elltoolboxcore/demo}
\begin{enumerate}
\item \hyperlink{/elltoolboxcore/demo/ell\_demo0}{ell\_demo0}
\fontfamily{pcr}
\selectfont
\begin{lstlisting}
%  Alex Kurzhanskiy <akurzhan@eecs.berkeley.edu>
% 
%

\end{lstlisting}
\fontfamily{\familydefault}
\selectfont
\item \hyperlink{/elltoolboxcore/demo/ell\_demo1}{ell\_demo1}
\fontfamily{pcr}
\selectfont
\begin{lstlisting}
%  Alex Kurzhanskiy <akurzhan@eecs.berkeley.edu>
% 
%

\end{lstlisting}
\fontfamily{\familydefault}
\selectfont
\item \hyperlink{/elltoolboxcore/demo/ell\_demo2}{ell\_demo2}
\fontfamily{pcr}
\selectfont
\begin{lstlisting}
%  Alex Kurzhanskiy <akurzhan@eecs.berkeley.edu>
% 
%

\end{lstlisting}
\fontfamily{\familydefault}
\selectfont
\item \hyperlink{/elltoolboxcore/demo/ell\_demo3}{ell\_demo3}
\fontfamily{pcr}
\selectfont
\begin{lstlisting}
%  Reachability Demo.
% 
%

\end{lstlisting}
\fontfamily{\familydefault}
\selectfont
\end{enumerate}
\subsection{/elltoolboxcore/graphics}
\begin{enumerate}
\item \hyperlink{/elltoolboxcore/graphics/ell\_plot}{ell\_plot}
\fontfamily{pcr}
\selectfont
\begin{lstlisting}
%     PLOT, PLOT3.
% 
%

\end{lstlisting}
\fontfamily{\familydefault}
\selectfont
\item \hyperlink{/elltoolboxcore/graphics/ell\_square\_facets}{ell\_square\_facets}
\fontfamily{pcr}
\selectfont
\begin{lstlisting}
%     PATCH, CONVHULLN.
% 
%

\end{lstlisting}
\fontfamily{\familydefault}
\selectfont
\item \hyperlink{/elltoolboxcore/graphics/ell\_triag\_facets}{ell\_triag\_facets}
\fontfamily{pcr}
\selectfont
\begin{lstlisting}
%     PATCH, CONVHULLN.
% 
%

\end{lstlisting}
\fontfamily{\familydefault}
\selectfont
\end{enumerate}
\subsection{/elltoolboxcore/solvers/gradient}
\begin{enumerate}
\item \hyperlink{/elltoolboxcore/solvers/gradient/ell\_nlfnlc}{ell\_nlfnlc}
\fontfamily{pcr}
\selectfont
\begin{lstlisting}
%     YALMIP, SEDUMI.
% 
%

\end{lstlisting}
\fontfamily{\familydefault}
\selectfont
\end{enumerate}
\subsection{/elltoolboxcore/solvers/gradient/private}
\begin{enumerate}
\item \hyperlink{/elltoolboxcore/solvers/gradient/private/compute\_direction}{compute\_direction}
\fontfamily{pcr}
\selectfont
\begin{lstlisting}
%  COMPUTE_DIRECTION - computes a search direction in a subspace defined by Z.
% 
%

\end{lstlisting}
\fontfamily{\familydefault}
\selectfont
\item \hyperlink{/elltoolboxcore/solvers/gradient/private/nlcp\_solve}{nlcp\_solve}
\fontfamily{pcr}
\selectfont
\begin{lstlisting}
%  NLCP_SOLVE - nonlinear function minimizer under nonlinear constraints.
% 
%

\end{lstlisting}
\fontfamily{\familydefault}
\selectfont
\item \hyperlink{/elltoolboxcore/solvers/gradient/private/qps}{qps}
\fontfamily{pcr}
\selectfont
\begin{lstlisting}
%             min 0.5*x'Hx + f'x   subject to:  Ax <= b 
%              x    
% 
%

\end{lstlisting}
\fontfamily{\familydefault}
\selectfont
\end{enumerate}
\section{Help list}
\subsection{/+elltool}
\begin{enumerate}
\item\hypertarget{/+elltool\copyconf}{copyconf}
\fontfamily{pcr}
\selectfont
\begin{lstlisting}
% COPYCONF copies configuration confName to configuration toConfName
% 
% Input:
%    regular:
%        confName:char[1,] - name of copied configuration
%        toConfName:char[1,] - name of new copy
% 
% $Author: Zakharov Eugene  <justenterrr@gmail.com> $    $Date: 17-november-2012 $
% $Copyright: Moscow State University,
%             Faculty of Computational Mathematics and Computer Science,
%             System Analysis Department 2012 $
% 
%

\end{lstlisting}
\fontfamily{\familydefault}
\selectfont
\item\hypertarget{/+elltool\editconf}{editconf}
\fontfamily{pcr}
\selectfont
\begin{lstlisting}
% EDITCONF edit configuration confName
% 
% Input:
%    regular:
%        confName:char[1,] - name of configuration to edit
% 
% $Author: Zakharov Eugene  <justenterrr@gmail.com> $    $Date: 17-november-2012 $
% $Copyright: Moscow State University,
%             Faculty of Computational Mathematics and Computer Science,
%             System Analysis Department 2012 $
% 
%

\end{lstlisting}
\fontfamily{\familydefault}
\selectfont
\item\hypertarget{/+elltool\listconf}{listconf}
\fontfamily{pcr}
\selectfont
\begin{lstlisting}
% LISTCONF gives a list of existing configurations
% 
% No input or output, just displays list of configuration
% 
% $Author: Zakharov Eugene  <justenterrr@gmail.com> $    $Date: 17-november-2012 $
% $Copyright: Moscow State University,
%             Faculty of Computational Mathematics and Computer Science,
%             System Analysis Department 2012 $
% 
%

\end{lstlisting}
\fontfamily{\familydefault}
\selectfont
\item\hypertarget{/+elltool\setconf}{setconf}
\fontfamily{pcr}
\selectfont
\begin{lstlisting}
% SETCONF selects the configuration confName as current
% 
% Input:
%    regular:
%        confName:char[1,] - name of configuration to set
% 
% $Author: Zakharov Eugene  <justenterrr@gmail.com> $    $Date: 17-november-2012 $
% $Copyright: Moscow State University,
%             Faculty of Computational Mathematics and Computer Science,
%             System Analysis Department 2012 $
% 
%

\end{lstlisting}
\fontfamily{\familydefault}
\selectfont
\end{enumerate}
\subsection{/+elltool/+conf}
\begin{enumerate}
\item\hypertarget{/+elltool/+conf\ConfRepoMgr}{ConfRepoMgr}
\fontfamily{pcr}
\selectfont
\begin{lstlisting}
% CONREPOMGR is analogue for elltool.test.configuration.AdaptiveConfRepoManager
% constructed to provide access for elltool.conf.Properties class to 
% local xml files, where information about properties is stored.
% 
%  $Author: <Zakharov Eugene>  <justenterrr@gmail.com> $    $Date: <5 november> $
%  $Copyright: Moscow State University,
%             Faculty of Computational Mathematics and Computer Science,
%             System Analysis Department <2012> $
% 
%

\end{lstlisting}
\fontfamily{\familydefault}
\selectfont
\end{enumerate}
\subsection{/+elltool/+conf/+test}
\begin{enumerate}
\item\hypertarget{/+elltool/+conf/+test\run\_tests}{run\_tests}
\fontfamily{pcr}
\selectfont
\begin{lstlisting}
%

\end{lstlisting}
\fontfamily{\familydefault}
\selectfont
\end{enumerate}
\subsection{/+elltool/+conf/+test/+mlunit}
\begin{enumerate}
\item\hypertarget{/+elltool/+conf/+test/+mlunit\PropertiesTestCase}{PropertiesTestCase}
\fontfamily{pcr}
\selectfont
\begin{lstlisting}
%  $Author: <Zakharov Eugene>  <justenterrr@gmail.com> $    $Date: <5 november> $
%  $Copyright: Moscow State University,
%             Faculty of Computational Mathematics and Computer Science,
%             System Analysis Department <2012> $
% 
%

\end{lstlisting}
\fontfamily{\familydefault}
\selectfont
\end{enumerate}
\subsection{/+elltool/+conf/@ConfPatchRepo}
\begin{enumerate}
\item\hypertarget{/+elltool/+conf/@ConfPatchRepo\ConfPatchRepo}{ConfPatchRepo}
\fontfamily{pcr}
\selectfont
\begin{lstlisting}
%

\end{lstlisting}
\fontfamily{\familydefault}
\selectfont
\item\hypertarget{/+elltool/+conf/@ConfPatchRepo\patch\_001\_dummy\_patch}{patch\_001\_dummy\_patch}
\fontfamily{pcr}
\selectfont
\begin{lstlisting}
%

\end{lstlisting}
\fontfamily{\familydefault}
\selectfont
\item\hypertarget{/+elltool/+conf/@ConfPatchRepo\patch\_002\_add\_log4j}{patch\_002\_add\_log4j}
\fontfamily{pcr}
\selectfont
\begin{lstlisting}
%

\end{lstlisting}
\fontfamily{\familydefault}
\selectfont
\end{enumerate}
\subsection{/+elltool/+conf/@Properties}
\begin{enumerate}
\item\hypertarget{/+elltool/+conf/@Properties\Properties}{Properties}
\fontfamily{pcr}
\selectfont
\begin{lstlisting}
% PROPERTIES is a static class, providing emulation of static properties
% for toolbox.
% 
% $Author: Zakharov Eugene  <justenterrr@gmail.com> $    $Date: 5-november-2012 $
% $Author: Peter Gagarinov  <pgagarinov@gmail.com> $    $Date: 25-november-2012 $    
% $Copyright: Moscow State University,
%             Faculty of Computational Mathematics and Computer Science,
%             System Analysis Department 2012 $
% 
%

\end{lstlisting}
\fontfamily{\familydefault}
\selectfont
\item\hypertarget{/+elltool/+conf/@Properties\parseProp}{parseProp}
\fontfamily{pcr}
\selectfont
\begin{lstlisting}
% PARSEPROP parses input into cell array with values of properties listed in
% neededPropNameList. Values are taken from args or, if there no value for
% some property in args, in current Properties.
% 
%  Input:
%    regular:
%        args:cell[1,] - cell array of arguments that should be parsed.
%        neededPropNameList:cell[1,] or empty cell - cell array of strings, containing
%            names of parameters, that output should consist of. Possible
%            properties:
%                version
%                isVerbose
%                absTol
%                relTol
%                nTimeGridPoints
%                ODESolverName
%                isODENormControl
%                isEnabledOdeSolverOptions
%                nPlot2dPoints
%                nPlot3dPoints
%            trying to specify other properties would be regarded as an
%            error.
% 
%  Output:
%    varargout:cell array[1,] - cell array with values of properties listed
%                               in neededPropNameList in the same order as they
%                               listed in neededPropNameList
% 
% $Author: Zakharov Eugene  <justenterrr@gmail.com> $    $Date: 5-november-2012 $
% $Copyright: Moscow State University,
%             Faculty of Computational Mathematics and Computer Science,
%             System Analysis Department 2012 $
% 
%

\end{lstlisting}
\fontfamily{\familydefault}
\selectfont
\end{enumerate}
\subsection{/+elltool/+core/+test}
\begin{enumerate}
\item\hypertarget{/+elltool/+core/+test\run\_tests}{run\_tests}
\fontfamily{pcr}
\selectfont
\begin{lstlisting}
%

\end{lstlisting}
\fontfamily{\familydefault}
\selectfont
\end{enumerate}
\subsection{/+elltool/+core/+test/+mlunit}
\begin{enumerate}
\item\hypertarget{/+elltool/+core/+test/+mlunit\EllSecTCMultiDim}{EllSecTCMultiDim}
\fontfamily{pcr}
\selectfont
\begin{lstlisting}
%  $Author: Igor Samokhin, Lomonosov Moscow State University,
%  Faculty of Computational Mathematics and Cybernetics, System Analysis
%  Department, 31-January-2013, <igorian.vmk@gmail.com>$
%  $Copyright: Moscow State University,
%             Faculty of Computational Mathematics and Computer Science,
%             System Analysis Department 2012 $
%

\end{lstlisting}
\fontfamily{\familydefault}
\selectfont
\item\hypertarget{/+elltool/+core/+test/+mlunit\EllTCMultiDim}{EllTCMultiDim}
\fontfamily{pcr}
\selectfont
\begin{lstlisting}
%  $Author: Igor Samokhin, Lomonosov Moscow State University,
%  Faculty of Computational Mathematics and Cybernetics, System Analysis
%  Department, 31-January-2013, <igorian.vmk@gmail.com>$
%  $Copyright: Moscow State University,
%             Faculty of Computational Mathematics and Computer Science,
%             System Analysis Department 2013 $
%

\end{lstlisting}
\fontfamily{\familydefault}
\selectfont
\item\hypertarget{/+elltool/+core/+test/+mlunit\ElliIntUnionTCMultiDim}{ElliIntUnionTCMultiDim}
\fontfamily{pcr}
\selectfont
\begin{lstlisting}
%  $Author: Igor Samokhin, Lomonosov Moscow State University,
%  Faculty of Computational Mathematics and Cybernetics, System Analysis
%  Department, 31-January-2013, <igorian.vmk@gmail.com>$
%  $Copyright: Moscow State University,
%             Faculty of Computational Mathematics and Computer Science,
%             System Analysis Department 2013 $
%

\end{lstlisting}
\fontfamily{\familydefault}
\selectfont
\item\hypertarget{/+elltool/+core/+test/+mlunit\EllipsoidIntUnionTC}{EllipsoidIntUnionTC}
\fontfamily{pcr}
\selectfont
\begin{lstlisting}
%  $Author: Vadim Kaushanskiy, Moscow State University by M.V. Lomonosov,
%  Faculty of Computational Mathematics and Cybernetics, System Analysis
%  Department, 24-December-2012, <vkaushanskiy@gmail.com>$
%

\end{lstlisting}
\fontfamily{\familydefault}
\selectfont
\item\hypertarget{/+elltool/+core/+test/+mlunit\EllipsoidSecTestCase}{EllipsoidSecTestCase}
\fontfamily{pcr}
\selectfont
\begin{lstlisting}
%  $Author: Igor Samokhin, Lomonosov Moscow State University,
%  Faculty of Computational Mathematics and Cybernetics, System Analysis
%  Department, 02-November-2012, <igorian.vmk@gmail.com>$
%  $Copyright: Moscow State University,
%             Faculty of Computational Mathematics and Computer Science,
%             System Analysis Department 2012 $
%

\end{lstlisting}
\fontfamily{\familydefault}
\selectfont
\item\hypertarget{/+elltool/+core/+test/+mlunit\EllipsoidTestCase}{EllipsoidTestCase}
\fontfamily{pcr}
\selectfont
\begin{lstlisting}
%

\end{lstlisting}
\fontfamily{\familydefault}
\selectfont
\item\hypertarget{/+elltool/+core/+test/+mlunit\GenEllipsoidPlotTestCase}{GenEllipsoidPlotTestCase}
\fontfamily{pcr}
\selectfont
\begin{lstlisting}
%

\end{lstlisting}
\fontfamily{\familydefault}
\selectfont
\item\hypertarget{/+elltool/+core/+test/+mlunit\GenEllipsoidTestCase}{GenEllipsoidTestCase}
\fontfamily{pcr}
\selectfont
\begin{lstlisting}
%

\end{lstlisting}
\fontfamily{\familydefault}
\selectfont
\item\hypertarget{/+elltool/+core/+test/+mlunit\HyperplaneTestCase}{HyperplaneTestCase}
\fontfamily{pcr}
\selectfont
\begin{lstlisting}
%  $Author: <Zakharov Eugene>  <justenterrr@gmail.com> $    $Date: <31 october> $
%  $Copyright: Moscow State University,
%             Faculty of Computational Mathematics and Computer Science,
%             System Analysis Department <2012> $
% 
%

\end{lstlisting}
\fontfamily{\familydefault}
\selectfont
\end{enumerate}
\subsection{/+elltool/+core/@GenEllipsoid}
\begin{enumerate}
\item\hypertarget{/+elltool/+core/@GenEllipsoid\GenEllipsoid}{GenEllipsoid}
\fontfamily{pcr}
\selectfont
\begin{lstlisting}
%  GENELLIPSOID - class of generalized ellipsoids
% 
%  Input:
%    Case1:
%      regular:
%        qVec: double[nDim,1] - ellipsoid center
%        qMat: double[nDim,nDim] / qVec: double[nDim,1] - ellipsoid matrix
%            or diagonal vector of eigenvalues, that may contain infinite
%            or zero elements
% 
%    Case2:
%      regular:
%        qMat: double[nDim,nDim] / qVec: double[nDim,1] - diagonal matrix or
%            vector, may contain infinite or zero elements
% 
%    Case3:
%      regular:
%        qVec: double[nDim,1] - ellipsoid center
%        dMat: double[nDim,nDim] / dVec: double[nDim,1] - diagonal matrix or
%            vector, may contain infinite or zero elements
%        wMat: double[nDim,nDim] - any square matrix
% 
% 
%  Output:
%    self: GenEllipsoid[1,1] - created generalized ellipsoid
% 
%  $Author: Vitaly Baranov  <vetbar42@gmail.com> $    $Date: Nov-2012$
%  $Copyright: Moscow State University,
%             Faculty of Computational Mathematics and Cybernetics,
%             System Analysis Department 2012 $
% 
%

\end{lstlisting}
\fontfamily{\familydefault}
\selectfont
\item\hypertarget{/+elltool/+core/@GenEllipsoid\checkBigger}{checkBigger}
\fontfamily{pcr}
\selectfont
\begin{lstlisting}
%

\end{lstlisting}
\fontfamily{\familydefault}
\selectfont
\item\hypertarget{/+elltool/+core/@GenEllipsoid\dimension}{dimension}
\fontfamily{pcr}
\selectfont
\begin{lstlisting}
%

\end{lstlisting}
\fontfamily{\familydefault}
\selectfont
\item\hypertarget{/+elltool/+core/@GenEllipsoid\eq}{eq}
\fontfamily{pcr}
\selectfont
\begin{lstlisting}
%  EQ - compares two arrays of ellipsoids
% 
%  Input:
%    regular:
%        ellFirstArr: ellipsoid: [nDims1,nDims2,...,nDimsN]/[1,1]- the first
%            array of ellipsoid objects
%        ellSecArr: ellipsoid: [nDims1,nDims2,...,nDimsN]/[1,1] - the second
%            array of ellipsoid objects
% 
%  Output:
%    isEqualArr: logical: [nDims1,nDims2,...,nDimsN]- array of comparison
%        results
% 
%    reportStr: char[1,] - comparison report
% 
%  $Author: Peter Gagarinov  <pgagarinov@gmail.com> $    $Date: Dec-2012$
%  $Copyright: Moscow State University,
%             Faculty of Computational Mathematics and Cybernetics,
%             System Analysis Department 2012 $
% 
%

\end{lstlisting}
\fontfamily{\familydefault}
\selectfont
\item\hypertarget{/+elltool/+core/@GenEllipsoid\findAllInfDir}{findAllInfDir}
\fontfamily{pcr}
\selectfont
\begin{lstlisting}
%

\end{lstlisting}
\fontfamily{\familydefault}
\selectfont
\item\hypertarget{/+elltool/+core/@GenEllipsoid\findBasRank}{findBasRank}
\fontfamily{pcr}
\selectfont
\begin{lstlisting}
%

\end{lstlisting}
\fontfamily{\familydefault}
\selectfont
\item\hypertarget{/+elltool/+core/@GenEllipsoid\findConstruction}{findConstruction}
\fontfamily{pcr}
\selectfont
\begin{lstlisting}
%

\end{lstlisting}
\fontfamily{\familydefault}
\selectfont
\item\hypertarget{/+elltool/+core/@GenEllipsoid\findDiffEaND}{findDiffEaND}
\fontfamily{pcr}
\selectfont
\begin{lstlisting}
%

\end{lstlisting}
\fontfamily{\familydefault}
\selectfont
\item\hypertarget{/+elltool/+core/@GenEllipsoid\findDiffFC}{findDiffFC}
\fontfamily{pcr}
\selectfont
\begin{lstlisting}
%

\end{lstlisting}
\fontfamily{\familydefault}
\selectfont
\item\hypertarget{/+elltool/+core/@GenEllipsoid\findDiffINFC}{findDiffINFC}
\fontfamily{pcr}
\selectfont
\begin{lstlisting}
%

\end{lstlisting}
\fontfamily{\familydefault}
\selectfont
\item\hypertarget{/+elltool/+core/@GenEllipsoid\findDiffIaND}{findDiffIaND}
\fontfamily{pcr}
\selectfont
\begin{lstlisting}
%

\end{lstlisting}
\fontfamily{\familydefault}
\selectfont
\item\hypertarget{/+elltool/+core/@GenEllipsoid\findMatProj}{findMatProj}
\fontfamily{pcr}
\selectfont
\begin{lstlisting}
%

\end{lstlisting}
\fontfamily{\familydefault}
\selectfont
\item\hypertarget{/+elltool/+core/@GenEllipsoid\findSpaceBas}{findSpaceBas}
\fontfamily{pcr}
\selectfont
\begin{lstlisting}
%

\end{lstlisting}
\fontfamily{\familydefault}
\selectfont
\item\hypertarget{/+elltool/+core/@GenEllipsoid\findSqrtOfMatrix}{findSqrtOfMatrix}
\fontfamily{pcr}
\selectfont
\begin{lstlisting}
%

\end{lstlisting}
\fontfamily{\familydefault}
\selectfont
\item\hypertarget{/+elltool/+core/@GenEllipsoid\getColorTable}{getColorTable}
\fontfamily{pcr}
\selectfont
\begin{lstlisting}
%

\end{lstlisting}
\fontfamily{\familydefault}
\selectfont
\item\hypertarget{/+elltool/+core/@GenEllipsoid\getIsGoodDirForMat}{getIsGoodDirForMat}
\fontfamily{pcr}
\selectfont
\begin{lstlisting}
%

\end{lstlisting}
\fontfamily{\familydefault}
\selectfont
\item\hypertarget{/+elltool/+core/@GenEllipsoid\inv}{inv}
\fontfamily{pcr}
\selectfont
\begin{lstlisting}
%  INV - create generalized ellipsoid whose matrix in pseudoinverse
%  to the matrix of input generalized ellipsoid
% 
%  Input:
%    regular:
%        ellObj: GenEllipsoid: [1,1] - generalized ellipsoid
% 
%  Output:
%    ellInvObj: GenEllipsoid: [1,1] - inverse generalized ellipsoid
% 
%  $Author: Vitaly Baranov  <vetbar42@gmail.com> $    $Date: Nov-2012$
%  $Copyright: Moscow State University,
%             Faculty of Computational Mathematics and Cybernetics,
%             System Analysis Department 2012 $
% 
% 
%

\end{lstlisting}
\fontfamily{\familydefault}
\selectfont
\item\hypertarget{/+elltool/+core/@GenEllipsoid\minkDiffEa}{minkDiffEa}
\fontfamily{pcr}
\selectfont
\begin{lstlisting}
%  MINKDIFFEA - computes tight external ellipsoidal approximation for
%  Minkowsky difference of two generalized ellipsoids
% 
%  Input:
%    regular:
%        ellObj1: GenEllipsoid: [1,1] - first generalized ellipsoid
%        ellObj2: GenEllipsoid: [1,1] - second generalized ellipsoid
%        dirMat: double[nDim,nDir] - matrix whose columns specify
%            directions for which approximations should be computed
%  Output:
%    resEllVec: GenEllipsoid[1,nDir] - vector of generalized ellipsoids of
%        external approximation of the dirrence of first and second generalized
%        ellipsoids (may contain empty ellipsoids if in specified
%        directions approximation cannot be computed)
% 
% 
%  $Author: Vitaly Baranov  <vetbar42@gmail.com> $    $Date: 2012-11$
%  $Copyright: Moscow State University,
%             Faculty of Computational Mathematics and Cybernetics,
%             System Analysis Department 2012 $
% 
% 
%

\end{lstlisting}
\fontfamily{\familydefault}
\selectfont
\item\hypertarget{/+elltool/+core/@GenEllipsoid\minkDiffIa}{minkDiffIa}
\fontfamily{pcr}
\selectfont
\begin{lstlisting}
%  MINKDIFFIA - computes tight internal ellipsoidal approximation for
%  Minkowsky difference of two generalized ellipsoids
% 
%  Input:
%    regular:
%        ellObj1: GenEllipsoid: [1,1] - first generalized ellipsoid
%        ellObj2: GenEllipsoid: [1,1] - second generalized ellipsoid
%        dirMat: double[nDim,nDir] - matrix whose columns specify
%            directions for which approximations should be computed
%  Output:
%    resEllVec: GenEllipsoid[1,nDir] - vector of generalized ellipsoids of
%        internal approximation of the dirrence of first and second generalized
%        ellipsoids
% 
% 
%  $Author: Vitaly Baranov  <vetbar42@gmail.com> $    $Date: 2012-11$
%  $Copyright: Moscow State University,
%             Faculty of Computational Mathematics and Cybernetics,
%             System Analysis Department 2012 $
% 
%  Bibliography:
%  V.V.Shiryaev, 'About internal ellipsoidal approximations of attainability
%  sets of linear systems under uncertanty'. Moscow University Vestnik,
%  Ser.15, Computational mathematics and cybernetics, 2012, N3, p. 20-27.
% 
%

\end{lstlisting}
\fontfamily{\familydefault}
\selectfont
\item\hypertarget{/+elltool/+core/@GenEllipsoid\minkSumEa}{minkSumEa}
\fontfamily{pcr}
\selectfont
\begin{lstlisting}
%  MINKSUMEA - computes tight external ellipsoidal approximation for
%  Minkowsky sum of the set of generalized ellipsoids
% 
%  Input:
%    regular:
%        ellObjVec: GenEllipsoid: [kSize,mSize] - vector of  generalized
%                                            ellipsoid
%        dirMat: double[nDim,nDir] - matrix whose columns specify
%            directions for which approximations should be computed
%  Output:
%    ellResVec: GenEllipsoid[1,nDir] - vector of generalized ellipsoids of
%        external approximation of the dirrence of first and second generalized
%        ellipsoids
% 
% 
%  $Author: Vitaly Baranov  <vetbar42@gmail.com> $    $Date: 2012-11$
%  $Copyright: Moscow State University,
%             Faculty of Computational Mathematics and Cybernetics,
%             System Analysis Department 2012 $
% 
% 
%

\end{lstlisting}
\fontfamily{\familydefault}
\selectfont
\item\hypertarget{/+elltool/+core/@GenEllipsoid\minkSumIa}{minkSumIa}
\fontfamily{pcr}
\selectfont
\begin{lstlisting}
%  MINKSUMIA - computes tight internal ellipsoidal approximation for
%  Minkowsky sum of the set of generalized ellipsoids
% 
%  Input:
%    regular:
%        ellObjVec: GenEllipsoid: [kSize,mSize] - vector of  generalized
%                                            ellipsoid
%        dirMat: double[nDim,nDir] - matrix whose columns specify
%            directions for which approximations should be computed
%  Output:
%    ellResVec: GenEllipsoid[1,nDir] - vector of generalized ellipsoids of
%        internal approximation of the dirrence of first and second
%        generalized ellipsoids
% 
%  $Author: Vitaly Baranov  <vetbar42@gmail.com> $    $Date: 2012-11$
%  $Copyright: Moscow State University,
%             Faculty of Computational Mathematics and Cybernetics,
%             System Analysis Department 2012 $
% 
% 
%

\end{lstlisting}
\fontfamily{\familydefault}
\selectfont
\item\hypertarget{/+elltool/+core/@GenEllipsoid\plot}{plot}
\fontfamily{pcr}
\selectfont
\begin{lstlisting}
%  PLOT - plots ellipsoids in 2D or 3D.
% 
% 
%  Usage:
%        plot(ell) - plots generic ellipsoid ell in default (red) color.
%        plot(ellArr) - plots an array of generic ellipsoids.
%        plot(ellArr, 'Property',PropValue,...) - plots ellArr with setting
%                                                 properties.
% 
%  Input:
%    regular:
%        ellArr:  elltool.core.GenEllipsoid: [dim11Size,dim12Size,...,dim1kSize] -
%                 array of 2D or 3D GenEllipsoids objects. All ellipsoids in ellArr
%                 must be either 2D or 3D simutaneously.
%    optional:
%        color1Spec: char[1,1] - color specification code, can be 'r','g',
%                                etc (any code supported by built-in Matlab function).
%        ell2Arr: elltool.core.GenEllipsoid: [dim21Size,dim22Size,...,dim2kSize] -
%                                            second ellipsoid array...
%        color2Spec: char[1,1] - same as color1Spec but for ell2Arr
%        ....
%        ellNArr: elltool.core.GenEllipsoid: [dimN1Size,dim22Size,...,dimNkSize] -
%                                             N-th ellipsoid array
%        colorNSpec - same as color1Spec but for ellNArr.
%    properties:
%        'newFigure': logical[1,1] - if 1, each plot command will open a new figure window.
%                     Default value is 0.
%        'fill': logical[1,1]/logical[dim11Size,dim12Size,...,dim1kSize]  -
%                if 1, ellipsoids in 2D will be filled with color. Default value is 0.
%        'lineWidth': double[1,1]/double[dim11Size,dim12Size,...,dim1kSize]  -
%                     line width for 1D and 2D plots. Default value is 1.
%        'color': double[1,3]/double[dim11Size,dim12Size,...,dim1kSize,3] -
%                 sets default colors in the form [x y z]. Default value is [1 0 0].
%        'shade': double[1,1]/double[dim11Size,dim12Size,...,dim1kSize]  -
%                 level of transparency between 0 and 1 (0 - transparent, 1 - opaque).
%                 Default value is 0.4.
%        'relDataPlotter' - relation data plotter object.
%        Notice that property vector could have different dimensions, only
%        total number of elements must be the same.
%  Output:
%    regular:
%        plObj: smartdb.disp.RelationDataPlotter[1,1] - returns the relation
%        data plotter object.
% 
%  Examples:
%        plot([ell1, ell2, ell3], 'color', [1, 0, 1; 0, 0, 1; 1, 0, 0]);
%        plot([ell1, ell2, ell3], 'color', [1; 0; 1; 0; 0; 1; 1; 0; 0]);
%        plot([ell1, ell2, ell3; ell1, ell2, ell3], 'shade', [1, 1, 1; 1, 1,
%        1]);
%        plot([ell1, ell2, ell3; ell1, ell2, ell3], 'shade', [1; 1; 1; 1; 1;
%        1]);
%        plot([ell1, ell2, ell3], 'shade', 0.5);
%        plot([ell1, ell2, ell3], 'lineWidth', 1.5);
%        plot([ell1, ell2, ell3], 'lineWidth', [1.5, 0.5, 3]);
%

\end{lstlisting}
\fontfamily{\familydefault}
\selectfont
\item\hypertarget{/+elltool/+core/@GenEllipsoid\rho}{rho}
\fontfamily{pcr}
\selectfont
\begin{lstlisting}
%

\end{lstlisting}
\fontfamily{\familydefault}
\selectfont
\end{enumerate}
\subsection{/+elltool/+cvx}
\begin{enumerate}
\item\hypertarget{/+elltool/+cvx\CVXController}{CVXController}
\fontfamily{pcr}
\selectfont
\begin{lstlisting}
%

\end{lstlisting}
\fontfamily{\familydefault}
\selectfont
\end{enumerate}
\subsection{/+elltool/+demo/+test}
\begin{enumerate}
\item\hypertarget{/+elltool/+demo/+test\run\_tests}{run\_tests}
\fontfamily{pcr}
\selectfont
\begin{lstlisting}
%

\end{lstlisting}
\fontfamily{\familydefault}
\selectfont
\end{enumerate}
\subsection{/+elltool/+demo/+test/+mlunit}
\begin{enumerate}
\item\hypertarget{/+elltool/+demo/+test/+mlunit\BasicTestCase}{BasicTestCase}
\fontfamily{pcr}
\selectfont
\begin{lstlisting}
%

\end{lstlisting}
\fontfamily{\familydefault}
\selectfont
\item\hypertarget{/+elltool/+demo/+test/+mlunit\ETManualTC}{ETManualTC}
\fontfamily{pcr}
\selectfont
\begin{lstlisting}
%

\end{lstlisting}
\fontfamily{\familydefault}
\selectfont
\end{enumerate}
\subsection{/+elltool/+doc}
\begin{enumerate}
\item\hypertarget{/+elltool/+doc\collecthelp}{collecthelp}
\fontfamily{pcr}
\selectfont
\begin{lstlisting}
%  COLLECTHELP collects helps of m files in given directory
% 
%  Usage: FuncData=collecthelp(dirName,varargin)
% 
%  Input:
%    regular
%        dirName: string - the path contained functions data;
%    properties:
%        ignorDirList: cell[1,nIgnor] of strings -
%            list of ignored dir names (by default is empty),
%        isClass: logic[1,1] - is current dir class or not,
%            by default false.
%        scriptNamePattern: string - regular expression
%            by default equals 's_\w+\.m'
%  Output:
%    FuncData: struct[1,1] with the following fields
%        funcName: cell[nElems,1] - list of function names
%        dirName: cell[nElems,1] - list of directory names
%        help: cell[nElems,1] - list of help headers
%        isClassMethod: logical[nElems,1] - a vector of "is class"
%            indicators
%        isScript: logical[nElems,1] - a vector of "is script" indicators
% 
% 
%  $Author: Peter Gagarinov  <pgagarinov@gmail.com> $	$Date: 2013-04-01 $
%  $Copyright: Moscow State University,
%             Faculty of Computational Mathematics and Computer Science,
%             System Analysis Department 2013 $
%

\end{lstlisting}
\fontfamily{\familydefault}
\selectfont
\item\hypertarget{/+elltool/+doc\run\_helpcollector}{run\_helpcollector}
\fontfamily{pcr}
\selectfont
\begin{lstlisting}
%

\end{lstlisting}
\fontfamily{\familydefault}
\selectfont
\end{enumerate}
\subsection{/+elltool/+linsys}
\begin{enumerate}
\item\hypertarget{/+elltool/+linsys\LinSys}{LinSys}
\fontfamily{pcr}
\selectfont
\begin{lstlisting}
%  Linear system object of the Ellipsoidal Toolbox.
% 
%  
%   LinSys         - Constructor of linear system object.
%   dimension      - Returns state space dimension, number of inputs, number of
%                    outputs and number of disturbance inputs.
%   isempty        - Checks if the linear system object is empty.
%   isdiscrete     - Returns 1 if linear system is discrete-time,
%                    0 - if continuous-time.
%   islti          - Returns 1 if the system is time-invariant, 0 - otherwise.
%   hasdisturbance - Returns 1 if unknown bounded disturbance is present,
%                    0 - if there is no disturbance, or disturbance vector is fixed.
%   hasnoise       - Returns 1 if unknown bounded noise at the output is present,
%                    0 - if there is no noise, or noise vector is fixed.
% 
% 
%  $Authors: Alex Kurzhanskiy <akurzhan@eecs.berkeley.edu>
%            Ivan Menshikov  <ivan.v.menshikov@gmail.com> $    $Date: 2012 $
%            Kirill Mayantsev  <kirill.mayantsev@gmail.com> $  $Date: March-2012 $
%  $Copyright: Moscow State University,
%             Faculty of Computational Mathematics and Computer Science,
%             System Analysis Department 2012 $
% 
%

\end{lstlisting}
\fontfamily{\familydefault}
\selectfont
\end{enumerate}
\subsection{/+elltool/+linsys/+test}
\begin{enumerate}
\item\hypertarget{/+elltool/+linsys/+test\run\_tests}{run\_tests}
\fontfamily{pcr}
\selectfont
\begin{lstlisting}
%

\end{lstlisting}
\fontfamily{\familydefault}
\selectfont
\end{enumerate}
\subsection{/+elltool/+linsys/+test/+mlunit}
\begin{enumerate}
\item\hypertarget{/+elltool/+linsys/+test/+mlunit\LinSysTestCase}{LinSysTestCase}
\fontfamily{pcr}
\selectfont
\begin{lstlisting}
%

\end{lstlisting}
\fontfamily{\familydefault}
\selectfont
\end{enumerate}
\subsection{/+elltool/+logging}
\begin{enumerate}
\item\hypertarget{/+elltool/+logging\Log4jConfigurator}{Log4jConfigurator}
\fontfamily{pcr}
\selectfont
\begin{lstlisting}
% LOG4JCONFIGURATOR simplifies log4j configuration, especially when
% Parallel Computing Toolbox is used. In the latter case the class forwards
% the logs of different processees in separate log files
% 
%  $Author: Peter Gagarinov  <pgagarinov@gmail.com> $	$Date: 2011-05-18 $ 
%  $Copyright: Moscow State University,
%             Faculty of Computational Mathematics and Computer Science,
%             System Analysis Department 2011 $
%

\end{lstlisting}
\fontfamily{\familydefault}
\selectfont
\end{enumerate}
\subsection{/+elltool/+reach}
\begin{enumerate}
\item\hypertarget{/+elltool/+reach\AReach}{AReach}
\fontfamily{pcr}
\selectfont
\begin{lstlisting}
%  $Author: Kirill Mayantsev  <kirill.mayantsev@gmail.com> $  $Date: March-2012 $
%  $Copyright: Moscow State University,
%             Faculty of Computational Mathematics and Computer Science,
%             System Analysis Department 2012 $
% 
%

\end{lstlisting}
\fontfamily{\familydefault}
\selectfont
\item\hypertarget{/+elltool/+reach\IReach}{IReach}
\fontfamily{pcr}
\selectfont
\begin{lstlisting}
%  $Author: Kirill Mayantsev  <kirill.mayantsev@gmail.com> $  $Date: March-2012 $
%  $Copyright: Moscow State University,
%             Faculty of Computational Mathematics and Computer Science,
%             System Analysis Department 2012 $
% 
%

\end{lstlisting}
\fontfamily{\familydefault}
\selectfont
\item\hypertarget{/+elltool/+reach\ReachContinuous}{ReachContinuous}
\fontfamily{pcr}
\selectfont
\begin{lstlisting}
%  Continuous reach set library of the Ellipsoidal Toolbox.
% 
%  
%  Constructor and data accessing functions:
%  -----------------------------------------
%   ReachContinuous - Constructor of the reach set object, performs the 
%                     computation of the specified reach set approximations.
%   dimension       - Returns the dimension of the reach set, which can be
%                     different from the state space dimension of the system
%                     if the reach set is a projection.
%   get_system      - Returns the linear system object, for which the reach set
%                     was computed.
%                     Warning: returns the last lin system.
%   get_directions  - Returns the values of the direction vectors corresponding
%                     to the values of the time grid.
%   get_center      - Returns points of the reach set center trajectory
%                     corresponding to the values of the time grid.
%   get_ea          - Returns external approximating ellipsoids corresponding
%                     to the values of the time grid.
%   get_ia          - Returns internal approximating ellipsoids corresponding
%                     to the values of the time grid.
%   get_goodcurves  - Returns points of the 'good curves' corresponding
%                     to the values of the time grid.
%   intersect       - Checks if external or internal reach set approximation
%                     intersects with given ellipsoid, hyperplane or polytope.
%   iscut           - Checks if given reach set object is a cut of another reach set.
%   isprojection    - Checks if given reach set object is a projection.
%   
% 
%  Reach set data manipulation and plotting functions:
%  ---------------------------------------------------
%   cut        - Extracts a piece of the reach set that corresponds to the
%                specified time value or time interval.
%   projection - Projects the reach set onto a given orthogonal basis.
%   evolve     - Computes further evolution in time for given reach set
%                for the same or different dynamical system.
%   plot_ea    - Plots external approximation of the reach set.
%   plot_ia    - Plots internal approximation of the reach set.
% 
% 
%  Overloaded functions:
%  ---------------------
%   display - Displays the reach set object.
%             Warning: displays only the last linear system.
%   
%     
%  $Authors: Alex Kurzhanskiy <akurzhan@eecs.berkeley.edu>
%            Kirill Mayantsev  <kirill.mayantsev@gmail.com> $  $Date: March-2012 $
%  $Copyright: Moscow State University,
%             Faculty of Computational Mathematics and Computer Science,
%             System Analysis Department 2012 $
%

\end{lstlisting}
\fontfamily{\familydefault}
\selectfont
\item\hypertarget{/+elltool/+reach\ReachDiscrete}{ReachDiscrete}
\fontfamily{pcr}
\selectfont
\begin{lstlisting}
%  Discrete reach set library of the Ellipsoidal Toolbox.
% 
%  
%  Constructor and data accessing functions:
%  -----------------------------------------
%   ReachDiscrete  - Constructor of the reach set object, performs the 
%                    computation of the specified reach set approximations.
%   dimension      - Returns the dimension of the reach set, which can be
%                    different from the state space dimension of the system
%                    if the reach set is a projection.
%   get_system     - Returns the linear system object, for which the reach set
%                    was computed.
%   get_directions - Returns the values of the direction vectors corresponding
%                    to the values of the time grid.
%   get_center     - Returns points of the reach set center trajectory
%                    corresponding to the values of the time grid.
%   get_ea         - Returns external approximating ellipsoids corresponding
%                    to the values of the time grid.
%   get_ia         - Returns internal approximating ellipsoids corresponding
%                    to the values of the time grid.
%   get_goodcurves - Returns points of the 'good curves' corresponding
%                    to the values of the time grid.
%                    This function does not work with projections.
%   intersect      - Checks if external or internal reach set approximation
%                    intersects with given ellipsoid, hyperplane or polytope.
%   iscut          - Checks if given reach set object is a cut of another reach set.
%   isprojection   - Checks if given reach set object is a projection.
%   
% 
%  Reach set data manipulation and plotting functions:
%  ---------------------------------------------------
%   cut        - Extracts a piece of the reach set that corresponds to the
%                specified time value or time interval.
%   projection - Projects the reach set onto a given orthogonal basis.
%   evolve     - Computes further evolution in time for given reach set
%                for the same or different dynamical system.
%   plot_ea    - Plots external approximation of the reach set.
%   plot_ia    - Plots internal approximation of the reach set.
% 
% 
%  Overloaded functions:
%  ---------------------
%   display - Displays the reach set object.
%   
% 
%  $Authors: Alex Kurzhanskiy <akurzhan@eecs.berkeley.edu>
%            Kirill Mayantsev  <kirill.mayantsev@gmail.com> $  $Date: March-2012 $
%  $Copyright: Moscow State University,
%             Faculty of Computational Mathematics and Computer Science,
%             System Analysis Department 2012 $
%

\end{lstlisting}
\fontfamily{\familydefault}
\selectfont
\end{enumerate}
\subsection{/+elltool/+reach/+test}
\begin{enumerate}
\item\hypertarget{/+elltool/+reach/+test\run\_continuous\_reach\_tests}{run\_continuous\_reach\_tests}
\fontfamily{pcr}
\selectfont
\begin{lstlisting}
%

\end{lstlisting}
\fontfamily{\familydefault}
\selectfont
\item\hypertarget{/+elltool/+reach/+test\run\_discrete\_reach\_tests}{run\_discrete\_reach\_tests}
\fontfamily{pcr}
\selectfont
\begin{lstlisting}
%

\end{lstlisting}
\fontfamily{\familydefault}
\selectfont
\end{enumerate}
\subsection{/+elltool/+reach/+test/+mlunit}
\begin{enumerate}
\item\hypertarget{/+elltool/+reach/+test/+mlunit\ContinuousReachFirstTestCase}{ContinuousReachFirstTestCase}
\fontfamily{pcr}
\selectfont
\begin{lstlisting}
%

\end{lstlisting}
\fontfamily{\familydefault}
\selectfont
\item\hypertarget{/+elltool/+reach/+test/+mlunit\ContinuousReachRegrTestCase}{ContinuousReachRegrTestCase}
\fontfamily{pcr}
\selectfont
\begin{lstlisting}
%

\end{lstlisting}
\fontfamily{\familydefault}
\selectfont
\item\hypertarget{/+elltool/+reach/+test/+mlunit\ContinuousReachTestCase}{ContinuousReachTestCase}
\fontfamily{pcr}
\selectfont
\begin{lstlisting}
%

\end{lstlisting}
\fontfamily{\familydefault}
\selectfont
\item\hypertarget{/+elltool/+reach/+test/+mlunit\DiscreteReachTestCase}{DiscreteReachTestCase}
\fontfamily{pcr}
\selectfont
\begin{lstlisting}
%

\end{lstlisting}
\fontfamily{\familydefault}
\selectfont
\item\hypertarget{/+elltool/+reach/+test/+mlunit\ReachFactory}{ReachFactory}
\fontfamily{pcr}
\selectfont
\begin{lstlisting}
%

\end{lstlisting}
\fontfamily{\familydefault}
\selectfont
\end{enumerate}
\subsection{/+elltool/+test}
\begin{enumerate}
\item\hypertarget{/+elltool/+test\TmpDataManager}{TmpDataManager}
\fontfamily{pcr}
\selectfont
\begin{lstlisting}
%  TMPDATAMANAGER provides a basic functionality for managing temporary
%  data folders, root folder name is determined automatically
% 
%

\end{lstlisting}
\fontfamily{\familydefault}
\selectfont
\item\hypertarget{/+elltool/+test\copyconf}{copyconf}
\fontfamily{pcr}
\selectfont
\begin{lstlisting}
%

\end{lstlisting}
\fontfamily{\familydefault}
\selectfont
\item\hypertarget{/+elltool/+test\editconf}{editconf}
\fontfamily{pcr}
\selectfont
\begin{lstlisting}
%

\end{lstlisting}
\fontfamily{\familydefault}
\selectfont
\item\hypertarget{/+elltool/+test\listconf}{listconf}
\fontfamily{pcr}
\selectfont
\begin{lstlisting}
%

\end{lstlisting}
\fontfamily{\familydefault}
\selectfont
\item\hypertarget{/+elltool/+test\run\_tests}{run\_tests}
\fontfamily{pcr}
\selectfont
\begin{lstlisting}
%

\end{lstlisting}
\fontfamily{\familydefault}
\selectfont
\item\hypertarget{/+elltool/+test\run\_tests\_remotely}{run\_tests\_remotely}
\fontfamily{pcr}
\selectfont
\begin{lstlisting}
%

\end{lstlisting}
\fontfamily{\familydefault}
\selectfont
\end{enumerate}
\subsection{/+elltool/+test/+configuration}
\begin{enumerate}
\item\hypertarget{/+elltool/+test/+configuration\AdaptiveConfRepoManager}{AdaptiveConfRepoManager}
\fontfamily{pcr}
\selectfont
\begin{lstlisting}
%  ADAPTIVECONFREPOMANAGER is a simplistic extension of
%  AdaptiveConfRepoManager that injects a configuration change
%  repository class equivolent.test.configuration.ConfPatchRepo
%  automatically
% 
% 
%  $Author: Peter Gagarinov <pgagarinov@gmail.com> $	$Date: 2011-05-18 $ 
%  $Copyright: Moscow State University,
%             Faculty of Computational Mathematics and Computer Science,
%             System Analysis Department 2011 $
% 
%

\end{lstlisting}
\fontfamily{\familydefault}
\selectfont
\end{enumerate}
\subsection{/+elltool/+test/+configuration/@ConfPatchRepo}
\begin{enumerate}
\item\hypertarget{/+elltool/+test/+configuration/@ConfPatchRepo\ConfPatchRepo}{ConfPatchRepo}
\fontfamily{pcr}
\selectfont
\begin{lstlisting}
%

\end{lstlisting}
\fontfamily{\familydefault}
\selectfont
\item\hypertarget{/+elltool/+test/+configuration/@ConfPatchRepo\patch\_001\_dummy\_patch}{patch\_001\_dummy\_patch}
\fontfamily{pcr}
\selectfont
\begin{lstlisting}
%

\end{lstlisting}
\fontfamily{\familydefault}
\selectfont
\end{enumerate}
\subsection{/+elltool/+test/+logging}
\begin{enumerate}
\item\hypertarget{/+elltool/+test/+logging\Log4jConfigurator}{Log4jConfigurator}
\fontfamily{pcr}
\selectfont
\begin{lstlisting}
% LOG4JCONFIGURATOR simplifies log4j configuration, especially when
% Parallel Computing Toolbox is used. In the latter case the class forwards
% the logs of different processees in separate log files
% 
%  $Author: Peter Gagarinov  <pgagarinov@gmail.com> $	$Date: 2011-05-18 $ 
%  $Copyright: Moscow State University,
%             Faculty of Computational Mathematics and Computer Science,
%             System Analysis Department 2011 $
%

\end{lstlisting}
\fontfamily{\familydefault}
\selectfont
\end{enumerate}
\subsection{/+gras/+ellapx/+enums}
\begin{enumerate}
\item\hypertarget{/+gras/+ellapx/+enums\EApproxType}{EApproxType}
\fontfamily{pcr}
\selectfont
\begin{lstlisting}
% APXTYPE Summary of this class goes here
%

\end{lstlisting}
\fontfamily{\familydefault}
\selectfont
\item\hypertarget{/+gras/+ellapx/+enums\EEllUnionTimeDirection}{EEllUnionTimeDirection}
\fontfamily{pcr}
\selectfont
\begin{lstlisting}
% APXTYPE Summary of this class goes here
%

\end{lstlisting}
\fontfamily{\familydefault}
\selectfont
\item\hypertarget{/+gras/+ellapx/+enums\EProjType}{EProjType}
\fontfamily{pcr}
\selectfont
\begin{lstlisting}
%

\end{lstlisting}
\fontfamily{\familydefault}
\selectfont
\end{enumerate}
\subsection{/+gras/+ellapx/+gen}
\begin{enumerate}
\item\hypertarget{/+gras/+ellapx/+gen\ATightEllApxBuilder}{ATightEllApxBuilder}
\fontfamily{pcr}
\selectfont
\begin{lstlisting}
%

\end{lstlisting}
\fontfamily{\familydefault}
\selectfont
\item\hypertarget{/+gras/+ellapx/+gen\EllApxCollectionBuilder}{EllApxCollectionBuilder}
\fontfamily{pcr}
\selectfont
\begin{lstlisting}
%

\end{lstlisting}
\fontfamily{\familydefault}
\selectfont
\item\hypertarget{/+gras/+ellapx/+gen\IEllApxBuilder}{IEllApxBuilder}
\fontfamily{pcr}
\selectfont
\begin{lstlisting}
%

\end{lstlisting}
\fontfamily{\familydefault}
\selectfont
\end{enumerate}
\subsection{/+gras/+ellapx/+lreachplain}
\begin{enumerate}
\item\hypertarget{/+gras/+ellapx/+lreachplain\ATightEllApxBuilder}{ATightEllApxBuilder}
\fontfamily{pcr}
\selectfont
\begin{lstlisting}
%

\end{lstlisting}
\fontfamily{\familydefault}
\selectfont
\item\hypertarget{/+gras/+ellapx/+lreachplain\ATightIntEllApxBuilder}{ATightIntEllApxBuilder}
\fontfamily{pcr}
\selectfont
\begin{lstlisting}
%

\end{lstlisting}
\fontfamily{\familydefault}
\selectfont
\item\hypertarget{/+gras/+ellapx/+lreachplain\EllTubeDynamicSpaceProjector}{EllTubeDynamicSpaceProjector}
\fontfamily{pcr}
\selectfont
\begin{lstlisting}
% IELLTUBEPROJECTOR Summary of this class goes here
%    Detailed explanation goes here
%

\end{lstlisting}
\fontfamily{\familydefault}
\selectfont
\item\hypertarget{/+gras/+ellapx/+lreachplain\ExtEllApxBuilder}{ExtEllApxBuilder}
\fontfamily{pcr}
\selectfont
\begin{lstlisting}
%

\end{lstlisting}
\fontfamily{\familydefault}
\selectfont
\item\hypertarget{/+gras/+ellapx/+lreachplain\GoodDirectionSet}{GoodDirectionSet}
\fontfamily{pcr}
\selectfont
\begin{lstlisting}
%

\end{lstlisting}
\fontfamily{\familydefault}
\selectfont
\item\hypertarget{/+gras/+ellapx/+lreachplain\IntEllApxBuilder}{IntEllApxBuilder}
\fontfamily{pcr}
\selectfont
\begin{lstlisting}
%

\end{lstlisting}
\fontfamily{\familydefault}
\selectfont
\item\hypertarget{/+gras/+ellapx/+lreachplain\IntProperEllApxBuilder}{IntProperEllApxBuilder}
\fontfamily{pcr}
\selectfont
\begin{lstlisting}
%

\end{lstlisting}
\fontfamily{\familydefault}
\selectfont
\end{enumerate}
\subsection{/+gras/+ellapx/+lreachplain/+probdef}
\begin{enumerate}
\item\hypertarget{/+gras/+ellapx/+lreachplain/+probdef\AReachContProblemDef}{AReachContProblemDef}
\fontfamily{pcr}
\selectfont
\begin{lstlisting}
%

\end{lstlisting}
\fontfamily{\familydefault}
\selectfont
\item\hypertarget{/+gras/+ellapx/+lreachplain/+probdef\IReachContProblemDef}{IReachContProblemDef}
\fontfamily{pcr}
\selectfont
\begin{lstlisting}
%

\end{lstlisting}
\fontfamily{\familydefault}
\selectfont
\item\hypertarget{/+gras/+ellapx/+lreachplain/+probdef\LReachContProblemDef}{LReachContProblemDef}
\fontfamily{pcr}
\selectfont
\begin{lstlisting}
%

\end{lstlisting}
\fontfamily{\familydefault}
\selectfont
\item\hypertarget{/+gras/+ellapx/+lreachplain/+probdef\ReachContLTIProblemDef}{ReachContLTIProblemDef}
\fontfamily{pcr}
\selectfont
\begin{lstlisting}
%

\end{lstlisting}
\fontfamily{\familydefault}
\selectfont
\end{enumerate}
\subsection{/+gras/+ellapx/+lreachplain/+probdyn}
\begin{enumerate}
\item\hypertarget{/+gras/+ellapx/+lreachplain/+probdyn\AReachProblemDynamics}{AReachProblemDynamics}
\fontfamily{pcr}
\selectfont
\begin{lstlisting}
%

\end{lstlisting}
\fontfamily{\familydefault}
\selectfont
\item\hypertarget{/+gras/+ellapx/+lreachplain/+probdyn\AReachProblemDynamicsInterp}{AReachProblemDynamicsInterp}
\fontfamily{pcr}
\selectfont
\begin{lstlisting}
%

\end{lstlisting}
\fontfamily{\familydefault}
\selectfont
\item\hypertarget{/+gras/+ellapx/+lreachplain/+probdyn\AReachProblemLTIDynamics}{AReachProblemLTIDynamics}
\fontfamily{pcr}
\selectfont
\begin{lstlisting}
%

\end{lstlisting}
\fontfamily{\familydefault}
\selectfont
\item\hypertarget{/+gras/+ellapx/+lreachplain/+probdyn\IReachProblemDynamics}{IReachProblemDynamics}
\fontfamily{pcr}
\selectfont
\begin{lstlisting}
%

\end{lstlisting}
\fontfamily{\familydefault}
\selectfont
\item\hypertarget{/+gras/+ellapx/+lreachplain/+probdyn\LReachProblemDynamicsFactory}{LReachProblemDynamicsFactory}
\fontfamily{pcr}
\selectfont
\begin{lstlisting}
%

\end{lstlisting}
\fontfamily{\familydefault}
\selectfont
\item\hypertarget{/+gras/+ellapx/+lreachplain/+probdyn\LReachProblemDynamicsInterp}{LReachProblemDynamicsInterp}
\fontfamily{pcr}
\selectfont
\begin{lstlisting}
%

\end{lstlisting}
\fontfamily{\familydefault}
\selectfont
\item\hypertarget{/+gras/+ellapx/+lreachplain/+probdyn\LReachProblemLTIDynamics}{LReachProblemLTIDynamics}
\fontfamily{pcr}
\selectfont
\begin{lstlisting}
%

\end{lstlisting}
\fontfamily{\familydefault}
\selectfont
\end{enumerate}
\subsection{/+gras/+ellapx/+lreachuncert}
\begin{enumerate}
\item\hypertarget{/+gras/+ellapx/+lreachuncert\ExtIntEllApxBuilder}{ExtIntEllApxBuilder}
\fontfamily{pcr}
\selectfont
\begin{lstlisting}
%

\end{lstlisting}
\fontfamily{\familydefault}
\selectfont
\end{enumerate}
\subsection{/+gras/+ellapx/+lreachuncert/+probdef}
\begin{enumerate}
\item\hypertarget{/+gras/+ellapx/+lreachuncert/+probdef\AReachContProblemDef}{AReachContProblemDef}
\fontfamily{pcr}
\selectfont
\begin{lstlisting}
%

\end{lstlisting}
\fontfamily{\familydefault}
\selectfont
\item\hypertarget{/+gras/+ellapx/+lreachuncert/+probdef\IReachContProblemDef}{IReachContProblemDef}
\fontfamily{pcr}
\selectfont
\begin{lstlisting}
%

\end{lstlisting}
\fontfamily{\familydefault}
\selectfont
\item\hypertarget{/+gras/+ellapx/+lreachuncert/+probdef\LReachContProblemDef}{LReachContProblemDef}
\fontfamily{pcr}
\selectfont
\begin{lstlisting}
%

\end{lstlisting}
\fontfamily{\familydefault}
\selectfont
\item\hypertarget{/+gras/+ellapx/+lreachuncert/+probdef\ReachContLTIProblemDef}{ReachContLTIProblemDef}
\fontfamily{pcr}
\selectfont
\begin{lstlisting}
%

\end{lstlisting}
\fontfamily{\familydefault}
\selectfont
\end{enumerate}
\subsection{/+gras/+ellapx/+lreachuncert/+probdyn}
\begin{enumerate}
\item\hypertarget{/+gras/+ellapx/+lreachuncert/+probdyn\AReachProblemDynamics}{AReachProblemDynamics}
\fontfamily{pcr}
\selectfont
\begin{lstlisting}
%

\end{lstlisting}
\fontfamily{\familydefault}
\selectfont
\item\hypertarget{/+gras/+ellapx/+lreachuncert/+probdyn\IReachProblemDynamics}{IReachProblemDynamics}
\fontfamily{pcr}
\selectfont
\begin{lstlisting}
%

\end{lstlisting}
\fontfamily{\familydefault}
\selectfont
\item\hypertarget{/+gras/+ellapx/+lreachuncert/+probdyn\LReachProblemDynamicsFactory}{LReachProblemDynamicsFactory}
\fontfamily{pcr}
\selectfont
\begin{lstlisting}
%

\end{lstlisting}
\fontfamily{\familydefault}
\selectfont
\item\hypertarget{/+gras/+ellapx/+lreachuncert/+probdyn\LReachProblemDynamicsInterp}{LReachProblemDynamicsInterp}
\fontfamily{pcr}
\selectfont
\begin{lstlisting}
%

\end{lstlisting}
\fontfamily{\familydefault}
\selectfont
\item\hypertarget{/+gras/+ellapx/+lreachuncert/+probdyn\LReachProblemLTIDynamics}{LReachProblemLTIDynamics}
\fontfamily{pcr}
\selectfont
\begin{lstlisting}
%

\end{lstlisting}
\fontfamily{\familydefault}
\selectfont
\end{enumerate}
\subsection{/+gras/+ellapx/+proj}
\begin{enumerate}
\item\hypertarget{/+gras/+ellapx/+proj\AEllTubePlainProjector}{AEllTubePlainProjector}
\fontfamily{pcr}
\selectfont
\begin{lstlisting}
% IELLTUBEPROJECTOR Summary of this class goes here
%    Detailed explanation goes here
%

\end{lstlisting}
\fontfamily{\familydefault}
\selectfont
\item\hypertarget{/+gras/+ellapx/+proj\EllTubeCollectionProjector}{EllTubeCollectionProjector}
\fontfamily{pcr}
\selectfont
\begin{lstlisting}
% IELLTUBEPROJECTOR Summary of this class goes here
%    Detailed explanation goes here
%

\end{lstlisting}
\fontfamily{\familydefault}
\selectfont
\item\hypertarget{/+gras/+ellapx/+proj\EllTubeStaticSpaceProjector}{EllTubeStaticSpaceProjector}
\fontfamily{pcr}
\selectfont
\begin{lstlisting}
% IELLTUBEPROJECTOR Summary of this class goes here
%    Detailed explanation goes here
%

\end{lstlisting}
\fontfamily{\familydefault}
\selectfont
\item\hypertarget{/+gras/+ellapx/+proj\IEllTubeProjector}{IEllTubeProjector}
\fontfamily{pcr}
\selectfont
\begin{lstlisting}
% IELLTUBEPROJECTOR Summary of this class goes here
%    Detailed explanation goes here
%

\end{lstlisting}
\fontfamily{\familydefault}
\selectfont
\end{enumerate}
\subsection{/+gras/+ellapx/+smartdb}
\begin{enumerate}
\item\hypertarget{/+gras/+ellapx/+smartdb\F}{F}
\fontfamily{pcr}
\selectfont
\begin{lstlisting}
% Standard fields
%

\end{lstlisting}
\fontfamily{\familydefault}
\selectfont
\item\hypertarget{/+gras/+ellapx/+smartdb\RelDispConfigurator}{RelDispConfigurator}
\fontfamily{pcr}
\selectfont
\begin{lstlisting}
%

\end{lstlisting}
\fontfamily{\familydefault}
\selectfont
\end{enumerate}
\subsection{/+gras/+ellapx/+smartdb/+rels}
\begin{enumerate}
\item\hypertarget{/+gras/+ellapx/+smartdb/+rels\EllTube}{EllTube}
\fontfamily{pcr}
\selectfont
\begin{lstlisting}
% TestRelation Summary of this class goes here
%    Detailed explanation goes here
%

\end{lstlisting}
\fontfamily{\familydefault}
\selectfont
\item\hypertarget{/+gras/+ellapx/+smartdb/+rels\EllTubeBasic}{EllTubeBasic}
\fontfamily{pcr}
\selectfont
\begin{lstlisting}
% TestRelation Summary of this class goes here
%    Detailed explanation goes here
%

\end{lstlisting}
\fontfamily{\familydefault}
\selectfont
\item\hypertarget{/+gras/+ellapx/+smartdb/+rels\EllTubeProj}{EllTubeProj}
\fontfamily{pcr}
\selectfont
\begin{lstlisting}
% TestRelation Summary of this class goes here
%    Detailed explanation goes here
%

\end{lstlisting}
\fontfamily{\familydefault}
\selectfont
\item\hypertarget{/+gras/+ellapx/+smartdb/+rels\EllTubeProjBasic}{EllTubeProjBasic}
\fontfamily{pcr}
\selectfont
\begin{lstlisting}
%

\end{lstlisting}
\fontfamily{\familydefault}
\selectfont
\item\hypertarget{/+gras/+ellapx/+smartdb/+rels\EllTubeTouchCurveBasic}{EllTubeTouchCurveBasic}
\fontfamily{pcr}
\selectfont
\begin{lstlisting}
% TestRelation Summary of this class goes here
%    Detailed explanation goes here
%

\end{lstlisting}
\fontfamily{\familydefault}
\selectfont
\item\hypertarget{/+gras/+ellapx/+smartdb/+rels\EllTubeTouchCurveProjBasic}{EllTubeTouchCurveProjBasic}
\fontfamily{pcr}
\selectfont
\begin{lstlisting}
%

\end{lstlisting}
\fontfamily{\familydefault}
\selectfont
\item\hypertarget{/+gras/+ellapx/+smartdb/+rels\EllUnionTube}{EllUnionTube}
\fontfamily{pcr}
\selectfont
\begin{lstlisting}
% TestRelation Summary of this class goes here
%    Detailed explanation goes here
%

\end{lstlisting}
\fontfamily{\familydefault}
\selectfont
\item\hypertarget{/+gras/+ellapx/+smartdb/+rels\EllUnionTubeBasic}{EllUnionTubeBasic}
\fontfamily{pcr}
\selectfont
\begin{lstlisting}
% TestRelation Summary of this class goes here
%    Detailed explanation goes here
%

\end{lstlisting}
\fontfamily{\familydefault}
\selectfont
\item\hypertarget{/+gras/+ellapx/+smartdb/+rels\EllUnionTubeStaticProj}{EllUnionTubeStaticProj}
\fontfamily{pcr}
\selectfont
\begin{lstlisting}
% TestRelation Summary of this class goes here
%    Detailed explanation goes here
% 
%

\end{lstlisting}
\fontfamily{\familydefault}
\selectfont
\item\hypertarget{/+gras/+ellapx/+smartdb/+rels\TypifiedByFieldCodeRel}{TypifiedByFieldCodeRel}
\fontfamily{pcr}
\selectfont
\begin{lstlisting}
% TestRelation Summary of this class goes here
%    Detailed explanation goes here
%

\end{lstlisting}
\fontfamily{\familydefault}
\selectfont
\end{enumerate}
\subsection{/+gras/+ellapx/+smartdb/+test}
\begin{enumerate}
\item\hypertarget{/+gras/+ellapx/+smartdb/+test\run\_tests}{run\_tests}
\fontfamily{pcr}
\selectfont
\begin{lstlisting}
%

\end{lstlisting}
\fontfamily{\familydefault}
\selectfont
\end{enumerate}
\subsection{/+gras/+ellapx/+smartdb/+test/+mlunit}
\begin{enumerate}
\item\hypertarget{/+gras/+ellapx/+smartdb/+test/+mlunit\SuiteEllTube}{SuiteEllTube}
\fontfamily{pcr}
\selectfont
\begin{lstlisting}
%

\end{lstlisting}
\fontfamily{\familydefault}
\selectfont
\end{enumerate}
\subsection{/+gras/+ellapx/+test}
\begin{enumerate}
\item\hypertarget{/+gras/+ellapx/+test\run\_tests}{run\_tests}
\fontfamily{pcr}
\selectfont
\begin{lstlisting}
%

\end{lstlisting}
\fontfamily{\familydefault}
\selectfont
\end{enumerate}
\subsection{/+gras/+ellapx/+uncertcalc}
\begin{enumerate}
\item\hypertarget{/+gras/+ellapx/+uncertcalc\ApproxProblemPropertyBuilder}{ApproxProblemPropertyBuilder}
\fontfamily{pcr}
\selectfont
\begin{lstlisting}
%

\end{lstlisting}
\fontfamily{\familydefault}
\selectfont
\item\hypertarget{/+gras/+ellapx/+uncertcalc\EllApxBuilder}{EllApxBuilder}
\fontfamily{pcr}
\selectfont
\begin{lstlisting}
% IELLTUBEPROJECTOR Summary of this class goes here
%    Detailed explanation goes here
%

\end{lstlisting}
\fontfamily{\familydefault}
\selectfont
\item\hypertarget{/+gras/+ellapx/+uncertcalc\EllTubeProjectorBuilder}{EllTubeProjectorBuilder}
\fontfamily{pcr}
\selectfont
\begin{lstlisting}
% IELLTUBEPROJECTOR Summary of this class goes here
%    Detailed explanation goes here
%

\end{lstlisting}
\fontfamily{\familydefault}
\selectfont
\item\hypertarget{/+gras/+ellapx/+uncertcalc\copyconf}{copyconf}
\fontfamily{pcr}
\selectfont
\begin{lstlisting}
%

\end{lstlisting}
\fontfamily{\familydefault}
\selectfont
\item\hypertarget{/+gras/+ellapx/+uncertcalc\copysysconf}{copysysconf}
\fontfamily{pcr}
\selectfont
\begin{lstlisting}
%

\end{lstlisting}
\fontfamily{\familydefault}
\selectfont
\item\hypertarget{/+gras/+ellapx/+uncertcalc\editconf}{editconf}
\fontfamily{pcr}
\selectfont
\begin{lstlisting}
%

\end{lstlisting}
\fontfamily{\familydefault}
\selectfont
\item\hypertarget{/+gras/+ellapx/+uncertcalc\editsysconf}{editsysconf}
\fontfamily{pcr}
\selectfont
\begin{lstlisting}
%

\end{lstlisting}
\fontfamily{\familydefault}
\selectfont
\item\hypertarget{/+gras/+ellapx/+uncertcalc\listconf}{listconf}
\fontfamily{pcr}
\selectfont
\begin{lstlisting}
%

\end{lstlisting}
\fontfamily{\familydefault}
\selectfont
\item\hypertarget{/+gras/+ellapx/+uncertcalc\listsysconf}{listsysconf}
\fontfamily{pcr}
\selectfont
\begin{lstlisting}
%

\end{lstlisting}
\fontfamily{\familydefault}
\selectfont
\item\hypertarget{/+gras/+ellapx/+uncertcalc\run}{run}
\fontfamily{pcr}
\selectfont
\begin{lstlisting}
%

\end{lstlisting}
\fontfamily{\familydefault}
\selectfont
\item\hypertarget{/+gras/+ellapx/+uncertcalc\updateallconf}{updateallconf}
\fontfamily{pcr}
\selectfont
\begin{lstlisting}
%

\end{lstlisting}
\fontfamily{\familydefault}
\selectfont
\end{enumerate}
\subsection{/+gras/+ellapx/+uncertcalc/+conf}
\begin{enumerate}
\item\hypertarget{/+gras/+ellapx/+uncertcalc/+conf\ConfRepoMgr}{ConfRepoMgr}
\fontfamily{pcr}
\selectfont
\begin{lstlisting}
%

\end{lstlisting}
\fontfamily{\familydefault}
\selectfont
\item\hypertarget{/+gras/+ellapx/+uncertcalc/+conf\IConfRepoMgr}{IConfRepoMgr}
\fontfamily{pcr}
\selectfont
\begin{lstlisting}
%

\end{lstlisting}
\fontfamily{\familydefault}
\selectfont
\end{enumerate}
\subsection{/+gras/+ellapx/+uncertcalc/+conf/+sysdef}
\begin{enumerate}
\item\hypertarget{/+gras/+ellapx/+uncertcalc/+conf/+sysdef\AConfRepoMgr}{AConfRepoMgr}
\fontfamily{pcr}
\selectfont
\begin{lstlisting}
%

\end{lstlisting}
\fontfamily{\familydefault}
\selectfont
\item\hypertarget{/+gras/+ellapx/+uncertcalc/+conf/+sysdef\ConfRepoMgr}{ConfRepoMgr}
\fontfamily{pcr}
\selectfont
\begin{lstlisting}
%

\end{lstlisting}
\fontfamily{\familydefault}
\selectfont
\end{enumerate}
\subsection{/+gras/+ellapx/+uncertcalc/+conf/+sysdef/+test}
\begin{enumerate}
\item\hypertarget{/+gras/+ellapx/+uncertcalc/+conf/+sysdef/+test\ConfRepoMgr}{ConfRepoMgr}
\fontfamily{pcr}
\selectfont
\begin{lstlisting}
%

\end{lstlisting}
\fontfamily{\familydefault}
\selectfont
\item\hypertarget{/+gras/+ellapx/+uncertcalc/+conf/+sysdef/+test\run\_tests}{run\_tests}
\fontfamily{pcr}
\selectfont
\begin{lstlisting}
%

\end{lstlisting}
\fontfamily{\familydefault}
\selectfont
\end{enumerate}
\subsection{/+gras/+ellapx/+uncertcalc/+conf/+sysdef/+test/+mlunit}
\begin{enumerate}
\item\hypertarget{/+gras/+ellapx/+uncertcalc/+conf/+sysdef/+test/+mlunit\SuiteBasic}{SuiteBasic}
\fontfamily{pcr}
\selectfont
\begin{lstlisting}
%

\end{lstlisting}
\fontfamily{\familydefault}
\selectfont
\end{enumerate}
\subsection{/+gras/+ellapx/+uncertcalc/+conf/+sysdef/@ConfPatchRepo}
\begin{enumerate}
\item\hypertarget{/+gras/+ellapx/+uncertcalc/+conf/+sysdef/@ConfPatchRepo\ConfPatchRepo}{ConfPatchRepo}
\fontfamily{pcr}
\selectfont
\begin{lstlisting}
%

\end{lstlisting}
\fontfamily{\familydefault}
\selectfont
\item\hypertarget{/+gras/+ellapx/+uncertcalc/+conf/+sysdef/@ConfPatchRepo\patch\_001\_remove\_garbage}{patch\_001\_remove\_garbage}
\fontfamily{pcr}
\selectfont
\begin{lstlisting}
%

\end{lstlisting}
\fontfamily{\familydefault}
\selectfont
\item\hypertarget{/+gras/+ellapx/+uncertcalc/+conf/+sysdef/@ConfPatchRepo\patch\_002\_add\_description}{patch\_002\_add\_description}
\fontfamily{pcr}
\selectfont
\begin{lstlisting}
%

\end{lstlisting}
\fontfamily{\familydefault}
\selectfont
\end{enumerate}
\subsection{/+gras/+ellapx/+uncertcalc/+conf/@ConfPatchRepo}
\begin{enumerate}
\item\hypertarget{/+gras/+ellapx/+uncertcalc/+conf/@ConfPatchRepo\ConfPatchRepo}{ConfPatchRepo}
\fontfamily{pcr}
\selectfont
\begin{lstlisting}
%

\end{lstlisting}
\fontfamily{\familydefault}
\selectfont
\item\hypertarget{/+gras/+ellapx/+uncertcalc/+conf/@ConfPatchRepo\patch\_001\_make\_proj\_spec\_logical}{patch\_001\_make\_proj\_spec\_logical}
\fontfamily{pcr}
\selectfont
\begin{lstlisting}
%

\end{lstlisting}
\fontfamily{\familydefault}
\selectfont
\item\hypertarget{/+gras/+ellapx/+uncertcalc/+conf/@ConfPatchRepo\patch\_002\_remove\_redundant\_stuff}{patch\_002\_remove\_redundant\_stuff}
\fontfamily{pcr}
\selectfont
\begin{lstlisting}
%

\end{lstlisting}
\fontfamily{\familydefault}
\selectfont
\item\hypertarget{/+gras/+ellapx/+uncertcalc/+conf/@ConfPatchRepo\patch\_003\_is\_plotting\_enabled}{patch\_003\_is\_plotting\_enabled}
\fontfamily{pcr}
\selectfont
\begin{lstlisting}
%

\end{lstlisting}
\fontfamily{\familydefault}
\selectfont
\item\hypertarget{/+gras/+ellapx/+uncertcalc/+conf/@ConfPatchRepo\patch\_004\_multiple\_int\_ell\_apx\_schemas}{patch\_004\_multiple\_int\_ell\_apx\_schemas}
\fontfamily{pcr}
\selectfont
\begin{lstlisting}
%

\end{lstlisting}
\fontfamily{\familydefault}
\selectfont
\item\hypertarget{/+gras/+ellapx/+uncertcalc/+conf/@ConfPatchRepo\patch\_005\_ext\_ell\_apx\_schema}{patch\_005\_ext\_ell\_apx\_schema}
\fontfamily{pcr}
\selectfont
\begin{lstlisting}
%

\end{lstlisting}
\fontfamily{\familydefault}
\selectfont
\item\hypertarget{/+gras/+ellapx/+uncertcalc/+conf/@ConfPatchRepo\patch\_006\_calc\_precision}{patch\_006\_calc\_precision}
\fontfamily{pcr}
\selectfont
\begin{lstlisting}
%

\end{lstlisting}
\fontfamily{\familydefault}
\selectfont
\item\hypertarget{/+gras/+ellapx/+uncertcalc/+conf/@ConfPatchRepo\patch\_007\_make\_space\_list\_a\_vector}{patch\_007\_make\_space\_list\_a\_vector}
\fontfamily{pcr}
\selectfont
\begin{lstlisting}
%

\end{lstlisting}
\fontfamily{\familydefault}
\selectfont
\item\hypertarget{/+gras/+ellapx/+uncertcalc/+conf/@ConfPatchRepo\patch\_008\_add\_reference\_to\_sysdef}{patch\_008\_add\_reference\_to\_sysdef}
\fontfamily{pcr}
\selectfont
\begin{lstlisting}
%

\end{lstlisting}
\fontfamily{\familydefault}
\selectfont
\item\hypertarget{/+gras/+ellapx/+uncertcalc/+conf/@ConfPatchRepo\patch\_009\_add\_scale\_factors}{patch\_009\_add\_scale\_factors}
\fontfamily{pcr}
\selectfont
\begin{lstlisting}
%

\end{lstlisting}
\fontfamily{\familydefault}
\selectfont
\item\hypertarget{/+gras/+ellapx/+uncertcalc/+conf/@ConfPatchRepo\patch\_010\_rename\_scale\_factors}{patch\_010\_rename\_scale\_factors}
\fontfamily{pcr}
\selectfont
\begin{lstlisting}
%

\end{lstlisting}
\fontfamily{\familydefault}
\selectfont
\item\hypertarget{/+gras/+ellapx/+uncertcalc/+conf/@ConfPatchRepo\patch\_011\_add\_view\_angle\_prop}{patch\_011\_add\_view\_angle\_prop}
\fontfamily{pcr}
\selectfont
\begin{lstlisting}
%

\end{lstlisting}
\fontfamily{\familydefault}
\selectfont
\item\hypertarget{/+gras/+ellapx/+uncertcalc/+conf/@ConfPatchRepo\patch\_012\_rename\_ell\_apx\_schemas}{patch\_012\_rename\_ell\_apx\_schemas}
\fontfamily{pcr}
\selectfont
\begin{lstlisting}
%

\end{lstlisting}
\fontfamily{\familydefault}
\selectfont
\item\hypertarget{/+gras/+ellapx/+uncertcalc/+conf/@ConfPatchRepo\patch\_013\_disable\_uncertainty\_regime\_by\_default}{patch\_013\_disable\_uncertainty\_regime\_by\_default}
\fontfamily{pcr}
\selectfont
\begin{lstlisting}
%

\end{lstlisting}
\fontfamily{\familydefault}
\selectfont
\item\hypertarget{/+gras/+ellapx/+uncertcalc/+conf/@ConfPatchRepo\patch\_014\_internal\_external\_apx\_params}{patch\_014\_internal\_external\_apx\_params}
\fontfamily{pcr}
\selectfont
\begin{lstlisting}
%

\end{lstlisting}
\fontfamily{\familydefault}
\selectfont
\item\hypertarget{/+gras/+ellapx/+uncertcalc/+conf/@ConfPatchRepo\patch\_015\_internal\_external\_apx\_addparams}{patch\_015\_internal\_external\_apx\_addparams}
\fontfamily{pcr}
\selectfont
\begin{lstlisting}
%

\end{lstlisting}
\fontfamily{\familydefault}
\selectfont
\item\hypertarget{/+gras/+ellapx/+uncertcalc/+conf/@ConfPatchRepo\patch\_016\_is\_good\_curves\_separately}{patch\_016\_is\_good\_curves\_separately}
\fontfamily{pcr}
\selectfont
\begin{lstlisting}
%

\end{lstlisting}
\fontfamily{\familydefault}
\selectfont
\item\hypertarget{/+gras/+ellapx/+uncertcalc/+conf/@ConfPatchRepo\patch\_017\_gen\_props\_mat\_calc\_mode}{patch\_017\_gen\_props\_mat\_calc\_mode}
\fontfamily{pcr}
\selectfont
\begin{lstlisting}
%

\end{lstlisting}
\fontfamily{\familydefault}
\selectfont
\item\hypertarget{/+gras/+ellapx/+uncertcalc/+conf/@ConfPatchRepo\patch\_018\_remove\_uncert\_int\_apx\_schema}{patch\_018\_remove\_uncert\_int\_apx\_schema}
\fontfamily{pcr}
\selectfont
\begin{lstlisting}
%

\end{lstlisting}
\fontfamily{\familydefault}
\selectfont
\end{enumerate}
\subsection{/+gras/+ellapx/+uncertcalc/+log}
\begin{enumerate}
\item\hypertarget{/+gras/+ellapx/+uncertcalc/+log\Log4jConfigurator}{Log4jConfigurator}
\fontfamily{pcr}
\selectfont
\begin{lstlisting}
% LOG4JCONFIGURATOR simplifies log4j configuration, especially when
% Parallel Computing Toolbox is used. In the latter case the class forwards
% the logs of different processees in separate log files
% 
%  $Author: Peter Gagarinov  <pgagarinov@gmail.com> $	$Date: 2011-05-18 $ 
%  $Copyright: Moscow State University,
%             Faculty of Computational Mathematics and Computer Science,
%             System Analysis Department 2011 $
% 
%

\end{lstlisting}
\fontfamily{\familydefault}
\selectfont
\end{enumerate}
\subsection{/+gras/+ellapx/+uncertcalc/+test}
\begin{enumerate}
\item\hypertarget{/+gras/+ellapx/+uncertcalc/+test\run\_tests}{run\_tests}
\fontfamily{pcr}
\selectfont
\begin{lstlisting}
%

\end{lstlisting}
\fontfamily{\familydefault}
\selectfont
\item\hypertarget{/+gras/+ellapx/+uncertcalc/+test\updateallconf}{updateallconf}
\fontfamily{pcr}
\selectfont
\begin{lstlisting}
%  UPDATEALLCONF updates all the configurations in the nested packages
% 
% 
%  $Author: Peter Gagarinov <pgagarinov@gmail.com> $	$Date: 2012-11-24 $ 
%  $Copyright: Moscow State University,
%             Faculty of Computational Mathematics and Computer Science,
%             System Analysis Department 2012 $
% 
%

\end{lstlisting}
\fontfamily{\familydefault}
\selectfont
\end{enumerate}
\subsection{/+gras/+ellapx/+uncertcalc/+test/+comp}
\begin{enumerate}
\item\hypertarget{/+gras/+ellapx/+uncertcalc/+test/+comp\copyconf}{copyconf}
\fontfamily{pcr}
\selectfont
\begin{lstlisting}
%

\end{lstlisting}
\fontfamily{\familydefault}
\selectfont
\item\hypertarget{/+gras/+ellapx/+uncertcalc/+test/+comp\editconf}{editconf}
\fontfamily{pcr}
\selectfont
\begin{lstlisting}
%

\end{lstlisting}
\fontfamily{\familydefault}
\selectfont
\item\hypertarget{/+gras/+ellapx/+uncertcalc/+test/+comp\editconftemplate}{editconftemplate}
\fontfamily{pcr}
\selectfont
\begin{lstlisting}
%

\end{lstlisting}
\fontfamily{\familydefault}
\selectfont
\item\hypertarget{/+gras/+ellapx/+uncertcalc/+test/+comp\listconfs}{listconfs}
\fontfamily{pcr}
\selectfont
\begin{lstlisting}
%  UPDATECONFTEMPLATE updates the specified template configuration
% 
% 
%  $Author: Peter Gagarinov <pgagarinov@gmail.com> $	$Date: 2011-09-09 $ 
%  $Copyright: Moscow State University,
%             Faculty of Computational Mathematics and Computer Science,
%             System Analysis Department 2011 $
% 
% 
%

\end{lstlisting}
\fontfamily{\familydefault}
\selectfont
\item\hypertarget{/+gras/+ellapx/+uncertcalc/+test/+comp\run\_tests}{run\_tests}
\fontfamily{pcr}
\selectfont
\begin{lstlisting}
%

\end{lstlisting}
\fontfamily{\familydefault}
\selectfont
\item\hypertarget{/+gras/+ellapx/+uncertcalc/+test/+comp\updateallconf}{updateallconf}
\fontfamily{pcr}
\selectfont
\begin{lstlisting}
%  UPDATEALLCONF updates all the configurations in the nested packages
% 
% 
%  $Author: Peter Gagarinov <pgagarinov@gmail.com> $	$Date: 2011-09-09 $ 
%  $Copyright: Moscow State University,
%             Faculty of Computational Mathematics and Computer Science,
%             System Analysis Department 2011 $
% 
% 
%

\end{lstlisting}
\fontfamily{\familydefault}
\selectfont
\item\hypertarget{/+gras/+ellapx/+uncertcalc/+test/+comp\updateconftemplate}{updateconftemplate}
\fontfamily{pcr}
\selectfont
\begin{lstlisting}
%  UPDATECONFTEMPLATE updates the specified template configuration
% 
% 
%  $Author: Peter Gagarinov <pgagarinov@gmail.com> $	$Date: 2011-09-09 $ 
%  $Copyright: Moscow State University,
%             Faculty of Computational Mathematics and Computer Science,
%             System Analysis Department 2011 $
% 
% 
%

\end{lstlisting}
\fontfamily{\familydefault}
\selectfont
\end{enumerate}
\subsection{/+gras/+ellapx/+uncertcalc/+test/+comp/+conf}
\begin{enumerate}
\item\hypertarget{/+gras/+ellapx/+uncertcalc/+test/+comp/+conf\ConfRepoMgr}{ConfRepoMgr}
\fontfamily{pcr}
\selectfont
\begin{lstlisting}
%

\end{lstlisting}
\fontfamily{\familydefault}
\selectfont
\end{enumerate}
\subsection{/+gras/+ellapx/+uncertcalc/+test/+comp/+conf/+sysdef}
\begin{enumerate}
\item\hypertarget{/+gras/+ellapx/+uncertcalc/+test/+comp/+conf/+sysdef\ConfRepoMgr}{ConfRepoMgr}
\fontfamily{pcr}
\selectfont
\begin{lstlisting}
%

\end{lstlisting}
\fontfamily{\familydefault}
\selectfont
\end{enumerate}
\subsection{/+gras/+ellapx/+uncertcalc/+test/+comp/+mlunit}
\begin{enumerate}
\item\hypertarget{/+gras/+ellapx/+uncertcalc/+test/+comp/+mlunit\SuiteCompare}{SuiteCompare}
\fontfamily{pcr}
\selectfont
\begin{lstlisting}
%

\end{lstlisting}
\fontfamily{\familydefault}
\selectfont
\end{enumerate}
\subsection{/+gras/+ellapx/+uncertcalc/+test/+regr}
\begin{enumerate}
\item\hypertarget{/+gras/+ellapx/+uncertcalc/+test/+regr\copyconf}{copyconf}
\fontfamily{pcr}
\selectfont
\begin{lstlisting}
%

\end{lstlisting}
\fontfamily{\familydefault}
\selectfont
\item\hypertarget{/+gras/+ellapx/+uncertcalc/+test/+regr\editconf}{editconf}
\fontfamily{pcr}
\selectfont
\begin{lstlisting}
%

\end{lstlisting}
\fontfamily{\familydefault}
\selectfont
\item\hypertarget{/+gras/+ellapx/+uncertcalc/+test/+regr\editconftemplate}{editconftemplate}
\fontfamily{pcr}
\selectfont
\begin{lstlisting}
%

\end{lstlisting}
\fontfamily{\familydefault}
\selectfont
\item\hypertarget{/+gras/+ellapx/+uncertcalc/+test/+regr\editsysconf}{editsysconf}
\fontfamily{pcr}
\selectfont
\begin{lstlisting}
%

\end{lstlisting}
\fontfamily{\familydefault}
\selectfont
\item\hypertarget{/+gras/+ellapx/+uncertcalc/+test/+regr\listconfs}{listconfs}
\fontfamily{pcr}
\selectfont
\begin{lstlisting}
%  UPDATECONFTEMPLATE updates the specified template configuration
% 
% 
%  $Author: Peter Gagarinov <pgagarinov@gmail.com> $	$Date: 2011-09-09 $ 
%  $Copyright: Moscow State University,
%             Faculty of Computational Mathematics and Computer Science,
%             System Analysis Department 2011 $
% 
% 
%

\end{lstlisting}
\fontfamily{\familydefault}
\selectfont
\item\hypertarget{/+gras/+ellapx/+uncertcalc/+test/+regr\run\_regr\_tests}{run\_regr\_tests}
\fontfamily{pcr}
\selectfont
\begin{lstlisting}
%

\end{lstlisting}
\fontfamily{\familydefault}
\selectfont
\item\hypertarget{/+gras/+ellapx/+uncertcalc/+test/+regr\run\_support\_function\_tests}{run\_support\_function\_tests}
\fontfamily{pcr}
\selectfont
\begin{lstlisting}
%  $Author: Kirill Mayantsev  <kirill.mayantsev@gmail.com> $  $Date: 2-11-2012 $
%  $Copyright: Moscow State University,
%              Faculty of Computational Mathematics and Computer Science,
%              System Analysis Department 2012 $
%

\end{lstlisting}
\fontfamily{\familydefault}
\selectfont
\item\hypertarget{/+gras/+ellapx/+uncertcalc/+test/+regr\run\_tests}{run\_tests}
\fontfamily{pcr}
\selectfont
\begin{lstlisting}
%

\end{lstlisting}
\fontfamily{\familydefault}
\selectfont
\item\hypertarget{/+gras/+ellapx/+uncertcalc/+test/+regr\updateallconf}{updateallconf}
\fontfamily{pcr}
\selectfont
\begin{lstlisting}
%  UPDATEALLCONF updates all the configurations in the nested packages
% 
% 
%  $Author: Peter Gagarinov <pgagarinov@gmail.com> $	$Date: 2011-09-09 $ 
%  $Copyright: Moscow State University,
%             Faculty of Computational Mathematics and Computer Science,
%             System Analysis Department 2011 $
% 
% 
%

\end{lstlisting}
\fontfamily{\familydefault}
\selectfont
\item\hypertarget{/+gras/+ellapx/+uncertcalc/+test/+regr\updateconftemplate}{updateconftemplate}
\fontfamily{pcr}
\selectfont
\begin{lstlisting}
%  UPDATECONFTEMPLATE updates the specified template configuration
% 
% 
%  $Author: Peter Gagarinov <pgagarinov@gmail.com> $	$Date: 2011-09-09 $ 
%  $Copyright: Moscow State University,
%             Faculty of Computational Mathematics and Computer Science,
%             System Analysis Department 2011 $
% 
% 
%

\end{lstlisting}
\fontfamily{\familydefault}
\selectfont
\end{enumerate}
\subsection{/+gras/+ellapx/+uncertcalc/+test/+regr/+conf}
\begin{enumerate}
\item\hypertarget{/+gras/+ellapx/+uncertcalc/+test/+regr/+conf\ConfRepoMgr}{ConfRepoMgr}
\fontfamily{pcr}
\selectfont
\begin{lstlisting}
%

\end{lstlisting}
\fontfamily{\familydefault}
\selectfont
\end{enumerate}
\subsection{/+gras/+ellapx/+uncertcalc/+test/+regr/+conf/+sysdef}
\begin{enumerate}
\item\hypertarget{/+gras/+ellapx/+uncertcalc/+test/+regr/+conf/+sysdef\ConfRepoMgr}{ConfRepoMgr}
\fontfamily{pcr}
\selectfont
\begin{lstlisting}
%

\end{lstlisting}
\fontfamily{\familydefault}
\selectfont
\end{enumerate}
\subsection{/+gras/+ellapx/+uncertcalc/+test/+regr/+mlunit}
\begin{enumerate}
\item\hypertarget{/+gras/+ellapx/+uncertcalc/+test/+regr/+mlunit\SuiteBasic}{SuiteBasic}
\fontfamily{pcr}
\selectfont
\begin{lstlisting}
%

\end{lstlisting}
\fontfamily{\familydefault}
\selectfont
\item\hypertarget{/+gras/+ellapx/+uncertcalc/+test/+regr/+mlunit\SuiteRegression}{SuiteRegression}
\fontfamily{pcr}
\selectfont
\begin{lstlisting}
%

\end{lstlisting}
\fontfamily{\familydefault}
\selectfont
\item\hypertarget{/+gras/+ellapx/+uncertcalc/+test/+regr/+mlunit\SuiteSupportFunction}{SuiteSupportFunction}
\fontfamily{pcr}
\selectfont
\begin{lstlisting}
%  $Author: Kirill Mayantsev  <kirill.mayantsev@gmail.com> $  $Date: 2-11-2012 $
%  $Copyright: Moscow State University,
%              Faculty of Computational Mathematics and Computer Science,
%              System Analysis Department 2012 $
%

\end{lstlisting}
\fontfamily{\familydefault}
\selectfont
\end{enumerate}
\subsection{/+gras/+gen}
\begin{enumerate}
\item\hypertarget{/+gras/+gen\MatVector}{MatVector}
\fontfamily{pcr}
\selectfont
\begin{lstlisting}
% MATVECTOR Summary of this class goes here
%    Detailed explanation goes here
%

\end{lstlisting}
\fontfamily{\familydefault}
\selectfont
\item\hypertarget{/+gras/+gen\ProgressCmdDisplayer}{ProgressCmdDisplayer}
\fontfamily{pcr}
\selectfont
\begin{lstlisting}
%

\end{lstlisting}
\fontfamily{\familydefault}
\selectfont
\item\hypertarget{/+gras/+gen\SquareMatVector}{SquareMatVector}
\fontfamily{pcr}
\selectfont
\begin{lstlisting}
% MATVECTOR Summary of this class goes here
%    Detailed explanation goes here
%

\end{lstlisting}
\fontfamily{\familydefault}
\selectfont
\item\hypertarget{/+gras/+gen\minadv}{minadv}
\fontfamily{pcr}
\selectfont
\begin{lstlisting}
%  MINADV works in the same way as the built-in min function but returns
%  indMinSize as a second argument which equals 1 if all minimum elements
%  can be taken from leftArray, 2 if they all can be taken from rightArray
%  and 0 otherwise
% 
%  Input:
%    regular:
%        leftArray: numeric[nElems1,...,nElemsK]
%        rightArray: numeric[nElems1,...,nElemsK]
%  Output:
%    minArray: numeric[nElems1,...,nElemsK] - array composed from minimum
%        elements 
%    indMinSide: double[1,1] - 1 if all minimum elements
%        can be taken from leftArray, 2 if they all can be taken from rightArray
%        and 0 otherwise
% 
%  $Author: Peter Gagarinov  <pgagarinov@gmail.com> $	$Date: 2011-05-29 $ 
%  $Copyright: Moscow State University,
%             Faculty of Computational Mathematics and Computer Science,
%             System Analysis Department 2011 $
% 
%

\end{lstlisting}
\fontfamily{\familydefault}
\selectfont
\item\hypertarget{/+gras/+gen\sortrowstol}{sortrowstol}
\fontfamily{pcr}
\selectfont
\begin{lstlisting}
%  SORTROWSTOL sorts rows of input numeric matrix in ascending order with a
%  specified precision i.e. sorting [1 2;1+1e-14 1] with tol>=1-14 would put
%  the first row on the second position while the built-in sortrows function
%  would keep the order of rows unchanged, the functions only looks at the
%  neighboring values and doesn't calculate pairwise distances (as in pdist
%  function) for speed
% 
%  Input:
%    regular:
%        inpMat: numeric[nRows,nCols] - input matrix
%        tol: numeric[1,1] - tolerance used for sorting, values of inpMat
%           are considered to be equal if their difference is less or equal
%           than tol by an absolute value
% 
%  Output:
%    resMat: numeric[nRows,nCols] - output of matrix
%    indSortVec: double[nRows,1] - vector of indices such that
%        inpMat(indSortVec,:)==resMat
%    indRevSortVec: double[nRows,1] - vector of indices such that
%        resMat(indRevSortVec,:)==inpMat
% 
%  $Author: Peter Gagarinov  <pgagarinov@gmail.com> $	$Date: 2011-05-29 $ 
%  $Copyright: Moscow State University,
%             Faculty of Computational Mathematics and Computer Science,
%             System Analysis Department 2011 $
% 
%  TODO add support for clustering method based on
%    clusterdata([1;1+1e-14;2;2+1e-14;2-1e-14],'criterion','distance','cutoff',1e-14)
%

\end{lstlisting}
\fontfamily{\familydefault}
\selectfont
\end{enumerate}
\subsection{/+gras/+gen/+test}
\begin{enumerate}
\item\hypertarget{/+gras/+gen/+test\run\_tests}{run\_tests}
\fontfamily{pcr}
\selectfont
\begin{lstlisting}
%

\end{lstlisting}
\fontfamily{\familydefault}
\selectfont
\end{enumerate}
\subsection{/+gras/+gen/+test/+mlunit}
\begin{enumerate}
\item\hypertarget{/+gras/+gen/+test/+mlunit\SuiteBasic}{SuiteBasic}
\fontfamily{pcr}
\selectfont
\begin{lstlisting}
%

\end{lstlisting}
\fontfamily{\familydefault}
\selectfont
\end{enumerate}
\subsection{/+gras/+geom}
\begin{enumerate}
\item\hypertarget{/+gras/+geom\circlepart}{circlepart}
\fontfamily{pcr}
\selectfont
\begin{lstlisting}
%  CIRCLEPART builds a partition of unit circle into a specified number of
%  points within a specified angle range
%  
%  Input:
%    regular:
%        nPoints: double[1,1] - number of points to partition the circle
%    optional
%        angleRangeVec: double[1,2] - angle range in radians, default value
%            is [0,2*PI]
%    
%  Output:
%    xVec: double[nPoints,1]/[nPoints,2] - coordinates on the unit circle,
%        both x and y coordinates are returned in different columns if the
%        second output argument is not specified, otherwise only x
%        coordinates are returned
%    yVec: double[nPoints,1] - y coordinates of the points on the unit
%        circle
%    
%  $Author: Peter Gagarinov  <pgagarinov@gmail.com> $	$Date: 2011-05-31$ 
%  $Copyright: Moscow State University,
%             Faculty of Computational Mathematics and Computer Science,
%             System Analysis Department 2011 $
% 
%

\end{lstlisting}
\fontfamily{\familydefault}
\selectfont
\end{enumerate}
\subsection{/+gras/+geom/+ell}
\begin{enumerate}
\item\hypertarget{/+gras/+geom/+ell\ellvolume}{ellvolume}
\fontfamily{pcr}
\selectfont
\begin{lstlisting}
%  ELLVOLUME calculates a volume of ellipsoid
%  
%  $Author: Peter Gagarinov  <pgagarinov@gmail.com> $	$Date: 2011-12-30 $ 
%  $Copyright: Moscow State University,
%             Faculty of Computational Mathematics and Computer Science,
%             System Analysis Department 2011 $
% 
%

\end{lstlisting}
\fontfamily{\familydefault}
\selectfont
\end{enumerate}
\subsection{/+gras/+geom/+ell/+test}
\begin{enumerate}
\item\hypertarget{/+gras/+geom/+ell/+test\run\_tests}{run\_tests}
\fontfamily{pcr}
\selectfont
\begin{lstlisting}
%

\end{lstlisting}
\fontfamily{\familydefault}
\selectfont
\end{enumerate}
\subsection{/+gras/+geom/+ell/+test/+mlunit}
\begin{enumerate}
\item\hypertarget{/+gras/+geom/+ell/+test/+mlunit\SuiteBasic}{SuiteBasic}
\fontfamily{pcr}
\selectfont
\begin{lstlisting}
%

\end{lstlisting}
\fontfamily{\familydefault}
\selectfont
\end{enumerate}
\subsection{/+gras/+geom/+sup}
\begin{enumerate}
\item\hypertarget{/+gras/+geom/+sup\sup2boundary2}{sup2boundary2}
\fontfamily{pcr}
\selectfont
\begin{lstlisting}
%  SUP2BOUNDARY2 approximates aMat boundary of 3d set using aMat support
%  function values defined for the directions from aMat triangulated unit
%  sphere
% 
%  Input:
%    regular:
%        dirMat: double[nDirs,2] - directions for on which support function
%            is defined
%        supVec: double[nDirs,1] - support function values
% 
%  Output:
%    xBoundMat: double[nFaces,2] - approximated coordinates of points 
%        on set boundary
% 
%  $Author: Peter Gagarinov  <pgagarinov@gmail.com> $	$Date: 2011-05-30 $ 
%  $Copyright: Moscow State University,
%             Faculty of Computational Mathematics and Computer Science,
%             System Analysis Department 2011 $
% 
%

\end{lstlisting}
\fontfamily{\familydefault}
\selectfont
\item\hypertarget{/+gras/+geom/+sup\sup2boundary3}{sup2boundary3}
\fontfamily{pcr}
\selectfont
\begin{lstlisting}
%  SUP2BOUNDARY3 approximates aMat boundary of 3d set using aMat support
%  function values defined for the directions from aMat triangulated unit
%  sphere
% 
%  Input:
%    regular:
%        dirMat: double[nDirs,3] - directions for on which support function
%            is defined
%        supVec: double[nDirs,1] - support function values
% 
%        faceMat: double[nFaces,3] - faces composing aMat triangulation of aMat
%            unit sphere on which dirMat is defined
% 
%  Output:
%    xBoundMat: double[nFaces,3] - approximated coordinates of points 
%        on set boundary
% 
%  $Author: Peter Gagarinov  <pgagarinov@gmail.com> $	$Date: 2011-05-30$ 
%  $Copyright: Moscow State University,
%             Faculty of Computational Mathematics and Computer Science,
%             System Analysis Department 2011 $
% 
%

\end{lstlisting}
\fontfamily{\familydefault}
\selectfont
\item\hypertarget{/+gras/+geom/+sup\supgeomdiff2d}{supgeomdiff2d}
\fontfamily{pcr}
\selectfont
\begin{lstlisting}
%  SUPGEOMDIFF2D calculates support function of two 2-dimensional
%  convex sets defined by their support functions
% 
%  Input:
%    rho1Vec: double [1,nDirs] - support function values for the
%        first set
%    rho2Vec: double [1,nDirs] - support function values for the
%        second set
%    lMat: double[nDims,nDirs] - set of directions for which the support
%        functions of two sets are defined
%  
%  Output:
%    rhoDiffVec: double[1,nDirs] - support function values for the geometric
%        difference of two sets
% 
%  $Author: Peter Gagarinov  <pgagarinov@gmail.com> $	$Date: 2013-01-22 $ 
%  $Copyright: Moscow State University,
%             Faculty of Computational Mathematics and Computer Science,
%             System Analysis Department 2013 $
% 
%

\end{lstlisting}
\fontfamily{\familydefault}
\selectfont
\end{enumerate}
\subsection{/+gras/+geom/+sup/+test}
\begin{enumerate}
\item\hypertarget{/+gras/+geom/+sup/+test\qint2}{qint2}
\fontfamily{pcr}
\selectfont
\begin{lstlisting}
%

\end{lstlisting}
\fontfamily{\familydefault}
\selectfont
\item\hypertarget{/+gras/+geom/+sup/+test\run\_tests}{run\_tests}
\fontfamily{pcr}
\selectfont
\begin{lstlisting}
%

\end{lstlisting}
\fontfamily{\familydefault}
\selectfont
\end{enumerate}
\subsection{/+gras/+geom/+sup/+test/+mlunit}
\begin{enumerate}
\item\hypertarget{/+gras/+geom/+sup/+test/+mlunit\SuiteBasic}{SuiteBasic}
\fontfamily{pcr}
\selectfont
\begin{lstlisting}
%

\end{lstlisting}
\fontfamily{\familydefault}
\selectfont
\end{enumerate}
\subsection{/+gras/+geom/+test}
\begin{enumerate}
\item\hypertarget{/+gras/+geom/+test\run\_tests}{run\_tests}
\fontfamily{pcr}
\selectfont
\begin{lstlisting}
%

\end{lstlisting}
\fontfamily{\familydefault}
\selectfont
\end{enumerate}
\subsection{/+gras/+geom/+tri}
\begin{enumerate}
\item\hypertarget{/+gras/+geom/+tri\elltubetri}{elltubetri}
\fontfamily{pcr}
\selectfont
\begin{lstlisting}
%  ELLTUBETRI builds a triangulation of ellipsoidal tube
% 
%  Input:
%    regular:
%        QArray: double[nDims,nDims,nTimes] - array of ellipsoidal
%            configuration matrices
%        aMat: double[nDims,nTimes] - array of ellipsoidal centers
%            timeVec: double[1,nDims] - time vector
%        timeVec: double[1,nTimes] - time vector for ellipsoidal tube
%        nSPoints: double[1,1] - number of points used for partitioning
%            [0,2pi] range when building tube triangulation
% 
%  Output:
%    vMat: double[nVertices,3] - array of vertex coordinates, the first
%    dimension is time.
%    fMat: double[nFaces,3] - face corners specified as row numbers in vMat
% 
%  $Author: Peter Gagarinov  <pgagarinov@gmail.com> $	$Date: 2009-07 $
%  $Copyright: Moscow State University,
%             Faculty of Computational Mathematics and Computer Science,
%             System Analysis Department 2011 $
%

\end{lstlisting}
\fontfamily{\familydefault}
\selectfont
\item\hypertarget{/+gras/+geom/+tri\icosahedron}{icosahedron}
\fontfamily{pcr}
\selectfont
\begin{lstlisting}
%  ICOSAHEDRON generates a triangulation corresponding to Icosahedron's
%  surface
% 
%  Input:
%    none
% 
%  Output:
%    vMat: double[12,3] - vertices of the Icosahedron
%    fMat: double[20,3] - indices matrix of face definitions
% 
%  $Author: Peter Gagarinov  <pgagarinov@gmail.com> $	$Date: 2011-05-27$ 
%  $Copyright: Moscow State University,
%             Faculty of Computational Mathematics and Computer Science,
%             System Analysis Department 2011 $
% 
%

\end{lstlisting}
\fontfamily{\familydefault}
\selectfont
\item\hypertarget{/+gras/+geom/+tri\isface}{isface}
\fontfamily{pcr}
\selectfont
\begin{lstlisting}
%  ISFACE checks if the specified faces belong to the given triangulation
% 
%  Input:
%    regular:
%        vMat: double[nVerts,3] - vertex coordinates
%        fMat: double[nFaces,3] - face definitions based on vertex numbers
%        fToCheckMat: double[nCheckFaces,3] - definitions of faces which
%           beloning is checked
% 
%  Output:
%    isFaceVec: logical[nCheckFaces,1] - contains true for the corresponding
%        face from fToCheckMat if the face belongs to the triangulation and
%        false otherwise
% 
%  $Author: Peter Gagarinov  <pgagarinov@gmail.com> $	$Date: 2011-05-27$ 
%  $Copyright: Moscow State University,
%             Faculty of Computational Mathematics and Computer Science,
%             System Analysis Department 2011 $
%

\end{lstlisting}
\fontfamily{\familydefault}
\selectfont
\item\hypertarget{/+gras/+geom/+tri\istriequal}{istriequal}
\fontfamily{pcr}
\selectfont
\begin{lstlisting}
%  ISTRIEQUAL checks if the matrices specify the same triangulation
%  (permunations of edge orders directions, vertices in faces do not count)
%  
%  Input:
%    regular:
%        v1Mat: double[n1Verts,3]
%        f1Mat: double[n1Faces,3]
%        v2Mat: double[n1Verts,3]
%        f2Mat: double[n1Faces,3]
%        maxTol: double[1,1]
% 
%  Output:
%    isPos: logical[1,1] - specifies if result of comparison is true
%    reportStr: char[1,] - describes a reason of negative result
% 
%  $Author: Peter Gagarinov  <pgagarinov@gmail.com> $	$Date: 2011-05-27$ 
%  $Copyright: Moscow State University,
%             Faculty of Computational Mathematics and Computer Science,
%             System Analysis Department 2011 $
% 
%

\end{lstlisting}
\fontfamily{\familydefault}
\selectfont
\item\hypertarget{/+gras/+geom/+tri\mapface2edge}{mapface2edge}
\fontfamily{pcr}
\selectfont
\begin{lstlisting}
%  MAPFACE2EDGE creates a mapping from faces to edges 
% 
%  Input:
%    regular:
%        vMat: double[nVerts,3] - coordinates of vertices
%        fMat: double[nFace,3] - indices of face vertices in vMat  
% 
%  Output:  
%    eMat: double[nEdges,2] - contains indices of vertices corresponding
%        to each edge
% 
%    f2eMat: double[nFaces,3] - contains indices of edges for
%        each face in this order (1-2, 2-3, 1-3)
%                
%    f2eIsDirMat: logical[nFaces,3] - contains true if face
%        references edge in a direct order (i.e. 1-2 for instance)
%        and false if reference is in an opposite order
% 
%  $Author: Peter Gagarinov  <pgagarinov@gmail.com> $	$Date: 2011-05-27$ 
%  $Copyright: Moscow State University,
%             Faculty of Computational Mathematics and Computer Science,
%             System Analysis Department 2011 $
% 
%

\end{lstlisting}
\fontfamily{\familydefault}
\selectfont
\item\hypertarget{/+gras/+geom/+tri\shrinkfacetri}{shrinkfacetri}
\fontfamily{pcr}
\selectfont
\begin{lstlisting}
%  SHRINKFACETRI shrinks faces of 3D triangulation space down to a
%  degree where a length of each edge is less or equal to a specified value
% 
%  Input:
%    regular:
%        vMat: double[nVerts,3] - coordinates of vertices
%        fMat: double[nFace,3] - indices of face vertices in vMat
%        maxEdgeLength: double[1,1] - maximum allowed edge length in the
%           resulting triangulation
% 
%    optional:
%        nMaxSteps: double[1,1] - maximum allowed step number - edge
%           shrinking is an iterative process so this parameter limits
%           a number of steps, by default the number of steps is unlimited
%           (nMaxSteps=Inf)
%        fVertAdjustFunc: function_handle[1,1] - function responsible for
%           transforming vertices on each step, the function accept vMat and
%           output vMat, by default the function is @deal (i.e. no
%           transformation is performed)
%  Output:
%        vMat: double[nNewVerts,3] - coordinates of vertices in the
%            resulting triangulation
%        fMat: double[nNewFace,3] - indices of face vertices in vMat
%            indices of face vertices in the resulting triangulation
% 
%        SStats: struct[1,1] - structure containing the statistics
%            related to the face shrink process, includes the following
%            fields:
%                nSteps: double[1,1] - number of performed steps
%                nVertVec: double[nSteps+1,1] - number of vertices on each
%                   step
%                nFaceVec: double[nSteps+1,1] - numbers of faces on each 
%                    step, the first item corresponds to zero step
%                nEdgeVec: double[nSteps+1,1] - numbers of edges on each 
%                    step
%                nEdgesToShrinkVec: double[nSteps+1,1] - numbers of edges to
%                   shrink on each step
%                maxEdgeLengthVec: double[nSteps+1,1] - vector of maximum
%                   edge lengths on each step
% 
%        eMat: double[nEdges,2] - contains indices of vertices corresponding
%           to each edge
% 
%        f2eMat: double[nFaces,3] - contains indices of edges for
%            each face in this order (1-2, 2-3, 1-3)
%                
%        f2eIsDirMat: logical[nFaces,3] - contains true if face
%            references edge in a direct order (i.e. 1-2 for instance)
%            and false if reference is in an opposite order
% 
%  $Author: Peter Gagarinov  <pgagarinov@gmail.com> $	$Date: 2011-05-28$ 
%  $Copyright: Moscow State University,
%             Faculty of Computational Mathematics and Computer Science,
%             System Analysis Department 2011 $
% 
%

\end{lstlisting}
\fontfamily{\familydefault}
\selectfont
\item\hypertarget{/+gras/+geom/+tri\spheretri}{spheretri}
\fontfamily{pcr}
\selectfont
\begin{lstlisting}
%  SPHERETRI builds a triangulation of a unit sphere based on recursive
%  partitioning each of Icosahedron faces into 4 triangles with vertices in
%  the middles of original face edgeMidMat
% 
%  Input:
%    depth: double[1,1] - depth of partitioning, use 1 for the first level of
%        Icosahedron partitioning, and greater value for a greater level
%        of partitioning
% 
%  Output:
%    vMat: double[nVerts,3] - (x,y,z) coordinates of triangulation
%        vertices
%    fMat: double[nFaces,3] - indices of face verties in vertMat
% 
%  $Author: Peter Gagarinov  <pgagarinov@gmail.com> $	$Date: 2011-05-27$ 
%  $Copyright: Moscow State University,
%             Faculty of Computational Mathematics and Computer Science,
%             System Analysis Department 2011 $
% 
%

\end{lstlisting}
\fontfamily{\familydefault}
\selectfont
\end{enumerate}
\subsection{/+gras/+geom/+tri/+test}
\begin{enumerate}
\item\hypertarget{/+gras/+geom/+tri/+test\run\_tests}{run\_tests}
\fontfamily{pcr}
\selectfont
\begin{lstlisting}
%

\end{lstlisting}
\fontfamily{\familydefault}
\selectfont
\item\hypertarget{/+gras/+geom/+tri/+test\spheretri}{spheretri}
\fontfamily{pcr}
\selectfont
\begin{lstlisting}
%  SPHERETRI builds a triangulation of a unit sphere based on recursive
%  partitioning each of Icosahedron faces into 4 triangles with vertices in
%  the middles of original face edgeMidMat
% 
%  Input:
%    depth: double[1,1] - depth of partitioning, use 1 for the first level of
%        Icosahedron partitioning, and greater value for a greater level
%        of partitioning
% 
%  Output:
%    vertMat: double[nVerts,3] - (x,y,z) coordinates of triangulation
%        vertices
%    faceMat: double[nFaces,3] - indices of face verties in vertMat
% 
%    edgeMidMat: double[nPrevStepEdges,3] - two first columns contain
%        indices of edges from the previous level of partitioning (for depth=1
%        it will be indices of Icosahedron edges) and the third column will
%        contain indices of the middles of these edges. Thus number of
%        actual edges is 4 * nPrevStepEdges
% 
%  $Author: Peter Gagarinov  <pgagarinov@gmail.com> $	$Date: 2011-05-21$ 
%  $Copyright: Moscow State University,
%             Faculty of Computational Mathematics and Computer Science,
%             System Analysis Department 2011 $
%

\end{lstlisting}
\fontfamily{\familydefault}
\selectfont
\end{enumerate}
\subsection{/+gras/+geom/+tri/+test/+mlunit}
\begin{enumerate}
\item\hypertarget{/+gras/+geom/+tri/+test/+mlunit\SuiteTri}{SuiteTri}
\fontfamily{pcr}
\selectfont
\begin{lstlisting}
%

\end{lstlisting}
\fontfamily{\familydefault}
\selectfont
\end{enumerate}
\subsection{/+gras/+geom/+tri/+test/srebuild3d}
\begin{enumerate}
\item\hypertarget{/+gras/+geom/+tri/+test/srebuild3d\build}{build}
\fontfamily{pcr}
\selectfont
\begin{lstlisting}
%

\end{lstlisting}
\fontfamily{\familydefault}
\selectfont
\end{enumerate}
\subsection{/+gras/+interp}
\begin{enumerate}
\item\hypertarget{/+gras/+interp\AMatrixCubicSpline}{AMatrixCubicSpline}
\fontfamily{pcr}
\selectfont
\begin{lstlisting}
%  $Author: Peter Gagarinov  <pgagarinov@gmail.com> $	$Date: 2011-08$
%  $Copyright: Moscow State University,
%             Faculty of Computational Mathematics and Computer Science,
%             System Analysis Department 2011 $
% 
%

\end{lstlisting}
\fontfamily{\familydefault}
\selectfont
\item\hypertarget{/+gras/+interp\MatrixColCubicSpline}{MatrixColCubicSpline}
\fontfamily{pcr}
\selectfont
\begin{lstlisting}
%  $Author: Peter Gagarinov  <pgagarinov@gmail.com> $	$Date: 2011-08$
%  $Copyright: Moscow State University,
%             Faculty of Computational Mathematics and Computer Science,
%             System Analysis Department 2011 $
% 
%

\end{lstlisting}
\fontfamily{\familydefault}
\selectfont
\item\hypertarget{/+gras/+interp\MatrixColTriuCubicSpline}{MatrixColTriuCubicSpline}
\fontfamily{pcr}
\selectfont
\begin{lstlisting}
%  $Author: Peter Gagarinov  <pgagarinov@gmail.com> $	$Date: 2011-08$
%  $Copyright: Moscow State University,
%             Faculty of Computational Mathematics and Computer Science,
%             System Analysis Department 2011 $
%     
%

\end{lstlisting}
\fontfamily{\familydefault}
\selectfont
\item\hypertarget{/+gras/+interp\MatrixColTriuSymmCubicSpline}{MatrixColTriuSymmCubicSpline}
\fontfamily{pcr}
\selectfont
\begin{lstlisting}
%  $Author: Peter Gagarinov  <pgagarinov@gmail.com> $	$Date: 2011-08$
%  $Copyright: Moscow State University,
%             Faculty of Computational Mathematics and Computer Science,
%             System Analysis Department 2011 $
%     
%

\end{lstlisting}
\fontfamily{\familydefault}
\selectfont
\item\hypertarget{/+gras/+interp\MatrixInterpolantFactory}{MatrixInterpolantFactory}
\fontfamily{pcr}
\selectfont
\begin{lstlisting}
%

\end{lstlisting}
\fontfamily{\familydefault}
\selectfont
\item\hypertarget{/+gras/+interp\MatrixRowCubicSpline}{MatrixRowCubicSpline}
\fontfamily{pcr}
\selectfont
\begin{lstlisting}
%  $Author: Peter Gagarinov  <pgagarinov@gmail.com> $	$Date: 2011-08$
%  $Copyright: Moscow State University,
%             Faculty of Computational Mathematics and Computer Science,
%             System Analysis Department 2011 $
%     
%

\end{lstlisting}
\fontfamily{\familydefault}
\selectfont
\item\hypertarget{/+gras/+interp\NNDefMatCholCubicSpline}{NNDefMatCholCubicSpline}
\fontfamily{pcr}
\selectfont
\begin{lstlisting}
%  $Author: Peter Gagarinov  <pgagarinov@gmail.com> $	$Date: 2011-10$
%  $Copyright: Moscow State University,
%             Faculty of Computational Mathematics and Computer Science,
%             System Analysis Department 2011 $    
%

\end{lstlisting}
\fontfamily{\familydefault}
\selectfont
\item\hypertarget{/+gras/+interp\PosDefMatCholCubicSpline}{PosDefMatCholCubicSpline}
\fontfamily{pcr}
\selectfont
\begin{lstlisting}
%  $Author: Peter Gagarinov  <pgagarinov@gmail.com> $	$Date: 2011-08$
%  $Copyright: Moscow State University,
%             Faculty of Computational Mathematics and Computer Science,
%             System Analysis Department 2011 $    
%

\end{lstlisting}
\fontfamily{\familydefault}
\selectfont
\item\hypertarget{/+gras/+interp\SplineMatrixOperations}{SplineMatrixOperations}
\fontfamily{pcr}
\selectfont
\begin{lstlisting}
%

\end{lstlisting}
\fontfamily{\familydefault}
\selectfont
\end{enumerate}
\subsection{/+gras/+interp/+test}
\begin{enumerate}
\item\hypertarget{/+gras/+interp/+test\run\_tests}{run\_tests}
\fontfamily{pcr}
\selectfont
\begin{lstlisting}
%

\end{lstlisting}
\fontfamily{\familydefault}
\selectfont
\end{enumerate}
\subsection{/+gras/+interp/+test/+mlunit}
\begin{enumerate}
\item\hypertarget{/+gras/+interp/+test/+mlunit\SuiteBasic}{SuiteBasic}
\fontfamily{pcr}
\selectfont
\begin{lstlisting}
%

\end{lstlisting}
\fontfamily{\familydefault}
\selectfont
\end{enumerate}
\subsection{/+gras/+la}
\begin{enumerate}
\item\hypertarget{/+gras/+la\ismatposdef}{ismatposdef}
\fontfamily{pcr}
\selectfont
\begin{lstlisting}
%  ISMATPOSDEF  checks if qMat is positive definite
% 
%  Input:
%    regular:
%        qMat: double[nDims, nDims] - inpute matrix
%        absTol: double - precision
% 
%    optional:
%        isFlagSemDefOn: logical[1,1] - if true than qMat is checked for 
%                    positive semi-definiteness
%  Output:
%    isPosDef: logical[1,1] - true iff matrix is positive definite
%  
% 
%  $Author: Vitaly Baranov  <vetbar42@gmail.com> $	$Date: 2013-01-Mar$
%  $Copyright: Lomonosov Moscow State University,
%              Faculty of Computational Mathematics and Cybernetics,
%              Department of System Analysis  2013 $
% 
% 
%

\end{lstlisting}
\fontfamily{\familydefault}
\selectfont
\item\hypertarget{/+gras/+la\ismatsymm}{ismatsymm}
\fontfamily{pcr}
\selectfont
\begin{lstlisting}
%  ISMATSYMM  checks if qMat is symmetric
% 
%  Input:
% 	regular:
%        qMat: double[nDims, nDims]
% 
%  Output:
%    isSymm: logical[1,1] - indicates whether a matrix is symmetric.
%  
% 
%  $Author: Rustam Guliev  <glvrst@gmail.com> $	$Date: 2012-16-11$
%  $Copyright: Moscow State University,
%             Faculty of Computational Mathematics and Cybernetics,
%             System Analysis Department 2012 $
% 
%

\end{lstlisting}
\fontfamily{\familydefault}
\selectfont
\item\hypertarget{/+gras/+la\matorth}{matorth}
\fontfamily{pcr}
\selectfont
\begin{lstlisting}
%  MATORTH generates an orthogonal matrix that contains in its first k
%  columns orthogonalized vectors specified on input as [n,k] matrix
% 
%  Input:
%    regular:
%        srcMat: double[nDims,1]
% 
%  Output:
%    oMat: double[nDims,nDims]
% 
%  $Author: Peter Gagarinov  <pgagarinov@gmail.com> $	$Date: 2012-06-25$
%  $Copyright: Moscow State University,
%             Faculty of Computational Mathematics and Computer Science,
%             System Analysis Department 2012 $
% 
%

\end{lstlisting}
\fontfamily{\familydefault}
\selectfont
\item\hypertarget{/+gras/+la\mlorthtransl}{mlorthtransl}
\fontfamily{pcr}
\selectfont
\begin{lstlisting}
%  MLORTHTRANSL generates a set of orthogonal matrices that translate each of
%  the given vectors into a corresponding another vector from another set
% 
%  Input:
%    regular:
%        srcMat: double[nDims,nVecs]
%        dstArray: double[nDims,nVecs,nElems]
% 
%  Output:
%    oArr: double[nDims,nDims,nElems,nVecs]
% 
%  $Author: Peter Gagarinov  <pgagarinov@gmail.com> $	$Date: 2011-05-01$
%  $Copyright: Moscow State University,
%             Faculty of Computational Mathematics and Computer Science,
%             System Analysis Department 2011 $
% 
%

\end{lstlisting}
\fontfamily{\familydefault}
\selectfont
\item\hypertarget{/+gras/+la\orthtransl}{orthtransl}
\fontfamily{pcr}
\selectfont
\begin{lstlisting}
%  ORTHTRANSL generates an orthogonal matrix that translates a specified
%  vector to another vector that is collinear to the second specified vector
% 
%  Input:
%    regular:
%        srcVec: double[nDims,1]
%        dstVec: double[nDims,1]
% 
%  Output:
%    oMat: double[nDims,nDims]
% 
%  $Author: Peter Gagarinov  <pgagarinov@gmail.com> $	$Date: 2012-11-28$
%  $Copyright: Moscow State University,
%             Faculty of Computational Mathematics and Computer Science,
%             System Analysis Department 2012 $
% 
%

\end{lstlisting}
\fontfamily{\familydefault}
\selectfont
\item\hypertarget{/+gras/+la\orthtranslhaus}{orthtranslhaus}
\fontfamily{pcr}
\selectfont
\begin{lstlisting}
%  ORTHTRANSLHAUS generates an orthogonal matrix that translates a specified
%  vector to another vector that is collinear to the second specified vector
%  using the Hausholder method:
%    w=srcVec-dstVec;
%    oMat=I-2*w*w.'./(w.'*w)
% 
%  Input:
%    regular:
%        srcVec: double[nDims,1]
%        dstVec: double[nDims,1]
% 
%  Output:
%    oMat: double[nDims,nDims]
%  $Author: Peter Gagarinov  <pgagarinov@gmail.com> $	$Date: 2011-05-15$
%  $Copyright: Moscow State University,
%             Faculty of Computational Mathematics and Computer Science,
%             System Analysis Department 2011 $
% 
%

\end{lstlisting}
\fontfamily{\familydefault}
\selectfont
\item\hypertarget{/+gras/+la\orthtranslmaxdir}{orthtranslmaxdir}
\fontfamily{pcr}
\selectfont
\begin{lstlisting}
%  ORTHTRANSLMAXDIR generates an orthogonal matrix oMat that translates
%  vector srcVec to another vector that is collinear to the second 
%  specified vector dstVec. The matrix is chosen to maximize 
%  (oMat*srcMaxVec,dstMaxVec)
% 
%  Input:
%    regular:
%        srcVec: double[nDims,1]
%        dstVec: double[nDims,1]
%        srcMaxVec: double[nDims,1]
%        dstMaxVec: double[nDims,1]
% 
%  Output:
%    oMat: double[nDims,nDims]
% 
%  References: see
% 
%  ISSN 0278-6419, Moscow University Computational Mathematics and Cybernetics, 
%  2007, Vol. 31, No. 1, pp. 11�20. � Allerton Press, Inc., 2007.
% 
%  "Computation of Projections of Reachability Tubes of Linear
%  Controlled Systems Based on Ellipsoidal Calculus Techniques"
%  P. V. Gagarinov
%  
%  $Author: Peter Gagarinov  <pgagarinov@gmail.com> $	$Date: 2011-05-03$
%  $Copyright: Moscow State University,
%             Faculty of Computational Mathematics and Computer Science,
%             System Analysis Department 2011 $
% 
%

\end{lstlisting}
\fontfamily{\familydefault}
\selectfont
\item\hypertarget{/+gras/+la\orthtranslmaxtr}{orthtranslmaxtr}
\fontfamily{pcr}
\selectfont
\begin{lstlisting}
%  ORTHTRANSLMAXVOL generates an orthogonal matrix oMat that translates 
%  a specified vector srcVec to another vector that is collinear to 
%  the second specified vector dstVec
%  The matrix S is chosen to maximize Tr(oMat*maxMat) where maxMat
%  is specified
% 
%  Input:
%    regular:
%        srcVec: double[nDims,1]
%        dstVec: double[nDims,1]
%        maxMat: double[nDims,nDims]
% 
%  Output:
%    oMat: double[nDims,nDims]
% 
%  References: see
% 
%  ISSN 0278-6419, Moscow University Computational Mathematics and Cybernetics, 
%  2007, Vol. 31, No. 1, pp. 11�20. � Allerton Press, Inc., 2007.
% 
%  "Computation of Projections of Reachability Tubes of Linear
%  Controlled Systems Based on Ellipsoidal Calculus Techniques"
%  P. V. Gagarinov
%  
%  $Author: Peter Gagarinov  <pgagarinov@gmail.com> $	$Date: 2011-05-03$
%  $Copyright: Moscow State University,
%             Faculty of Computational Mathematics and Computer Science,
%             System Analysis Department 2011 $
% 
%

\end{lstlisting}
\fontfamily{\familydefault}
\selectfont
\item\hypertarget{/+gras/+la\sqrtm}{sqrtm}
\fontfamily{pcr}
\selectfont
\begin{lstlisting}
%  SQRTM generates a square root from matrix QMat 
%  Input:
%       QMat: double[nDims, nDims]
%  Output:
%    QsqrtMat: double[nDims,nDims]
% 
%  
%  $Author: Vadim Kaushanskiy  <vkaushanskiy@gmail.com> $	$Date: 2012-01-11$
%  $Copyright: Moscow State University,
%             Faculty of Computational Mathematics and Cybernetics,
%             System Analysis Department 2012 $
%

\end{lstlisting}
\fontfamily{\familydefault}
\selectfont
\end{enumerate}
\subsection{/+gras/+la/+test}
\begin{enumerate}
\item\hypertarget{/+gras/+la/+test\run\_tests}{run\_tests}
\fontfamily{pcr}
\selectfont
\begin{lstlisting}
%

\end{lstlisting}
\fontfamily{\familydefault}
\selectfont
\end{enumerate}
\subsection{/+gras/+la/+test/+mlunit}
\begin{enumerate}
\item\hypertarget{/+gras/+la/+test/+mlunit\BasicTestCase}{BasicTestCase}
\fontfamily{pcr}
\selectfont
\begin{lstlisting}
%  $Author: Vadim Kaushanskiy, Moscow State University by M.V. Lomonosov,
%  Faculty of Computational Mathematics and Cybernetics, System Analysis
%  Department, 1-November-2012, <vkaushanskiy@gmail.com>$
%

\end{lstlisting}
\fontfamily{\familydefault}
\selectfont
\item\hypertarget{/+gras/+la/+test/+mlunit\SuiteOrthTransl}{SuiteOrthTransl}
\fontfamily{pcr}
\selectfont
\begin{lstlisting}
%

\end{lstlisting}
\fontfamily{\familydefault}
\selectfont
\end{enumerate}
\subsection{/+gras/+mat}
\begin{enumerate}
\item\hypertarget{/+gras/+mat\AConstMatrixFunction}{AConstMatrixFunction}
\fontfamily{pcr}
\selectfont
\begin{lstlisting}
%

\end{lstlisting}
\fontfamily{\familydefault}
\selectfont
\item\hypertarget{/+gras/+mat\AMatrixBinaryOpFunc}{AMatrixBinaryOpFunc}
\fontfamily{pcr}
\selectfont
\begin{lstlisting}
%

\end{lstlisting}
\fontfamily{\familydefault}
\selectfont
\item\hypertarget{/+gras/+mat\AMatrixOpFunc}{AMatrixOpFunc}
\fontfamily{pcr}
\selectfont
\begin{lstlisting}
%

\end{lstlisting}
\fontfamily{\familydefault}
\selectfont
\item\hypertarget{/+gras/+mat\AMatrixOperations}{AMatrixOperations}
\fontfamily{pcr}
\selectfont
\begin{lstlisting}
%

\end{lstlisting}
\fontfamily{\familydefault}
\selectfont
\item\hypertarget{/+gras/+mat\AMatrixTernaryOpFunc}{AMatrixTernaryOpFunc}
\fontfamily{pcr}
\selectfont
\begin{lstlisting}
%

\end{lstlisting}
\fontfamily{\familydefault}
\selectfont
\item\hypertarget{/+gras/+mat\AMatrixUnaryOpFunc}{AMatrixUnaryOpFunc}
\fontfamily{pcr}
\selectfont
\begin{lstlisting}
%

\end{lstlisting}
\fontfamily{\familydefault}
\selectfont
\item\hypertarget{/+gras/+mat\CompositeMatrixOperations}{CompositeMatrixOperations}
\fontfamily{pcr}
\selectfont
\begin{lstlisting}
%

\end{lstlisting}
\fontfamily{\familydefault}
\selectfont
\item\hypertarget{/+gras/+mat\ConstMatrixFunctionFactory}{ConstMatrixFunctionFactory}
\fontfamily{pcr}
\selectfont
\begin{lstlisting}
%

\end{lstlisting}
\fontfamily{\familydefault}
\selectfont
\item\hypertarget{/+gras/+mat\IMatrixFunction}{IMatrixFunction}
\fontfamily{pcr}
\selectfont
\begin{lstlisting}
%

\end{lstlisting}
\fontfamily{\familydefault}
\selectfont
\item\hypertarget{/+gras/+mat\IMatrixOperations}{IMatrixOperations}
\fontfamily{pcr}
\selectfont
\begin{lstlisting}
%

\end{lstlisting}
\fontfamily{\familydefault}
\selectfont
\item\hypertarget{/+gras/+mat\MatrixOperationsFactory}{MatrixOperationsFactory}
\fontfamily{pcr}
\selectfont
\begin{lstlisting}
%

\end{lstlisting}
\fontfamily{\familydefault}
\selectfont
\end{enumerate}
\subsection{/+gras/+mat/+fcnlib}
\begin{enumerate}
\item\hypertarget{/+gras/+mat/+fcnlib\ConstColFunction}{ConstColFunction}
\fontfamily{pcr}
\selectfont
\begin{lstlisting}
%

\end{lstlisting}
\fontfamily{\familydefault}
\selectfont
\item\hypertarget{/+gras/+mat/+fcnlib\ConstMatrixFunction}{ConstMatrixFunction}
\fontfamily{pcr}
\selectfont
\begin{lstlisting}
%

\end{lstlisting}
\fontfamily{\familydefault}
\selectfont
\item\hypertarget{/+gras/+mat/+fcnlib\ConstRowFunction}{ConstRowFunction}
\fontfamily{pcr}
\selectfont
\begin{lstlisting}
%

\end{lstlisting}
\fontfamily{\familydefault}
\selectfont
\item\hypertarget{/+gras/+mat/+fcnlib\MatrixBinaryTimesFunc}{MatrixBinaryTimesFunc}
\fontfamily{pcr}
\selectfont
\begin{lstlisting}
%

\end{lstlisting}
\fontfamily{\familydefault}
\selectfont
\item\hypertarget{/+gras/+mat/+fcnlib\MatrixExpFunc}{MatrixExpFunc}
\fontfamily{pcr}
\selectfont
\begin{lstlisting}
%

\end{lstlisting}
\fontfamily{\familydefault}
\selectfont
\item\hypertarget{/+gras/+mat/+fcnlib\MatrixExpTimeFunc}{MatrixExpTimeFunc}
\fontfamily{pcr}
\selectfont
\begin{lstlisting}
%

\end{lstlisting}
\fontfamily{\familydefault}
\selectfont
\item\hypertarget{/+gras/+mat/+fcnlib\MatrixInvFunc}{MatrixInvFunc}
\fontfamily{pcr}
\selectfont
\begin{lstlisting}
%

\end{lstlisting}
\fontfamily{\familydefault}
\selectfont
\item\hypertarget{/+gras/+mat/+fcnlib\MatrixLRDivideVecFunc}{MatrixLRDivideVecFunc}
\fontfamily{pcr}
\selectfont
\begin{lstlisting}
%

\end{lstlisting}
\fontfamily{\familydefault}
\selectfont
\item\hypertarget{/+gras/+mat/+fcnlib\MatrixLRTimesFunc}{MatrixLRTimesFunc}
\fontfamily{pcr}
\selectfont
\begin{lstlisting}
%

\end{lstlisting}
\fontfamily{\familydefault}
\selectfont
\item\hypertarget{/+gras/+mat/+fcnlib\MatrixMakeSymmetricFunc}{MatrixMakeSymmetricFunc}
\fontfamily{pcr}
\selectfont
\begin{lstlisting}
%

\end{lstlisting}
\fontfamily{\familydefault}
\selectfont
\item\hypertarget{/+gras/+mat/+fcnlib\MatrixMinEigValFunc}{MatrixMinEigValFunc}
\fontfamily{pcr}
\selectfont
\begin{lstlisting}
%

\end{lstlisting}
\fontfamily{\familydefault}
\selectfont
\item\hypertarget{/+gras/+mat/+fcnlib\MatrixMinusFunc}{MatrixMinusFunc}
\fontfamily{pcr}
\selectfont
\begin{lstlisting}
%

\end{lstlisting}
\fontfamily{\familydefault}
\selectfont
\item\hypertarget{/+gras/+mat/+fcnlib\MatrixPInvFunc}{MatrixPInvFunc}
\fontfamily{pcr}
\selectfont
\begin{lstlisting}
%

\end{lstlisting}
\fontfamily{\familydefault}
\selectfont
\item\hypertarget{/+gras/+mat/+fcnlib\MatrixPlusFunc}{MatrixPlusFunc}
\fontfamily{pcr}
\selectfont
\begin{lstlisting}
%

\end{lstlisting}
\fontfamily{\familydefault}
\selectfont
\item\hypertarget{/+gras/+mat/+fcnlib\MatrixSqrtFunc}{MatrixSqrtFunc}
\fontfamily{pcr}
\selectfont
\begin{lstlisting}
%

\end{lstlisting}
\fontfamily{\familydefault}
\selectfont
\item\hypertarget{/+gras/+mat/+fcnlib\MatrixTernaryTimesFunc}{MatrixTernaryTimesFunc}
\fontfamily{pcr}
\selectfont
\begin{lstlisting}
%

\end{lstlisting}
\fontfamily{\familydefault}
\selectfont
\item\hypertarget{/+gras/+mat/+fcnlib\MatrixTransposeFunc}{MatrixTransposeFunc}
\fontfamily{pcr}
\selectfont
\begin{lstlisting}
%

\end{lstlisting}
\fontfamily{\familydefault}
\selectfont
\item\hypertarget{/+gras/+mat/+fcnlib\MatrixTriuFunc}{MatrixTriuFunc}
\fontfamily{pcr}
\selectfont
\begin{lstlisting}
%

\end{lstlisting}
\fontfamily{\familydefault}
\selectfont
\item\hypertarget{/+gras/+mat/+fcnlib\QuadraticFormSqrtFunc}{QuadraticFormSqrtFunc}
\fontfamily{pcr}
\selectfont
\begin{lstlisting}
%

\end{lstlisting}
\fontfamily{\familydefault}
\selectfont
\end{enumerate}
\subsection{/+gras/+mat/+symb}
\begin{enumerate}
\item\hypertarget{/+gras/+mat/+symb\MatrixSFBinaryProd}{MatrixSFBinaryProd}
\fontfamily{pcr}
\selectfont
\begin{lstlisting}
%  $Author: Peter Gagarinov  <pgagarinov@gmail.com> $	$Date: 2011-12-12$
%  $Copyright: Moscow State University,
%             Faculty of Computational Mathematics and Computer Science,
%             System Analysis Department 2011 $
%     
%

\end{lstlisting}
\fontfamily{\familydefault}
\selectfont
\item\hypertarget{/+gras/+mat/+symb\MatrixSFBinaryProdByVec}{MatrixSFBinaryProdByVec}
\fontfamily{pcr}
\selectfont
\begin{lstlisting}
%  $Author: Peter Gagarinov  <pgagarinov@gmail.com> $	$Date: 2011-12-12$
%  $Copyright: Moscow State University,
%             Faculty of Computational Mathematics and Computer Science,
%             System Analysis Department 2011 $
%     
%

\end{lstlisting}
\fontfamily{\familydefault}
\selectfont
\item\hypertarget{/+gras/+mat/+symb\MatrixSFTripleProd}{MatrixSFTripleProd}
\fontfamily{pcr}
\selectfont
\begin{lstlisting}
%

\end{lstlisting}
\fontfamily{\familydefault}
\selectfont
\item\hypertarget{/+gras/+mat/+symb\MatrixSymbFormulaBased}{MatrixSymbFormulaBased}
\fontfamily{pcr}
\selectfont
\begin{lstlisting}
%  $Author: Peter Gagarinov  <pgagarinov@gmail.com> $	$Date: 2011-12-12$
%  $Copyright: Moscow State University,
%             Faculty of Computational Mathematics and Computer Science,
%             System Analysis Department 2011 $    
%

\end{lstlisting}
\fontfamily{\familydefault}
\selectfont
\item\hypertarget{/+gras/+mat/+symb\iscellofstringconst}{iscellofstringconst}
\fontfamily{pcr}
\selectfont
\begin{lstlisting}
%

\end{lstlisting}
\fontfamily{\familydefault}
\selectfont
\end{enumerate}
\subsection{/+gras/+mat/+test}
\begin{enumerate}
\item\hypertarget{/+gras/+mat/+test\run\_tests}{run\_tests}
\fontfamily{pcr}
\selectfont
\begin{lstlisting}
%

\end{lstlisting}
\fontfamily{\familydefault}
\selectfont
\end{enumerate}
\subsection{/+gras/+mat/+test/+mlunit}
\begin{enumerate}
\item\hypertarget{/+gras/+mat/+test/+mlunit\SuiteBasic}{SuiteBasic}
\fontfamily{pcr}
\selectfont
\begin{lstlisting}
%

\end{lstlisting}
\fontfamily{\familydefault}
\selectfont
\item\hypertarget{/+gras/+mat/+test/+mlunit\SuiteOp}{SuiteOp}
\fontfamily{pcr}
\selectfont
\begin{lstlisting}
%

\end{lstlisting}
\fontfamily{\familydefault}
\selectfont
\end{enumerate}
\subsection{/+gras/+ode}
\begin{enumerate}
\item\hypertarget{/+gras/+ode\MatrixODESolver}{MatrixODESolver}
\fontfamily{pcr}
\selectfont
\begin{lstlisting}
%

\end{lstlisting}
\fontfamily{\familydefault}
\selectfont
\item\hypertarget{/+gras/+ode\MatrixSysODESolver}{MatrixSysODESolver}
\fontfamily{pcr}
\selectfont
\begin{lstlisting}
%

\end{lstlisting}
\fontfamily{\familydefault}
\selectfont
\item\hypertarget{/+gras/+ode\ode113reg}{ode113reg}
\fontfamily{pcr}
\selectfont
\begin{lstlisting}
%

\end{lstlisting}
\fontfamily{\familydefault}
\selectfont
\item\hypertarget{/+gras/+ode\ode45reg}{ode45reg}
\fontfamily{pcr}
\selectfont
\begin{lstlisting}
%  ODE45REG is an extension of built-in ode45 solver capable of solving ODEs
%  with right hand-side functions having a limited definition area
% 
%  Input:
%    regular:
%        fOdeDeriv: function_handle[1,1] - function responsible for
%            calculating the right-hand side function as f=fOdeDeriv(t,y)
%        fOdeReg: function_handle[1,1] - function responsible for
%            regularizing the phase variables as
%            [isStrictViolation,yReg]=fOdeReg(t,y) where isStrictViolation
%            is supposed to be true when y is outside of definition area of
%            the right-hand side function
%        tspan: double[1,2]/double[1,nPoints] time range, same meaning 
%            as in ode45
%        y0: double[1,nDims] - initial state, same meaning as in ode45
%        
%    optional:
%        options: odeset[1,1] - options generated by odeset function, same
%            meaning as in ode45
% 
%    properties:
%        regMaxStepTol: double[1,1] - maximum allowed regularization size
%            calculated as max(abs(yReg-y)) allowed per step
%        regAbsTol: double[1,1] - maximum regularization tolerance
%            calculated as max(abs(yReg-y)) that is allowed to consider the
%            integration step to be successful. If the tolerance level is
%            not achieved the regularization continues in the iterative
%            manner via correcting dyReg or decreasing the step size
%        nMaxRegSteps: double[1,1] - maximum number of allowed
%            regularization steps, if regAbsTol is not achieved in 
%            nMaxRegSteps(or less) the integration process fails
% 
%  Output:
%    tout: double[nPoints,1] - time grid, same meaning as in ode45
%    yout: double[nPoints,nDims] - solution, same meaning as in ode45
%    dyRegMat: double[nPoints,nDims] - regularizing derivative addition
%        to the right-hand side function value performed at each step,
%        basically yout is a solution of dot(y)=fOdeDeriv(t,y)+dyRegMat(t,y)
% 
%  $Author: Peter Gagarinov  <pgagarinov@gmail.com> $	$Date: 2011$
%  $Copyright: Moscow State University,
%             Faculty of Computational Mathematics and Computer Science,
%             System Analysis Department 2011 $
% 
%

\end{lstlisting}
\fontfamily{\familydefault}
\selectfont
\end{enumerate}
\subsection{/+gras/+ode/+test}
\begin{enumerate}
\item\hypertarget{/+gras/+ode/+test\run\_tests}{run\_tests}
\fontfamily{pcr}
\selectfont
\begin{lstlisting}
%

\end{lstlisting}
\fontfamily{\familydefault}
\selectfont
\end{enumerate}
\subsection{/+gras/+ode/+test/+mlunit}
\begin{enumerate}
\item\hypertarget{/+gras/+ode/+test/+mlunit\SuiteBasic}{SuiteBasic}
\fontfamily{pcr}
\selectfont
\begin{lstlisting}
%

\end{lstlisting}
\fontfamily{\familydefault}
\selectfont
\end{enumerate}
\subsection{/+gras/+ode/private}
\begin{enumerate}
\item\hypertarget{/+gras/+ode/private\odearguments}{odearguments}
\fontfamily{pcr}
\selectfont
\begin{lstlisting}
%

\end{lstlisting}
\fontfamily{\familydefault}
\selectfont
\item\hypertarget{/+gras/+ode/private\odenonnegative}{odenonnegative}
\fontfamily{pcr}
\selectfont
\begin{lstlisting}
%

\end{lstlisting}
\fontfamily{\familydefault}
\selectfont
\end{enumerate}
\subsection{/+gras/+test}
\begin{enumerate}
\item\hypertarget{/+gras/+test\TmpDataManager}{TmpDataManager}
\fontfamily{pcr}
\selectfont
\begin{lstlisting}
%  TMPDATAMANAGER provides a basic functionality for managing temporary
%  data folders, root folder name is determined automatically
% 
%

\end{lstlisting}
\fontfamily{\familydefault}
\selectfont
\item\hypertarget{/+gras/+test\editconf}{editconf}
\fontfamily{pcr}
\selectfont
\begin{lstlisting}
%

\end{lstlisting}
\fontfamily{\familydefault}
\selectfont
\item\hypertarget{/+gras/+test\run\_tests}{run\_tests}
\fontfamily{pcr}
\selectfont
\begin{lstlisting}
%

\end{lstlisting}
\fontfamily{\familydefault}
\selectfont
\item\hypertarget{/+gras/+test\run\_tests\_remotely}{run\_tests\_remotely}
\fontfamily{pcr}
\selectfont
\begin{lstlisting}
%

\end{lstlisting}
\fontfamily{\familydefault}
\selectfont
\end{enumerate}
\subsection{/+gras/+test/+configuration}
\begin{enumerate}
\item\hypertarget{/+gras/+test/+configuration\AdaptiveConfRepoManager}{AdaptiveConfRepoManager}
\fontfamily{pcr}
\selectfont
\begin{lstlisting}
%  ADAPTIVECONFREPOMANAGER is a simplistic extension of
%  AdaptiveConfRepoManager that injects a configuration change
%  repository class equivolent.test.configuration.ConfPatchRepo
%  automatically
% 
% 
%  $Author: Peter Gagarinov <pgagarinov@gmail.com> $	$Date: 2011-05-18 $ 
%  $Copyright: Moscow State University,
%             Faculty of Computational Mathematics and Computer Science,
%             System Analysis Department 2011 $
% 
%

\end{lstlisting}
\fontfamily{\familydefault}
\selectfont
\end{enumerate}
\subsection{/+gras/+test/+configuration/@ConfPatchRepo}
\begin{enumerate}
\item\hypertarget{/+gras/+test/+configuration/@ConfPatchRepo\ConfPatchRepo}{ConfPatchRepo}
\fontfamily{pcr}
\selectfont
\begin{lstlisting}
%

\end{lstlisting}
\fontfamily{\familydefault}
\selectfont
\item\hypertarget{/+gras/+test/+configuration/@ConfPatchRepo\patch\_001\_dummy\_patch}{patch\_001\_dummy\_patch}
\fontfamily{pcr}
\selectfont
\begin{lstlisting}
%

\end{lstlisting}
\fontfamily{\familydefault}
\selectfont
\end{enumerate}
\subsection{/+gras/+test/+logging}
\begin{enumerate}
\item\hypertarget{/+gras/+test/+logging\Log4jConfigurator}{Log4jConfigurator}
\fontfamily{pcr}
\selectfont
\begin{lstlisting}
% LOG4JCONFIGURATOR simplifies log4j configuration, especially when
% Parallel Computing Toolbox is used. In the latter case the class forwards
% the logs of different processees in separate log files
% 
%  $Author: Peter Gagarinov  <pgagarinov@gmail.com> $	$Date: 2011-05-18$
%  $Copyright: Moscow State University,
%             Faculty of Computational Mathematics and Computer Science,
%             System Analysis Department 2011 $
% 
%

\end{lstlisting}
\fontfamily{\familydefault}
\selectfont
\end{enumerate}
\subsection{/+gras/+test/+mlunit}
\begin{enumerate}
\item\hypertarget{/+gras/+test/+mlunit\SuiteBasic}{SuiteBasic}
\fontfamily{pcr}
\selectfont
\begin{lstlisting}
%

\end{lstlisting}
\fontfamily{\familydefault}
\selectfont
\end{enumerate}
\subsection{/elltoolboxcore/@ellipsoid}
\begin{enumerate}
\item\hypertarget{/elltoolboxcore/@ellipsoid\checkIsMe}{checkIsMe}
\fontfamily{pcr}
\selectfont
\begin{lstlisting}
%

\end{lstlisting}
\fontfamily{\familydefault}
\selectfont
\item\hypertarget{/elltoolboxcore/@ellipsoid\contains}{contains}
\fontfamily{pcr}
\selectfont
\begin{lstlisting}
%  CONTAINS - checks if one ellipsoid contains the other.
%             The condition for E1 = firstEllArr to contain
%             E2 = secondEllArr is
%             min(rho(l | E1) - rho(l | E2)) > 0, subject to <l, l> = 1.
% 
%  Input:
%    regular:
%        firstEllArr: ellipsoid [nDims1,nDims2,...,nDimsN]/[1,1] - first 
%            array of ellipsoids.
%        secondEllArr: ellipsoid [nDims1,nDims2,...,nDimsN]/[1,1] - second
%            array of ellipsoids.
% 
%  Output:
%    resArr: logical[nDims1,nDims2,...,nDimsN],
%        resArr(iCount) = true - firstEllArr(iCount)
%        contains secondEllArr(iCount), false - otherwise.
% 
% 
%  $Author: Alex Kurzhanskiy <akurzhan@eecs.berkeley.edu>
%  $Copyright:  The Regents of the University of California 2004-2008 $
% 
%  $Author: Guliev Rustam <glvrst@gmail.com> $   $Date: Dec-2012$
%  $Copyright: Moscow State University,
%              Faculty of Computational Mathematics and Cybernetics,
%              Science, System Analysis Department 2012 $
% 
%

\end{lstlisting}
\fontfamily{\familydefault}
\selectfont
\item\hypertarget{/elltoolboxcore/@ellipsoid\contents}{contents}
\fontfamily{pcr}
\selectfont
\begin{lstlisting}
%  Ellipsoid library of the Ellipsoidal Toolbox.
% 
%  
%  Constructor and data accessing functions:
%  -----------------------------------------
%   ellipsoid    - Constructor of ellipsoid object.
%   double       - Returns parameters of ellipsoid, i.e. center and shape matrix.
%   parameters   - Same function as 'double' (legacy matter).
%   dimension    - Returns dimension of ellipsoid and its rank.
%   isdegenerate - Checks if ellipsoid is degenerate.
%   isempty      - Checks if ellipsoid is empty.
%   maxeig       - Returns the biggest eigenvalue of the ellipsoid.
%   mineig       - Returns the smallest eigenvalue of the ellipsoid.
%   trace        - Returns the trace of the ellipsoid.
%   volume       - Returns the volume of the ellipsoid.
% 
% 
%  Overloaded operators and functions:
%  -----------------------------------
%   eq      - Checks if two ellipsoids are equal.
%   ne      - The opposite of 'eq'.
%   gt, ge  - E1 > E2 (E1 >= E2) checks if, given the same center ellipsoid E1
%             contains E2.
%   lt, le  - E1 < E2 (E1 <= E2) checks if, given the same center ellipsoid E2
%             contains E1.
%   mtimes  - Given matrix A in R^(mxn) and ellipsoid E in R^n, returns (A * E).
%   plus    - Given vector b in R^n and ellipsoid E in R^n, returns (E + b).
%   minus   - Given vector b in R^n and ellipsoid E in R^n, returns (E - b).
%   uminus  - Changes the sign of the center of ellipsoid.
%   display - Displays the details about given ellipsoid object.
%   inv     - inverts the shape matrix of the ellipsoid.
%   plot    - Plots ellipsoid in 1D, 2D and 3D.
% 
% 
%  Geometry functions:
%  -------------------
%   move2origin        - Moves the center of ellipsoid to the origin.
%   shape              - Same as 'mtimes', but modifies only shape matrix of the
%                        ellipsoid leaving its center as is.
%   rho                - Computes the value of support function and corresponding
%                        boundary point of the ellipsoid in the given direction.
%   polar              - Computes the polar ellipsoid to an ellipsoid that contains
%                        the origin.
%   projection         - Projects the ellipsoid onto a subspace specified by 
%                        orthogonal basis vectors.
%   minksum            - Computes and plots the geometric (Minkowski) sum of given
%                        ellipsoids in 1D, 2D and 3D.
%   minksum_ea         - Computes the external ellipsoidal approximation of
%                        geometric sum of given ellipsoids in given direction.
%   minksum_ia         - Computes the internal ellipsoidal approximation of
%                        geometric sum of given ellipsoids in given direction.
%   minkdiff           - Computes and plots the geometric (Minkowski) difference of
%                        given ellipsoids in 1D, 2D and 3D.
%   minkdiff_ea        - Computes the external ellipsoidal approximation of
%                        geometric difference of two ellipsoids in given direction.
%   minkdiff_ia        - Computes the internal ellipsoidal approximation of
%                        geometric difference of two ellipsoids in given direction.
%   minkpm             - Computes and plots the geometric (Minkowski) difference
%                        of a geometric sum of ellipsoids and a single ellipsoid
%                        in 1D, 2D and 3D.
%   minkpm_ea          - Computes the external ellipsoidal approximation of
%                        the geometric difference of a geometric sum of ellipsoids
%                        and a single ellipsoid in given direction.
%   minkpm_ia          - Computes the internal ellipsoidal approximation of
%                        the geometric difference of a geometric sum of ellipsoids
%                        and a single ellipsoid in given direction.
%   minkmp             - Computes and plots the geometric (Minkowski) sum
%                        of a geometric difference of two single ellipsoids
%                        and a geometric sum of ellipsoids in 1D, 2D and 3D.
%   minkmp_ea          - Computes the external ellipsoidal approximation of
%                        the geometric sum of a geometric difference of
%                        two single ellipsoids and a geometric sum of ellipsoids
%                        in given direction.
%   minkmp_ia          - Computes the internal ellipsoidal approximation of
%                        the geometric sum of a geometric difference of
%                        two single ellipsoids and a geometric sum of ellipsoids
%                        in given direction.
%   isbaddirection     - Checks if ellipsoidal approximation of geometric difference
%                        of two ellipsoids in the given direction can be computed.
%   isinside           - Checks if the union or intersection of ellipsoids or
%                        polytopes lies inside the intersection of given ellipsoids.
%   isinternal         - Checks if given vector belongs to the union or intersection
%                        of given ellipsoids.
%   distance           - Computes the distance from ellipsoid to given point,
%                        ellipsoid, hyperplane or polytope.
%   intersect          - Checks if the union or intersection of ellipsoids intersects
%                        with given ellipsoid, hyperplane, or polytope.
%   intersection_ea    - Computes the minimal volume ellipsoid containing intersection
%                        of two ellipsoids, ellipsoid and halfspace, or ellipsoid
%                        and polytope.
%   intersection_ia    - Computes the maximal ellipsoid contained inside the
%                        intersection of two ellipsoids, ellipsoid and halfspace
%                        or ellipsoid and polytope.
%   ellintersection_ia - Computes maximum volume ellipsoid that is contained
%                        in the intersection of given ellipsoids (can be more than 2).
%   ellunion_ea        - Computes minimum volume ellipsoid that contains
%                        the union of given ellipsoids.
%   hpintersection     - Computes the intersection of ellipsoid with hyperplane.
% 
% 
%  Author:
%  -------
%     Alex Kurzhanskiy <akurzhan@eecs.berkeley.edu>
% 
%

\end{lstlisting}
\fontfamily{\familydefault}
\selectfont
\item\hypertarget{/elltoolboxcore/@ellipsoid\dimension}{dimension}
\fontfamily{pcr}
\selectfont
\begin{lstlisting}
%  DIMENSION - returns the dimension of the space in which the ellipsoid
%              is defined and the actual dimension of the ellipsoid.
% 
%  Input:
%    regular:
%        myEllArr: ellipsoid[nDims1,nDims2,...,nDimsN] - array of ellipsoids.
% 
%  Output:
%    regular:
%        dimArr: double[nDims1,nDims2,...,nDimsN] - space dimensions.
% 
%    optional:
%        rankArr: double[nDims1,nDims2,...,nDimsN] - dimensions of the
%            ellipsoids in myEllArr.
% 
%  $Author: Alex Kurzhanskiy <akurzhan@eecs.berkeley.edu>
%  $Copyright:  The Regents of the University of California 2004-2008 $
% 
%  $Author: Guliev Rustam <glvrst@gmail.com> $   $Date: Dec-2012$
%  $Copyright: Moscow State University,
%              Faculty of Computational Mathematics and Cybernetics,
%              Science, System Analysis Department 2012 $
% 
%

\end{lstlisting}
\fontfamily{\familydefault}
\selectfont
\item\hypertarget{/elltoolboxcore/@ellipsoid\disp}{disp}
\fontfamily{pcr}
\selectfont
\begin{lstlisting}
%  DISP - Displays ellipsoid object.
% 
%  Input:
%    regular:
%        myEllMat: ellipsoid [mRows, nCols] - matrix of ellipsoids.
% 
%  $Author: Alex Kurzhanskiy <akurzhan@eecs.berkeley.edu>
%  $Copyright:  The Regents of the University of California 2004-2008 $
%

\end{lstlisting}
\fontfamily{\familydefault}
\selectfont
\item\hypertarget{/elltoolboxcore/@ellipsoid\display}{display}
\fontfamily{pcr}
\selectfont
\begin{lstlisting}
%  DISPLAY - Displays the details of the ellipsoid object.
% 
%  Input:
%    regular:
%        myEllMat: ellipsoid [mRows, nCols] - matrix of ellipsoids.
% 
%  $Author: Alex Kurzhanskiy <akurzhan@eecs.berkeley.edu>
%  $Copyright:  The Regents of the University of California 2004-2008 $
%

\end{lstlisting}
\fontfamily{\familydefault}
\selectfont
\item\hypertarget{/elltoolboxcore/@ellipsoid\distance}{distance}
\fontfamily{pcr}
\selectfont
\begin{lstlisting}
%  DISTANCE - computes distance between the given ellipsoid (or array of
%             ellipsoids) to the specified object (or arrays of objects):
%             vector, ellipsoid, hyperplane or polytope.
% 
%  Input:
%    regular:
%        ellObjArr: ellipsoid [nDims1, nDims2,..., nDimsN] - array of
%            ellipsoids of the same dimension.
%        objArray: double / ellipsoid / hyperplane /
%            polytope [nDims1, nDims2,..., nDimsN] - array of vectors or
%            ellipsoids or hyperplanes or polytopes. If number of elements
%            in objArray is more than 1, then it must be equal to the number
%            of elements in ellObjArr.
% 
%    optional:
%        isFlagOn: logical[1,1] - if true then distance is computed in
%            ellipsoidal metric, if false - in Euclidean metric
%            (by default isFlagOn=false).
% 
%  Output:
%    regular:
%        distValArray: double [nDims1, nDims2,..., nDimsN] - array of
%            pairwise calculated distances.
%            Negative distance value means
%                for ellipsoid and vector: vector belongs to the ellipsoid,
%                for ellipsoid and hyperplane: ellipsoid
%                    intersects the hyperplane.
%            Zero distance value means
%                for ellipsoid and vector: vector is a
%                    boundary point of the ellipsoid,
%                for ellipsoid and hyperplane: ellipsoid
%                    touches the hyperplane.
%    optional:
%        statusArray: double [nDims1, nDims2,..., nDimsN] - array of time of
%            computation of ellipsoids-vectors or ellipsoids-ellipsoids
%            distances, or status of cvx solver for ellipsoids-polytopes
%            distances.
% 
% 
%  $Author: Alex Kurzhanskiy  <akurzhan@eecs.berkeley.edu> $    $Date: 2004-2008 $
%  $Copyright:  The Regents of the University of California 2004-2008 $
% 
%  $Author:  Vitaly Baranov  <vetbar42@gmail.com> $    $Date: 31-10-2012 $
%  $Copyright: Lomonosov Moscow State University,
%             Faculty of Computational Mathematics and Cybernetics,
%             System Analysis Department 2012 $
%  Literature:
%     1. Lin, A. and Han, S. On the Distance between Two Ellipsoids.
%        SIAM Journal on Optimization, 2002, Vol. 13, No. 1 : pp. 298-308
%     2. Stanley Chan, "Numerical method for Finding Minimum Distance to an
%        Ellipsoid". http://videoprocessing.ucsd.edu/~stanleychan/publication/unpublished/Ellipse.pdf
% 
%

\end{lstlisting}
\fontfamily{\familydefault}
\selectfont
\item\hypertarget{/elltoolboxcore/@ellipsoid\double}{double}
\fontfamily{pcr}
\selectfont
\begin{lstlisting}
%  DOUBLE - returns parameters of the ellipsoid.
% 
%  Input:
%    regular:
%        myEll: ellipsoid [1, 1] - single ellipsoid of dimention nDims.
% 
%  Output:
%    myEllCentVec: double[nDims, 1] - center of the ellipsoid myEll.
%    myEllShMat: double[nDims, nDims] - shape matrix
%        of the ellipsoid myEll.
% 
%  $Author: Alex Kurzhanskiy <akurzhan@eecs.berkeley.edu>
%  $Copyright:  The Regents of the University of California 2004-2008 $
% 
%  $Author: Guliev Rustam <glvrst@gmail.com> $   $Date: Dec-2012$
%  $Copyright: Moscow State University,
%              Faculty of Computational Mathematics and Cybernetics,
%              Science, System Analysis Department 2012 $
% 
%

\end{lstlisting}
\fontfamily{\familydefault}
\selectfont
\item\hypertarget{/elltoolboxcore/@ellipsoid\ellbndr\_2d}{ellbndr\_2d}
\fontfamily{pcr}
\selectfont
\begin{lstlisting}
%

\end{lstlisting}
\fontfamily{\familydefault}
\selectfont
\item\hypertarget{/elltoolboxcore/@ellipsoid\ellbndr\_3d}{ellbndr\_3d}
\fontfamily{pcr}
\selectfont
\begin{lstlisting}
%

\end{lstlisting}
\fontfamily{\familydefault}
\selectfont
\item\hypertarget{/elltoolboxcore/@ellipsoid\ellintersection\_ia}{ellintersection\_ia}
\fontfamily{pcr}
\selectfont
\begin{lstlisting}
%  ELLINTERSECTION_IA - computes maximum volume ellipsoid that is
%                       contained in the intersection of
%                       given ellipsoids.
% 
%  Input:
%    regular:
%        inpEllArr: ellipsoid [nDims1,nDims2,...,nDimsN] - array of
%            ellipsoids of the same dimentions.
% 
%  Output:
%    outEll: ellipsoid [1, 1] - resulting maximum volume ellipsoid.
% 
%  $Author: Alex Kurzhanskiy <akurzhan@eecs.berkeley.edu>
%  $Copyright:  The Regents of the University of California 2004-2008 $
%     Alex Kurzhanskiy <akurzhan@eecs.berkeley.edu>
% 
%  $Author: Vadim Kaushanskiy <vkaushanskiy@gmail.com>$ $Date: 10-11-2012$
%  $Copyright: Moscow State University,
%             Faculty of Computational Mathematics and Computer Science,
%             System Analysis Department 2012 $
%

\end{lstlisting}
\fontfamily{\familydefault}
\selectfont
\item\hypertarget{/elltoolboxcore/@ellipsoid\ellipsoid}{ellipsoid}
\fontfamily{pcr}
\selectfont
\begin{lstlisting}
%  ELLIPSOID - constructor of the ellipsoid object.
% 
%    Ellipsoid E = { x in R^n : <(x - q), Q^(-1)(x - q)> <= 1 },
%        with current "Properties"..
%        Here q is a vector in R^n, and Q in R^(nxn) is positive
%            semi-definite matrix
% 
%    ell = ELLIPSOID - Creates an empty ellipsoid
% 
%    ell = ELLIPSOID(shMat) - creates an ellipsoid with shape
%        matrix shMat, centered at 0
% 
% 	ell = ELLIPSOID(centVec, shMat) - creates an ellipsoid with
%        shape matrix shMat and center centVec
% 
%    ell = ELLIPSOID(centVec, shMat, 'propName1', propVal1,...,
%        'propNameN',propValN) - creates an ellipsoid with shape
%        matrix shMat, center centVec and propName1 = propVal1,...,
%        propNameN = propValN. In other cases "Properties"
%        are taken from current values stored in
%        elltool.conf.Properties.
% 
%    These parameters can be accessed by DOUBLE(E) function call.
%    Also, DIMENSION(E) function call returns the dimension of
%    the space in which ellipsoid E is defined and the actual
%    dimension of the ellipsoid; function ISEMPTY(E) checks if
%    ellipsoid E is empty; function ISDEGENERATE(E) checks if
%    ellipsoid E is degenerate.
% 
%  Input:
%    Case1:
%      regular:
%        shMat: double [nDim, nDim] - shape matrix of an ellipsoid
% 
%    Case2:
%      regular:
%        centVec: double [nDim,1] - center of an ellipsoid
%        shMat: double [nDim, nDim] - shape matrix of an ellipsoid
% 
%    properties:
%        absTol: double [1,1] - absolute tolerance with default
%            value 10^(-7)
%        relTol: double [1,1] - relative tolerance with default
%            value 10^(-5)
%        nPlot2dPoints: double [1,1] - number of points for 2D plot
%            with default value 200
%        nPlot3dPoints: double [1,1] - number of points for 3D plot
%            with default value 200.
% 
%  Output:
%    ell: ellipsoid [1,1] - ellipsoid with specified properties.
% 
%  $Author: Alex Kurzhanskiy <akurzhan@eecs.berkeley.edu>
%  $Copyright: The Regents of the University
%    of California 2004-2008 $
% 
%  $Author: Guliev Rustam <glvrst@gmail.com> $   $Date: Dec-2012$
%  $Copyright: Moscow State University,
%              Faculty of Computational Mathematics and Cybernetics,
%              Science, System Analysis Department 2012 $
% 
%

\end{lstlisting}
\fontfamily{\familydefault}
\selectfont
\item\hypertarget{/elltoolboxcore/@ellipsoid\ellunion\_ea}{ellunion\_ea}
\fontfamily{pcr}
\selectfont
\begin{lstlisting}
%  ELLUNION_EA - computes minimum volume ellipsoid that contains union
%                of given ellipsoids.
% 
%  Input:
%    regular:
%        inpEllMat: ellipsoid [nDims1,nDims2,...,nDimsN] - array of
%            ellipsoids of the same dimentions.
% 
%  Output:
%    outEll: ellipsoid [1, 1] - resulting minimum volume ellipsoid.
% 
%  $Author: Alex Kurzhanskiy <akurzhan@eecs.berkeley.edu>
%  $Copyright:  The Regents of the University of California 2004-2008 $
% 
%  $Author: Vadim Kaushanskiy <vkaushanskiy@gmail.com>$ $Date: 10-11-2012$
%  $Copyright: Moscow State University,
%             Faculty of Computational Mathematics and Computer Science,
%             System Analysis Department 2012 $
%

\end{lstlisting}
\fontfamily{\familydefault}
\selectfont
\item\hypertarget{/elltoolboxcore/@ellipsoid\eq}{eq}
\fontfamily{pcr}
\selectfont
\begin{lstlisting}
%  EQ - compares two arrays of ellipsoids
% 
%  Input:
%    regular:
%        ellFirstArr: ellipsoid: [nDims1,nDims2,...,nDimsN]/[1,1]- the first
%            array of ellipsoid objects
%        ellSecArr: ellipsoid: [nDims1,nDims2,...,nDimsN]/[1,1] - the second
%            array of ellipsoid objects
% 
%  Output:
%    isEqualArr: logical: [nDims1,nDims2,...,nDimsN]- array of comparison
%        results
% 
%    reportStr: char[1,] - comparison report
% 
%  $Author: Vadim Kaushansky  <vkaushanskiy@gmail.com> $    $Date: Nov-2012$
%  $Copyright: Moscow State University,
%             Faculty of Computational Mathematics and Cybernetics,
%             System Analysis Department 2012 $
%  $Author: Peter Gagarinov  <pgagarinov@gmail.com> $    $Date: Dec-2012$
%  $Copyright: Moscow State University,
%             Faculty of Computational Mathematics and Cybernetics,
%             System Analysis Department 2012 $
%

\end{lstlisting}
\fontfamily{\familydefault}
\selectfont
\item\hypertarget{/elltoolboxcore/@ellipsoid\ge}{ge}
\fontfamily{pcr}
\selectfont
\begin{lstlisting}
%  GE - checks if the first ellipsoid is bigger than the second one.
%       Same as GT.
% 
%  Input:
%    regular:
%        firsrEllArr: ellipsoid [nDims1,nDims2,...,nDimsN]/[1,1] - array 
%            of ellipsoids.
%        secondEllArr: ellipsoid [nDims1,nDims2,...,nDimsN]/[1,1] - array
%            of ellipsoids of the corresponding dimensions.
% 
%  Output:
%    isPositiveArr: logical [nDims1,nDims2,...,nDimsN],
%        isPositiveArr(iCount) = true - if firsrEllArr(iCount)
%        contains secondEllArr(iCount)
%        when both have same center, false - otherwise.
% 
%  $Author: Alex Kurzhanskiy <akurzhan@eecs.berkeley.edu>
%  $Copyright:  The Regents of the University of California 2004-2008 $
%

\end{lstlisting}
\fontfamily{\familydefault}
\selectfont
\item\hypertarget{/elltoolboxcore/@ellipsoid\getAbsTol}{getAbsTol}
\fontfamily{pcr}
\selectfont
\begin{lstlisting}
%  GETABSTOL - gives array the same size as ellArr with values of absTol
%    properties for each ellipsoid in ellArr
% 
%  Input:
%    regular:
%        ellArr: ellipsoid[nDim1, nDim2, ...] - multidimension array
%            of ellipsoids
% 
%  Output:
%    absTolArr: double[nDim1, nDim2,...] - multidimension array of absTol
%        properties for ellipsoids in ellArr
% 
%  $Author: Zakharov Eugene  <justenterrr@gmail.com> $
%    $Date: 17-november-2012$
%  $Copyright: Moscow State University,
%             Faculty of Computational Arrhematics and Computer Science,
%             System Analysis Department 2012 $
% 
%

\end{lstlisting}
\fontfamily{\familydefault}
\selectfont
\item\hypertarget{/elltoolboxcore/@ellipsoid\getNPlot2dPoints}{getNPlot2dPoints}
\fontfamily{pcr}
\selectfont
\begin{lstlisting}
%  GETNPLOT2DPOINTS - gives value of nPlot2dPoints property
%    of ellipsoids in ellArr
% 
%  Input:
%    regular:
%        ellArr: ellipsoid[nDim1, nDim2,...] - mltidimensional array
%            of ellipsoids
% 
%  Output:
%        nPlot2dPointsArr: double[nDim1, nDim2,...] - multidimension array
%            of nPlot2dPoints property for ellipsoids in ellArr
% 
%  $Author: Zakharov Eugene  <justenterrr@gmail.com> $ 
%    $Date: 17-november-2012$
%  $Copyright: Moscow State University,
%             Faculty of Computational Arrhematics and Computer Science,
%             System Analysis Department 2012 $
% 
%

\end{lstlisting}
\fontfamily{\familydefault}
\selectfont
\item\hypertarget{/elltoolboxcore/@ellipsoid\getNPlot3dPoints}{getNPlot3dPoints}
\fontfamily{pcr}
\selectfont
\begin{lstlisting}
%  GETNPLOT3DPOINTS - gives value of nPlot3dPoints property
%    of ellipsoids in ellArr
% 
%  Input:
%    regular:
%        ellArr: ellipsoid[nDim1, nDim2,...] - mltidimensional array
%            of ellipsoids
% 
%  Output:
%        nPlot2dPointsArr: double[nDim1, nDim2,...] - multidimension array
%            of nPlot3dPoints property for ellipsoids in ellArr
% 
%  $Author: Zakharov Eugene  <justenterrr@gmail.com> $
%    $Date: 17-november-2012$
%  $Copyright: Moscow State University,
%             Faculty of Computational Arrhematics and Computer Science,
%             System Analysis Department 2012 $
% 
%

\end{lstlisting}
\fontfamily{\familydefault}
\selectfont
\item\hypertarget{/elltoolboxcore/@ellipsoid\getProperty}{getProperty}
\fontfamily{pcr}
\selectfont
\begin{lstlisting}
%

\end{lstlisting}
\fontfamily{\familydefault}
\selectfont
\item\hypertarget{/elltoolboxcore/@ellipsoid\getRelTol}{getRelTol}
\fontfamily{pcr}
\selectfont
\begin{lstlisting}
%  GETRELTOL - gives array the same size as ellArr with values of relTol
%              properties for each ellipsoid in ellArr
% 
%  Input:
%    regular:
%        ellArr: ellipsoid[nDim1, nDim2,...] - multidimension array
%            of ellipsoids
% 
%  Output:
%    relTolArr: double[nDim1, nDim2,...] - multidimension array of relTol
%        properties for ellipsoids in ellArr
% 
%  $Author: Zakharov Eugene <justenterrr@gmail.com> $
%    $Date: 17-november-2012$
%  $Copyright: Moscow State University,
%             Faculty of Computational Arrhematics and Computer Science,
%             System Analysis Department 2012 $
% 
%

\end{lstlisting}
\fontfamily{\familydefault}
\selectfont
\item\hypertarget{/elltoolboxcore/@ellipsoid\gt}{gt}
\fontfamily{pcr}
\selectfont
\begin{lstlisting}
%  GT - checks if the first ellipsoid is bigger than the second one.
% 
%  Input:
%    regular:
%        firsrEllArr: ellipsoid [nDims1,nDims2,...,nDimsN]/[1,1] - array
%            of ellipsoids.
%        secondEllArr: ellipsoid [nDims1,nDims2,...,nDimsN]/[1,1] - array
%            of ellipsoids of the corresponding dimensions.
% 
%  Output:
%    isPositiveArr: logical [nDims1,nDims2,...,nDimsN],
%        isPositiveArr(iCount) = true - if firsrEllArr(iCount)
%        contains secondEllArr(iCount)
%        when both have same center, false - otherwise.
% 
%  $Author: Alex Kurzhanskiy <akurzhan@eecs.berkeley.edu>
%  $Copyright:  The Regents of the University of California 2004-2008 $
% 
%  $Author: Guliev Rustam <glvrst@gmail.com> $   $Date: Dec-2012$
%  $Copyright: Moscow State University,
%              Faculty of Computational Mathematics and Cybernetics,
%              Science, System Analysis Department 2012 $
% 
%

\end{lstlisting}
\fontfamily{\familydefault}
\selectfont
\item\hypertarget{/elltoolboxcore/@ellipsoid\hpintersection}{hpintersection}
\fontfamily{pcr}
\selectfont
\begin{lstlisting}
%  HPINTERSECTION - computes the intersection of ellipsoid with hyperplane.
% 
%  Input:
%    regular:
%        myEllArr: ellipsoid [nDims1,nDims2,...,nDimsN]/[1,1] - array
%            of ellipsoids.
%        myHypArr: hyperplane [nDims1,nDims2,...,nDimsN]/[1,1] - array
%            of hyperplanes of the same size.
% 
%  Output:
%    intEllArr: ellipsoid [nDims1,nDims2,...,nDimsN] - array of ellipsoids
%        resulting from intersections.
% 
%    isnIntersectedArr: logical [nDims1,nDims2,...,nDimsN].
%        isnIntersectedArr(iCount) = true, if myEllArr(iCount)
%        doesn't intersect myHipArr(iCount),
%        isnIntersectedArr(iCount) = false, otherwise.
% 
%  $Author: Alex Kurzhanskiy <akurzhan@eecs.berkeley.edu>
%  $Copyright:  The Regents of the University of California 2004-2008 $
% 
%  $Author: Guliev Rustam <glvrst@gmail.com> $   $Date: Dec-2012$
%  $Copyright: Moscow State University,
%              Faculty of Computational Mathematics and Cybernetics,
%              Science, System Analysis Department 2012 $
% 
%

\end{lstlisting}
\fontfamily{\familydefault}
\selectfont
\item\hypertarget{/elltoolboxcore/@ellipsoid\intersect}{intersect}
\fontfamily{pcr}
\selectfont
\begin{lstlisting}
%  INTERSECT - checks if the union or intersection of ellipsoids intersects
%              given ellipsoid, hyperplane or polytope.
% 
%    resArr = INTERSECT(myEllArr, objArr, mode) - Checks if the union
%        (mode = 'u') or intersection (mode = 'i') of ellipsoids
%        in myEllArr intersects with objects in objArr.
%        objArr can be array of ellipsoids, array of hyperplanes,
%        or array of polytopes.
%        Ellipsoids, hyperplanes or polytopes in objMat must have
%        the same dimension as ellipsoids in myEllArr.
%        mode = 'u' (default) - union of ellipsoids in myEllArr.
%        mode = 'i' - intersection.
% 
%    If we need to check the intersection of union of ellipsoids in
%    myEllArr (mode = 'u'), or if myEllMat is a single ellipsoid,
%    it can be done by calling distance function for each of the
%    ellipsoids in myEllArr and objMat, and if it returns negative value,
%    the intersection is nonempty. Checking if the intersection of
%    ellipsoids in myEllArr (with size of myEllMat greater than 1)
%    intersects with ellipsoids or hyperplanes in objArr is more
%    difficult. This problem can be formulated as quadratically
%    constrained quadratic programming (QCQP) problem.
% 
%    Let objArr(iObj) = E(q, Q) be an ellipsoid with center q and shape matrix Q.
%    To check if this ellipsoid intersects (or touches) the intersection
%    of ellipsoids in meEllArr: E(q1, Q1), E(q2, Q2), ..., E(qn, Qn),
%    we define the QCQP problem:
%                      J(x) = <(x - q), Q^(-1)(x - q)> --> min
%    with constraints:
%                       <(x - q1), Q1^(-1)(x - q1)> <= 1   (1)
%                       <(x - q2), Q2^(-1)(x - q2)> <= 1   (2)
%                       ................................
%                       <(x - qn), Qn^(-1)(x - qn)> <= 1   (n)
% 
%    If this problem is feasible, i.e. inequalities (1)-(n) do not
%    contradict, or, in other words, intersection of ellipsoids
%    E(q1, Q1), E(q2, Q2), ..., E(qn, Qn) is nonempty, then we can find
%    vector y such that it satisfies inequalities (1)-(n) and minimizes
%    function J. If J(y) <= 1, then ellipsoid E(q, Q) intersects or touches
%    the given intersection, otherwise, it does not. To check if E(q, Q)
%    intersects the union of E(q1, Q1), E(q2, Q2), ..., E(qn, Qn),
%    we compute the distances from this ellipsoids to those in the union.
%    If at least one such distance is negative,
%    then E(q, Q) does intersect the union.
% 
%    If we check the intersection of ellipsoids with hyperplane
%    objArr = H(v, c), it is enough to check the feasibility
%    of the problem
%                        1'x --> min
%    with constraints (1)-(n), plus
%                      <v, x> - c = 0.
% 
%    Checking the intersection of ellipsoids with polytope
%    objArr = P(A, b) reduces to checking if there any x, satisfying
%    constraints (1)-(n) and
%                         Ax <= b.
% 
%  Input:
%    regular:
%        myEllArr: ellipsoid [nDims1,nDims2,...,nDimsN] - array of ellipsoids.
%        objArr: ellipsoid / hyperplane /
%            / polytope [nDims1,nDims2,...,nDimsN] - array of ellipsoids or
%            hyperplanes or polytopes of the same sizes.
% 
%    optional:
%        mode: char[1, 1] - 'u' or 'i', go to description.
% 
%            note: If mode == 'u', then mRows, nCols should be equal to 1.
% 
%  Output:
%    resArr: double[nDims1,nDims2,...,nDimsN] - return:
%        resArr(iCount) = -1 in case parameter mode is set
%            to 'i' and the intersection of ellipsoids in myEllArr
%            is empty.
%        resArr(iCount) = 0 if the union or intersection of
%            ellipsoids in myEllArr does not intersect the object
%            in objArr(iCount).
%        resArr(iCount) = 1 if the union or intersection of
%            ellipsoids in myEllArr and the object in objArr(iCount)
%            have nonempty intersection.
%    statusArr: double[0, 0]/double[nDims1,nDims2,...,nDimsN] - status
%        variable. statusArr is empty if mode = 'u'.
% 
%  $Author: Alex Kurzhanskiy <akurzhan@eecs.berkeley.edu>
%  $Copyright:  The Regents of the University of California 2004-2008 $
% 
%  $Author: Guliev Rustam <glvrst@gmail.com> $   $Date: Dec-2012$
%  $Copyright: Moscow State University,
%              Faculty of Computational Mathematics and Cybernetics,
%              Science, System Analysis Department 2012 $
% 
%  $Author: <Zakharov Eugene>  <justenterrr@gmail.com> $    $Date: March-2013 $
%  $Copyright: Moscow State University,
%             Faculty of Computational Mathematics and Computer Science,
%             System Analysis Department$
% 
%

\end{lstlisting}
\fontfamily{\familydefault}
\selectfont
\item\hypertarget{/elltoolboxcore/@ellipsoid\intersection\_ea}{intersection\_ea}
\fontfamily{pcr}
\selectfont
\begin{lstlisting}
%  INTERSECTION_EA - external ellipsoidal approximation of the
%                    intersection of two ellipsoids, or ellipsoid and
%                    halfspace, or ellipsoid and polytope.
% 
%    outEllArr = INTERSECTION_EA(myEllArr, objArr) Given two ellipsoidal
%        matrixes of equal sizes, myEllArr and objArr = ellArr, or,
%        alternatively, myEllArr or ellMat must be a single ellipsoid,
%        computes the ellipsoid that contains the intersection of two
%        corresponding ellipsoids from myEllArr and from ellArr.
%    outEllArr = INTERSECTION_EA(myEllArr, objArr) Given matrix of
%        ellipsoids myEllArr and matrix of hyperplanes objArr = hypArr
%        whose sizes match, computes the external ellipsoidal
%        approximations of intersections of ellipsoids
%        and halfspaces defined by hyperplanes in hypArr.
%        If v is normal vector of hyperplane and c - shift,
%        then this hyperplane defines halfspace
%                <v, x> <= c.
%    outEllArr = INTERSECTION_EA(myEllArr, objArr) Given matrix of
%        ellipsoids myEllArr and matrix of polytopes objArr = polyArr
%        whose sizes match, computes the external ellipsoidal
%        approximations of intersections of ellipsoids myEllMat and
%        polytopes polyArr.
% 
%    The method used to compute the minimal volume overapproximating
%    ellipsoid is described in "Ellipsoidal Calculus Based on
%    Propagation and Fusion" by Lluis Ros, Assumpta Sabater and
%    Federico Thomas; IEEE Transactions on Systems, Man and Cybernetics,
%    Vol.32, No.4, pp.430-442, 2002. For more information, visit
%    http://www-iri.upc.es/people/ros/ellipsoids.html
% 
%  Input:
%    regular:
%        myEllArr: ellipsoid [nDims1,nDims2,...,nDimsN]/[1,1] - array
%            of ellipsoids.
%        objArr: ellipsoid / hyperplane /
%            / polytope [nDims1,nDims2,...,nDimsN]/[1,1]  - array of
%            ellipsoids or hyperplanes or polytopes of the same sizes.
% 
%  Output:
%     outEllArr: ellipsoid [nDims1,nDims2,...,nDimsN] - array of external
%        approximating ellipsoids; entries can be empty ellipsoids
%        if the corresponding intersection is empty.
% 
%  $Author: Alex Kurzhanskiy <akurzhan@eecs.berkeley.edu>
%  $Copyright:  The Regents of the University of California 2004-2008 $
% 
%  $Author: Guliev Rustam <glvrst@gmail.com> $   $Date: Dec-2012$
%  $Copyright: Moscow State University,
%              Faculty of Computational Mathematics and Cybernetics,
%              Science, System Analysis Department 2012 $
% 
%

\end{lstlisting}
\fontfamily{\familydefault}
\selectfont
\item\hypertarget{/elltoolboxcore/@ellipsoid\intersection\_ia}{intersection\_ia}
\fontfamily{pcr}
\selectfont
\begin{lstlisting}
%  INTERSECTION_IA - internal ellipsoidal approximation of the
%                    intersection of ellipsoid and ellipsoid,
%                    or ellipsoid and halfspace, or ellipsoid
%                    and polytope.
% 
%    outEllArr = INTERSECTION_IA(myEllArr, objArr) - Given two
%        ellipsoidal matrixes of equal sizes, myEllArr and
%        objArr = ellArr, or, alternatively, myEllMat or ellMat must be
%        a single ellipsoid, comuptes the internal ellipsoidal
%        approximations of intersections of two corresponding ellipsoids
%        from myEllMat and from ellMat.
%    outEllArr = INTERSECTION_IA(myEllArr, objArr) - Given matrix of
%        ellipsoids myEllArr and matrix of hyperplanes objArr = hypArr
%        whose sizes match, computes the internal ellipsoidal
%        approximations of intersections of ellipsoids and halfspaces
%        defined by hyperplanes in hypMat.
%        If v is normal vector of hyperplane and c - shift,
%        then this hyperplane defines halfspace
%                   <v, x> <= c.
%    outEllArr = INTERSECTION_IA(myEllArr, objArr) - Given matrix of
%        ellipsoids  myEllArr and matrix of polytopes objArr = polyArr
%        whose sizes match, computes the internal ellipsoidal
%        approximations of intersections of ellipsoids myEllArr
%        and polytopes polyArr.
% 
%    The method used to compute the minimal volume overapproximating
%    ellipsoid is described in "Ellipsoidal Calculus Based on
%    Propagation and Fusion" by Lluis Ros, Assumpta Sabater and
%    Federico Thomas; IEEE Transactions on Systems, Man and Cybernetics,
%    Vol.32, No.4, pp.430-442, 2002. For more information, visit
%    http://www-iri.upc.es/people/ros/ellipsoids.html
% 
%  Input:
%    regular:
%        myEllArr: ellipsoid [nDims1,nDims2,...,nDimsN]/[1,1] - array
%            of ellipsoids.
%        objArr: ellipsoid / hyperplane /
%            / polytope [nDims1,nDims2,...,nDimsN]/[1,1]  - array of
%            ellipsoids or hyperplanes or polytopes of the same sizes.
% 
%  Output:
%     outEllArr: ellipsoid [nDims1,nDims2,...,nDimsN] - array of internal
%        approximating ellipsoids; entries can be empty ellipsoids
%        if the corresponding intersection is empty.
% 
%  $Author: Alex Kurzhanskiy <akurzhan@eecs.berkeley.edu>
%  $Copyright:  The Regents of the University of California 2004-2008 $
% 
%  $Author: Guliev Rustam <glvrst@gmail.com> $   $Date: Dec-2012$
%  $Copyright: Moscow State University,
%              Faculty of Computational Mathematics and Cybernetics,
%              Science, System Analysis Department 2012 $
% 
%

\end{lstlisting}
\fontfamily{\familydefault}
\selectfont
\item\hypertarget{/elltoolboxcore/@ellipsoid\inv}{inv}
\fontfamily{pcr}
\selectfont
\begin{lstlisting}
%  INV - inverts shape matrices of ellipsoids in the given array.
% 
%    invEllArr = INV(myEllArr)  Inverts shape matrices of ellipsoids
%        in the array myEllMat. In case shape matrix is sigular, it is
%        regularized before inversion.
% 
%  Input:
%    regular:
%        myEllArr: ellipsoid [nDims1,nDims2,...,nDimsN] - array of ellipsoids.
% 
%  Output:
%     invEllArr: ellipsoid [nDims1,nDims2,...,nDimsN] - array of ellipsoids
%        with inverted shape matrices.
% 
%  $Author: Alex Kurzhanskiy <akurzhan@eecs.berkeley.edu>
%  $Copyright:  The Regents of the University of California 2004-2008 $
% 
%  $Author: Guliev Rustam <glvrst@gmail.com> $   $Date: Dec-2012$
%  $Copyright: Moscow State University,
%              Faculty of Computational Mathematics and Cybernetics,
%              Science, System Analysis Department 2012 $
% 
%

\end{lstlisting}
\fontfamily{\familydefault}
\selectfont
\item\hypertarget{/elltoolboxcore/@ellipsoid\isbaddirection}{isbaddirection}
\fontfamily{pcr}
\selectfont
\begin{lstlisting}
%  ISBADDIRECTION - checks if ellipsoidal approximations of geometric
%                   difference of two ellipsoids can be computed for
%                   given directions.
%    isBadDirVec = ISBADDIRECTION(fstEll, secEll, dirsMat) - Checks if
%        it is possible to build ellipsoidal approximation of the
%        geometric difference of two ellipsoids fstEll - secEll in
%        directions specified by matrix dirsMat (columns of dirsMat
%        are direction vectors). Type 'help minkdiff_ea' or
%        'help minkdiff_ia' for more information.
% 
%  Input:
%    regular:
%        fstEll: ellipsoid [1, 1] - first ellipsoid. Suppose nDim - space
%            dimension.
%        secEll: ellipsoid [1, 1] - second ellipsoid of the same dimention.
%        dirsMat: numeric[nDims, nCols] - matrix whose columns are
%            direction vectors that need to be checked.
%        absTol: double [1,1] - absolute tolerance
% 
%  Output:
%     isBadDirVec: logical[1, nCols] - array of true or false with length
%        being equal to the number of columns in matrix dirsMat.
%        true marks direction vector as bad - ellipsoidal approximation
%        cannot be computed for this direction. false means the opposite.
% 
%  $Author: Alex Kurzhanskiy <akurzhan@eecs.berkeley.edu>
%  $Copyright:  The Regents of the University of California 2004-2008 $
% 
%  $Author: Guliev Rustam <glvrst@gmail.com> $   $Date: Dec-2012$
%  $Copyright: Moscow State University,
%              Faculty of Computational Mathematics and Cybernetics,
%              Science, System Analysis Department 2012 $
% 
%

\end{lstlisting}
\fontfamily{\familydefault}
\selectfont
\item\hypertarget{/elltoolboxcore/@ellipsoid\isbaddirectionmat}{isbaddirectionmat}
\fontfamily{pcr}
\selectfont
\begin{lstlisting}
%

\end{lstlisting}
\fontfamily{\familydefault}
\selectfont
\item\hypertarget{/elltoolboxcore/@ellipsoid\isbigger}{isbigger}
\fontfamily{pcr}
\selectfont
\begin{lstlisting}
%  ISBIGGER - checks if one ellipsoid would contain the other if their
%             centers would coincide.
% 
%    isPositive = ISBIGGER(fstEll, secEll) - Given two single ellipsoids
%        of the same dimension, fstEll and secEll, check if fstEll
%        would contain secEll inside if they were both
%        centered at origin.
% 
%  Input:
%    regular:
%        fstEll: ellipsoid [1, 1] - first ellipsoid.
%        secEll: ellipsoid [1, 1] - second ellipsoid
%            of the same dimention.
% 
%  Output:
%    isPositive: logical[1, 1], true - if ellipsoid fstEll
%        would contain secEll inside, false - otherwise.
% 
%  $Author: Alex Kurzhanskiy <akurzhan@eecs.berkeley.edu>
%  $Copyright:  The Regents of the University of California 2004-2008 $
%

\end{lstlisting}
\fontfamily{\familydefault}
\selectfont
\item\hypertarget{/elltoolboxcore/@ellipsoid\isdegenerate}{isdegenerate}
\fontfamily{pcr}
\selectfont
\begin{lstlisting}
%  ISDEGENERATE - checks if the ellipsoid is degenerate.
% 
%  Input:
%    regular:
%        myEllArr: ellipsoid[nDims1,nDims2,...,nDimsN] - array of ellipsoids.
% 
%  Output:
%    isPositiveArr: logical[nDims1,nDims2,...,nDimsN], 
%        isPositiveArr(iCount) = true if ellipsoid myEllMat(iCount) 
%        is degenerate, false - otherwise.
% 
%  $Author: Alex Kurzhanskiy <akurzhan@eecs.berkeley.edu>
%  $Copyright:  The Regents of the University of California 2004-2008 $
% 
%  $Author: Guliev Rustam <glvrst@gmail.com> $   $Date: Dec-2012$
%  $Copyright: Moscow State University,
%              Faculty of Computational Mathematics and Cybernetics,
%              Science, System Analysis Department 2012 $
% 
%

\end{lstlisting}
\fontfamily{\familydefault}
\selectfont
\item\hypertarget{/elltoolboxcore/@ellipsoid\isempty}{isempty}
\fontfamily{pcr}
\selectfont
\begin{lstlisting}
%  ISEMPTY - checks if the ellipsoid object is empty.
% 
%  Input:
%    regular:
%        myEllArr: ellipsoid [nDims1,nDims2,...,nDimsN] - array of ellipsoids.
% 
%  Output:
%    isPositiveArr: logical[nDims1,nDims2,...,nDimsN], 
%        isPositiveArr(iCount) = true - if ellipsoid
%        myEllMat(iCount) is empty, false - otherwise.
% 
%  $Author: Alex Kurzhanskiy <akurzhan@eecs.berkeley.edu>
%  $Copyright:  The Regents of the University of California 2004-2008 $
% 
%  $Author: Guliev Rustam <glvrst@gmail.com> $   $Date: Dec-2012$
%  $Copyright: Moscow State University,
%              Faculty of Computational Mathematics and Cybernetics,
%              Science, System Analysis Department 2012 $
% 
%

\end{lstlisting}
\fontfamily{\familydefault}
\selectfont
\item\hypertarget{/elltoolboxcore/@ellipsoid\isinside}{isinside}
\fontfamily{pcr}
\selectfont
\begin{lstlisting}
%  ISINSIDE - checks if the intersection of ellipsoids contains the
%             union or intersection of given ellipsoids or polytopes.
% 
%    res = ISINSIDE(fstEllArr, secEllArr, mode) Checks if the union
%        (mode = 'u') or intersection (mode = 'i') of ellipsoids in
%        secEllArr lies inside the intersection of ellipsoids in
%        fstEllArr. Ellipsoids in fstEllArr and secEllArr must be
%        of the same dimension. mode = 'u' (default) - union of
%        ellipsoids in secEllArr. mode = 'i' - intersection.
%    res = ISINSIDE(fstEllArr, secPolyArr, mode) Checks if the union
%        (mode = 'u') or intersection (mode = 'i')  of polytopes in
%        secPolyArr lies inside the intersection of ellipsoids in
%        fstEllArr. Ellipsoids in fstEllArr and polytopes in secPolyArr
%        must be of the same dimension. mode = 'u' (default) - union of
%        polytopes in secPolyMat. mode = 'i' - intersection.
% 
%    To check if the union of ellipsoids secEllArr belongs to the
%    intersection of ellipsoids fstEllArr, it is enough to check that
%    every ellipsoid of secEllMat is contained in every
%    ellipsoid of fstEllArr.
%    Checking if the intersection of ellipsoids in secEllMat is inside
%    intersection fstEllMat can be formulated as quadratically
%    constrained quadratic programming (QCQP) problem.
% 
%    Let fstEllArr(iEll) = E(q, Q) be an ellipsoid with center q and shape
%    matrix Q. To check if this ellipsoid contains the intersection of
%    ellipsoids in secObjArr:
%    E(q1, Q1), E(q2, Q2), ..., E(qn, Qn), we define the QCQP problem:
%                      J(x) = <(x - q), Q^(-1)(x - q)> --> max
%    with constraints:
%                      <(x - q1), Q1^(-1)(x - q1)> <= 1   (1)
%                      <(x - q2), Q2^(-1)(x - q2)> <= 1   (2)
%                      ................................
%                      <(x - qn), Qn^(-1)(x - qn)> <= 1   (n)
% 
%    If this problem is feasible, i.e. inequalities (1)-(n) do not
%    contradict, or, in other words, intersection of ellipsoids
%    E(q1, Q1), E(q2, Q2), ..., E(qn, Qn) is nonempty, then we can find
%    vector y such that it satisfies inequalities (1)-(n)
%    and maximizes function J. If J(y) <= 1, then ellipsoid E(q, Q)
%    contains the given intersection, otherwise, it does not.
% 
%    The intersection of polytopes is a polytope, which is computed
%    by the standard routine of MPT. If the vertices of this polytope
%    belong to the intersection of ellipsoids, then the polytope itself
%    belongs to this intersection.
%    Checking if the union of polytopes belongs to the intersection
%    of ellipsoids is the same as checking if its convex hull belongs
%    to this intersection.
% 
%  Input:
%    regular:
%        fstEllArr: ellipsoid [nDims1,nDims2,...,nDimsN] - array of ellipsoids
%            of the same size.
%        secEllArr: ellipsoid /
%            polytope [nDims1,nDims2,...,nDimsN] - array of ellipsoids or
%            polytopes of the same sizes.
% 
%            note: if mode == 'i', then fstEllArr, secEllVec should be
%                array.
% 
%    optional:
%        mode: char[1, 1] - 'u' or 'i', go to description.
% 
%  Output:
%    res: double[1, 1] - result:
%        -1 - problem is infeasible, for example, if s = 'i',
%            but the intersection of ellipsoids in E2 is an empty set;
%        0 - intersection is empty;
%        1 - if intersection is nonempty.
%    status: double[0, 0]/double[1, 1] - status variable. status is empty
%        if mode == 'u' or mSecRows == nSecCols == 1.
% 
%  $Author: Alex Kurzhanskiy <akurzhan@eecs.berkeley.edu>
%  $Copyright:  The Regents of the University of California 2004-2008 $
% 
%  $Author: Vadim Kaushanskiy <vkaushanskiy@gmail.com>$ $Date: 10-11-2012$
%  $Copyright: Moscow State University,
%             Faculty of Computational Mathematics and Computer Science,
%             System Analysis Department 2012 $
%

\end{lstlisting}
\fontfamily{\familydefault}
\selectfont
\item\hypertarget{/elltoolboxcore/@ellipsoid\isinternal}{isinternal}
\fontfamily{pcr}
\selectfont
\begin{lstlisting}
%  ISINTERNAL - checks if given points belong to the union or intersection
%               of ellipsoids in the given array.
% 
%    isPositiveVec = ISINTERNAL(myEllArr,  matrixOfVecMat, mode) - Checks
%        if vectors specified as columns of matrix matrixOfVecMat
%        belong to the union (mode = 'u'), or intersection (mode = 'i')
%        of the ellipsoids in myEllArr. If myEllArr is a single
%        ellipsoid, then this function checks if points in matrixOfVecMat
%        belong to myEllArr or not. Ellipsoids in myEllArr must be
%        of the same dimension. Column size of matrix  matrixOfVecMat
%        should match the dimension of ellipsoids.
% 
%     Let myEllArr(iEll) = E(q, Q) be an ellipsoid with center q and shape
%     matrix Q. Checking if given vector matrixOfVecMat = x belongs
%     to E(q, Q) is equivalent to checking if inequality
%                     <(x - q), Q^(-1)(x - q)> <= 1
%     holds.
%     If x belongs to at least one of the ellipsoids in the array, then it
%     belongs to the union of these ellipsoids. If x belongs to all
%     ellipsoids in the array,
%     then it belongs to the intersection of these ellipsoids.
%     The default value of the specifier s = 'u'.
% 
%     WARNING: be careful with degenerate ellipsoids.
% 
%  Input:
%    regular:
%        myEllArr: ellipsoid [nDims1,nDims2,...,nDimsN] - array
%            of ellipsoids.
%        matrixOfVecMat: double [mRows, nColsOfVec] - matrix which
%            specifiy points.
% 
%    optional:
%        mode: char[1, 1] - 'u' or 'i', go to description.
% 
%  Output:
%     isPositiveVec: logical[1, nColsOfVec] -
%        true - if vector belongs to the union or intersection
%        of ellipsoids, false - otherwise.
% 
%  $Author: Alex Kurzhanskiy <akurzhan@eecs.berkeley.edu>
%  $Copyright:  The Regents of the University of California 2004-2008 $
%

\end{lstlisting}
\fontfamily{\familydefault}
\selectfont
\item\hypertarget{/elltoolboxcore/@ellipsoid\le}{le}
\fontfamily{pcr}
\selectfont
\begin{lstlisting}
%  LE - checks if the second ellipsoid is bigger than the first one.
%       Same as LT.
% 
%  Input:
%    regular:
%        fstEllArr: ellipsoid [nDims1,nDims2,...,nDimsN] - array of ellipsoids.
%        secEllArr: ellipsoid [nDims1,nDims2,...,nDimsN] - array of ellipsoids
%            of the corresponding dimensions.
% 
%  Output:
%    isPositiveArr: logical[nDims1,nDims2,...,nDimsN],
%        isPositive(iCount) = true - if secEllArr(iCount)
%        contains fstEllArr(iCount)
%        when both have same center, false - otherwise.
% 
%  $Author: Alex Kurzhanskiy <akurzhan@eecs.berkeley.edu>
%  $Copyright:  The Regents of the University of California 2004-2008 $
%

\end{lstlisting}
\fontfamily{\familydefault}
\selectfont
\item\hypertarget{/elltoolboxcore/@ellipsoid\lt}{lt}
\fontfamily{pcr}
\selectfont
\begin{lstlisting}
%  LT - checks if the second ellipsoid is bigger than the first one.
%       The opposite of GT.
% 
%  Input:
%    regular:
%        fstEllArr: ellipsoid [nDims1,nDims2,...,nDimsN] - array of ellipsoids.
%        secEllArr: ellipsoid [nDims1,nDims2,...,nDimsN] - array of ellipsoids
%            of the corresponding dimensions.
% 
%  Output:
%    isPositiveArr: logical[nDims1,nDims2,...,nDimsN],
%        isPositiveArr(iCount) = true - if secEllArr(iCount)
%        contains fstEllArr(iCount)
%        when both have same center, false - otherwise.
% 
%  $Author: Alex Kurzhanskiy <akurzhan@eecs.berkeley.edu>
%  $Copyright:  The Regents of the University of California 2004-2008 $
%

\end{lstlisting}
\fontfamily{\familydefault}
\selectfont
\item\hypertarget{/elltoolboxcore/@ellipsoid\maxeig}{maxeig}
\fontfamily{pcr}
\selectfont
\begin{lstlisting}
%  MAXEIG - return the maximal eigenvalue of the ellipsoid.
% 
%  Input:
%    regular:
%        inpEllArr: ellipsoid [nDims1,nDims2,...,nDimsN] - array of ellipsoids.
% 
%  Output:
%    maxEigArr: double[nDims1,nDims2,...,nDimsN] - array of maximal eigenvalues
%        of ellipsoids in the input matrix inpEllMat.
% 
%  $Author: Alex Kurzhanskiy <akurzhan@eecs.berkeley.edu>
%  $Copyright:  The Regents of the University of California 2004-2008 $
% 
%  $Author: Guliev Rustam <glvrst@gmail.com> $   $Date: Dec-2012$
%  $Copyright: Moscow State University,
%              Faculty of Computational Mathematics and Cybernetics,
%              Science, System Analysis Department 2012 $
% 
%

\end{lstlisting}
\fontfamily{\familydefault}
\selectfont
\item\hypertarget{/elltoolboxcore/@ellipsoid\mineig}{mineig}
\fontfamily{pcr}
\selectfont
\begin{lstlisting}
%  MINEIG - return the minimal eigenvalue of the ellipsoid.
% 
%  Input:
% 	regular:
%        inpEllArr: ellipsoid [nDims1,nDims2,...,nDimsN] - array of ellipsoids.
% 
%  Output:
% 	minEigArr: double[nDims1,nDims2,...,nDimsN] - array of minimal eigenvalues
%        of ellipsoids in the input array inpEllMat.
% 
%  $Author: Alex Kurzhanskiy <akurzhan@eecs.berkeley.edu>
%  $Copyright:  The Regents of the University of California 2004-2008 $
% 
%  $Author: Guliev Rustam <glvrst@gmail.com> $   $Date: Dec-2012$
%  $Copyright: Moscow State University,
%              Faculty of Computational Mathematics and Cybernetics,
%              Science, System Analysis Department 2012 $
% 
%

\end{lstlisting}
\fontfamily{\familydefault}
\selectfont
\item\hypertarget{/elltoolboxcore/@ellipsoid\minkdiff}{minkdiff}
\fontfamily{pcr}
\selectfont
\begin{lstlisting}
%  MINKDIFF - computes geometric (Minkowski) difference of two
%             ellipsoids in 2D or 3D.
% 
%    MINKDIFF(fstEll, secEll, Options) - Computes geometric difference
%        of two ellipsoids fstEll - secEll, if 1 <= dimension(fstEll) =
%        = dimension(secEll) <= 3, and plots it if no output arguments
%        are specified.
%    [centVec, boundPointMat] = MINKDIFF(fstEll, secEll)  Computes
%        geometric difference of two ellipsoids fstEll - secEll.
%        Here centVec is the center, and boundPointMat - matrix
%        whose colums are boundary points.
%    MINKDIFF(fstEll, secEll)  Plots geometric difference of two
%        ellipsoids fstEll - secEll in default (red) color.
%    MINKDIFF(fstEll, secEll, Options)  Plots geometric difference
%        fstEll - secEll using options given in the Options structure.
% 
%    In order for the geometric difference to be nonempty set,
%    ellipsoid fstEll must be bigger than secEll in the sense that
%    if fstEll and secEll had the same center, secEll would be
%    contained inside fstEll.
% 
%  Input:
%    regular:
%        fstEll: ellipsoid [1, 1] - first ellipsoid. Suppose
%            nDim - space dimension, nDim = 2 or 3.
%        secEll: ellipsoid [1, 1] - second ellipsoid
%            of the same dimention.
% 
%    optional:
%        Options: structure[1, 1] - fields:
%            show_all: double[1, 1] - if 1, displays
%                also ellipsoids fstEll and secEll.
%            newfigure: double[1, 1] - if 1, each plot
%                command will open a new figure window.
%            fill: double[1, 1] - if 1, the resulting
%                set in 2D will be filled with color.
%            color: double[1, 3] - sets default colors
%                in the form [x y z].
%            shade: double[1, 1] = 0-1 - level of transparency
%                (0 - transparent, 1 - opaque).
% 
%  Output:
%    centVec: double[nDim, 1]/double[0, 0] - center of the resulting set.
%        centVec may be empty if ellipsoid fsrEll isn't bigger
%        than secEll.
%    boundPointMat: double[nDim, nBoundPoints]/double[0, 0] - set of
%        boundary points (vertices) of resulting set. boundPointMat
%        may be empty if  ellipsoid fstEll isn't bigger than secEll.
% 
%  $Author: Alex Kurzhanskiy <akurzhan@eecs.berkeley.edu>
%  $Copyright:  The Regents of the University of California 2004-2008 $
%

\end{lstlisting}
\fontfamily{\familydefault}
\selectfont
\item\hypertarget{/elltoolboxcore/@ellipsoid\minkdiff\_ea}{minkdiff\_ea}
\fontfamily{pcr}
\selectfont
\begin{lstlisting}
%  MINKDIFF_EA - computation of external approximating ellipsoids
%                of the geometric difference of two ellipsoids along
%                given directions.
% 
%    extApprEllVec = MINKDIFF_EA(fstEll, secEll, directionsMat) -
%        Computes external approximating ellipsoids of the
%        geometric difference of two ellipsoids fstEll - secEll
%        along directions specified by columns of matrix directionsMat
% 
%    First condition for the approximations to be computed, is that
%    ellipsoid fstEll = E1 must be bigger than ellipsoid secEll = E2
%    in the sense that if they had the same center, E2 would be contained
%    inside E1. Otherwise, the geometric difference E1 - E2
%    is an empty set.
%    Second condition for the approximation in the given direction l
%    to exist, is the following. Given
%        P = sqrt(<l, Q1 l>)/sqrt(<l, Q2 l>)
%    where Q1 is the shape matrix of ellipsoid E1, and
%    Q2 - shape matrix of E2, and R being minimal root of the equation
%        det(Q1 - R Q2) = 0,
%    parameter P should be less than R.
%    If both of these conditions are satisfied, then external
%    approximating ellipsoid is defined by its shape matrix
%        Q = (Q1^(1/2) + S Q2^(1/2))' (Q1^(1/2) + S Q2^(1/2)),
%    where S is orthogonal matrix such that vectors
%        Q1^(1/2)l and SQ2^(1/2)l
%    are parallel, and its center
%        q = q1 - q2,
%    where q1 is center of ellipsoid E1 and q2 - center of E2.
% 
%  Input:
%    regular:
%        fstEll: ellipsoid [1, 1] - first ellipsoid. Suppose
%            nDim - space dimension.
%        secEll: ellipsoid [1, 1] - second ellipsoid
%            of the same dimention.
%        directionsMat: double[nDim, nCols] - matrix whose columns
%            specify the directions for which the approximations
%            should be computed.
% 
%  Output:
%    extApprEllVec: ellipsoid [1, nCols] - array of external
%        approximating ellipsoids (empty, if for all specified
%        directions approximations cannot be computed).
% 
%  $Author: Alex Kurzhanskiy <akurzhan@eecs.berkeley.edu>
%  $Copyright:  The Regents of the University of California 2004-2008 $
% 
%  $Author: Guliev Rustam <glvrst@gmail.com> $   $Date: Dec-2012$
%  $Copyright: Moscow State University,
%              Faculty of Computational Mathematics and Cybernetics,
%              Science, System Analysis Department 2012 $
% 
%

\end{lstlisting}
\fontfamily{\familydefault}
\selectfont
\item\hypertarget{/elltoolboxcore/@ellipsoid\minkdiff\_ia}{minkdiff\_ia}
\fontfamily{pcr}
\selectfont
\begin{lstlisting}
%  MINKDIFF_IA - computation of internal approximating ellipsoids
%                of the geometric difference of two ellipsoids along
%                given directions.
% 
%    intApprEllVec = MINKDIFF_IA(fstEll, secEll, directionsMat) -
%        Computes internal approximating ellipsoids of the geometric
%        difference of two ellipsoids fstEll - secEll along directions
%        specified by columns of matrix directionsMat.
% 
%    First condition for the approximations to be computed, is that
%    ellipsoid fstEll = E1 must be bigger than ellipsoid secEll = E2
%    in the sense that if they had the same center, E2 would be contained
%    inside E1. Otherwise, the geometric difference E1 - E2 is an
%    empty set. Second condition for the approximation in the given
%    direction l to exist, is the following. Given
%        P = sqrt(<l, Q1 l>)/sqrt(<l, Q2 l>)
%    where Q1 is the shape matrix of ellipsoid E1,
%    and Q2 - shape matrix of E2, and R being minimal root of the equation
%        det(Q1 - R Q2) = 0,
%    parameter P should be less than R.
%    If these two conditions are satisfied, then internal approximating
%    ellipsoid for the geometric difference E1 - E2 along the
%    direction l is defined by its shape matrix
%        Q = (1 - (1/P)) Q1 + (1 - P) Q2
%    and its center
%        q = q1 - q2,
%    where q1 is center of E1 and q2 - center of E2.
% 
%  Input:
%    regular:
%        fstEll: ellipsoid [1, 1] - first ellipsoid. Suppose
%            nDim - space dimension.
%        secEll: ellipsoid [1, 1] - second ellipsoid
%            of the same dimention.
%        directionsMat: double[nDim, nCols] - matrix whose columns
%            specify the directions for which the approximations
%            should be computed.
% 
%  Output:
%    intApprEllVec: ellipsoid [1, nCols] - array of internal
%        approximating ellipsoids (empty, if for all specified directions
%        approximations cannot be computed).
% 
%  $Author: Alex Kurzhanskiy <akurzhan@eecs.berkeley.edu>
%  $Copyright:  The Regents of the University of California 2004-2008 $
% 
%  $Author: Guliev Rustam <glvrst@gmail.com> $   $Date: Dec-2012$
%  $Copyright: Moscow State University,
%              Faculty of Computational Mathematics and Cybernetics,
%              Science, System Analysis Department 2012 $
% 
%

\end{lstlisting}
\fontfamily{\familydefault}
\selectfont
\item\hypertarget{/elltoolboxcore/@ellipsoid\minkmp}{minkmp}
\fontfamily{pcr}
\selectfont
\begin{lstlisting}
%  MINKMP - computes and plots geometric (Minkowski) sum of the
%           geometric difference of two ellipsoids and the geometric
%           sum of n ellipsoids in 2D or 3D:
%           (E - Em) + (E1 + E2 + ... + En),
%           where E = firstEll, Em = secondEll,
%           E1, E2, ..., En - are ellipsoids in sumEllArr
% 
%    MINKMP(firstEll, secondEll, sumEllArr, Options) - Computes
%        geometric sum of the geometric difference of two ellipsoids
%        firstEll - secondEll and the geometric sum of ellipsoids in
%        the ellipsoidal array sumEllArr, if
%        1 <= dimension(firstEll) = dimension(secondEll) =
%        = dimension(sumEllArr) <= 3, and plots it if no output
%        arguments are specified.
% 
%    [centVec, boundPntMat] = MINKMP(firstEll, secondEll, sumEllArr) -
%        computes: (firstEll - secondEll) +
%        + (geometric sum of ellipsoids in sumEllArr).
%        Here centVec is the center, and
%        boundPntMat - array of boundary points.
%    MINKMP(firstEll, secondEll, sumEllArr) - plots
%        (firstEll - secondEll) +
%        +(geometric sum of ellipsoids in sumEllArr)
%        in default (red) color.
%    MINKMP(firstEll, secondEll, sumEllArr, Options) - plots
%        (firstEll - secondEll) +
%        +(geometric sum of ellipsoids in sumEllArr)
%        using options given in the Options structure.
% 
%  Input:
%    regular:
%        firstEll: ellipsoid [1, 1] - first ellipsoid. Suppose
%            nDim - space dimension, nDim = 2 or 3.
%        secondEll: ellipsoid [1, 1] - second ellipsoid
%            of the same dimention.
%        sumEllArr: ellipsoid [nDims1, nDims2,...,nDimsN] - array of 
%            ellipsoids.
% 
%    optional:
%        Options: structure[1, 1] - fields:
%            show_all: double[1, 1] - if 1, displays
%                also ellipsoids fstEll and secEll.
%            newfigure: double[1, 1] - if 1, each plot
%                command will open a new figure window.
%            fill: double[1, 1] - if 1, the resulting
%                set in 2D will be filled with color.
%            color: double[1, 3] - sets default colors
%                in the form [x y z].
%            shade: double[1, 1] = 0-1 - level of transparency
%                (0 - transparent, 1 - opaque).
% 
%  Output:
%    centerVec: double[nDim, 1] - center of the resulting set.
%    boundarPointsMat: double[nDim, nBoundPoints] - set of boundary
%        points (vertices) of resulting set.
% 
%  $Author: Alex Kurzhanskiy <akurzhan@eecs.berkeley.edu>
%  $Copyright:  The Regents of the University of California 2004-2008 $
% 
%  $Author: Guliev Rustam <glvrst@gmail.com> $   $Date: Nov-2012$
%  $Copyright: Moscow State University,
%              Faculty of Computational Mathematics and Cybernetics,
%              Science, System Analysis Department 2012 $
% 
%

\end{lstlisting}
\fontfamily{\familydefault}
\selectfont
\item\hypertarget{/elltoolboxcore/@ellipsoid\minkmp\_ea}{minkmp\_ea}
\fontfamily{pcr}
\selectfont
\begin{lstlisting}
%  MINKMP_EA - computation of external approximating ellipsoids
%              of (E - Em) + (E1 + ... + En) along given directions.
%              where E = fstEll, Em = secEll,
%              E1, E2, ..., En - are ellipsoids in sumEllArr
% 
%    extApprEllVec = MINKMP_EA(fstEll, secEll, sumEllArr, dirMat) -
%        Computes external approximating
%        ellipsoids of (E - Em) + (E1 + E2 + ... + En),
%        where E1, E2, ..., En are ellipsoids in array sumEllArr,
%        E = fstEll, Em = secEll,
%        along directions specified by columns of matrix dirMat.
% 
%  Input:
%    regular:
%        fstEll: ellipsoid [1, 1] - first ellipsoid. Suppose
%            nDims - space dimension.
%        secEll: ellipsoid [1, 1] - second ellipsoid
%            of the same dimention.
%        sumEllArr: ellipsoid [nDims1, nDims2,...,nDimsN] - array of 
%            ellipsoids of the same dimentions nDims.
%        dirMat: double[nDims, nCols] - matrix whose columns specify the
%            directions for which the approximations should be computed.
% 
%  Output:
%    extApprEllVec: ellipsoid [1, nCols] - array of external
%        approximating ellipsoids (empty, if for all specified
%        directions approximations cannot be computed).
% 
%  $Author: Alex Kurzhanskiy <akurzhan@eecs.berkeley.edu>
%  $Copyright:  The Regents of the University of California 2004-2008 $
% 
%  $Author: Guliev Rustam <glvrst@gmail.com> $   $Date: Dec-2012$
%  $Copyright: Moscow State University,
%              Faculty of Computational Mathematics and Cybernetics,
%              Science, System Analysis Department 2012 $
% 
%

\end{lstlisting}
\fontfamily{\familydefault}
\selectfont
\item\hypertarget{/elltoolboxcore/@ellipsoid\minkmp\_ia}{minkmp\_ia}
\fontfamily{pcr}
\selectfont
\begin{lstlisting}
%  MINKMP_IA - computation of internal approximating ellipsoids
%              of (E - Em) + (E1 + ... + En) along given directions.
%              where E = fstEll, Em = secEll,
%              E1, E2, ..., En - are ellipsoids in sumEllArr
% 
%    intApprEllVec = MINKMP_IA(fstEll, secEll, sumEllArr, dirMat) -
%        Computes internal approximating
%        ellipsoids of (E - Em) + (E1 + E2 + ... + En),
%        where E1, E2, ..., En are ellipsoids in array sumEllArr,
%        E = fstEll, Em = secEll,
%        along directions specified by columns of matrix dirMat.
% 
%  Input:
%    regular:
%        fstEll: ellipsoid [1, 1] - first ellipsoid. Suppose
%            nDim - space dimension.
%        secEll: ellipsoid [1, 1] - second ellipsoid
%            of the same dimention.
%        sumEllArr: ellipsoid [nDims1, nDims2,...,nDimsN] - array of 
%            ellipsoids of the same dimentions.
%        dirMat: double[nDim, nCols] - matrix whose columns specify the
%            directions for which the approximations should be computed.
% 
%  Output:
%    intApprEllVec: ellipsoid [1, nCols] - array of internal
%        approximating ellipsoids (empty, if for all specified
%        directions approximations cannot be computed).
% 
%  $Author: Alex Kurzhanskiy <akurzhan@eecs.berkeley.edu>
%  $Copyright:  The Regents of the University of California 2004-2008 $
% 
%  $Author: Guliev Rustam <glvrst@gmail.com> $   $Date: Dec-2012$
%  $Copyright: Moscow State University,
%              Faculty of Computational Mathematics and Cybernetics,
%              Science, System Analysis Department 2012 $
% 
%

\end{lstlisting}
\fontfamily{\familydefault}
\selectfont
\item\hypertarget{/elltoolboxcore/@ellipsoid\minkpm}{minkpm}
\fontfamily{pcr}
\selectfont
\begin{lstlisting}
%  MINKPM - computes and plots geometric (Minkowski) difference
%           of the geometric sum of ellipsoids and a single ellipsoid
%           in 2D or 3D: (E1 + E2 + ... + En) - E,
%           where E = inpEll,
%           E1, E2, ... En - are ellipsoids in inpEllArr.
% 
%    MINKPM(inpEllArr, inpEll, OPTIONS)  Computes geometric difference
%        of the geometric sum of ellipsoids in inpEllArr and
%        ellipsoid inpEll, if
%        1 <= dimension(inpEllArr) = dimension(inpArr) <= 3,
%        and plots it if no output arguments are specified.
% 
%    [centVec, boundPointMat] = MINKPM(inpEllArr, inpEll) - computes
%        (geometric sum of ellipsoids in inpEllArr) - inpEll.
%        Here centVec is the center, and boundPointMat - array
%        of boundary points.
%    MINKPM(inpEllArr, inpEll) - plots (geometric sum of ellipsoids
%        in inpEllArr) - inpEll in default (red) color.
%    MINKPM(inpEllArr, inpEll, Options) - plots
%        (geometric sum of ellipsoids in inpEllArr) - inpEll using
%        options given in the Options structure.
% 
%  Input:
%    regular:
%        inpEllArr: ellipsoid [nDims1, nDims2,...,nDimsN] - array of 
%            ellipsoids of the same dimentions 2D or 3D.
%        inpEll: ellipsoid [1, 1] - ellipsoid of the same
%            dimention 2D or 3D.
% 
%    optional:
%        Options: structure[1, 1] - fields:
%            show_all: double[1, 1] - if 1, displays
%                also ellipsoids fstEll and secEll.
%            newfigure: double[1, 1] - if 1, each plot
%                command will open a new figure window.
%            fill: double[1, 1] - if 1, the resulting
%                set in 2D will be filled with color.
%            color: double[1, 3] - sets default colors
%                in the form [x y z].
%            shade: double[1, 1] = 0-1 - level of transparency
%                (0 - transparent, 1 - opaque).
% 
%  Output:
%     centVec: double[nDim, 1]/double[0, 0] - center of the resulting set.
%        centerVec may be empty.
%     boundPointMat: double[nDim, ]/double[0, 0] - set of boundary
%        points (vertices) of resulting set. boundPointMat may be empty.
% 
%  $Author: Alex Kurzhanskiy <akurzhan@eecs.berkeley.edu>
%  $Copyright:  The Regents of the University of California 2004-2008 $
% 
%  $Author: Guliev Rustam <glvrst@gmail.com> $   $Date: Dec-2012$
%  $Copyright: Moscow State University,
%              Faculty of Computational Mathematics and Cybernetics,
%              Science, System Analysis Department 2012 $
% 
%

\end{lstlisting}
\fontfamily{\familydefault}
\selectfont
\item\hypertarget{/elltoolboxcore/@ellipsoid\minkpm\_ea}{minkpm\_ea}
\fontfamily{pcr}
\selectfont
\begin{lstlisting}
%  MINKPM_EA - computation of external approximating ellipsoids
%              of (E1 + E2 + ... + En) - E along given directions.
%              where E = inpEll,
%              E1, E2, ... En - are ellipsoids in inpEllArr.
% 
%    ExtApprEllVec = MINKPM_EA(inpEllArr, inpEll, dirMat) - Computes
%        external approximating ellipsoids of
%        (E1 + E2 + ... + En) - E, where E1, E2, ..., En are ellipsoids
%        in array inpEllArr, E = inpEll,
%        along directions specified by columns of matrix dirMat.
% 
%  Input:
%    regular:
%        inpEllArr: ellipsoid [nDims1, nDims2,...,nDimsN] -
%            array of ellipsoids of the same dimentions.
%        inpEll: ellipsoid [1, 1] - ellipsoid of the same dimention.
%        dirMat: double[nDim, nCols] - matrix whose columns specify
%            the directions for which the approximations
%            should be computed.
% 
%  Output:
%    extApprEllVec: ellipsoid [1, nCols]/[0, 0] - array of external
%        approximating ellipsoids. Empty, if for all specified
%        directions approximations cannot be computed.
% 
%  $Author: Alex Kurzhanskiy <akurzhan@eecs.berkeley.edu>
%  $Copyright:  The Regents of the University of California 2004-2008 $
% 
%  $Author: Guliev Rustam <glvrst@gmail.com> $   $Date: Dec-2012$
%  $Copyright: Moscow State University,
%              Faculty of Computational Mathematics and Cybernetics,
%              Science, System Analysis Department 2012 $
% 
%

\end{lstlisting}
\fontfamily{\familydefault}
\selectfont
\item\hypertarget{/elltoolboxcore/@ellipsoid\minkpm\_ia}{minkpm\_ia}
\fontfamily{pcr}
\selectfont
\begin{lstlisting}
%  MINKPM_IA - computation of internal approximating ellipsoids
%              of (E1 + E2 + ... + En) - E along given directions.
%              where E = inpEll,
%              E1, E2, ... En - are ellipsoids in inpEllArr.
% 
%    intApprEllVec = MINKPM_IA(inpEllArr, inpEll, dirMat) - Computes
%        internal approximating ellipsoids of
%        (E1 + E2 + ... + En) - E, where E1, E2, ..., En are ellipsoids
%        in array inpEllArr, E = inpEll,
%        along directions specified by columns of matrix dirArr.
% 
%  Input:
%    regular:
%        inpEllArr: ellipsoid [nDims1, nDims2,...,nDimsN] -
%            array of ellipsoids of the same dimentions.
%        inpEll: ellipsoid [1, 1] - ellipsoid of the same dimention.
%        dirMat: double[nDim, nCols] - matrix whose columns specify
%            the directions for which the approximations
%            should be computed.
% 
%  Output:
%    intApprEllVec: ellipsoid [1, nCols]/[0, 0] - array of internal
%        approximating ellipsoids. Empty, if for all specified
%        directions approximations cannot be computed.
% 
%  $Author: Alex Kurzhanskiy <akurzhan@eecs.berkeley.edu>
%  $Copyright:  The Regents of the University of California 2004-2008 $
% 
%  $Author: Guliev Rustam <glvrst@gmail.com> $   $Date: Dec-2012$
%  $Copyright: Moscow State University,
%              Faculty of Computational Mathematics and Cybernetics,
%              Science, System Analysis Department 2012 $
% 
%

\end{lstlisting}
\fontfamily{\familydefault}
\selectfont
\item\hypertarget{/elltoolboxcore/@ellipsoid\minksum}{minksum}
\fontfamily{pcr}
\selectfont
\begin{lstlisting}
%  MINKSUM - computes geometric (Minkowski) sum of ellipsoids in 2D or 3D.
% 
%    MINKSUM(inpEllArr, Options) - Computes geometric sum of ellipsoids
%        in the array inpEllArr, if
%        1 <= min(dimension(inpEllArr)) = max(dimension(inpEllArr)) <= 3,
%        and plots it if no output arguments are specified.
% 
%    [centVec, boundPointMat] = MINKSUM(inpEllArr) - Computes
%        geometric sum of ellipsoids in inpEllArr. Here centVec is
%        the center, and boundPointMat - array of boundary points.
%    MINKSUM(inpEllArr) - Plots geometric sum of ellipsoids in
%        inpEllArr in default (red) color.
%    MINKSUM(inpEllArr, Options) - Plots geometric sum of inpEllMat
%        using options given in the Options structure.
% 
%  Input:
%    regular:
%        inpEllArr: ellipsoid [nDims1, nDims2,...,nDimsN] - array of 
%            ellipsoids of the same dimentions 2D or 3D.
% 
%    optional:
%        Options: structure[1, 1] - fields:
%            show_all: double[1, 1] - if 1, displays
%                also ellipsoids fstEll and secEll.
%            newfigure: double[1, 1] - if 1, each plot
%                command will open a new figure window.
%            fill: double[1, 1] - if 1, the resulting
%                set in 2D will be filled with color.
%            color: double[1, 3] - sets default colors
%                in the form [x y z].
%            shade: double[1, 1] = 0-1 - level of transparency
%                (0 - transparent, 1 - opaque).
% 
%  Output:
%    centVec: double[nDim, 1] - center of the resulting set.
%    boundPointMat: double[nDim, nBoundPoints] - set of boundary
%        points (vertices) of resulting set.
% 
%  $Author: Alex Kurzhanskiy <akurzhan@eecs.berkeley.edu>
%  $Copyright:  The Regents of the University of California 2004-2008 $
% 
%  $Author: Guliev Rustam <glvrst@gmail.com> $   $Date: Dec-2012$
%  $Copyright: Moscow State University,
%              Faculty of Computational Mathematics and Cybernetics,
%              Science, System Analysis Department 2012 $
% 
%

\end{lstlisting}
\fontfamily{\familydefault}
\selectfont
\item\hypertarget{/elltoolboxcore/@ellipsoid\minksum\_ea}{minksum\_ea}
\fontfamily{pcr}
\selectfont
\begin{lstlisting}
%  MINKSUM_EA - computation of external approximating ellipsoids
%               of the geometric sum of ellipsoids along given directions.
% 
%    extApprEllVec = MINKSUM_EA(inpEllArr, dirMat) - Computes
%        tight external approximating ellipsoids for the geometric
%        sum of the ellipsoids in the array inpEllArr along directions
%        specified by columns of dirMat.
%        If ellipsoids in inpEllArr are n-dimensional, matrix
%        dirMat must have dimension (n x k) where k can be
%        arbitrarily chosen.
%        In this case, the output of the function will contain k
%        ellipsoids computed for k directions specified in dirMat.
% 
%    Let inpEllArr consists of E(q1, Q1), E(q2, Q2), ..., E(qm, Qm) -
%    ellipsoids in R^n, and dirMat(:, iCol) = l - some vector in R^n.
%    Then tight external approximating ellipsoid E(q, Q) for the
%    geometric sum E(q1, Q1) + E(q2, Q2) + ... + E(qm, Qm)
%    along direction l, is such that
%        rho(l | E(q, Q)) = rho(l | (E(q1, Q1) + ... + E(qm, Qm)))
%    and is defined as follows:
%        q = q1 + q2 + ... + qm,
%        Q = (p1 + ... + pm)((1/p1)Q1 + ... + (1/pm)Qm),
%    where
%        p1 = sqrt(<l, Q1l>), ..., pm = sqrt(<l, Qml>).
% 
%  Input:
%    regular:
%        inpEllArr: ellipsoid [nDims1, nDims2,...,nDimsN] - array
%            of ellipsoids of the same dimentions.
%        dirMat: double[nDims, nCols] - matrix whose columns specify
%            the directions for which the approximations
%            should be computed.
% 
%  Output:
%    extApprEllVec: ellipsoid [1, nCols] - array of external
%        approximating ellipsoids.
% 
%  $Author: Alex Kurzhanskiy <akurzhan@eecs.berkeley.edu>
%  $Copyright:  The Regents of the University of California 2004-2008 $
% 
%  $Author: Guliev Rustam <glvrst@gmail.com> $   $Date: Dec-2012$
%  $Copyright: Moscow State University,
%              Faculty of Computational Mathematics and Cybernetics,
%              Science, System Analysis Department 2012 $
% 
%

\end{lstlisting}
\fontfamily{\familydefault}
\selectfont
\item\hypertarget{/elltoolboxcore/@ellipsoid\minksum\_ia}{minksum\_ia}
\fontfamily{pcr}
\selectfont
\begin{lstlisting}
%  MINKSUM_IA - computation of internal approximating ellipsoids
%               of the geometric sum of ellipsoids along given directions.
% 
%    intApprEllVec = MINKSUM_IA(inpEllArr, dirMat) - Computes
%        tight internal approximating ellipsoids for the geometric
%        sum of the ellipsoids in the array inpEllArr along directions
%        specified by columns of dirMat. If ellipsoids in
%        inpEllArr are n-dimensional, matrix dirMat must have
%        dimension (n x k) where k can be arbitrarily chosen.
%        In this case, the output of the function will contain k
%        ellipsoids computed for k directions specified in dirMat.
% 
%    Let inpEllArr consists of E(q1, Q1), E(q2, Q2), ..., E(qm, Qm) -
%    ellipsoids in R^n, and dirMat(:, iCol) = l - some vector in R^n.
%    Then tight internal approximating ellipsoid E(q, Q) for the
%    geometric sum E(q1, Q1) + E(q2, Q2) + ... + E(qm, Qm) along
%    direction l, is such that
%        rho(l | E(q, Q)) = rho(l | (E(q1, Q1) + ... + E(qm, Qm)))
%    and is defined as follows:
%        q = q1 + q2 + ... + qm,
%        Q = (S1 Q1^(1/2) + ... + Sm Qm^(1/2))' *
%            * (S1 Q1^(1/2) + ... + Sm Qm^(1/2)),
%    where S1 = I (identity), and S2, ..., Sm are orthogonal
%    matrices such that vectors
%    (S1 Q1^(1/2) l), ..., (Sm Qm^(1/2) l) are parallel.
% 
%  Input:
%    regular:
%        inpEllArr: ellipsoid [nDims1, nDims2,...,nDimsN] - array
%            of ellipsoids of the same dimentions.
%        dirMat: double[nDim, nCols] - matrix whose columns specify the
%            directions for which the approximations should be computed.
% 
%  Output:
%    intApprEllVec: ellipsoid [1, nCols] - array of internal
%        approximating ellipsoids.
% 
%  $Author: Alex Kurzhanskiy <akurzhan@eecs.berkeley.edu>
%  $Copyright:  The Regents of the University of California 2004-2008 $
% 
%  $Author: Guliev Rustam <glvrst@gmail.com> $   $Date: Dec-2012$
%  $Copyright: Moscow State University,
%              Faculty of Computational Mathematics and Cybernetics,
%              Science, System Analysis Department 2012 $
% 
%

\end{lstlisting}
\fontfamily{\familydefault}
\selectfont
\item\hypertarget{/elltoolboxcore/@ellipsoid\minus}{minus}
\fontfamily{pcr}
\selectfont
\begin{lstlisting}
%  MINUS - overloaded operator '-'
% 
%    outEllArr = MINUS(inpEllArr, inpVec) implements E(q, Q) - b
%        for each ellipsoid E(q, Q) in inpEllArr.
%    outEllArr = MINUS(inpVec, inpEllArr) implements b - E(q, Q)
%        for each ellipsoid E(q, Q) in inpEllArr.
% 
%    Operation E - b where E = inpEll is an ellipsoid in R^n,
%    and b = inpVec - vector in R^n. If E(q, Q) is an ellipsoid
%    with center q and shape matrix Q, then
%    E(q, Q) - b = E(q - b, Q).
% 
%  Input:
%    regular:
%        inpEllArr: ellipsoid [nDims1,nDims2,...,nDimsN] - array of 
%            ellipsoids of the same dimentions nDims.
%        inpVec: double[nDims, 1] - vector.
% 
%  Output:
% 	outEllVec: ellipsoid [nDims1,nDims2,...,nDimsN] - array of ellipsoids 
%        with same shapes as inpEllVec, but with centers shifted by vectors 
%        in -inpVec.
% 
%  $Author: Alex Kurzhanskiy <akurzhan@eecs.berkeley.edu>
%  $Copyright:  The Regents of the University of California 2004-2008 $
% 
%  $Author: Guliev Rustam <glvrst@gmail.com> $   $Date: Dec-2012$
%  $Copyright: Moscow State University,
%              Faculty of Computational Mathematics and Cybernetics,
%              Science, System Analysis Department 2012 $
% 
%

\end{lstlisting}
\fontfamily{\familydefault}
\selectfont
\item\hypertarget{/elltoolboxcore/@ellipsoid\move2origin}{move2origin}
\fontfamily{pcr}
\selectfont
\begin{lstlisting}
%  MOVE2ORIGIN - moves ellipsoids in the given array to the origin.
% 
%    outEllArr = MOVE2ORIGIN(inpEll) - Replaces the centers of
%        ellipsoids in inpEllArr with zero vectors.
% 
%  Input:
%    regular:
%        inpEllArr: ellipsoid [nDims1,nDims2,...,nDimsN] - array of 
%            ellipsoids.
% 
%  Output:
%    outEllArr: ellipsoid [nDims1,nDims2,...,nDimsN] - array of ellipsoids
%        with the same shapes as in inpEllArr centered at the origin.
% 
%  $Author: Alex Kurzhanskiy <akurzhan@eecs.berkeley.edu>
%  $Copyright:  The Regents of the University of California 2004-2008 $
% 
%  $Author: Guliev Rustam <glvrst@gmail.com> $   $Date: Dec-2012$
%  $Copyright: Moscow State University,
%              Faculty of Computational Mathematics and Cybernetics,
%              Science, System Analysis Department 2012 $
% 
%

\end{lstlisting}
\fontfamily{\familydefault}
\selectfont
\item\hypertarget{/elltoolboxcore/@ellipsoid\mtimes}{mtimes}
\fontfamily{pcr}
\selectfont
\begin{lstlisting}
%  MTIMES - overloaded operator '*'.
% 
%    Multiplication of the ellipsoid by a matrix or a scalar.
%    If inpEllVec(iEll) = E(q, Q) is an ellipsoid, and
%    multMat = A - matrix of suitable dimensions,
%    then A E(q, Q) = E(Aq, AQA').
% 
%  Input:
%    regular:
%        multMat: double[mRows, nDims]/[1, 1] - scalar or
%            matrix in R^{mRows x nDim}
%        inpEllVec: ellipsoid [1, nCols] - array of ellipsoids.
% 
%  Output:
%    outEllVec: ellipsoid [1, nCols] - resulting ellipsoids.
% 
%  $Author: Alex Kurzhanskiy <akurzhan@eecs.berkeley.edu>
%  $Copyright:  The Regents of the University of California 2004-2008 $
% 
%  $Author: Guliev Rustam <glvrst@gmail.com> $   $Date: Dec-2012$
%  $Copyright: Moscow State University,
%              Faculty of Computational Mathematics and Cybernetics,
%              Science, System Analysis Department 2012 $
% 
%

\end{lstlisting}
\fontfamily{\familydefault}
\selectfont
\item\hypertarget{/elltoolboxcore/@ellipsoid\my\_color\_table}{my\_color\_table}
\fontfamily{pcr}
\selectfont
\begin{lstlisting}
%

\end{lstlisting}
\fontfamily{\familydefault}
\selectfont
\item\hypertarget{/elltoolboxcore/@ellipsoid\ne}{ne}
\fontfamily{pcr}
\selectfont
\begin{lstlisting}
%  NE - the opposite of EQ
% 
%  Input:
%    regular:
%        ellFirstArr: ellipsoid: [nDims1,nDims2,...,nDimsN]/[1,1]- the first
%            array of ellipsoid objects
%        ellSecArr: ellipsoid: [nDims1,nDims2,...,nDimsN]/[1,1] - the second
%            array of ellipsoid objects
% 
%  Output:
%    isNeqArr: logical: [nDims1,nDims2,...,nDimsN]- array of comparison
%        results
% 
%    reportStr: char[1,] - comparison report
% 
%  $Author: Alex Kurzhanskiy <akurzhan@eecs.berkeley.edu>
%  $Copyright:  The Regents of the University of California 2004-2008 $
%

\end{lstlisting}
\fontfamily{\familydefault}
\selectfont
\item\hypertarget{/elltoolboxcore/@ellipsoid\parameters}{parameters}
\fontfamily{pcr}
\selectfont
\begin{lstlisting}
%  PARAMETERS - returns parameters of the ellipsoid.
% 
%  Input:
%    regular:
%        myEll: ellipsoid [1, 1] - single ellipsoid of dimention nDims.
% 
%  Output:
%    myEllCenterVec: double[nDims, 1] - center of the ellipsoid myEll.
%    myEllShapeMat: double[nDims, nDims] - shape matrix
%        of the ellipsoid myEll.
% 
%  $Author: Alex Kurzhanskiy <akurzhan@eecs.berkeley.edu>
%  $Copyright:  The Regents of the University of California 2004-2008 $
% 
%  $Author: Guliev Rustam <glvrst@gmail.com> $   $Date: Nov-2012$
%  $Copyright: Moscow State University,
%              Faculty of Computational Mathematics and Cybernetics,
%              Science, System Analysis Department 2012 $
% 
%

\end{lstlisting}
\fontfamily{\familydefault}
\selectfont
\item\hypertarget{/elltoolboxcore/@ellipsoid\plot}{plot}
\fontfamily{pcr}
\selectfont
\begin{lstlisting}
%  PLOT - plots ellipsoids in 2D or 3D.
% 
% 
%  Description:
%  ------------
% 
%  PLOT(E, OPTIONS) plots ellipsoid E if 1 <= dimension(E) <= 3.
% 
%                   PLOT(E)  Plots E in default (red) color.
%               PLOT(EA, E)  Plots array of ellipsoids EA and single ellipsoid E.
%    PLOT(E1, 'g', E2, 'b')  Plots E1 in green and E2 in blue color.
%         PLOT(EA, Options)  Plots EA using options given in the Options structure.
% 
% 
%  Options.newfigure    - if 1, each plot command will open a new figure window.
%  Options.fill         - if 1, ellipsoids in 2D will be filled with color.
%  Options.width        - line width for 1D and 2D plots.
%  Options.color        - sets default colors in the form [x y z].
%  Options.shade = 0-1  - level of transparency (0 - transparent, 1 - opaque).
% 
% 
%  Output:
%  -------
% 
%     None.
% 
% 
%  See also:
%  ---------
% 
%     ELLIPSOID/ELLIPSOID.
% 
%

\end{lstlisting}
\fontfamily{\familydefault}
\selectfont
\item\hypertarget{/elltoolboxcore/@ellipsoid\plot3}{plot3}
\fontfamily{pcr}
\selectfont
\begin{lstlisting}
%  PLOT3 - plots ellipsoids in 2D or 3D.
% 
% 
%  Description:
%  ------------
% 
%  PLOT3(E, OPTIONS) plots ellipsoid E if 1 <= dimension(E) <= 3.
% 
%                  PLOT3(E)  Plots E in default (red) color.
%              PLOT3(EA, E)  Plots array of ellipsoids EA and single ellipsoid E.
%   PLOT3(E1, 'g', E2, 'b')  Plots E1 in green and E2 in blue color.
%        PLOT3(EA, Options)  Plots EA using options given in the Options structure.
% 
% 
%  Options.newfigure    - if 1, each plot command will open a new figure window.
%  Options.fill         - if 1, ellipsoids in 2D will be filled with color.
%  Options.width        - line width for 1D and 2D plots.
%  Options.color        - sets default colors in the form [x y z].
%  Options.shade = 0-1  - level of transparency (0 - transparent, 1 - opaque).
% 
% 
%  Output:
%  -------
% 
%     None.
% 
% 
%  See also:
%  ---------
% 
%     ELLIPSOID/ELLIPSOID, ELLIPSOID/PLOT.
% 
%

\end{lstlisting}
\fontfamily{\familydefault}
\selectfont
\item\hypertarget{/elltoolboxcore/@ellipsoid\plus}{plus}
\fontfamily{pcr}
\selectfont
\begin{lstlisting}
%  PLUS - overloaded operator '+'
% 
%    outEllArr = PLUS(inpEllArr, inpVec) implements E(q, Q) + b
%        for each ellipsoid E(q, Q) in inpEllArr.
%    outEllArr = PLUS(inpVec, inpEllArr) implements b + E(q, Q)
%        for each ellipsoid E(q, Q) in inpEllArr.
% 
% 	Operation E + b (or b+E) where E = inpEll is an ellipsoid in R^n,
%    and b=inpVec - vector in R^n. If E(q, Q) is an ellipsoid
%    with center q and shape matrix Q, then
%    E(q, Q) + b = b + E(q,Q) = E(q + b, Q).
% 
%  Input:
%    regular:
%        ellArr: ellipsoid [nDims1,nDims2,...,nDimsN] - array of ellipsoids
%            of the same dimentions nDims.
%        bVec: double[nDims, 1] - vector.
% 
%  Output:
%    outEllArr: ellipsoid [nDims1,nDims2,...,nDimsN] - array of ellipsoids 
%        with same shapes as ellVec, but with centers shifted by vectors 
%        in inpVec.
% 
%  $Author: Alex Kurzhanskiy <akurzhan@eecs.berkeley.edu>
%  $Copyright:  The Regents of the University of California 2004-2008 $
% 
%  $Author: Guliev Rustam <glvrst@gmail.com> $   $Date: Dec-2012$
%  $Copyright: Moscow State University,
%              Faculty of Computational Mathematics and Cybernetics,
%              Science, System Analysis Department 2012 $
% 
%

\end{lstlisting}
\fontfamily{\familydefault}
\selectfont
\item\hypertarget{/elltoolboxcore/@ellipsoid\polar}{polar}
\fontfamily{pcr}
\selectfont
\begin{lstlisting}
%  POLAR - computes the polar ellipsoids.
% 
%    polEllArr = POLAR(ellArr)  Computes the polar ellipsoids for those
%        ellipsoids in ellArr, for which the origin is an interior point.
%        For those ellipsoids in E, for which this condition does not hold,
%        an empty ellipsoid is returned.
% 
%    Given ellipsoid E(q, Q) where q is its center, and Q - its shape matrix,
%    the polar set to E(q, Q) is defined as follows:
%    P = { l in R^n  | <l, q> + sqrt(<l, Q l>) <= 1 }
%    If the origin is an interior point of ellipsoid E(q, Q),
%    then its polar set P is an ellipsoid.
% 
%  Input:
%    regular:
%        ellArr: ellipsoid [nDims1,nDims2,...,nDimsN] - array
%            of ellipsoids.
% 
%  Output:
%    polEllArr: ellipsoid [nDims1,nDims2,...,nDimsN] - array of
%     	polar ellipsoids.
% 
%  $Author: Alex Kurzhanskiy <akurzhan@eecs.berkeley.edu>
%  $Copyright:  The Regents of the University of California 2004-2008 $
% 
%  $Author: Guliev Rustam <glvrst@gmail.com> $   $Date: Dec-2012$
%  $Copyright: Moscow State University,
%              Faculty of Computational Mathematics and Cybernetics,
%              Science, System Analysis Department 2012 $
% 
%

\end{lstlisting}
\fontfamily{\familydefault}
\selectfont
\item\hypertarget{/elltoolboxcore/@ellipsoid\projection}{projection}
\fontfamily{pcr}
\selectfont
\begin{lstlisting}
%  PROJECTION - computes projection of the ellipsoid onto the given subspace.
% 
%    projEllArr = projection(ellArr, basisMat)  Computes projection of the 
%        ellipsoid ellArr onto a subspace, specified by orthogonal 
%        basis vectors basisMat. ellArr can be an array of ellipsoids of 
%        the same dimension. Columns of B must be orthogonal vectors.
% 
%  Input:
%    regular:
%        ellArr: ellipsoid [nDims1,nDims2,...,nDimsN] - array
%            of ellipsoids.
%        basisMat: double[nDim, nSubSpDim] - matrix of orthogonal basis
%            vectors
% 
%  Output:
%    projEllArr: ellipsoid [nDims1,nDims2,...,nDimsN] - array of
%        projected ellipsoids, generally, of lower dimension.
% 
%  $Author: Alex Kurzhanskiy <akurzhan@eecs.berkeley.edu>
%  $Copyright:  The Regents of the University of California 2004-2008 $
% 
%  $Author: Guliev Rustam <glvrst@gmail.com> $   $Date: Dec-2012$
%  $Copyright: Moscow State University,
%              Faculty of Computational Mathematics and Cybernetics,
%              Science, System Analysis Department 2012 $
% 
%

\end{lstlisting}
\fontfamily{\familydefault}
\selectfont
\item\hypertarget{/elltoolboxcore/@ellipsoid\regularize}{regularize}
\fontfamily{pcr}
\selectfont
\begin{lstlisting}
%

\end{lstlisting}
\fontfamily{\familydefault}
\selectfont
\item\hypertarget{/elltoolboxcore/@ellipsoid\rho}{rho}
\fontfamily{pcr}
\selectfont
\begin{lstlisting}
%  RHO - computes the values of the support function for given ellipsoid
% 	and given direction.
% 
% 	supArr = RHO(ellArr, dirsMat)  Computes the support function of the 
%        ellipsoid ellArr in directions specified by the columns of matrix 
%        dirsMat. Or, if ellArr is array of ellipsoids, dirsMat is expected 
%        to be a single vector.
% 
% 	[supArr, bpMat] = RHO(ellArr, dirstMat)  Computes the support function 
%        of the ellipsoid ellArr in directions specified by the columns of 
%        matrix dirsMat, and boundary points bpMat of this ellipsoid that 
%        correspond to directions in dirsMat. Or, if ellArr is array of 
%        ellipsoids, and dirsMat - single vector, then support functions and 
%        corresponding boundary points are computed for all the given 
%        ellipsoids in the array in the specified direction dirsMat.
% 
% 	The support function is defined as
%    (1)  rho(l | E) = sup { <l, x> : x belongs to E }.
% 	For ellipsoid E(q,Q), where q is its center and Q - shape matrix,
%    it is simplified to
%    (2)  rho(l | E) = <q, l> + sqrt(<l, Ql>)
%    Vector x, at which the maximum at (1) is achieved is defined by
%    (3)  q + Ql/sqrt(<l, Ql>)
% 
%  Input:
%    regular:
%        ellArr: ellipsoid [nDims1,nDims2,...,nDimsN]/[1,1] - array
%            of ellipsoids.
%        dirsMat: double[nDim,nDirs]/[nDim,1] - matrix of directions.
% 
%  Output:
% 	supArr: double [nDims1,nDims2,...,nDimsN]/[1,nDirs] - support function 
%        of the ellArr in directions specified by the columns of matrix 
%        dirsMat. Or, if ellArr is array of ellipsoids, support function of
%        each ellipsoid in ellArr specified by dirsMat direction.
% 
%    bpMat: double [nDim,nDims1*nDims2*...*nDimsN]/[nDim,nDirs] - matrix of
%        boundary points
% 
%  $Author: Alex Kurzhanskiy <akurzhan@eecs.berkeley.edu>
%  $Copyright:  The Regesnts of the University of California 2004-2008 $
% 
%  $Author: Guliev Rustam <glvrst@gmail.com> $   $Date: Dec-2012$
%  $Copyright: Moscow State University,
%              Faculty of Computational Mathematics and Cybernetics,
%              Science, System Analysis Department 2012 $
% 
%

\end{lstlisting}
\fontfamily{\familydefault}
\selectfont
\item\hypertarget{/elltoolboxcore/@ellipsoid\rm\_bad\_directions}{rm\_bad\_directions}
\fontfamily{pcr}
\selectfont
\begin{lstlisting}
%

\end{lstlisting}
\fontfamily{\familydefault}
\selectfont
\item\hypertarget{/elltoolboxcore/@ellipsoid\shape}{shape}
\fontfamily{pcr}
\selectfont
\begin{lstlisting}
%  SHAPE - modifies the shape matrix of the ellipsoid without
%    changing its center.
% 
% 	modEllArr = SHAPE(ellArr, modMat)  Modifies the shape matrices of
%        the ellipsoids in the ellipsoidal array ellArr. The centers
%        remain untouched - that is the difference of the function SHAPE and
%        linear transformation modMat*ellArr. modMat is expected to be a
%        scalar or a square matrix of suitable dimension.
% 
%  Input:
%    regular:
%        ellArr: ellipsoid [nDims1,nDims2,...,nDimsN] - array
%            of ellipsoids.
%        modMat: double[nDim, nDim]/[1,1] - square matrix or scalar
% 
%  Output:
% 	modEllArr: ellipsoid [nDims1,nDims2,...,nDimsN] - array of modified
%        ellipsoids.
% 
%  $Author: Alex Kurzhanskiy <akurzhan@eecs.berkeley.edu>
%  $Copyright:  The Regents of the University of California 2004-2008 $
% 
%  $Author: Guliev Rustam <glvrst@gmail.com> $   $Date: Dec-2012$
%  $Copyright: Moscow State University,
%              Faculty of Computational Mathematics and Cybernetics,
%              Science, System Analysis Department 2012 $
% 
%

\end{lstlisting}
\fontfamily{\familydefault}
\selectfont
\item\hypertarget{/elltoolboxcore/@ellipsoid\trace}{trace}
\fontfamily{pcr}
\selectfont
\begin{lstlisting}
%  TRACE - returns the trace of the ellipsoid.
% 
%     trArr = TRACE(ellArr)  Computes the trace of ellipsoids in
%        ellipsoidal array ellArr.
% 
%  Input:
%    regular:
%        ellArr: ellipsoid [nDims1,nDims2,...,nDimsN] - array
%            of ellipsoids.
% 
%  Output:
% 	trArr: double [nDims1,nDims2,...,nDimsN] - array of trace values, 
%        same size as ellArr.
% 
%  $Author: Alex Kurzhanskiy <akurzhan@eecs.berkeley.edu>
%  $Copyright:  The Regents of the University of California 2004-2008 $
% 
%  $Author: Guliev Rustam <glvrst@gmail.com> $   $Date: Dec-2012$
%  $Copyright: Moscow State University,
%              Faculty of Computational Mathematics and Cybernetics,
%              Science, System Analysis Department 2012 $
% 
%

\end{lstlisting}
\fontfamily{\familydefault}
\selectfont
\item\hypertarget{/elltoolboxcore/@ellipsoid\uminus}{uminus}
\fontfamily{pcr}
\selectfont
\begin{lstlisting}
%  UMINUS - changes the sign of the center of ellipsoid.
% 
%  Input:
% 	regular:
%        ellArr: ellipsoid [nDims1,nDims2,...,nDimsN] - array
%            of ellipsoids.
% 
%  Output:
% 	outEllArr: ellipsoid [nDims1,nDims2,...,nDimsN] - array of
%     	ellipsoids, same size as ellArr.
% 
%  $Author: Alex Kurzhanskiy <akurzhan@eecs.berkeley.edu>
%  $Copyright:  The Regents of the University of California 2004-2008 $
% 
%  $Author: Guliev Rustam <glvrst@gmail.com> $   $Date: Dec-2012$
%  $Copyright: Moscow State University,
%              Faculty of Computational Mathematics and Cybernetics,
%              Science, System Analysis Department 2012 $
% 
%

\end{lstlisting}
\fontfamily{\familydefault}
\selectfont
\item\hypertarget{/elltoolboxcore/@ellipsoid\volume}{volume}
\fontfamily{pcr}
\selectfont
\begin{lstlisting}
%  VOLUME - returns the volume of the ellipsoid.
% 
% 	volArr = VOLUME(ellArr)  Computes the volume of ellipsoids in
%        ellipsoidal array ellArr.
% 
% 	The volume of ellipsoid E(q, Q) with center q and shape matrix Q 
% 	is given by V = S sqrt(det(Q)) where S is the volume of unit ball.
% 
%  Input:
%    regular:
%        ellArr: ellipsoid [nDims1,nDims2,...,nDimsN] - array
%            of ellipsoids.
% 
%  Output:
% 	volArr: double [nDims1,nDims2,...,nDimsN] - array of
%    	volume values, same size as ellArr.
% 
%  $Author: Alex Kurzhanskiy <akurzhan@eecs.berkeley.edu>
%  $Copyright:  The Regents of the University of California 2004-2008 $
% 
%  $Author: Guliev Rustam <glvrst@gmail.com> $   $Date: Dec-2012$
%  $Copyright: Moscow State University,
%              Faculty of Computational Mathematics and Cybernetics,
%              Science, System Analysis Department 2012 $
% s
%

\end{lstlisting}
\fontfamily{\familydefault}
\selectfont
\end{enumerate}
\subsection{/elltoolboxcore/@hyperplane}
\begin{enumerate}
\item\hypertarget{/elltoolboxcore/@hyperplane\checkIsMe}{checkIsMe}
\fontfamily{pcr}
\selectfont
\begin{lstlisting}
%  CHECKISME - determine whether input object is hyperplane. And display
%              message and abort function if input object
%              is not hyperplane
% 
%  Input:
%    regular:
%        someObjArr: any[] - any type array of objects.
% 
%  $Author: Aushkap Nikolay <n.aushkap@gmail.com> $  $Date: 30-11-2012$
%  $Copyright: Moscow State University,
%    Faculty of Computational Mathematics and Computer Science,
%    System Analysis Department 2012 $
%

\end{lstlisting}
\fontfamily{\familydefault}
\selectfont
\item\hypertarget{/elltoolboxcore/@hyperplane\contains}{contains}
\fontfamily{pcr}
\selectfont
\begin{lstlisting}
%  CONTAINS - checks if given vectors belong to the hyperplanes.
% 
%    isPosArr = CONTAINS(myHypArr, xArr) - Checks if vectors specified
%        by columns xArr(:, hpDim1, hpDim2, ...) belong
%        to hyperplanes in myHypArr.
% 
%  Input:
%    regular:
%        myHypArr: hyperplane [nCols, 1]/[1, nCols]/
%            /[hpDim1, hpDim2, ...]/[1, 1] - array of hyperplanes
%            of the same dimentions nDims.
%        xArr: double[nDims, nCols]/[nDims, hpDim1, hpDim2, ...]/
%            /[nDims, 1]/[nDims, nVecArrDim1, nVecArrDim2, ...] - array
%            whose columns represent the vectors needed to be checked.
% 
%            note: if size of myHypArr is [hpDim1, hpDim2, ...], then
%                size of xArr is [nDims, hpDim1, hpDim2, ...]
%                or [nDims, 1], if size of myHypArr [1, 1], then xArr
%                can be any size [nDims, nVecArrDim1, nVecArrDim2, ...],
%                in this case output variable will has
%                size [1, nVecArrDim1, nVecArrDim2, ...]. If size of
%                xArr is [nDims, nCols], then size of myHypArr may be
%                [nCols, 1] or [1, nCols] or [1, 1], output variable
%                will has size respectively
%                [nCols, 1] or [1, nCols] or [nCols, 1].
% 
%  Output:
%    isPosArr: logical[hpDim1, hpDim2,...] /
%        / logical[1, nVecArrDim1, nVecArrDim2, ...],
%        isPosArr(iDim1, iDim2, ...) = true - myHypArr(iDim1, iDim2, ...)
%        contains xArr(:, iDim1, iDim2, ...), false - otherwise.
% 
% 
%  $Author: Alex Kurzhanskiy <akurzhan@eecs.berkeley.edu>
%  $Copyright:  The Regents of the University of California 2004-2008 $
% 
%  $Authors:
%    Zakharov Eugene <justenterrr@gmail.com>$ $Date: <31 october>$
%    Aushkap Nikolay <n.aushkap@gmail.com> $  $Date: 30-11-2012$
%  $Copyright: Moscow State University,
%    Faculty of Computational Mathematics and Computer Science,
%    System Analysis Department 2012 $
%

\end{lstlisting}
\fontfamily{\familydefault}
\selectfont
\item\hypertarget{/elltoolboxcore/@hyperplane\contents}{contents}
\fontfamily{pcr}
\selectfont
\begin{lstlisting}
%  Hyperplane object of the Ellipsoidal Toolbox.
% 
%  
%  Functions:
%  ----------
%   hyperplane - Constructor of hyperplane object.
%   double     - Returns parameters of hyperplane, i.e. normal vector and shift.
%   parameters - Same function as 'double' (legacy matter).
%   dimension  - Returns dimension of hyperplane.
%   isempty    - Checks if hyperplane is empty.
%   isparallel - Checks if one hyperplane is parallel to the other one.
%   contains   - Check if hyperplane contains given point.
% 
% 
%  Overloaded operators and functions:
%  -----------------------------------
%   eq      - Checks if two hyperplanes are equal.
%   ne      - The opposite of 'eq'.
%   uminus  - Switches signs of normal and shift parameters to the opposite.
%   display - Displays the details about given hyperplane object.
%   plot    - Plots hyperplane in 2D and 3D.
% 
% 
%  Author:
%  -------
%     Alex Kurzhanskiy <akurzhan@eecs.berkeley.edu>
% 
%

\end{lstlisting}
\fontfamily{\familydefault}
\selectfont
\item\hypertarget{/elltoolboxcore/@hyperplane\dimension}{dimension}
\fontfamily{pcr}
\selectfont
\begin{lstlisting}
%  DIMENSION - returns dimensions of hyperplanes in the array.
% 
%    dimsArr = DIMENSION(hypArr) - returns dimensions of hyperplanes
%        described by hyperplane structures in the array hypArr.
% 
%  Input:
%    regular:
%        hypArr: hyperplane [nDims1, nDims2, ...] - array
%            of hyperplanes.
% 
%  Output:
%        dimsArr: double[nDims1, nDims2, ...] - dimensions
%            of hyperplanes.
% 
%  $Author: Alex Kurzhanskiy <akurzhan@eecs.berkeley.edu>
%  $Copyright:  The Regents of the University of California 2004-2008 $
% 
%  $Author: Aushkap Nikolay <n.aushkap@gmail.com> $  $Date: 30-11-2012$
%  $Copyright: Moscow State University,
%    Faculty of Computational Mathematics and Computer Science,
%    System Analysis Department 2012 $
%

\end{lstlisting}
\fontfamily{\familydefault}
\selectfont
\item\hypertarget{/elltoolboxcore/@hyperplane\display}{display}
\fontfamily{pcr}
\selectfont
\begin{lstlisting}
%  DISPLAY - Displays hyperplane object.
% 
%  Input:
%    regular:
%        myHypArr: hyperplane [hpDim1, hpDim2, ...] - array
%            of hyperplanes.
% 
%  $Author: Alex Kurzhanskiy <akurzhan@eecs.berkeley.edu>
%  $Copyright:  The Regents of the University of California 2004-2008 $
% 
%  $Author: Aushkap Nikolay <n.aushkap@gmail.com> $  $Date: 07-12-2012$
%  $Copyright: Moscow State University,
%    Faculty of Computational Mathematics and Computer Science,
%    System Analysis Department 2012 $
%

\end{lstlisting}
\fontfamily{\familydefault}
\selectfont
\item\hypertarget{/elltoolboxcore/@hyperplane\double}{double}
\fontfamily{pcr}
\selectfont
\begin{lstlisting}
%  DOUBLE - return parameters of hyperplane - normal vector and shift.
% 
%    [normVec, hypScal] = DOUBLE(myHyp) - returns normal vector
%        and scalar value of the hyperplane.
% 
%  Input:
%    regular:
%        myHyp: hyperplane [1, 1] - single hyperplane of dimention nDims.
% 
%  Output:
%    normVec: double[nDims, 1] - normal vector of the hyperplane myHyp.
%    hypScal: double[1, 1] - scalar of the hyperplane myHyp.
% 
%  $Author: Alex Kurzhanskiy <akurzhan@eecs.berkeley.edu>
%  $Copyright:  The Regents of the University of California 2004-2008 $
% 
%  $Author: Aushkap Nikolay <n.aushkap@gmail.com> $  $Date: 30-11-2012$
%  $Copyright: Moscow State University,
%    Faculty of Computational Mathematics and Computer Science,
%    System Analysis Department 2012 $
%

\end{lstlisting}
\fontfamily{\familydefault}
\selectfont
\item\hypertarget{/elltoolboxcore/@hyperplane\eq}{eq}
\fontfamily{pcr}
\selectfont
\begin{lstlisting}
%  EQ - check if two hyperplanes are the same.
% 
%  Input:
%    regular:
%        fstHypArr: hyperplane [nDims1, nDims2, ...]/hyperplane [1, 1] -
%            first array of hyperplanes.
%        secHypArr: hyperplane [nDims1, nDims2, ...]/hyperplane [1, 1] -
%            second array of hyperplanes.
% 
%  Output:
%    isPosArr: logical[nDims1, nDims2, ...] - true -
%        if fstHypArr(iDim1, iDim2, ...) == secHypArr(iDim1, iDim2, ...),
%        false - otherwise. If size of fstHypArr is [1, 1], then checks
%        if fstHypArr == secHypArr(iDim1, iDim2, ...)
%        for all iDim1, iDim2, ... , and vice versa.
%    reportStr: char[1,] - comparison report
% 
% 
%  $Author: Vadim Kaushansky  <vkaushanskiy@gmail.com> $ $Date: Nov-2012$
%  $Copyright: Moscow State University,
%             Faculty of Computational Mathematics and Cybernetics,
%             System Analysis Department 2012 $
% 
%  $Authors:
%    Peter Gagarinov  <pgagarinov@gmail.com> $ $Date: Dec-2012$
%    Aushkap Nikolay <n.aushkap@gmail.com> $ $Date: Dec-2012$
%  $Copyright: Moscow State University,
%    Faculty of Computational Mathematics and Computer Science,
%    System Analysis Department 2012 $
%

\end{lstlisting}
\fontfamily{\familydefault}
\selectfont
\item\hypertarget{/elltoolboxcore/@hyperplane\getAbsTol}{getAbsTol}
\fontfamily{pcr}
\selectfont
\begin{lstlisting}
%  GETABSTOL - gives array the same size as hplaneArr with values
%              of absTol properties for each hyperplane in hplaneArr.
%  
%  Input:
%    regular:
%        hplaneArr: hyperplane[nDims1, nDims2,...] - hyperplane array.
%  
%  Output:
%    absTolArr: double[nDims1, nDims2, ...] - array of absTol properties
%        for hyperplanes in hplaneArr.
%  
%  $Author: Zakharov Eugene <justenterrr@gmail.com>$ $Date: 17-11-2012$
%  $Copyright: Moscow State University,
%    Faculty of Computational Mathematics and Computer Science,
%    System Analysis Department 2012 $
%  
%

\end{lstlisting}
\fontfamily{\familydefault}
\selectfont
\item\hypertarget{/elltoolboxcore/@hyperplane\hyperplane}{hyperplane}
\fontfamily{pcr}
\selectfont
\begin{lstlisting}
%  HYPERPLANE - creates hyperplane structure
%               (or array of hyperplane structures).
% 
%    Hyperplane H = { x in R^n : <v, x> = c },
%    with current "Properties"..
%    Here v must be vector in R^n, and c - scalar.
% 
%    hypH = HYPERPLANE - create empty hyperplane.
% 
%    hypH = HYPERPLANE(hypNormVec) - create
%        hyperplane object hypH with properties:
%            hypH.normal = hypNormVec,
%            hypH.shift = 0.
% 
%    hypH = HYPERPLANE(hypNormVec, hypConst) - create
%        hyperplane object hypH with properties:
%            hypH.normal = hypNormVec,
%            hypH.shift = hypConst.
% 
%    hypH = HYPERPLANE(hypNormVec, hypConst, ...
%        'absTol', absTolVal) - create
%        hyperplane object hypH with properties:
%            hypH.normal = hypNormVec,
%            hypH.shift = hypConst.
%            hypH.absTol = absTolVal
% 
%    hypObjArr = HYPERPLANE(hypNormArr, hypConstArr) - create
%        array of hyperplanes object just as
%        hyperplane(hypNormVec, hypConst).
% 
%    hypObjArr = HYPERPLANE(hypNormArr, hypConstArr, ...
%        'absTol', absTolValArr) - create
%        array of hyperplanes object just as
%        hyperplane(hypNormVec, hypConst, 'absTol', absTolVal).
% 
%  Input:
%    Case1:
%      regular:
%        hypNormArr: double[hpDims, nDims1, nDims2,...] -
%            array of vectors in R^hpDims. There hpDims -
%            hyperplane dimension.
% 
%    Case2:
%      regular:
%        hypNormArr: double[hpDims, nCols] /
%            / [hpDims, nDims1, nDims2,...] /
%            / [hpDims, 1] - array of vectors
%            in R^hpDims. There hpDims - hyperplane dimension.
%        hypConstArr: double[1, nCols] / [nCols, 1] /
%            / [nDims1, nDims2,...] /
%            / [nVecArrDim1, nVecArrDim2,...] -
%            array of scalar.
% 
%    Case3:
%      regular:
%        hypNormArr: double[hpDims, nCols] /
%            / [hpDims, nDims1, nDims2,...] /
%            / [hpDims, 1] - array of vectors
%            in R^hpDims. There hpDims - hyperplane dimension.
%        hypConstArr: double[1, nCols] / [nCols, 1] /
%            / [nDims1, nDims2,...] /
%            / [nVecArrDim1, nVecArrDim2,...] -
%            array of scalar.
%        absTolValArr: double[1, 1] - value of
%            absTol propeties.
% 
%      properties:
%        propMode: char[1,] - property mode, the following
%            modes are supported:
%            'absTol' - name of absTol properties.
% 
%            note: if size of hypNormArr is
%                [hpDims, nDims1, nDims2,...], then size of
%                hypConstArr is [nDims1, nDims2, ...] or
%                [1, 1], if size of hypNormArr [hpDims, 1],
%                then hypConstArr can be any size
%                [nVecArrDim1, nVecArrDim2, ...],
%                in this case output variable will has
%                size [nVecArrDim1, nVecArrDim2, ...].
%                If size of hypNormArr is [hpDims, nCols],
%                then size of hypConstArr may be
%                [1, nCols] or [nCols, 1],
%                output variable will has size
%                respectively [1, nCols] or [nCols, 1].
% 
%  Output:
%    hypObjArr: hyperplane [nDims1, nDims2...] /
%        / hyperplane [nVecArrDim1, nVecArrDim2, ...] -
%        array of hyperplane structure hypH:
%            hypH.normal - vector in R^hpDims,
%            hypH.shift  - scalar.
% 
%  $Author: Alex Kurzhanskiy <akurzhan@eecs.berkeley.edu>
%  $Copyright: The Regents of the University
%    of California 2004-2008 $
% 
%  $Author: Aushkap Nikolay <n.aushkap@gmail.com> $
%    $Date: 30-11-2012$
%  $Copyright: Moscow State University,
%    Faculty of Computational Mathematics and Computer
%    Science, System Analysis Department 2012 $
%

\end{lstlisting}
\fontfamily{\familydefault}
\selectfont
\item\hypertarget{/elltoolboxcore/@hyperplane\isempty}{isempty}
\fontfamily{pcr}
\selectfont
\begin{lstlisting}
%  ISEMPTY - checks if hyperplanes in H are empty.
% 
%  Input:
%    regular:
%        myHypArr: hyperplane [nDims1, nDims2, ...] - array
%            of hyperplanes.
% 
%  Output:
%    isPositiveArr: logical[nDims1, nDims2, ...],
%        isPositiveArr(iDim1, iDim2, ...) = true - if ellipsoid
%        myHypArr(iDim1, iDim2, ...) is empty, false - otherwise.
% 
%  $Author: Alex Kurzhanskiy <akurzhan@eecs.berkeley.edu>
%  $Copyright:  The Regents of the University of California 2004-2008 $
% 
%  $Author: Aushkap Nikolay <n.aushkap@gmail.com> $  $Date: 30-11-2012$
%  $Copyright: Moscow State University,
%    Faculty of Computational Mathematics and Computer Science,
%    System Analysis Department 2012 $
%

\end{lstlisting}
\fontfamily{\familydefault}
\selectfont
\item\hypertarget{/elltoolboxcore/@hyperplane\isparallel}{isparallel}
\fontfamily{pcr}
\selectfont
\begin{lstlisting}
%  ISPARALLEL - check if two hyperplanes are parallel.
% 
%    isResArr = ISPARALLEL(fstHypArr, secHypArr) - Checks if hyperplanes
%        in fstHypArr are parallel to hyperplanes in secHypArr and
%        returns array of true and false of the size corresponding
%        to the sizes of fstHypArr and secHypArr.
% 
%  Input:
%    regular:
%        fstHypArr: hyperplane [nDims1, nDims2, ...] - first array
%            of hyperplanes
%        secHypArr: hyperplane [nDims1, nDims2, ...] - second array
%            of hyperplanes
% 
%  Output:
%    isPosArr: logical[nDims1, nDims2, ...] - 
%        isPosArr(iFstDim, iSecDim, ...) = true - 
%        if fstHypArr(iFstDim, iSecDim, ...) is parallel 
%        secHypArr(iFstDim, iSecDim, ...), false - otherwise.
% 
%  $Author: Alex Kurzhanskiy <akurzhan@eecs.berkeley.edu>
%  $Copyright:  The Regents of the University of California 2004-2008 $
% 
%  $Author: Aushkap Nikolay <n.aushkap@gmail.com> $  $Date: 30-11-2012$
%  $Copyright: Moscow State University,
%    Faculty of Computational Mathematics and Computer Science,
%    System Analysis Department 2012 $
%

\end{lstlisting}
\fontfamily{\familydefault}
\selectfont
\item\hypertarget{/elltoolboxcore/@hyperplane\ne}{ne}
\fontfamily{pcr}
\selectfont
\begin{lstlisting}
%  NE - The opposite of EQ.
% 
%  Input:
%    regular:
%        fstHypArr: hyperplane [nDims1, nDims2, ...]/hyperplane [1, 1] -
%            first array of hyperplanes.
%        secHypArr: hyperplane [nDims1, nDims2, ...]/hyperplane [1, 1] -
%            second array of hyperplanes.
% 
%  Output:
%    isPosArr: logical[nDims1, nDims2, ...] - false -
%        if fstHypArr(iDim1, iDim2, ...) == secHypArr(iDim1, iDim2, ...),
%        true - otherwise. If size of fstHypArr is [1, 1], then checks
%        if fstHypArr == secHypArr(iDim1, iDim2, ...)
%        for all iDim1, iDim2, ... , and vice versa.
%    reportStr: char[1,] - comparison report
% 
%  $Author: Alex Kurzhanskiy <akurzhan@eecs.berkeley.edu>
%  $Copyright:  The Regents of the University of California 2004-2008 $
% 
%  $Authors:
%    Peter Gagarinov  <pgagarinov@gmail.com> $ $Date: Dec-2012$
%    Aushkap Nikolay <n.aushkap@gmail.com> $ $Date: Dec-2012$
%  $Copyright: Moscow State University,
%    Faculty of Computational Mathematics and Computer Science,
%    System Analysis Department 2012 $
%

\end{lstlisting}
\fontfamily{\familydefault}
\selectfont
\item\hypertarget{/elltoolboxcore/@hyperplane\parameters}{parameters}
\fontfamily{pcr}
\selectfont
\begin{lstlisting}
%  PARAMETERS - return parameters of hyperplane - normal vector and shift.
% 
%    [normVec, hypScal] = PARAMETERS(myHyp) - returns normal vector
%        and scalar value of the hyperplane.
% 
%  Input:
%    regular:
%        myHyp: hyperplane [1, 1] - single hyperplane of dimention nDims.
% 
%  Output:
%    normVec: double[nDims, 1] - normal vector of the hyperplane myHyp.
%    hypScal: double[1, 1] - scalar of the hyperplane myHyp.
% 
%  $Author: Alex Kurzhanskiy <akurzhan@eecs.berkeley.edu>
%  $Copyright:  The Regents of the University of California 2004-2008 $
% 
%  $Author: Aushkap Nikolay <n.aushkap@gmail.com> $  $Date: 30-11-2012$
%  $Copyright: Moscow State University,
%    Faculty of Computational Mathematics and Computer Science,
%    System Analysis Department 2012 $
%

\end{lstlisting}
\fontfamily{\familydefault}
\selectfont
\item\hypertarget{/elltoolboxcore/@hyperplane\plot}{plot}
\fontfamily{pcr}
\selectfont
\begin{lstlisting}
%  PLOT - plots hyperplanes in 2D or 3D.
% 
% 
%  Usage:
%        plot(h) - plots hyperplane H in default (red) color.
%        plot(hM) -plots hyperplanes contained in hyperplane matrix.
%        plot(hM1, 'cSpec1', hM2, 'cSpec1',...) - plots hyperplanes in h1 in
%            cSpec1 color, hyperplanes in h2 in cSpec2 color, etc.
%        plot(hM1, hM2,..., hMn, option) - plots h1,...,hn using options given
%            in the option structure.
% 
%  Input:
%    regular:
%        hMat: hyperplane[m,n] - matrix of 2D or 3D hyperplanes. All hyperplanes
%              in hM must be either 2D or 3D simutaneously.
%    optional:
%        colorSpec: char[1,1] - specify wich color hyperplane plots will
%                   have
%        option: structure[1,1], containing some of follwing fields:
%            option.newfigure: boolean[1,1]   - if 1, each plot command will open a new figure window.
%            option.size: double[1,1] - length of the line segment in 2D, or square diagonal in 3D.
%            option.center: double[1,1] - center of the line segment in 2D, of the square in 3D.
%            option.width: double[1,1] - specifies the width (in points) of the line for 2D plots.
%            option.color: double[1,3] - sets default colors in the form [x y z], .
%            option.shade = 0-1 - level of transparency (0 - transparent, 1 - opaque).
%            NOTE: if using options and colorSpec simutaneously, option.color is
%            ignored
% 
%  Output:
%    regular:
%        figHandleVec: double[1,n] - array with handles of figures hyperplanes
%        were plotted in. Where n is number of figures.
% 
%  
%  $Author: Alex Kurzhanskiy <akurzhan@eecs.berkeley.edu>
%  $Copyright:  The Regents of the University of California 2004-2008 $
% 
%  $Author: <Zakharov Eugene>  <justenterrr@gmail.com> $    $Date: <1 november> $
%  $Copyright: Moscow State University,
%             Faculty of Computational Mathematics and Computer Science,
%             System Analysis Department <2012> $
%

\end{lstlisting}
\fontfamily{\familydefault}
\selectfont
\item\hypertarget{/elltoolboxcore/@hyperplane\uminus}{uminus}
\fontfamily{pcr}
\selectfont
\begin{lstlisting}
%  UMINUS - switch signs of normal vector and the shift scalar
%           to the opposite.
% 
%  Input:
%    regular:
%        inpHypArr: hyperplane [nDims1, nDims2, ...] - array
%            of hyperplanes.
% 
%  Output:
%    outHypArr: hyperplane [nDims1, nDims2, ...] - array
%        of the same hyperplanes as in inpHypArr whose
%        normals and scalars are multiplied by -1.
% 
%  $Author: Alex Kurzhanskiy <akurzhan@eecs.berkeley.edu>
%  $Copyright:  The Regents of the University of California 2004-2008 $
% 
%  $Author: Aushkap Nikolay <n.aushkap@gmail.com> $  $Date: 30-11-2012$
%  $Copyright: Moscow State University,
%    Faculty of Computational Mathematics and Computer Science,
%    System Analysis Department 2012 $
%

\end{lstlisting}
\fontfamily{\familydefault}
\selectfont
\end{enumerate}
\subsection{/elltoolboxcore/auxiliary}
\begin{enumerate}
\item\hypertarget{/elltoolboxcore/auxiliary\ell\_enclose}{ell\_enclose}
\fontfamily{pcr}
\selectfont
\begin{lstlisting}
%  ELL_ENCLOSE - computes minimum volume ellipsoid that contains given vectors.
% 
% 
%  Description:
%  ------------
% 
%     E = ELL_ENCLOSE(V)  Given vectors specified as columns of matrix V,
%                         compute minimum volume ellipsoid E that contains them.
% 
% 
%  Output:
%  -------
% 
%     E - computed ellipsoid.
% 
% 
%  See also:
%  ---------
% 
%     ELLIPSOID/ISINTERNAL, ELLUNION_EA;
%     POLYTOPE/getOutterEllipsoid.
% 
%

\end{lstlisting}
\fontfamily{\familydefault}
\selectfont
\item\hypertarget{/elltoolboxcore/auxiliary\ell\_fusionlambda}{ell\_fusionlambda}
\fontfamily{pcr}
\selectfont
\begin{lstlisting}
%  ELL_FUSIONLAMBDA - function whose root in the interval (0, 1) determines
%                     the minimal volume ellipsoid overapproximating the
%                     intersection of two ellipsoids.
% 
%  This function is called from ELLIPSOID/INTERSECTION_EA by FZERO.
% 
%

\end{lstlisting}
\fontfamily{\familydefault}
\selectfont
\item\hypertarget{/elltoolboxcore/auxiliary\ell\_inv}{ell\_inv}
\fontfamily{pcr}
\selectfont
\begin{lstlisting}
%  ELL_INV - computes matrix inverse treating ill-conditioned matrices properly.
% 
% 
%  Description:
%  ------------
% 
%     I = ELL_INV(A)  Given two square nonsingular matrix A, returns its inverse.
% 
% 
%  Output:
%  -------
% 
%     I - inverse of matrix A.
% 
% 
%  See also:
%  ---------
% 
%     INV, COND.
% 
%

\end{lstlisting}
\fontfamily{\familydefault}
\selectfont
\item\hypertarget{/elltoolboxcore/auxiliary\ell\_simdiag}{ell\_simdiag}
\fontfamily{pcr}
\selectfont
\begin{lstlisting}
%  ELL_SIMDIAG - computes the transformation matrix that simultaneously
%                diagonalizes two symmetric matrices.
% 
% 
%  Description:
%  ------------
% 
%     T = ELL_SIMDIAG(A, B)  Given two symmetric matrices, A and B, with A being
%                            positive definite, find nonsingular transformation
%                            matrix T such that
%                                        T A T' = I
%                                        T B T' = D
%                            where I is identity matrix, and D is diagonal. 
% 
%     General info.
%     Two matrices are said to be simultaneously diagonalizable if they are
%     diagonalized by a same invertible matrix. That is, they share full rank
%     of linearly independent eigenvectors. Two square matrices of the same
%     dimension are simultaneously diagonalizable if and only if they are
%     diagonalizable and commutative, or these matrices are symmetric and
%     one of them is positive definite.
% 
% 
%  Output:
%  -------
% 
%     T - tranformation matrix.
% 
% 
%  See also:
%  ---------
% 
%     SVD, GSVD.
% 
%

\end{lstlisting}
\fontfamily{\familydefault}
\selectfont
\item\hypertarget{/elltoolboxcore/auxiliary\ell\_unitball}{ell\_unitball}
\fontfamily{pcr}
\selectfont
\begin{lstlisting}
%  ELL_UNITBALL - creates unit ball object
% 
% 
%  Description:
%  ------------
% 
%     B = ELL_UNITBALL(N)  Creates an ellipsoid in R^N with identity shape matrix,
%                          centered at the origin.
% 
% 
%  Output:
%  -------
% 
%     B = { x : <x, x> <= 1 } - unit ball.
% 
% 
%  See also:
%  ---------
% 
%     ELLIPSOID/ELLIPSOID
% 
%

\end{lstlisting}
\fontfamily{\familydefault}
\selectfont
\item\hypertarget{/elltoolboxcore/auxiliary\ell\_valign}{ell\_valign}
\fontfamily{pcr}
\selectfont
\begin{lstlisting}
%  ELL_VALIGN - given two vectors in R^n, computes orthogonal matrix that rotates
%               the second vector making it parallel to the first one.
% 
% 
%  Description:
%  ------------
% 
%     T = ELL_VALIGN(v, x)  Given vectors v and x in R^n, compute orthogonal
%                           matrix T (TT' = T'T = I), such that
%                                 T x = a v,
%                           where a is some scalar. Actually,
%                                 a = |x|/|v|
%                           Here |.| denotes euclidean norm.
% 
%     Let SVD of v be
%                     v = U1 * S1 * V1',
%     and SVD of x
%                     x = U2 * S2 * V2'.
%     Then we can find T from the matrix equation
%                     T U2 V2' = U1 V1',
%     or,
%                     T = U1 V1' (U2 V2')^(-1) = U1 V1' V2 U2'.
% 
% 
%  Output:
%  -------
% 
%     T - resulting orthogonal matrix.
% 
% 
%  See also:
%  ---------
% 
%     SVD, GSVD.
% 
%

\end{lstlisting}
\fontfamily{\familydefault}
\selectfont
\item\hypertarget{/elltoolboxcore/auxiliary\hyperplane2polytope}{hyperplane2polytope}
\fontfamily{pcr}
\selectfont
\begin{lstlisting}
%  HYPERPLANE2POLYTOPE - converts array of hyperplanes into polytope
% 
% 
%  Description:
%  ------------
% 
%     P = HYPERPLANE2POLYTOPE(HA)  Given array of hyperplane objects HA, 
%                                  returns polytope object.
%                                  Requires Multi-Parametric Toolbox.
% 
% 
%  Output:
%  -------
% 
%     P - polytope.
% 
% 
%  See also:
%  ---------
% 
%     HYPERPLANE/HYPERPLANE, POLYTOPE/POLYTOPE, POLYTOPE2HYPERPLANE.
% 
%

\end{lstlisting}
\fontfamily{\familydefault}
\selectfont
\item\hypertarget{/elltoolboxcore/auxiliary\polytope2hyperplane}{polytope2hyperplane}
\fontfamily{pcr}
\selectfont
\begin{lstlisting}
%  POLYTOPE2HYPERPLANE - converts given polytope object into the array
%                        of hyperplanes.
% 
% 
%  Description:
%  ------------
% 
%     HA = POLYTOPE2HYPERPLANE(P)  Given polytope object P, returns array of
%                                  hyperplane objects HA.
%                                  Requires Multi-Parametric Toolbox.
% 
% 
%  Output:
%  -------
% 
%     HA - array of hyperplanes.
% 
% 
%  See also:
%  ---------
% 
%     POLYTOPE/POLYTOPE, HYPERPLANE/HYPERPLANE, HYPERPLANE2POLYTOPE.
% 
%

\end{lstlisting}
\fontfamily{\familydefault}
\selectfont
\end{enumerate}
\subsection{/elltoolboxcore/control/auxiliary}
\begin{enumerate}
\item\hypertarget{/elltoolboxcore/control/auxiliary\ell\_center\_ode}{ell\_center\_ode}
\fontfamily{pcr}
\selectfont
\begin{lstlisting}
%  ELL_CENTER_ODE - ODE for the center of the reach set.
% 
%

\end{lstlisting}
\fontfamily{\familydefault}
\selectfont
\item\hypertarget{/elltoolboxcore/control/auxiliary\ell\_eedist\_ode}{ell\_eedist\_ode}
\fontfamily{pcr}
\selectfont
\begin{lstlisting}
%  ELL_EEDIST_ODE - ODE for the shape matrix of the external ellipsoid
%                   for system with disturbance.
% 
%

\end{lstlisting}
\fontfamily{\familydefault}
\selectfont
\item\hypertarget{/elltoolboxcore/control/auxiliary\ell\_eesm\_ode}{ell\_eesm\_ode}
\fontfamily{pcr}
\selectfont
\begin{lstlisting}
%  ELL_EESM_ODE - ODE for the shape matrix of the external ellipsoid.
% 
%

\end{lstlisting}
\fontfamily{\familydefault}
\selectfont
\item\hypertarget{/elltoolboxcore/control/auxiliary\ell\_iedist\_ode}{ell\_iedist\_ode}
\fontfamily{pcr}
\selectfont
\begin{lstlisting}
%  ELL_IEDIST_ODE - ODE for the shape matrix of the internal ellipsoid
%                   for system with disturbance.
% 
%

\end{lstlisting}
\fontfamily{\familydefault}
\selectfont
\item\hypertarget{/elltoolboxcore/control/auxiliary\ell\_iesm\_ode}{ell\_iesm\_ode}
\fontfamily{pcr}
\selectfont
\begin{lstlisting}
%  ELL_IESM_ODE - ODE for the shape matrix of the internal ellipsoid.
% 
%

\end{lstlisting}
\fontfamily{\familydefault}
\selectfont
\item\hypertarget{/elltoolboxcore/control/auxiliary\ell\_ode\_solver}{ell\_ode\_solver}
\fontfamily{pcr}
\selectfont
\begin{lstlisting}
%  ELL_ODE_SOLVER - caller for particular ODE solver.
% 
%

\end{lstlisting}
\fontfamily{\familydefault}
\selectfont
\item\hypertarget{/elltoolboxcore/control/auxiliary\ell\_regularize}{ell\_regularize}
\fontfamily{pcr}
\selectfont
\begin{lstlisting}
%  ELL_REGULARIZE - regularization of singular matrix.
% 
%

\end{lstlisting}
\fontfamily{\familydefault}
\selectfont
\item\hypertarget{/elltoolboxcore/control/auxiliary\ell\_stm\_ode}{ell\_stm\_ode}
\fontfamily{pcr}
\selectfont
\begin{lstlisting}
%  ELL_STM_ODE - ODE for state transition matrix.
% 
%

\end{lstlisting}
\fontfamily{\familydefault}
\selectfont
\item\hypertarget{/elltoolboxcore/control/auxiliary\ell\_value\_extract}{ell\_value\_extract}
\fontfamily{pcr}
\selectfont
\begin{lstlisting}
%  ELL_VALUE_EXTRACT - extracts matrix value from ppform or vector array.
% 
%

\end{lstlisting}
\fontfamily{\familydefault}
\selectfont
\item\hypertarget{/elltoolboxcore/control/auxiliary\iesm\_ode}{iesm\_ode}
\fontfamily{pcr}
\selectfont
\begin{lstlisting}
%  ELL_IESM_ODE - ODE for the shape matrix of the internal ellipsoid.
% 
%

\end{lstlisting}
\fontfamily{\familydefault}
\selectfont
\end{enumerate}
\subsection{/elltoolboxcore/demo}
\begin{enumerate}
\item\hypertarget{/elltoolboxcore/demo\ell\_demo0}{ell\_demo0}
\fontfamily{pcr}
\selectfont
\begin{lstlisting}
%  Demo of the ellipsoidal calculus.
% 
% 
%  Author:
%  -------
% 
%  Alex Kurzhanskiy <akurzhan@eecs.berkeley.edu>
% 
%

\end{lstlisting}
\fontfamily{\familydefault}
\selectfont
\item\hypertarget{/elltoolboxcore/demo\ell\_demo1}{ell\_demo1}
\fontfamily{pcr}
\selectfont
\begin{lstlisting}
%  Demo of the ellipsoidal calculus.
% 
% 
%  Author:
%  -------
% 
%  Alex Kurzhanskiy <akurzhan@eecs.berkeley.edu>
% 
%

\end{lstlisting}
\fontfamily{\familydefault}
\selectfont
\item\hypertarget{/elltoolboxcore/demo\ell\_demo2}{ell\_demo2}
\fontfamily{pcr}
\selectfont
\begin{lstlisting}
%  Demo of the ellipsoid visualization.
% 
% 
%  Author:
%  -------
% 
%  Alex Kurzhanskiy <akurzhan@eecs.berkeley.edu>
% 
%

\end{lstlisting}
\fontfamily{\familydefault}
\selectfont
\item\hypertarget{/elltoolboxcore/demo\ell\_demo3}{ell\_demo3}
\fontfamily{pcr}
\selectfont
\begin{lstlisting}
%  Reachability Demo.
% 
%

\end{lstlisting}
\fontfamily{\familydefault}
\selectfont
\end{enumerate}
\subsection{/elltoolboxcore/graphics}
\begin{enumerate}
\item\hypertarget{/elltoolboxcore/graphics\ell\_plot}{ell\_plot}
\fontfamily{pcr}
\selectfont
\begin{lstlisting}
%  Description:
%  ------------
% 
%     Wrapper for PLOT and PLOT3 functions.
%     First argument must be a vector, or an array of vectors, in 1D, 2D or 3D.
%     Other arguments are the same as for PLOT and PLOT3 functions.
% 
% 
%  Output:
%  -------
% 
%     Plot handle.
% 
% 
%  See also:
%  ---------
% 
%     PLOT, PLOT3.
% 
%

\end{lstlisting}
\fontfamily{\familydefault}
\selectfont
\item\hypertarget{/elltoolboxcore/graphics\ell\_square\_facets}{ell\_square\_facets}
\fontfamily{pcr}
\selectfont
\begin{lstlisting}
%  ELL_SQUARE_FACETS - generates square facets to be used in PATCH function call.
% 
% 
%  Description:
%  ------------
% 
%     ELL_SQUARE_FACETS(M, N)  Generates square facets for the PATCH call.
% 
% 
%  Output:
%  -------
% 
%     Array of facets.
% 
% 
%  See also:
%  ---------
% 
%     PATCH, CONVHULLN.
% 
%

\end{lstlisting}
\fontfamily{\familydefault}
\selectfont
\item\hypertarget{/elltoolboxcore/graphics\ell\_triag\_facets}{ell\_triag\_facets}
\fontfamily{pcr}
\selectfont
\begin{lstlisting}
%  ELL_TRIAG_FACETS - generates triangular facets to be used in PATCH function call.
% 
% 
%  Description:
%  ------------
% 
%     ELL_TRIAG_FACETS(M, N)  Generates triangular facets for the PATCH call.
% 
% 
%  Output:
%  -------
% 
%     Array of facets.
% 
% 
%  See also:
%  ---------
% 
%     PATCH, CONVHULLN.
% 
%

\end{lstlisting}
\fontfamily{\familydefault}
\selectfont
\end{enumerate}
\subsection{/elltoolboxcore/solvers/gradient}
\begin{enumerate}
\item\hypertarget{/elltoolboxcore/solvers/gradient\ell\_nlfnlc}{ell\_nlfnlc}
\fontfamily{pcr}
\selectfont
\begin{lstlisting}
%  ELL_NLFNLC - computes minimum of nonlinear function with nonlinear constraints.
% 
% 
%  Description:
%  ------------
% 
%     [X, FVAL] = ELL_NLFNLC(OBJFUN, X0, NLCF)  Find minimum of the function
%                 specified by OBJFUN (inline or function handler) with nonlinear
%                 constraints specified by NLCF (inline or function handler) using
%                 gradient optimization method starting at initial vector X0.
%     [X, FVAL] = ELL_NLFNLC(OBJFUN, X0, NLCF, OPTIONS, P1, P2, ...)
%                 In OPTIONS parameter the user can specify if he wants
%                 to provide his own gradient values by setting
%                            OPTIONS.fungrad = 1,
%                 and/or
%                            OPTIONS.congrad = 1
%                 P1, P2, ... are optional parameters that are passed to
%                 the objective function OBJFUN and to the constraint function NLCF.
% 
%     Function OBJFUN takes X as input parameter and returns value of the nonlinear
%     objective function at that point. If OPTIONS.fungrad is set to 1, it
%     also returns the value of the gradient of this function at that point.
%     Function NLCF takes vector X as input and returns a pair of matrices
%     [A, B], that describes nonlinear constraints on X in the form
%                           A X <= 0,
%                           B X  = 0.
%     Either A or B (but not both) can be empty. If OPTIONS.congrad is set to 1,
%     then NLCF returns [A, B, C, D] where C is partial derivatives of the
%     constraint vector of inequalities A, and D - partial derivatives of 
%     constraint vector of equalities B.
% 
%     Example of how ELL_NLFNLC function is used can be found in ELLIPSOID/DISTANCE.
% 
% 
%  Output:
%  -------
% 
%     X    - vector at which the minimum is achieved,
%     FVAL - value of objective function at minimum.
% 
% 
%  See also:
%  ---------
% 
%     YALMIP, SEDUMI.
% 
%

\end{lstlisting}
\fontfamily{\familydefault}
\selectfont
\end{enumerate}
\subsection{/elltoolboxcore/solvers/gradient/private}
\begin{enumerate}
\item\hypertarget{/elltoolboxcore/solvers/gradient/private\compute\_direction}{compute\_direction}
\fontfamily{pcr}
\selectfont
\begin{lstlisting}
%  COMPUTE_DIRECTION - computes a search direction in a subspace defined by Z.
% 
%

\end{lstlisting}
\fontfamily{\familydefault}
\selectfont
\item\hypertarget{/elltoolboxcore/solvers/gradient/private\nlcp\_solve}{nlcp\_solve}
\fontfamily{pcr}
\selectfont
\begin{lstlisting}
%  NLCP_SOLVE - nonlinear function minimizer under nonlinear constraints.
% 
%

\end{lstlisting}
\fontfamily{\familydefault}
\selectfont
\item\hypertarget{/elltoolboxcore/solvers/gradient/private\qps}{qps}
\fontfamily{pcr}
\selectfont
\begin{lstlisting}
%  QPS - Quadratic programming problem.
% 
%             min 0.5*x'Hx + f'x   subject to:  Ax <= b 
%              x    
% 
%

\end{lstlisting}
\fontfamily{\familydefault}
\selectfont
\end{enumerate}
\end{document}
